\documentclass{beamer}

\newcommand{\week}{Banks 3 of 6}

\title{Bank Earnings}
\subtitle{Mishkin Chapter 9}
\author{Econ 357}
\date{\week}

% Reference the shared preamble
\setbeamertemplate{frametitle}{
  \vspace{0.5em}
  \insertframetitle
  \par
  \vspace{0.5em}
  \hrule
  \vspace{0.3em}
  {\small\color{gray}\insertframesubtitle}
}

\setbeamertemplate{navigation symbols}{}
\setbeamertemplate{itemize item}{\textbullet} % main bullet: filled dot
\setbeamertemplate{itemize subitem}{\normalsize$\circ$} % sub-bullet: empty dot
\setbeamertemplate{itemize subsubitem}{\scriptsize--} % sub-sub-bullet: dash


% Font changes
\usepackage[scaled=0.92]{helvet}
\renewcommand{\familydefault}{\sfdefault}

% Packages
\usepackage{tikz}
\usepackage{booktabs}
\usepackage{xcolor}
\usepackage{array}           % Enhanced column types for tables
\usepackage{multirow}        % Spanning multiple rows in tables
\usepackage{makecell}        % Line breaks and formatting in table cells
\usepackage{siunitx}         % Proper formatting of numbers and units
\usepackage{amsmath}         % Enhanced math environments
\usepackage{amsfonts}        % Additional math fonts
\usepackage{amssymb}         % Additional math symbols
\usepackage{url}             % Better URL formatting
\usepackage{graphicx}        % Enhanced graphics support
\usepackage{tabularray}
\UseTblrLibrary{booktabs, siunitx, varwidth}
% For financial presentations specifically
\usepackage{eurosym}         % Euro symbol
\usepackage{textcomp}        % Additional text symbols
\usepackage{hyperref}        % Hyperlinks (should be loaded last)

% Define a footnote
\renewcommand{\footnoterule}{\vspace*{-3pt}\hrule width 2in height 0.4pt\vspace*{2.6pt}}

% Define a Foundation Slide
\newenvironment{foundframe}[1][t]{
    \setbeamercolor{background canvas}{bg=gray!8}
    \setbeamercolor{frametitle}{fg=gray!80!black,bg=gray!25}
    \setbeamercolor{framesubtitle}{fg=gray!70!black,bg=gray!15}
    \setbeamercolor{item}{fg=gray!80!black}
    \setbeamercolor{enumerate item}{fg=gray!80!black}
    
    % Modify the frametitle template for this frame type
    \setbeamertemplate{frametitle}{
        \vspace{0.5em}
        \begin{minipage}[t]{0.75\textwidth}
            \insertframetitle
            \par
            \vspace{0.5em}
            \hrule
            \vspace{0.3em}
            {\small\color{gray}\insertframesubtitle}
        \end{minipage}%
        \hfill
        \begin{minipage}[t]{0.2\textwidth}
            \raggedleft
            \colorbox{gray!30}{%
                \scriptsize\bfseries\color{gray!80!black}%
                   \hspace{3pt}\begin{tabular}{c}Foundation\\Material\end{tabular}\hspace{3pt}%
            }
        \end{minipage}
        \vspace{0.3em}
    }
    
    \begin{frame}[#1]
}{
    \end{frame}
}

% Define Practice Slide
\newenvironment{practiceframe}[1][t]{
    \setbeamercolor{background canvas}{bg=white}
    \setbeamercolor{frametitle}{fg=blue!80!black,bg=blue!15}
    \setbeamercolor{framesubtitle}{fg=blue!70!black,bg=blue!10}
    \setbeamercolor{item}{fg=blue!80!black}
    \setbeamercolor{enumerate item}{fg=blue!80!black}
    \setbeamercolor{normal text}{fg=blue!90!black}
    
    % Modify the frametitle template for this frame type
    \setbeamertemplate{frametitle}{
        \vspace{0.5em}
        \begin{minipage}[t]{0.75\textwidth}
            \insertframetitle
            \par
            \vspace{0.5em}
            \hrule
            \vspace{0.3em}
            {\small\color{blue!70!black}\insertframesubtitle}
        \end{minipage}%
        \hfill
        \begin{minipage}[t]{0.2\textwidth}
            \raggedleft
            \colorbox{blue!20}{%
                \scriptsize\bfseries\color{blue!80!black}%
                   \hspace{3pt}\begin{tabular}{c}Practice\\Questions\end{tabular}\hspace{3pt}%
            }
        \end{minipage}
        \vspace{0.3em}
    }
    
    \begin{frame}[#1]
}{
    \end{frame}
}

% Define Excel Slide
\newenvironment{excelframe}[1][t]{
    \setbeamercolor{background canvas}{bg=white}
    \setbeamercolor{frametitle}{fg=blue!80!black,bg=blue!15}
    \setbeamercolor{framesubtitle}{fg=blue!70!black,bg=blue!10}
    \setbeamercolor{item}{fg=blue!80!black}
    \setbeamercolor{enumerate item}{fg=blue!80!black}
    \setbeamercolor{normal text}{fg=blue!90!black}
    
    % Modify the frametitle template for this frame type
    \setbeamertemplate{frametitle}{
        \vspace{0.5em}
        \begin{minipage}[t]{0.75\textwidth}
            \insertframetitle
            \par
            \vspace{0.5em}
            \hrule
            \vspace{0.3em}
            {\small\color{blue!70!black}\insertframesubtitle}
        \end{minipage}%
        \hfill
        \begin{minipage}[t]{0.2\textwidth}
            \raggedleft
            \colorbox{green!10}{%
                \scriptsize\bfseries\color{blue!80!black}%
                   \hspace{3pt}\begin{tabular}{c}MS Excel\end{tabular}\hspace{3pt}%
            }
        \end{minipage}
        \vspace{0.3em}
    }
    
    \begin{frame}[#1]
}{
    \end{frame}
}

% Define Caution Slide
\newenvironment{cautionframe}[1][t]{
    \setbeamercolor{background canvas}{bg=white}
    \setbeamercolor{frametitle}{fg=blue!80!black,bg=blue!15}
    \setbeamercolor{framesubtitle}{fg=blue!70!black,bg=blue!10}
    \setbeamercolor{item}{fg=blue!80!black}
    \setbeamercolor{enumerate item}{fg=blue!80!black}
    \setbeamercolor{normal text}{fg=blue!90!black}
    
    % Modify the frametitle template for this frame type
    \setbeamertemplate{frametitle}{
        \vspace{0.5em}
        \begin{minipage}[t]{0.75\textwidth}
            \insertframetitle
            \par
            \vspace{0.5em}
            \hrule
            \vspace{0.3em}
            {\small\color{blue!70!black}\insertframesubtitle}
        \end{minipage}%
        \hfill
        \begin{minipage}[t]{0.2\textwidth}
            \raggedleft
            \colorbox{red!10}{%
                \scriptsize\bfseries\color{blue!80!black}%
                   \hspace{3pt}\begin{tabular}{c}Caution\end{tabular}\hspace{3pt}%
            }
        \end{minipage}
        \vspace{0.3em}
    }
    
    \begin{frame}[#1]
}{
    \end{frame}
}

% Add to footnotes
\makeatletter
\newcommand\blfootnote[1]{%
  \begingroup
  \renewcommand\thefootnote{}%
  \renewcommand\@makefntext[1]{\raggedright\leftskip=0pt ##1}%
  \footnote{\scriptsize #1}%
  \addtocounter{footnote}{-1}%
  \endgroup
}
\makeatother

% Set the footer -- change 
\setbeamertemplate{footline}{
  \leavevmode%
  \vspace{2ex}
  \hbox{%
    % Left box: Econ 457
    \begin{beamercolorbox}[wd=.4\paperwidth,ht=2.5ex,dp=1ex,left]{author in head/foot}%
      \hspace{1em}Econ 357
    \end{beamercolorbox}%
    % Middle box: Week
    \begin{beamercolorbox}[wd=.2\paperwidth,ht=2.5ex,dp=1ex,center]{date in head/foot}%
      \centering\week
    \end{beamercolorbox}%
    % Right box: Slide numbers
    \begin{beamercolorbox}[wd=.4\paperwidth,ht=2.5ex,dp=1ex,center]{date in head/foot}%
      \hfill\insertframenumber{} 
    \end{beamercolorbox}%
  }%
  \vskip0pt%
}

\begin{document}

\frame{\titlepage}

\begin{frame}
    \frametitle{Outline for Banks}
        \begin{enumerate}
            \item Role of Banks
            \item Bank Balance Sheets
            \item \fbox{Bank Earnings}
            \item Bank Management (and Investment Banks)
            \item Bank Failures
            \item Bank Regulation
        \end{enumerate}
\end{frame}


\begin{frame}
    \frametitle{Outline for Today's Lecture}
    
    \begin{enumerate}
        \item Review: Bank Balance Sheets
        \item Required Ratios
        \item Balance Sheet Changes
        \begin{itemize}
        \item Practice
        \end{itemize}
        \item How Banks Make Money
        \begin{itemize}
            \item Net Interest Margin
            \item Impact of Leverage
            \item Practice
        \end{itemize}
    \end{enumerate}

\end{frame}

\begin{frame}[t]
    \frametitle{1. Review: Bank Balance Sheets}
    \framesubtitle{Assets}

    \centering
    \includegraphics[width=0.8\textwidth]{Econ357/figures/03_banks_02_assets.png}

    \blfootnote{Source: FDIC Quarterly Banking Profile, 2024-q2}
    
\end{frame}

\begin{frame}[t]
    \frametitle{1. Review: Bank Balance Sheets}
    \framesubtitle{Liabilities}

    \centering
    \includegraphics[width=0.8\textwidth]{Econ357/figures/03_banks_02_liabilities.png}

    \blfootnote{Source: FDIC Quarterly Banking Profile, 2024-q2}
    
\end{frame}

\begin{frame}[t]
    \frametitle{1. Review: Bank Balance Sheets}
    \framesubtitle{Capital}

    A hypothetical bank's capital stack may look like this:

    \centering
    \includegraphics[width=\textwidth]{Econ357/figures/03_banks_02_bank_capital_stack.png}

\end{frame}

\begin{frame}[t]
    \frametitle{2. Required Ratios}

    There are two important limits that banks must adhere to:
    \vspace{1em}

    \begin{enumerate}
        \item \textbf{Reserve Requirements}: Banks are required to maintain a minimum amount of reserves.   Usually reserves must be maintained at $\approx$10\% deposits.
        $$\text{Required Reserves} = \frac{\text{Reserves}}{\text{Deposits}}$$
        \item \textbf{Capital Ratios}: Banks are required to maintain a minimum amount of capital.   (You could alternatively think about this as a restriction on bank leverage)  The minimum capital ratio for most banks is around 10\% and could be as high as 15\% for larger banks.
        $$\text{Capital Ratio} = \frac{\text{Capital}}{\text{Assets}}$$
    \end{enumerate}
    
\end{frame}

\begin{frame}[t]
    \frametitle{3. Bank Balance Sheet Changes}

    What happens when you deposit money at a bank?
    
    \begin{table}
    \centering
    \caption{\textbf{Your Bank}}
    \begin{tblr}{
      colspec = { Q[c,wd=2cm] Q[c,wd=1cm] |Q[c,wd=2cm] Q[c,wd=1cm] }, vline{3} = {1-Z}{}
    }
    \toprule
    Assets && Liabilities \\
    \midrule
    Cash & +\$100 & Deposits & +\$100 \\
    \end{tblr}
    \end{table}

\end{frame}

\begin{frame}[t]
    \frametitle{3. Bank Balance Sheet Changes}

    What happens when you write a check to somebody else?

    \begin{table}
    \centering
    \caption{\textbf{Your Bank}}
    \begin{tblr}{
      colspec = { Q[c,wd=2cm] Q[c,wd=1cm] |Q[c,wd=2cm] Q[c,wd=1cm] }, vline{3} = {1-Z}{}
    }
    \toprule
    Assets && Liabilities \\
    \midrule
    Cash & -\$100 & Deposits & -\$100 \\
    \end{tblr}
    \end{table}

        \begin{table}
    \centering
    \caption{\textbf{Recipients' Bank}}
    \begin{tblr}{
      colspec = { Q[l,wd=2cm] Q[c,wd=1cm] |Q[l,wd=2cm] Q[c,wd=1cm] }, vline{3} = {1-Z}{}
    }
    \toprule
    Assets && Liabilities \\
    \midrule
    Cash & +\$100 & Deposits & +\$100 \\
    \end{tblr}
    \end{table}

\end{frame}

\begin{frame}[t]
    \frametitle{3. Bank Balance Sheet Changes}

    What happens you (“A”) deposit money at a bank and the bank then makes a loan to somebody else (“B”) that uses the same bank?   

    {\small
    \begin{table}
    \centering
    \caption{\textbf{Step 1}}
    \begin{tblr}{
      colspec = { Q[l,wd=3cm] Q[c,wd=1cm] |Q[l,wd=3cm] Q[c,wd=1cm] }, vline{3} = {1-Z}{}
    }
    \toprule
    Assets && Liabilities \\
    \midrule
    Cash & +\$100 & Deposit from A & +\$100 \\
    \end{tblr}
    \end{table}
    \vspace{-1em}

        \begin{table}
    \centering
    \caption{\textbf{Step 2}}
    \begin{tblr}{
      colspec = { Q[l,wd=3cm] Q[c,wd=1cm] |Q[l,wd=3cm] Q[c,wd=1cm] }, vline{3} = {1-Z}{}
    }
    \toprule
    Assets && Liabilities \\
    \midrule
    Loan to B & +\$100 & Deposit from A & +\$100 \\
    Cash & +\$100 & Deposit from B & +\$100 \\
    \end{tblr}
    \end{table}
    }

\end{frame}

\begin{frame}[t]
    \frametitle{3. Bank Balance Sheet Changes}

    In the previous example the bank literally created money.   It received a \$100 deposit, but ended up with \$200 in deposits.\\
    \vspace{1em}
    Required reserves limit this process and slow down money creation.\\
    \vspace{1em}
    To see how, the two next slide shows the same process but now with the requirement that the bank hold reserves equal to at least 10\% of deposits and then with capital ratio requirement.   In both case, the bank is limited in how much lending it can do, which slows down the money creation process.  

\end{frame}

\begin{frame}[t]
    \frametitle{3. Bank Balance Sheet Changes}

    Assume the bank must maintain required reserves of at least 10\% of deposits.

    {\small
    \begin{table}
    \centering

    \begin{tblr}{
      colspec = { Q[l,wd=3cm] Q[c,wd=1cm] |Q[l,wd=3cm] Q[c,wd=1cm] }, vline{3} = {1-Z}{}
    }
    \toprule
    Assets && Liabilities \\
    \midrule
    Cash & +\$90 & Deposit from A & +\$100 \\
    Reserves & +\$10 & && \\
    \end{tblr}
    \end{table}

    \begin{table}
    \centering
    \begin{tblr}{
      colspec = { Q[l,wd=3cm] Q[c,wd=1cm] |Q[l,wd=3cm] Q[c,wd=1cm] }, vline{3} = {1-Z}{}
    }
    \toprule
    Assets && Liabilities \\
    \midrule
    Loan to B & +\$90 & Deposit from A & +\$100 \\
    Cash & +\$81 & Deposit from B & +\$90 \\
    Reserves & +\$19
    \end{tblr}
    \end{table}
    }
    Because of the required reserves, the bank only created \$90 (instead of \$100 in the first example).

\end{frame}

\begin{frame}[t]
    \frametitle{3. Bank Balance Sheet Changes}

    Assume the bank starts with \$15 in capital and must maintain a capital ratio of at least 10\%.


    {\small
    \begin{table}
    \centering
    \begin{tblr}{
      colspec = { Q[l,wd=3cm] Q[c,wd=1cm] |Q[l,wd=3cm] Q[c,wd=1cm] }, vline{3} = {1-Z}{}
    }
    \toprule
    Assets && Liabilities \\
    \midrule
    Cash & +\$90 & Deposit from A & +\$100 \\
    Reserves & +\$10 & && \\
    && Capital & +\$15\\
    \end{tblr}
    \end{table}

    \begin{table}
    \centering
    \begin{tblr}{
      colspec = { Q[l,wd=3cm] Q[c,wd=1cm] |Q[l,wd=3cm] Q[c,wd=1cm] }, vline{3} = {1-Z}{}
    }
    \toprule
    Assets && Liabilities \\
    \midrule
    Loan to B & +\$50 & Deposit from A & +\$100 \\
    Cash & +\$85 & Deposit from B & +\$50 \\
    Reserves & +\$15 \\
        && Capital & +\$15\\

    \end{tblr}
    \end{table}
    }

\end{frame}


\begin{frame}[t]
    \frametitle{3. Bank Balance Sheet Changes}

    A couple more examples:
    \vspace{1em}
    
    Deposit outflows:
    \begin{itemize}
        \item Use most liquid assets
        \item Need to maintain reserve requirement
    \end{itemize}
    
    Loan Losses
    \begin{itemize}
        \item Regulators may force raising more capital
        \item This involves getting new owners to contribute cash in exchange for equity. What are the pros/cons of that?
        \item If losses are large enough, may have negative capital (will discuss this more in a later class...)
    \end{itemize}

\end{frame}

\begin{practiceframe}[t]
    \frametitle{3. Bank Balance Sheet Changes}
    \vspace{-1em}
    Consider a bank with the following: \$30 million in capital, \$20 million in debt, \$200 million in deposits, \$50 million in securities, \$150 million in loans, and the remainder of its assets in cash/reserves. The banks regulators require it to maintain a capital to asset ratio of 10\% and a reserve to deposit ratio of 10\%.
    \vspace{1em}
    \begin{enumerate}
        \item Write down the bank’s balance sheet using a “T-chart”. Verify that it is meeting both of its regulatory ratios.
        \item What happens if there is a deposit outflow of \$40 million? Write down the bank balance sheet. Is the bank currently meeting its regulatory ratios?
        \item What happens if instead of a deposit outflow,the bank experienced losses on its loans. Specifically, what happens if 10\% of the banks’ loans default? Write down the bank balance sheet.
        \item Is the bank currently meeting its regulatory ratios?
    \end{enumerate}
    
\end{practiceframe}

\end{document}