\documentclass{beamer}

\newcommand{\week}{Week 1-a}

\title{Money Credit and Banking}
\author{Econ 357}
\date{\week}

% Reference the shared preamble
\setbeamertemplate{frametitle}{
  \vspace{0.5em}
  \insertframetitle
  \par
  \vspace{0.5em}
  \hrule
  \vspace{0.3em}
  {\small\color{gray}\insertframesubtitle}
}

\setbeamertemplate{navigation symbols}{}
\setbeamertemplate{itemize item}{\textbullet} % main bullet: filled dot
\setbeamertemplate{itemize subitem}{\normalsize$\circ$} % sub-bullet: empty dot
\setbeamertemplate{itemize subsubitem}{\scriptsize--} % sub-sub-bullet: dash


% Font changes
\usepackage[scaled=0.92]{helvet}
\renewcommand{\familydefault}{\sfdefault}

% Packages
\usepackage{tikz}
\usepackage{booktabs}
\usepackage{xcolor}
\usepackage{array}           % Enhanced column types for tables
\usepackage{multirow}        % Spanning multiple rows in tables
\usepackage{makecell}        % Line breaks and formatting in table cells
\usepackage{siunitx}         % Proper formatting of numbers and units
\usepackage{amsmath}         % Enhanced math environments
\usepackage{amsfonts}        % Additional math fonts
\usepackage{amssymb}         % Additional math symbols
\usepackage{url}             % Better URL formatting
\usepackage{graphicx}        % Enhanced graphics support
\usepackage{tabularray}
\UseTblrLibrary{booktabs, siunitx, varwidth}
% For financial presentations specifically
\usepackage{eurosym}         % Euro symbol
\usepackage{textcomp}        % Additional text symbols
\usepackage{hyperref}        % Hyperlinks (should be loaded last)

% Define a footnote
\renewcommand{\footnoterule}{\vspace*{-3pt}\hrule width 2in height 0.4pt\vspace*{2.6pt}}

% Define a Foundation Slide
\newenvironment{foundframe}[1][t]{
    \setbeamercolor{background canvas}{bg=gray!8}
    \setbeamercolor{frametitle}{fg=gray!80!black,bg=gray!25}
    \setbeamercolor{framesubtitle}{fg=gray!70!black,bg=gray!15}
    \setbeamercolor{item}{fg=gray!80!black}
    \setbeamercolor{enumerate item}{fg=gray!80!black}
    
    % Modify the frametitle template for this frame type
    \setbeamertemplate{frametitle}{
        \vspace{0.5em}
        \begin{minipage}[t]{0.75\textwidth}
            \insertframetitle
            \par
            \vspace{0.5em}
            \hrule
            \vspace{0.3em}
            {\small\color{gray}\insertframesubtitle}
        \end{minipage}%
        \hfill
        \begin{minipage}[t]{0.2\textwidth}
            \raggedleft
            \colorbox{gray!30}{%
                \scriptsize\bfseries\color{gray!80!black}%
                   \hspace{3pt}\begin{tabular}{c}Foundation\\Material\end{tabular}\hspace{3pt}%
            }
        \end{minipage}
        \vspace{0.3em}
    }
    
    \begin{frame}[#1]
}{
    \end{frame}
}

% Define Practice Slide
\newenvironment{practiceframe}[1][t]{
    \setbeamercolor{background canvas}{bg=white}
    \setbeamercolor{frametitle}{fg=blue!80!black,bg=blue!15}
    \setbeamercolor{framesubtitle}{fg=blue!70!black,bg=blue!10}
    \setbeamercolor{item}{fg=blue!80!black}
    \setbeamercolor{enumerate item}{fg=blue!80!black}
    \setbeamercolor{normal text}{fg=blue!90!black}
    
    % Modify the frametitle template for this frame type
    \setbeamertemplate{frametitle}{
        \vspace{0.5em}
        \begin{minipage}[t]{0.75\textwidth}
            \insertframetitle
            \par
            \vspace{0.5em}
            \hrule
            \vspace{0.3em}
            {\small\color{blue!70!black}\insertframesubtitle}
        \end{minipage}%
        \hfill
        \begin{minipage}[t]{0.2\textwidth}
            \raggedleft
            \colorbox{blue!20}{%
                \scriptsize\bfseries\color{blue!80!black}%
                   \hspace{3pt}\begin{tabular}{c}Practice\\Questions\end{tabular}\hspace{3pt}%
            }
        \end{minipage}
        \vspace{0.3em}
    }
    
    \begin{frame}[#1]
}{
    \end{frame}
}

% Define Excel Slide
\newenvironment{excelframe}[1][t]{
    \setbeamercolor{background canvas}{bg=white}
    \setbeamercolor{frametitle}{fg=blue!80!black,bg=blue!15}
    \setbeamercolor{framesubtitle}{fg=blue!70!black,bg=blue!10}
    \setbeamercolor{item}{fg=blue!80!black}
    \setbeamercolor{enumerate item}{fg=blue!80!black}
    \setbeamercolor{normal text}{fg=blue!90!black}
    
    % Modify the frametitle template for this frame type
    \setbeamertemplate{frametitle}{
        \vspace{0.5em}
        \begin{minipage}[t]{0.75\textwidth}
            \insertframetitle
            \par
            \vspace{0.5em}
            \hrule
            \vspace{0.3em}
            {\small\color{blue!70!black}\insertframesubtitle}
        \end{minipage}%
        \hfill
        \begin{minipage}[t]{0.2\textwidth}
            \raggedleft
            \colorbox{green!10}{%
                \scriptsize\bfseries\color{blue!80!black}%
                   \hspace{3pt}\begin{tabular}{c}MS Excel\end{tabular}\hspace{3pt}%
            }
        \end{minipage}
        \vspace{0.3em}
    }
    
    \begin{frame}[#1]
}{
    \end{frame}
}

% Define Caution Slide
\newenvironment{cautionframe}[1][t]{
    \setbeamercolor{background canvas}{bg=white}
    \setbeamercolor{frametitle}{fg=blue!80!black,bg=blue!15}
    \setbeamercolor{framesubtitle}{fg=blue!70!black,bg=blue!10}
    \setbeamercolor{item}{fg=blue!80!black}
    \setbeamercolor{enumerate item}{fg=blue!80!black}
    \setbeamercolor{normal text}{fg=blue!90!black}
    
    % Modify the frametitle template for this frame type
    \setbeamertemplate{frametitle}{
        \vspace{0.5em}
        \begin{minipage}[t]{0.75\textwidth}
            \insertframetitle
            \par
            \vspace{0.5em}
            \hrule
            \vspace{0.3em}
            {\small\color{blue!70!black}\insertframesubtitle}
        \end{minipage}%
        \hfill
        \begin{minipage}[t]{0.2\textwidth}
            \raggedleft
            \colorbox{red!10}{%
                \scriptsize\bfseries\color{blue!80!black}%
                   \hspace{3pt}\begin{tabular}{c}Caution\end{tabular}\hspace{3pt}%
            }
        \end{minipage}
        \vspace{0.3em}
    }
    
    \begin{frame}[#1]
}{
    \end{frame}
}

% Add to footnotes
\makeatletter
\newcommand\blfootnote[1]{%
  \begingroup
  \renewcommand\thefootnote{}%
  \renewcommand\@makefntext[1]{\raggedright\leftskip=0pt ##1}%
  \footnote{\scriptsize #1}%
  \addtocounter{footnote}{-1}%
  \endgroup
}
\makeatother

% Set the footer -- change 
\setbeamertemplate{footline}{
  \leavevmode%
  \vspace{2ex}
  \hbox{%
    % Left box: Econ 457
    \begin{beamercolorbox}[wd=.4\paperwidth,ht=2.5ex,dp=1ex,left]{author in head/foot}%
      \hspace{1em}Econ 357
    \end{beamercolorbox}%
    % Middle box: Week
    \begin{beamercolorbox}[wd=.2\paperwidth,ht=2.5ex,dp=1ex,center]{date in head/foot}%
      \centering\week
    \end{beamercolorbox}%
    % Right box: Slide numbers
    \begin{beamercolorbox}[wd=.4\paperwidth,ht=2.5ex,dp=1ex,center]{date in head/foot}%
      \hfill\insertframenumber{} 
    \end{beamercolorbox}%
  }%
  \vskip0pt%
}

\begin{document}

\frame{\titlepage}

% 1
\begin{frame}
    \frametitle{Outline}
    \begin{enumerate}
        \item Introductions
        \item Economics 357
        \item Real v. Financial Assets
        \item Sectors in the Financial System
        \item Role of Markets
    \end{enumerate}
\end{frame}

% 2
\begin{frame}[t]
    \frametitle{1. Introductions}
    John Bellows (me)
    \begin{itemize}
        \item Western Asset Management, 2012-2024
        \begin{itemize}
            \item Portfolio Manager, Fixed Income Portfolios
            \item Investment Strategy Committees, Responsible for Fed View
        \end{itemize}
        \item US Department of Treasury, 2009-2011
        \item PhD in Economics, UC Berkeley, 2009
    \end{itemize}
    \vspace{2em}
    Quiz (not graded, do your best!)
\end{frame}

\begin{frame}[t]
    \frametitle{1. Introductions}

    \centering
    \includegraphics[width=0.8\textwidth]{figures/JB Photos.png}

\end{frame}

\begin{frame}[t]
    \frametitle{1. Introductions}

    \textbf{Office Hours:} \\
    Location: KAP 308\\
    Time: Thursdays, 2:00-3:15 pm\\
    Additional office hours may be available, please send me an email with a request.   I'll do my best to be accomodating.\\
    \vspace{1em}
    \textbf{Email policy:} Best to talk to me during office hours or after class, especially for questions about class material or test preparation.   I will do my best to respond to emails within 24 hours.   

\end{frame}

\begin{frame}[t]
    \frametitle{1. Introductions}

    Answers to the Quiz
    \begin{enumerate}
        \item Secretary of the Treasury:
        \item Federal Reserve Chair:
        \item Current Level of Fed Funds Rate: 
        \item Rate of 10-year UST notes since 2020: 
        \item S\&P Year-to-date:
        \item Annualized S\&P gain since 2020: 
    \end{enumerate}

\end{frame}

\begin{frame}[t]
    \frametitle{2. Economics 457}
    \framesubtitle{Material Covered}

    Goals for the class:
    \begin{itemize}
        \item Develop some basic tools used in finance
        \item Apply these tools
        \item Appreciate some of the cool results in finance
        \item Understand why finance matters
    \end{itemize}
    \vspace{1em}

    My own experience includes academia, public policy, and also investing.   The material in this class will touch on all of these at different points.

\end{frame}

\begin{frame}[t]
    \frametitle{2. Economics 457}
    \framesubtitle{Material Covered}

    \vspace{-1em}

        \begin{table}
        \centering
        \footnotesize  % Global font size for the table
        \begin{tblr}{
            colspec = {Q[l,wd=2cm] Q[c,wd=2cm] Q[l,wd=6cm]}
        }
        \toprule
        Subject & Weeks & Sub-topics \\
        \midrule
        Intro & 1 \& 2 & Measuring Returns, Distribution of Returns, Evaluating Returns \\
        Portfolio Construction & 3, 4, 5 & Capital Allocation, Diversification, Index Model \\
        Market Equilibrium & 6 \& 7 & CAPM, Fama-French Factors \\
        Fixed Income & 8 \& 9 & Prices, Yields, Yield Curve, Duration and Convexity \\
        Equity & 10 \& 11 & Dividend Discount Models, Price-Earnings Ratios, Efficient Markets, Equity Risk Premium\\
        Derivatives & 12, 13, 14 & Futures, Swaps, Options \\
        \bottomrule
        \end{tblr}
    \end{table}
    \vfill
    \footnotesize
    Note, this is subject to change throughout the semester.

\end{frame}

\begin{frame}[t]
    \frametitle{2. Economics 457}
    \framesubtitle{Material Covered}

    \textbf{Textbook:} Bodie, Z., Kane, A., \& Marcus, A. J. (2024). \textit{Investments} (13th ed.). New York: McGraw-Hill/Irwin.\\
    The textbook is very good and I will follow it closely, for the most part. It also has lots of practice problems, which are useful for exam preparation.
    \vspace{1em}
    
    \textbf{Lectures:} Slides will be posted on Brightspace prior to each lecture. The slides will not be comprehensive, however, as I will undoubtedly say or discuss things in class that are not covered in the slides. I will also occasionally write on the chalkboard. I will do my best to record each lecture, but sometimes I forget, so please don't rely on that.

\end{frame}


\begin{frame}[t]
    \frametitle{2. Economics 457}
    \framesubtitle{Resources}

    USC library access to financial news:\\
    https://libguides.usc.edu/bizinfo/news\\
    \vspace{1em}
    \centering
    \includegraphics[width=0.7\textwidth]{figures/intro_a_library.png}

\end{frame}

\begin{frame}[t]
    \frametitle{2. Economics 457}
    \framesubtitle{Grading}

    \begin{table}
        \centering
        \caption{Grades for Econ 457}
        \begin{tblr}{
            colspec = {Q[l,wd=2cm] Q[c,wd=2.cm]}
        }
        \toprule
        Item & Percent of Total \\
        \midrule
        Homework & 25\% \\
        Mid-term 1 & 20\% \\
        Mid-term 2 & 20\% \\
        Final & 35\% \\
        \bottomrule
        \end{tblr}
    \end{table}

\end{frame}

\begin{frame}[t]
    \frametitle{2. Economics 457}
    \framesubtitle{Grading}

    Homework
    \begin{itemize}
        \item Goal is to familiarize students with Microsoft Excel, while practicing concepts from class.
        \item 10 assignments.   Best 8 will be counted. (Full points will be given on 2 for free)
        \item 2.5 points per assigment.   2 for turning it on time, 0.5 for getting it (mostly) correct.
        \item Posted and submitted on Brightspace.
        \item Posted on Fridays.  Due at the start of class the following Thursday.   (Brightspace timestamps submissions, late submissions will not be accepted.)
        \item If you'd like to do these in Python/Jupyter, please talk to me after class or in office hours.
    \end{itemize}
\end{frame}

\begin{frame}[t]
    \frametitle{3. Real v. Financial Assets}
    \framesubtitle{Definitions and Examples}
    \textbf{Real Assets:}
    \begin{itemize}
        \item \textit{Definition:} An asset that can be used to produce goods or services
        \item \textit{Examples:} Land, buidlings, machine, patents, human capital
    \end{itemize}
    \vspace{1em}

    \textbf{Financial Assets:}
    \begin{itemize}
        \item \textit{Definition:} Claims to the income created by real assets
        \item \textit{Examples:} Stocks, bonds, derivatives
        \item \textit{Important:} Financial Assets are always both somebody's asset AND somebody else's liability.
    \end{itemize}

\end{frame}

%5 
\begin{frame}[t]
    \frametitle{3. Real v. Financial Assets}
    \framesubtitle{Balance Sheets (examples)}
    \textbf{Households:}\\
    \vspace{1em}
    \centering
    \begin{tabular}{p{0.4\textwidth}|p{0.4\textwidth}}
        \hline
        \textbf{Assets} & \textbf{Liabilities} \\
        \hline
         & \\
         &  \\
         &  \\
         &  \\
         &  \\
         &  \\
    \end{tabular}
\end{frame}

\begin{frame}[t]
    \frametitle{3. Real v. Financial Assets}
    \framesubtitle{Balance Sheets (examples)}
    \textbf{Non-financial Corporations:}\\
    \vspace{1em}
    \centering
    \begin{tabular}{p{0.4\textwidth}|p{0.4\textwidth}}
        \hline
        \textbf{Assets} & \textbf{Liabilities} \\
        \hline
         & \\
         &  \\
         &  \\
         &  \\
         &  \\
         &  \\
    \end{tabular}
\end{frame}

%6
\begin{frame}[t]
    \frametitle{4. Sectors of Financial System}
    \framesubtitle{Household Net Worth}
    \centering
    \includegraphics[width=0.95\textwidth]{figures/intro_a_fig1.png}

    \vspace{1em}
    \begin{minipage}{0.95\textwidth}
        \centering
        \colorbox{white}{\parbox{0.9\textwidth}{\scriptsize\strut Source: Federal Reserve\strut}}
    \end{minipage}
\end{frame}

%6
\begin{frame}[t]
    \frametitle{4. Sectors of Financial System}
    \framesubtitle{Household Real v. Financial Assets}
    \centering
    \includegraphics[width=0.95\textwidth]{figures/intro_a_fig2.png}

    \vspace{1em}
    \begin{minipage}{0.95\textwidth}
        \centering
        \colorbox{white}{\parbox{0.9\textwidth}{\scriptsize\strut Source: Federal Reserve\strut}}
    \end{minipage}
\end{frame}

%6
\begin{frame}[t]
    \frametitle{4. Sectors of Financial System}
    \framesubtitle{Household Non-Financial Assets}
    \centering
    \includegraphics[width=0.95\textwidth]{figures/intro_a_fig3.png}

    \vspace{1em}
    \begin{minipage}{0.95\textwidth}
        \centering
        \colorbox{white}{\parbox{0.9\textwidth}{\scriptsize\strut Source: Federal Reserve\strut}}
    \end{minipage}
\end{frame}

%7
\begin{frame}[t]
    \frametitle{4. Sectors of Financial System}
    \framesubtitle{Household Financial Assets}
    \centering
    \includegraphics[width=0.95\textwidth]{figures/intro_a_fig4.png}

    \vspace{1em}
    \begin{minipage}{0.95\textwidth}
        \centering
        \colorbox{white}{\parbox{0.9\textwidth}{\scriptsize\strut Source: Federal Reserve\strut}}
    \end{minipage}
\end{frame}

%7
\begin{frame}[t]
    \frametitle{4. Sectors of Financial System}
    \framesubtitle{Household Liabilities}
    \centering
    \includegraphics[width=0.95\textwidth]{figures/intro_a_fig5.png}

    \vspace{1em}
    \begin{minipage}{0.95\textwidth}
        \centering
        \colorbox{white}{\parbox{0.9\textwidth}{\scriptsize\strut Source: Federal Reserve\strut}}
    \end{minipage}
\end{frame}

\begin{frame}[t]
    \frametitle{4. Sectors of the Financial System}
    \framesubtitle{Example: Financing a student loan}
    Scenario 1: 529 Plans\\

    \vspace{4em}
    
    Scenario 2: Student Loans\\
    
    \vspace{4em}
    
    \textit{Key Point: financial assets always show up in two places.  They are somebody's asset and somebody else's liability}
\end{frame}


\begin{frame}[t]
    \frametitle{5. Role of Markets}
    \framesubtitle{Some Observations About Markets}
    \begin{itemize}
        \item Markets match buyers and sellers!
        \item Markets facilitate speculation
        \item Markets often require intermediaries
        \begin{itemize}
            \item Reduce transaction costs (e.g. search costs)
            \item Solve moral hazard and adverse selection problems 
        \end{itemize}
        \item Markets often require regulators
        \item Markets are innovative
    \end{itemize}
\end{frame}


\begin{frame}[t]
    \frametitle{5. Role of Markets}
    \framesubtitle{Financial Product Innovation}
    Examples:
    \begin{itemize}
        \item Financing college education: student loans
        \item Money Market Instruments: Money Market Mutual Funds, Eurodollars
        \item Mortgages: 30y fixed rate loans, NINJA loans
        \item Corporate Debt Financing: HY Bonds, Bank Loans, "Private Credit" 
        \item Corporate Equity Financing: Private Equity (lots of flavors), Mutual Funds, ETFs
    \end{itemize}
\end{frame}

\end{document}