\documentclass{beamer}

\newcommand{\week}{Financial Markets 2 of 8}

\title{Financial Market Returns}
\subtitle{Mishkin Chapter 4}
\author{Econ 357}
\date{\week}

% Reference the shared preamble
\setbeamertemplate{frametitle}{
  \vspace{0.5em}
  \insertframetitle
  \par
  \vspace{0.5em}
  \hrule
  \vspace{0.3em}
  {\small\color{gray}\insertframesubtitle}
}

\setbeamertemplate{navigation symbols}{}
\setbeamertemplate{itemize item}{\textbullet} % main bullet: filled dot
\setbeamertemplate{itemize subitem}{\normalsize$\circ$} % sub-bullet: empty dot
\setbeamertemplate{itemize subsubitem}{\scriptsize--} % sub-sub-bullet: dash


% Font changes
\usepackage[scaled=0.92]{helvet}
\renewcommand{\familydefault}{\sfdefault}

% Packages
\usepackage{tikz}
\usepackage{booktabs}
\usepackage{xcolor}
\usepackage{array}           % Enhanced column types for tables
\usepackage{multirow}        % Spanning multiple rows in tables
\usepackage{makecell}        % Line breaks and formatting in table cells
\usepackage{siunitx}         % Proper formatting of numbers and units
\usepackage{amsmath}         % Enhanced math environments
\usepackage{amsfonts}        % Additional math fonts
\usepackage{amssymb}         % Additional math symbols
\usepackage{url}             % Better URL formatting
\usepackage{graphicx}        % Enhanced graphics support
\usepackage{tabularray}
\UseTblrLibrary{booktabs, siunitx, varwidth}
% For financial presentations specifically
\usepackage{eurosym}         % Euro symbol
\usepackage{textcomp}        % Additional text symbols
\usepackage{hyperref}        % Hyperlinks (should be loaded last)

% Define a footnote
\renewcommand{\footnoterule}{\vspace*{-3pt}\hrule width 2in height 0.4pt\vspace*{2.6pt}}

% Define a Foundation Slide
\newenvironment{foundframe}[1][t]{
    \setbeamercolor{background canvas}{bg=gray!8}
    \setbeamercolor{frametitle}{fg=gray!80!black,bg=gray!25}
    \setbeamercolor{framesubtitle}{fg=gray!70!black,bg=gray!15}
    \setbeamercolor{item}{fg=gray!80!black}
    \setbeamercolor{enumerate item}{fg=gray!80!black}
    
    % Modify the frametitle template for this frame type
    \setbeamertemplate{frametitle}{
        \vspace{0.5em}
        \begin{minipage}[t]{0.75\textwidth}
            \insertframetitle
            \par
            \vspace{0.5em}
            \hrule
            \vspace{0.3em}
            {\small\color{gray}\insertframesubtitle}
        \end{minipage}%
        \hfill
        \begin{minipage}[t]{0.2\textwidth}
            \raggedleft
            \colorbox{gray!30}{%
                \scriptsize\bfseries\color{gray!80!black}%
                   \hspace{3pt}\begin{tabular}{c}Foundation\\Material\end{tabular}\hspace{3pt}%
            }
        \end{minipage}
        \vspace{0.3em}
    }
    
    \begin{frame}[#1]
}{
    \end{frame}
}

% Define Practice Slide
\newenvironment{practiceframe}[1][t]{
    \setbeamercolor{background canvas}{bg=white}
    \setbeamercolor{frametitle}{fg=blue!80!black,bg=blue!15}
    \setbeamercolor{framesubtitle}{fg=blue!70!black,bg=blue!10}
    \setbeamercolor{item}{fg=blue!80!black}
    \setbeamercolor{enumerate item}{fg=blue!80!black}
    \setbeamercolor{normal text}{fg=blue!90!black}
    
    % Modify the frametitle template for this frame type
    \setbeamertemplate{frametitle}{
        \vspace{0.5em}
        \begin{minipage}[t]{0.75\textwidth}
            \insertframetitle
            \par
            \vspace{0.5em}
            \hrule
            \vspace{0.3em}
            {\small\color{blue!70!black}\insertframesubtitle}
        \end{minipage}%
        \hfill
        \begin{minipage}[t]{0.2\textwidth}
            \raggedleft
            \colorbox{blue!20}{%
                \scriptsize\bfseries\color{blue!80!black}%
                   \hspace{3pt}\begin{tabular}{c}Practice\\Questions\end{tabular}\hspace{3pt}%
            }
        \end{minipage}
        \vspace{0.3em}
    }
    
    \begin{frame}[#1]
}{
    \end{frame}
}

% Define Excel Slide
\newenvironment{excelframe}[1][t]{
    \setbeamercolor{background canvas}{bg=white}
    \setbeamercolor{frametitle}{fg=blue!80!black,bg=blue!15}
    \setbeamercolor{framesubtitle}{fg=blue!70!black,bg=blue!10}
    \setbeamercolor{item}{fg=blue!80!black}
    \setbeamercolor{enumerate item}{fg=blue!80!black}
    \setbeamercolor{normal text}{fg=blue!90!black}
    
    % Modify the frametitle template for this frame type
    \setbeamertemplate{frametitle}{
        \vspace{0.5em}
        \begin{minipage}[t]{0.75\textwidth}
            \insertframetitle
            \par
            \vspace{0.5em}
            \hrule
            \vspace{0.3em}
            {\small\color{blue!70!black}\insertframesubtitle}
        \end{minipage}%
        \hfill
        \begin{minipage}[t]{0.2\textwidth}
            \raggedleft
            \colorbox{green!10}{%
                \scriptsize\bfseries\color{blue!80!black}%
                   \hspace{3pt}\begin{tabular}{c}MS Excel\end{tabular}\hspace{3pt}%
            }
        \end{minipage}
        \vspace{0.3em}
    }
    
    \begin{frame}[#1]
}{
    \end{frame}
}

% Define Caution Slide
\newenvironment{cautionframe}[1][t]{
    \setbeamercolor{background canvas}{bg=white}
    \setbeamercolor{frametitle}{fg=blue!80!black,bg=blue!15}
    \setbeamercolor{framesubtitle}{fg=blue!70!black,bg=blue!10}
    \setbeamercolor{item}{fg=blue!80!black}
    \setbeamercolor{enumerate item}{fg=blue!80!black}
    \setbeamercolor{normal text}{fg=blue!90!black}
    
    % Modify the frametitle template for this frame type
    \setbeamertemplate{frametitle}{
        \vspace{0.5em}
        \begin{minipage}[t]{0.75\textwidth}
            \insertframetitle
            \par
            \vspace{0.5em}
            \hrule
            \vspace{0.3em}
            {\small\color{blue!70!black}\insertframesubtitle}
        \end{minipage}%
        \hfill
        \begin{minipage}[t]{0.2\textwidth}
            \raggedleft
            \colorbox{red!10}{%
                \scriptsize\bfseries\color{blue!80!black}%
                   \hspace{3pt}\begin{tabular}{c}Caution\end{tabular}\hspace{3pt}%
            }
        \end{minipage}
        \vspace{0.3em}
    }
    
    \begin{frame}[#1]
}{
    \end{frame}
}

% Add to footnotes
\makeatletter
\newcommand\blfootnote[1]{%
  \begingroup
  \renewcommand\thefootnote{}%
  \renewcommand\@makefntext[1]{\raggedright\leftskip=0pt ##1}%
  \footnote{\scriptsize #1}%
  \addtocounter{footnote}{-1}%
  \endgroup
}
\makeatother

% Set the footer -- change 
\setbeamertemplate{footline}{
  \leavevmode%
  \vspace{2ex}
  \hbox{%
    % Left box: Econ 457
    \begin{beamercolorbox}[wd=.4\paperwidth,ht=2.5ex,dp=1ex,left]{author in head/foot}%
      \hspace{1em}Econ 357
    \end{beamercolorbox}%
    % Middle box: Week
    \begin{beamercolorbox}[wd=.2\paperwidth,ht=2.5ex,dp=1ex,center]{date in head/foot}%
      \centering\week
    \end{beamercolorbox}%
    % Right box: Slide numbers
    \begin{beamercolorbox}[wd=.4\paperwidth,ht=2.5ex,dp=1ex,center]{date in head/foot}%
      \hfill\insertframenumber{} 
    \end{beamercolorbox}%
  }%
  \vskip0pt%
}

\begin{document}

\frame{\titlepage}



\begin{frame}
    \frametitle{Outline for Financial Markets}
        \begin{enumerate}
            \item Financial Market Returns
            \item \fbox{Bonds 1: Discounting, Prices and Yields}
            \item Bonds 2: Nominal v. Real Yields, Supply and Demand
            \item Bonds 3: Market for Money
            \item Bonds 4: Risk and Term Structure
            \item Equities 1: Dividend Discount Model
            \item Equities 2: P/E Ratio
            \item Equities 3: Theories of Stock Pricing
            
        \end{enumerate}
\end{frame}


\begin{frame}
    \frametitle{Outline for Today's Lecture}
    \begin{enumerate}
        \item Discounting Future Values
        \item Bonds
        \begin{itemize}
            \item Definition
            \item Types of Bonds
            \item Pricing and Yields
        \end{itemize}
    \end{enumerate}

\end{frame}

\begin{frame}[t]
    \frametitle{1. Discounting Future Values}

    Generally people place lower value on cash flows that are expected in the future.  
    
    \vspace{1em}
    
    Why?
    
    \vspace{1em}
    \begin{enumerate}
        \item Behavioral
        \item Time Value of Money
        \item Risk
        \item Inflation
    \end{enumerate}

\end{frame}

\begin{frame}[t]
    \frametitle{1. Discounting Future Values}
    \framesubtitle{Time Value of Money}

    Easy example to start:
    \begin{itemize}
        \item Assume you have access to a guaranteed return of 10\% per year.
        \item How much would you accept one year from today in exchange for \$100 today?
        \item Can also reverse the question: How much would you accept today in exchange for \$110 one year from today?
    \end{itemize}

\end{frame}

\begin{frame}[t]
    \frametitle{1. Discounting Future Values}
    \framesubtitle{Risk}

    For concave utility functions, a certain outcome has a higher utility than an uncertain outcome.

    INSERT CHART

\end{frame}

\begin{frame}[t]
    \frametitle{1. Discounting Future Values}
    \framesubtitle{Inflation}

    Prices of the things (goods and services) you want to buy tend to increase over time.

    \vspace{1em}

    \$1 will buy more things (goods and services) today than the same \$1 will buy 1 or more years from now.

    \vspace{1em}

    In order for you to be indifferent between a value today and a value in the future, the future value must increase by at least as much as prices of things (goods and services) increase.

\end{frame}    

\begin{frame}[t]
    \frametitle{1. Discounting Future Values}
    \framesubtitle{Formula}

    Generic formula to equate today values ($v_0$) with future values ($v_N$) where $N$ is the number of period in between today and the future:

    $$V_0 = \frac{V_N}{(1+r)^N}$$

    Where $r$ is referred to as the "discount rate".   

    \vspace{1em}
    It's easy to rearrange this equation and to find the value in the future that would be equivalent to a value today that is expected to grow at a certain rate:

    $$V_0 (1+r)^N = V_N$$
    
\end{frame}

\begin{frame}[t]
    \frametitle{1. Discounting Future Values}
    \framesubtitle{Rule of 72}

    \begin{quote}
    Compound interest is the eigth wonder of the world.  He who understands it, earns it.   He who doesn't, pays it.\\
    \vspace{1em}\\
     \hspace*{5mm} --- \textit{Albert Einstein}
    \end{quote}

    \vspace{1em}

    \textbf{Rule of 72}: Divide 72 by the annual rate of return to determine how many years is required to double an investment.

    \vspace{1em}

    Example: An investment with an expected annual total return of 14\% will double in value approximately every five years ($72/14 \approx 5$)

\end{frame}

\begin{frame}[t]
    \frametitle{1. Discounting Future Values}
    \framesubtitle{Rule of 72, cont}

    INSERT CHART FROM AARP

\end{frame}

\end{document}