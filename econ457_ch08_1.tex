\documentclass{beamer}

\newcommand{\week}{Week 6-a}

\title{Index Model}
\subtitle{Reference: Bodie et al, Ch 8}
\author{Econ 457}
\date{\week}

% Reference the shared preamble
\setbeamertemplate{frametitle}{
  \vspace{0.5em}
  \insertframetitle
  \par
  \vspace{0.5em}
  \hrule
  \vspace{0.3em}
  {\small\color{gray}\insertframesubtitle}
}

\setbeamertemplate{navigation symbols}{}
\setbeamertemplate{itemize item}{\textbullet} % main bullet: filled dot
\setbeamertemplate{itemize subitem}{\normalsize$\circ$} % sub-bullet: empty dot
\setbeamertemplate{itemize subsubitem}{\scriptsize--} % sub-sub-bullet: dash


% Font changes
\usepackage[scaled=0.92]{helvet}
\renewcommand{\familydefault}{\sfdefault}

% Packages
\usepackage{tikz}
\usepackage{booktabs}
\usepackage{xcolor}
\usepackage{array}           % Enhanced column types for tables
\usepackage{multirow}        % Spanning multiple rows in tables
\usepackage{makecell}        % Line breaks and formatting in table cells
\usepackage{siunitx}         % Proper formatting of numbers and units
\usepackage{amsmath}         % Enhanced math environments
\usepackage{amsfonts}        % Additional math fonts
\usepackage{amssymb}         % Additional math symbols
\usepackage{url}             % Better URL formatting
\usepackage{graphicx}        % Enhanced graphics support
\usepackage{tabularray}
\UseTblrLibrary{booktabs, siunitx, varwidth}
% For financial presentations specifically
\usepackage{eurosym}         % Euro symbol
\usepackage{textcomp}        % Additional text symbols
\usepackage{hyperref}        % Hyperlinks (should be loaded last)

% Define a footnote
\renewcommand{\footnoterule}{\vspace*{-3pt}\hrule width 2in height 0.4pt\vspace*{2.6pt}}

% Define a Foundation Slide
\newenvironment{foundframe}[1][t]{
    \setbeamercolor{background canvas}{bg=gray!8}
    \setbeamercolor{frametitle}{fg=gray!80!black,bg=gray!25}
    \setbeamercolor{framesubtitle}{fg=gray!70!black,bg=gray!15}
    \setbeamercolor{item}{fg=gray!80!black}
    \setbeamercolor{enumerate item}{fg=gray!80!black}
    
    % Modify the frametitle template for this frame type
    \setbeamertemplate{frametitle}{
        \vspace{0.5em}
        \begin{minipage}[t]{0.75\textwidth}
            \insertframetitle
            \par
            \vspace{0.5em}
            \hrule
            \vspace{0.3em}
            {\small\color{gray}\insertframesubtitle}
        \end{minipage}%
        \hfill
        \begin{minipage}[t]{0.2\textwidth}
            \raggedleft
            \colorbox{gray!30}{%
                \scriptsize\bfseries\color{gray!80!black}%
                   \hspace{3pt}\begin{tabular}{c}Foundation\\Material\end{tabular}\hspace{3pt}%
            }
        \end{minipage}
        \vspace{0.3em}
    }
    
    \begin{frame}[#1]
}{
    \end{frame}
}

% Define Practice Slide
\newenvironment{practiceframe}[1][t]{
    \setbeamercolor{background canvas}{bg=white}
    \setbeamercolor{frametitle}{fg=blue!80!black,bg=blue!15}
    \setbeamercolor{framesubtitle}{fg=blue!70!black,bg=blue!10}
    \setbeamercolor{item}{fg=blue!80!black}
    \setbeamercolor{enumerate item}{fg=blue!80!black}
    \setbeamercolor{normal text}{fg=blue!90!black}
    
    % Modify the frametitle template for this frame type
    \setbeamertemplate{frametitle}{
        \vspace{0.5em}
        \begin{minipage}[t]{0.75\textwidth}
            \insertframetitle
            \par
            \vspace{0.5em}
            \hrule
            \vspace{0.3em}
            {\small\color{blue!70!black}\insertframesubtitle}
        \end{minipage}%
        \hfill
        \begin{minipage}[t]{0.2\textwidth}
            \raggedleft
            \colorbox{blue!20}{%
                \scriptsize\bfseries\color{blue!80!black}%
                   \hspace{3pt}\begin{tabular}{c}Practice\\Questions\end{tabular}\hspace{3pt}%
            }
        \end{minipage}
        \vspace{0.3em}
    }
    
    \begin{frame}[#1]
}{
    \end{frame}
}

% Define Excel Slide
\newenvironment{excelframe}[1][t]{
    \setbeamercolor{background canvas}{bg=white}
    \setbeamercolor{frametitle}{fg=blue!80!black,bg=blue!15}
    \setbeamercolor{framesubtitle}{fg=blue!70!black,bg=blue!10}
    \setbeamercolor{item}{fg=blue!80!black}
    \setbeamercolor{enumerate item}{fg=blue!80!black}
    \setbeamercolor{normal text}{fg=blue!90!black}
    
    % Modify the frametitle template for this frame type
    \setbeamertemplate{frametitle}{
        \vspace{0.5em}
        \begin{minipage}[t]{0.75\textwidth}
            \insertframetitle
            \par
            \vspace{0.5em}
            \hrule
            \vspace{0.3em}
            {\small\color{blue!70!black}\insertframesubtitle}
        \end{minipage}%
        \hfill
        \begin{minipage}[t]{0.2\textwidth}
            \raggedleft
            \colorbox{green!10}{%
                \scriptsize\bfseries\color{blue!80!black}%
                   \hspace{3pt}\begin{tabular}{c}MS Excel\end{tabular}\hspace{3pt}%
            }
        \end{minipage}
        \vspace{0.3em}
    }
    
    \begin{frame}[#1]
}{
    \end{frame}
}

% Define Caution Slide
\newenvironment{cautionframe}[1][t]{
    \setbeamercolor{background canvas}{bg=white}
    \setbeamercolor{frametitle}{fg=blue!80!black,bg=blue!15}
    \setbeamercolor{framesubtitle}{fg=blue!70!black,bg=blue!10}
    \setbeamercolor{item}{fg=blue!80!black}
    \setbeamercolor{enumerate item}{fg=blue!80!black}
    \setbeamercolor{normal text}{fg=blue!90!black}
    
    % Modify the frametitle template for this frame type
    \setbeamertemplate{frametitle}{
        \vspace{0.5em}
        \begin{minipage}[t]{0.75\textwidth}
            \insertframetitle
            \par
            \vspace{0.5em}
            \hrule
            \vspace{0.3em}
            {\small\color{blue!70!black}\insertframesubtitle}
        \end{minipage}%
        \hfill
        \begin{minipage}[t]{0.2\textwidth}
            \raggedleft
            \colorbox{red!10}{%
                \scriptsize\bfseries\color{blue!80!black}%
                   \hspace{3pt}\begin{tabular}{c}Caution\end{tabular}\hspace{3pt}%
            }
        \end{minipage}
        \vspace{0.3em}
    }
    
    \begin{frame}[#1]
}{
    \end{frame}
}

% Add to footnotes
\makeatletter
\newcommand\blfootnote[1]{%
  \begingroup
  \renewcommand\thefootnote{}%
  \renewcommand\@makefntext[1]{\raggedright\leftskip=0pt ##1}%
  \footnote{\scriptsize #1}%
  \addtocounter{footnote}{-1}%
  \endgroup
}
\makeatother

% Set the footer -- change 
\setbeamertemplate{footline}{
  \leavevmode%
  \vspace{2ex}
  \hbox{%
    % Left box: Econ 457
    \begin{beamercolorbox}[wd=.4\paperwidth,ht=2.5ex,dp=1ex,left]{author in head/foot}%
      \hspace{1em}Econ 457
    \end{beamercolorbox}%
    % Middle box: Week
    \begin{beamercolorbox}[wd=.2\paperwidth,ht=2.5ex,dp=1ex,center]{date in head/foot}%
      \centering\week
    \end{beamercolorbox}%
    % Right box: Slide numbers
    \begin{beamercolorbox}[wd=.4\paperwidth,ht=2.5ex,dp=1ex,center]{date in head/foot}%
      \hfill\insertframenumber{} 
    \end{beamercolorbox}%
  }%
  \vskip0pt%
}

\begin{document}

\frame{\titlepage}

\begin{frame}
    \frametitle{Outline}

    \begin{enumerate}
        \item Law of Large Numbers
        \item Diversification
        \item Example: Stocks and Portfolio Correlation
        \item Index Model
        \item Information Ratio
        \item Subjects Covered on the Midterm
    \end{enumerate}
\end{frame}

\begin{foundframe}[t]
    \frametitle{1. Law of Large Numbers}
    \framesubtitle{}

    \textbf{Intuition:} As the number of independent random variables increases, their average converges to the expected value.
    \vspace{1em}

    \textbf{Mathematical Statement:}
    For independent random variables $X_1, X_2, \ldots, X_n$ with mean $\mu$ and finite variance $\sigma^2$:
    $$\bar{X}_n = \frac{1}{n}\sum_{i=1}^{n} X_i \xrightarrow{p} \mu \text{ as } n \to \infty$$
    \vspace{1em}

\end{foundframe}

\begin{frame}[t]
    \frametitle{2. Diversification}
    \framesubtitle{Benefit of Diversification}

    Consider $n$ securities have returns $r_i$ and standard deviations $\sigma_i$ for $i=1...n$.
    For a portfolio constucted with weights $(w_i,...,w_n)$ the variance is given by:
    $$\sigma_p^2 = \sum_{i=1}^{n}\sum_{j=1}^{n}w_i w_j Cov(r_i,r_j)$$
    For an equally weighted portfolio $w_i = w_n = \frac{1}{n}$ this becomes:
    $$\sigma_p^2 = \sum_{i=1}^{n}\frac{1}{n^2}\sigma_i^2 + \sum_{j=1,j \neq i}^{n}\sum_{i=1}^{n}\frac{1}{n^2}Cov(r_i,r_j)$$
    Continued...

\end{frame}

\begin{frame}[t]
    \frametitle{2. Diversification}
    \framesubtitle{Benefit of Diversification}

    If the securities are independent (i.e. $Cov(r_i,r_j) = 0$), then the variance of the portfolio can be reduced to:
    $$\sigma_p^2 = \frac{1}{n}\sum_{i=1}^{n}\frac{1}{n}\sigma_i^2$$
    Note that $\frac{1}{n}$ was factored out and moved outside the summation.   
    By the Law of Large Numbers, the term 
    $\sum_{i=1}^{n}\frac{1}{n}\sigma_i^2$ will converge to the average $\sigma_i^2$ across all $i$, denoted by
    $\overline{\sigma^2}$.   We now have:
    $$\sigma_p^2 = \frac{1}{n}\overline{\sigma^2}$$
    This approaches $0$ as $n$ gets large.   \textit{We conclude is is possible 
    to construct a portfolio of close to zero 
    variance by including a large number of independent securities}

\end{frame}

\begin{frame}[t]
    \frametitle{2. Diversification}
    \framesubtitle{Limits of Diversification}

    Question: what is the expected return of this portfolio?\\
    \vspace{1em}

    Answer: By CAPM, if the portfolio has zero variance, then the fair expected return is the 
    same as the risk-free rate.\\
    \vspace{1em}

    So, while it may be possible to keep adding independent securities, and by doing so keep reducing 
    the variance of the portfolio, the expected return of this portfolio may end up being unattractive.

\end{frame}

\begin{frame}[t]
    \frametitle{2. Diversification}
    \framesubtitle{Limits of Diversification}

    Question: is it realistic to expect to find a large number of independent securities?\\
    \vspace{1em}

    Answer: Probably not.   For example, most stocks probably have some correlation with the broader 
    index.   A portfolio of stocks will therefore also have some correlation with the broader index\\
    \vspace{1em}
    
    The following slides show how correlation of a portfolio with the broader market affects the 
    portfolio's variance.
  
\end{frame}

\begin{frame}[t]
    \frametitle{2. Diversification}
    \framesubtitle{Limits of Diversification} 

    We can represent down the expected risk premium for a single stock as follows:
    $$\mathbb{E}[R_i] = \alpha_i + \beta_i\mathbb{E}[R_m] + \epsilon_i$$
    Similarly, we can represent the expected return of a portfolio of stocks as follows:
    $$\mathbb{E}[R_p] = \alpha_p + \beta_p\mathbb{E}[R_m] + \epsilon_p$$
    The first term is the \textit{nonmarket premium}.   
    The term $\beta_p\mathbb{E}[R_m]$ is known as the \textit{systematic risk}.   
    And the final term, $\epsilon_p$ is known as the \textit{idiosyncratic risk}.
    
 \end{frame}  
    
\begin{frame}[t]
    \frametitle{2. Diversification}
    \framesubtitle{Limits of Diversification} 

    What is the variance of the portfolio represented by the equation on the previous slide?  
    A couple of observations: 
    \begin{enumerate}
        \item $\alpha_p$ is assumed to constant, therefore doesn't contribute to the variance.
        \item The error terms are assumed to be indpendent of the market returns, with a variance 
        given by $\sigma(\epsilon_p)^2$
        \item The variance of the portfolio is therefore: $\beta_i^2\sigma_m^2 + \sigma(\epsilon_p)^2$
        \item $\sigma(\epsilon_p)^2$ will become negligible as $n$ becomes large, by the Law of Large 
        Numbers.
    \end{enumerate}

    Conclude: the variance of a portfolio of stocks will approach $\beta_i^2\sigma_m^2$ 
        as $n$ becomes large.

\end{frame}

\begin{frame}[t]
    \frametitle{3. Example: Stocks and Portfolio Diversification}
    \framesubtitle{10 Stocks from the S\&P}

    \textbf{Selected Stocks:} Apple (AAPL), Microsoft (MSFT), Johnson \& Johnson (JNJ), 
    JPMorgan Chase (JPM), Procter \& Gamble (PG), Exxon Mobil (XOM), 
    Home Depot (HD), Coca-Cola (KO), Disney (DIS), Boeing (BA)\\
    \vspace{1em}

    \begin{itemize}
        \item \textbf{Sector Diversity:} 8 different GICS sectors represented
        \item \textbf{Different Risk Profiles:}
        \begin{itemize}
            \item Defensive: JNJ, PG, KO (stable earnings, lower volatility)
            \item Cyclical: BA, XOM, HD (economic cycle sensitive)
            \item Growth: AAPL, MSFT (high growth potential)
        \end{itemize}
        \item \textbf{Large, Liquid Names:} All major S\&P 500 components with long price histories and high trading volumes
    \end{itemize}

\end{frame}

\begin{frame}[t]
    \frametitle{3. Example: Stocks and Portfolio Diversification}
    \framesubtitle{10 Stocks from the S\&P}

    \centering
    \includegraphics[width=0.7\textwidth]{figures/ch08_1_diversification_benefits.png}
    \vspace{1em}

    \footnotesize
    \raggedright
    Monthly returns since 2010.   Portfolios created from non-randomly selected stocks (previous slide). 
    Stocks added to the portfolio in order.   While a different order of adding stocks may change 
    the chart slightly, the broader point is that it doesn't take all that many stocks to replicate 
    the volatility of the S\&P 500.

\end{frame}


\begin{frame}[t]
    \frametitle{3. Example: Stocks and Portfolio Diversification}
    \framesubtitle{Example: DIS}

    \centering
    \includegraphics[width=0.8\textwidth]{figures/ch9_dissp_20_25_wreg.png}

\end{frame}




\begin{frame}[t]
    \frametitle{4. The Index Model}
    \framesubtitle{A Strong Assumption}

    Returning to the model from above.  
    Let the rate of excess return on a stock ($R_i$) be given by
    $$R_i = \alpha_i + \beta_i \cdot R_M + \epsilon_i$$
    Where $R_M$ is the excess return of the market, $\alpha_i$ is firm-specific 
    excess returns and $\epsilon_i$ are mean-zero and \textit{independent} error terms.\\
    \vspace{1em}
    This model makes the following very strong assumption: the only correlation between 
    two stocks is the through their correlations with the market.   After controlling for the 
    market moves, the rest of the volatility in any two stocks will be independent.   
    This is obviously a very strong assumption and will almost never hold.

\end{frame}

\begin{frame}[t]
    \frametitle{4. The Index Model}
    \framesubtitle{A Strong Assumption}

    Show: If the excess return of all stocks is represented by the following equation:
    $$R_i = \alpha_i + \beta_i \cdot R_M + \epsilon_i$$
    then the only correlation between two stocks is through their correlation with the market.
    Observations:\\
    \vspace{1em}
    \begin{enumerate}
        \item $Cov(a + b, c) = Cov(a,c) + Cov(b,c)$ for any $a$, $b$, and $c$.
        \item $Cov(a,X) = 0$ if $a$ is a constant and $X$ is a random variable
        \item $Cov(X,\epsilon) = 0$ if $\epsilon$ is independent of $X$
    \end{enumerate}

    Applying these rules -- and remembering that $\alpha_i$ is constant and $\epsilon_i$ is 
    independent -- we have:
    $$Cov(R_i,R_j) = \beta_i \beta_j \sigma_m^2$$

\end{frame}

\begin{frame}[t]
    \frametitle{4. The Index Model}
    \framesubtitle{A Strong Assumption}

    Why would we make such an assumption?\\
    \vspace{1em}
    \textbf{Answer}: Because it significantly reduces the 
    number of parameters required to conduct an optimization over multiple assets.\\
    \vspace{1em}

\end{frame}

\begin{frame}[t]
    \frametitle{4. The Index Model}
    \framesubtitle{A Strong Assumption}

    The covariance matrix used for a n-asset optimization is given by:
    $$
    \boldsymbol{\Sigma} = \begin{pmatrix}
    \sigma_{11} & \sigma_{12} & \sigma_{13} & \cdots & \sigma_{1n} \\
    \sigma_{21} & \sigma_{22} & \sigma_{23} & \cdots & \sigma_{2n} \\
    \sigma_{31} & \sigma_{32} & \sigma_{33} & \cdots & \sigma_{3n} \\
    \vdots & \vdots & \vdots & \ddots & \vdots \\
    \sigma_{n1} & \sigma_{n2} & \sigma_{n3} & \cdots & \sigma_{nn}
    \end{pmatrix}
    $$
    Where $\sigma_{i,j} = Cov(r_i,r_j)$.\\
    \vspace{1em}
    This matrix requires $\frac{n(n+1)}{2}$ unique parameters to estimate.  For $n = 100$ stocks, this means estimating $\frac{100 \times 101}{2} = 5,050$ parameters!

\end{frame}

\begin{frame}[t]
    \frametitle{4. The Index Model}
    \framesubtitle{A Strong Assumption}

    If, instead, we assume that each stock is only correlated to all other stocks 
    through the correlation with the market, then we could construct the covariance matrix on the previous slide using the following:
    $$
    \boldsymbol{\Sigma} = \begin{pmatrix}
    \beta_1^2\sigma_{m}^2 + \sigma(\epsilon_1)^2 &      &   & \cdots &      \\
    \beta_1 \beta_2 \sigma_{m}^2 & \beta_2^2 \sigma_m^2 + \sigma(\epsilon_2)^2&  & \cdots &  \\
    \vdots & \vdots & \vdots & \ddots & \vdots \\
    \beta_1 \beta_n \sigma_m^2 & &  & \cdots & \beta_n^2 \sigma_{m}^2 + + \sigma(\epsilon_n)^2 \\
    \end{pmatrix}
    $$
    Thus, in order to estimate the Index model, we require $n$ estimates of $\beta$s, $n$ estimates of $\sigma(\epsilon_i)$, as well 
    as $n$ estimates of firm-specific nonsystemic returns $\alpha_i$.  For 100 stocks, we now only need 300 estimates, as well as 
    estimates of the market risk premium and market variance ($3n + 2$ is still a lot!).

\end{frame}

\begin{frame}[t]
    \frametitle{4. The Index Model}
    \framesubtitle{The Index as an Investable Asset}

    Note that we are implicitly assuming that you could also invest in the index.\\
    \vspace{1em}
    Why go through all of this work, rather than just invest in the index and be done?\\
    \vspace{1em}
    \textit{Answer:} Analysts/investors may have their own views on the alpha (firm-specific return factors) of 
    certain stocks.   They then need an optimization procedure to incorporate their estimates 
    into an optimal portfolio.

\end{frame}

\begin{frame}[t]
    \frametitle{4. The Index Model}
    \framesubtitle{Procedure for Estimating the Index Model}

    Steps to follow:
    \begin{enumerate}
        \item Estimate expected risk premium and variance for the market: $\mathbb{E}[R_m]$ and $\sigma_m^2$
        \item Estimate $\beta$ for each stock you are actively overweighting / underweighting.   These 
        will probably be close to historic $\beta$.
        \item Estimate $\sigma_i^2$ for each stock you are actively overweighting / underweighting.
        \item Based on your own analysis, come up with expectations for the non-systemic returns of each 
        stock you are overweighting / underweighting ($\alpha_i$).
        \item Follow the steps from the Markowitz optimization to construct your optimal portfolio.
    \end{enumerate}
    \vspace{1em}
    Continued...

\end{frame}

\begin{frame}[t]
    \frametitle{4. The Index Model}
    \framesubtitle{Procedure for Estimating the Index Model}

    Continued...As you might expect, the result of the Markowitz optimization procedure will have you overweighting 
    the stocks you expect to have positive $\alpha$ and underweighting the stocks you expect to have 
    negative $\alpha$.\\
    \vspace{1em}
    The optimization also takes into account the variance ($\sigma(\epsilon_i)^2$) of each stock.   The weight 
    on each stock is therefore a function of both the expected return and the variance:
     $$w_i^0 = \frac{\alpha_i}{\sigma(\epsilon_i)^2}$$
    This is the starting weight for the security in the portfolio.   The final weight adjusts for 
    the beta of the active portfolio and the other overweights/underweights (refer to book for details, 
    won't be on the exam).

\end{frame}

\begin{frame}[t]
    \frametitle{5. The Information Ratio}
    \framesubtitle{}

    In the lecture on evaluating returns, we defined the \textit{Information Ratio} as
    $$\frac{\mathbb{E}[r - r_{index}]}{\sigma_{active}}$$
    This looks like the Sharpe Ratio, but instead of the risk-free rate we are comparing 
    returns to the index, and evaluating the variance relative to the variance of the index.\\
    \vspace{1em}
    The procedure for estiamting the Index Model, discussed in the previous slides,
     produces the following intuitive result: in order to maximize the Informatio Ratio,
     stocks should be over- or underweighted according to the ratio of their non-market returns ($\alpha$) 
     and their estimated variances: $\frac{\alpha_i}{\sigma(\epsilon_i)^2}$

\end{frame}

\begin{practiceframe}[t]
    \frametitle{6. Subjects Covered on the Midterm}
    \framesubtitle{}

    Midterm Topics:
    \begin{enumerate}
        \item Financial Market Returns
            \begin{enumerate}
                \item Measurement (\textit{Bodi et al} Ch 5, class lecture 1-b)
                \item Evaluation (\textit{Bodi et al} Ch 5, class lecture 2-a)
                \item Risk (\textit{Bodi et al} Ch 5, class lecture 2-b)
            \end{enumerate}
        \item Capital Allocation (\textit{Bodi et al} Ch 6, class lectures 3-a and 3-b)
        \item Diversification (\textit{Bodi et al} Ch 7, class lectures 4-a and 4-b)
        \item CAPM (\textit{Bodi et al} Ch 9, class lectures 5-a and 5-b)
        \item Index Model (\textit{Bodi et al} Ch 8, class lectures 6-a)
    \end{enumerate}

\end{practiceframe}

\end{document}