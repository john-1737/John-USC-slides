\documentclass{beamer}

\newcommand{\week}{Week 15-b}

\title{Swaps}
\subtitle{Reference: Bodie et al, Ch 23}
\author{Econ 457}
\date{\week}

% Reference the shared preamble
\setbeamertemplate{frametitle}{
  \vspace{0.5em}
  \insertframetitle
  \par
  \vspace{0.5em}
  \hrule
  \vspace{0.3em}
  {\small\color{gray}\insertframesubtitle}
}

\setbeamertemplate{navigation symbols}{}
\setbeamertemplate{itemize item}{\textbullet} % main bullet: filled dot
\setbeamertemplate{itemize subitem}{\normalsize$\circ$} % sub-bullet: empty dot
\setbeamertemplate{itemize subsubitem}{\scriptsize--} % sub-sub-bullet: dash


% Font changes
\usepackage[scaled=0.92]{helvet}
\renewcommand{\familydefault}{\sfdefault}

% Packages
\usepackage{tikz}
\usepackage{booktabs}
\usepackage{xcolor}
\usepackage{array}           % Enhanced column types for tables
\usepackage{multirow}        % Spanning multiple rows in tables
\usepackage{makecell}        % Line breaks and formatting in table cells
\usepackage{siunitx}         % Proper formatting of numbers and units
\usepackage{amsmath}         % Enhanced math environments
\usepackage{amsfonts}        % Additional math fonts
\usepackage{amssymb}         % Additional math symbols
\usepackage{url}             % Better URL formatting
\usepackage{graphicx}        % Enhanced graphics support
\usepackage{tabularray}
\UseTblrLibrary{booktabs, siunitx, varwidth}
% For financial presentations specifically
\usepackage{eurosym}         % Euro symbol
\usepackage{textcomp}        % Additional text symbols
\usepackage{hyperref}        % Hyperlinks (should be loaded last)

% Define a footnote
\renewcommand{\footnoterule}{\vspace*{-3pt}\hrule width 2in height 0.4pt\vspace*{2.6pt}}

% Define a Foundation Slide
\newenvironment{foundframe}[1][t]{
    \setbeamercolor{background canvas}{bg=gray!8}
    \setbeamercolor{frametitle}{fg=gray!80!black,bg=gray!25}
    \setbeamercolor{framesubtitle}{fg=gray!70!black,bg=gray!15}
    \setbeamercolor{item}{fg=gray!80!black}
    \setbeamercolor{enumerate item}{fg=gray!80!black}
    
    % Modify the frametitle template for this frame type
    \setbeamertemplate{frametitle}{
        \vspace{0.5em}
        \begin{minipage}[t]{0.75\textwidth}
            \insertframetitle
            \par
            \vspace{0.5em}
            \hrule
            \vspace{0.3em}
            {\small\color{gray}\insertframesubtitle}
        \end{minipage}%
        \hfill
        \begin{minipage}[t]{0.2\textwidth}
            \raggedleft
            \colorbox{gray!30}{%
                \scriptsize\bfseries\color{gray!80!black}%
                   \hspace{3pt}\begin{tabular}{c}Foundation\\Material\end{tabular}\hspace{3pt}%
            }
        \end{minipage}
        \vspace{0.3em}
    }
    
    \begin{frame}[#1]
}{
    \end{frame}
}

% Define Practice Slide
\newenvironment{practiceframe}[1][t]{
    \setbeamercolor{background canvas}{bg=white}
    \setbeamercolor{frametitle}{fg=blue!80!black,bg=blue!15}
    \setbeamercolor{framesubtitle}{fg=blue!70!black,bg=blue!10}
    \setbeamercolor{item}{fg=blue!80!black}
    \setbeamercolor{enumerate item}{fg=blue!80!black}
    \setbeamercolor{normal text}{fg=blue!90!black}
    
    % Modify the frametitle template for this frame type
    \setbeamertemplate{frametitle}{
        \vspace{0.5em}
        \begin{minipage}[t]{0.75\textwidth}
            \insertframetitle
            \par
            \vspace{0.5em}
            \hrule
            \vspace{0.3em}
            {\small\color{blue!70!black}\insertframesubtitle}
        \end{minipage}%
        \hfill
        \begin{minipage}[t]{0.2\textwidth}
            \raggedleft
            \colorbox{blue!20}{%
                \scriptsize\bfseries\color{blue!80!black}%
                   \hspace{3pt}\begin{tabular}{c}Practice\\Questions\end{tabular}\hspace{3pt}%
            }
        \end{minipage}
        \vspace{0.3em}
    }
    
    \begin{frame}[#1]
}{
    \end{frame}
}

% Define Excel Slide
\newenvironment{excelframe}[1][t]{
    \setbeamercolor{background canvas}{bg=white}
    \setbeamercolor{frametitle}{fg=blue!80!black,bg=blue!15}
    \setbeamercolor{framesubtitle}{fg=blue!70!black,bg=blue!10}
    \setbeamercolor{item}{fg=blue!80!black}
    \setbeamercolor{enumerate item}{fg=blue!80!black}
    \setbeamercolor{normal text}{fg=blue!90!black}
    
    % Modify the frametitle template for this frame type
    \setbeamertemplate{frametitle}{
        \vspace{0.5em}
        \begin{minipage}[t]{0.75\textwidth}
            \insertframetitle
            \par
            \vspace{0.5em}
            \hrule
            \vspace{0.3em}
            {\small\color{blue!70!black}\insertframesubtitle}
        \end{minipage}%
        \hfill
        \begin{minipage}[t]{0.2\textwidth}
            \raggedleft
            \colorbox{green!10}{%
                \scriptsize\bfseries\color{blue!80!black}%
                   \hspace{3pt}\begin{tabular}{c}MS Excel\end{tabular}\hspace{3pt}%
            }
        \end{minipage}
        \vspace{0.3em}
    }
    
    \begin{frame}[#1]
}{
    \end{frame}
}

% Define Caution Slide
\newenvironment{cautionframe}[1][t]{
    \setbeamercolor{background canvas}{bg=white}
    \setbeamercolor{frametitle}{fg=blue!80!black,bg=blue!15}
    \setbeamercolor{framesubtitle}{fg=blue!70!black,bg=blue!10}
    \setbeamercolor{item}{fg=blue!80!black}
    \setbeamercolor{enumerate item}{fg=blue!80!black}
    \setbeamercolor{normal text}{fg=blue!90!black}
    
    % Modify the frametitle template for this frame type
    \setbeamertemplate{frametitle}{
        \vspace{0.5em}
        \begin{minipage}[t]{0.75\textwidth}
            \insertframetitle
            \par
            \vspace{0.5em}
            \hrule
            \vspace{0.3em}
            {\small\color{blue!70!black}\insertframesubtitle}
        \end{minipage}%
        \hfill
        \begin{minipage}[t]{0.2\textwidth}
            \raggedleft
            \colorbox{red!10}{%
                \scriptsize\bfseries\color{blue!80!black}%
                   \hspace{3pt}\begin{tabular}{c}Caution\end{tabular}\hspace{3pt}%
            }
        \end{minipage}
        \vspace{0.3em}
    }
    
    \begin{frame}[#1]
}{
    \end{frame}
}

% Add to footnotes
\makeatletter
\newcommand\blfootnote[1]{%
  \begingroup
  \renewcommand\thefootnote{}%
  \renewcommand\@makefntext[1]{\raggedright\leftskip=0pt ##1}%
  \footnote{\scriptsize #1}%
  \addtocounter{footnote}{-1}%
  \endgroup
}
\makeatother

% Set the footer -- change 
\setbeamertemplate{footline}{
  \leavevmode%
  \vspace{2ex}
  \hbox{%
    % Left box: Econ 457
    \begin{beamercolorbox}[wd=.4\paperwidth,ht=2.5ex,dp=1ex,left]{author in head/foot}%
      \hspace{1em}Econ 457
    \end{beamercolorbox}%
    % Middle box: Week
    \begin{beamercolorbox}[wd=.2\paperwidth,ht=2.5ex,dp=1ex,center]{date in head/foot}%
      \centering\week
    \end{beamercolorbox}%
    % Right box: Slide numbers
    \begin{beamercolorbox}[wd=.4\paperwidth,ht=2.5ex,dp=1ex,center]{date in head/foot}%
      \hfill\insertframenumber{} 
    \end{beamercolorbox}%
  }%
  \vskip0pt%
}

\begin{document}

\frame{\titlepage}

\begin{frame}
    \frametitle{Outline}

    End of semester\\
    \vspace{1em}

    \begin{enumerate}
        \item Swaps
        \item OTC and Cleared Swaps
        \item Types of Swaps
            \begin{itemize}
                \item Interest Rate Swaps
                \item Credit Default Swaps
            \end{itemize}
        \item AIG
            \begin{itemize}
                \item Securitization
                \item CDOs
                \item AIG Financial Products
                \item AIG Failure and Bailout
            \end{itemize}
    \end{enumerate}

\end{frame}

\begin{frame}[t]
    \frametitle{End of Semester}
    \framesubtitle{}

    Please fill out your course evaluations\\
    \vspace{1em}
    I appreciate your candid feedback.

\end{frame}

\begin{frame}[t]
    \frametitle{End of Semester}
    \framesubtitle{Final}

    \textbf{Date:} Tuesday, December 16, 2:00 - 4:00 pm\\
    \vspace{1em}
    
    \textbf{Location:} Normal Classroom
    \vspace{1em}

    \textbf{Content:}
    \begin{itemize}
        \item Material covered in first midterm: 40\%
        \item Material Covered in second midterm: 40\%
        \item Options, swaps, and futures: 20\%
    \end{itemize}

    \textit{Please bring a caluclator}

\end{frame}

\begin{frame}[t]
    \frametitle{1. Swaps}
    \framesubtitle{What is a Swap?}

    \textbf{Definition:} A swap is a derivative contract where 
    two parties agree to exchange sequences of cash flows 
    for a set period of time.

    \vspace{1em}
    \begin{itemize}
        \item \textbf{Bilateral agreement} between two counterparties
        \item \textbf{Exchange of cash flows} based on different underlying variables, 
        specified at the time the swap initiated
        \item \textbf{No principal exchange} at initiation (typically)
        \item \textbf{Zero initial value} when fairly priced
        \item \textbf{Customizable terms} (notional amount, maturity, payment frequency)
    \end{itemize}

\end{frame}

\begin{frame}[t]
    \frametitle{2. OTC and Cleared Swaps}
    \framesubtitle{OTC}

    \textbf{Over-the-Counter (OTC) Markets:}
    \begin{itemize}
        \item \textbf{Definition:} Trades executed directly between counterparties, not on exchanges
        \item \textbf{Characteristics:} Customizable terms, bilateral negotiation
        \item \textbf{Counterparty risk:} If the counterparty fails, the swap may be worth zero, regardless of
         the underlying variables.
    \end{itemize}
    \vspace{1em}

    \textbf{Pre-2008:} Most swaps were OTC with significant counterparty exposure

\end{frame}

\begin{frame}[t]
    \frametitle{2. OTC and Cleared Swaps}
    \framesubtitle{Cleared Swaps}

    \textbf{Central Clearing (Post-Dodd Frank):}
    \begin{itemize}
        \item \textbf{Central Counterparty (CCP):} Clearinghouse becomes counterparty to both sides
        \item \textbf{Benefits:} 
            \begin{itemize}
                \item Eliminates bilateral counterparty risk
                \item Standardizes terms and margining
                \item Provides transparency and netting
            \end{itemize}
        \item \textbf{Requirements:} 
            \begin{itemize}
                \item Initial and variation margin posting
                \item Daily mark-to-market settlements
                \item Standardized contract terms
            \end{itemize}
    \end{itemize}

    \vspace{1em}
    \textbf{Today:} Most interest rate swaps are centrally cleared, significantly reducing systemic risk

\end{frame}

\begin{frame}[t]
    \frametitle{2. Types of Swaps}
        \framesubtitle{Interest Rate Swaps}

        Example:  \\

        \textbf{Company A}: Paying floating (SOFR) and receiving fixed (6.95\%).\\

        \textbf{Company B}: Paying fixed (7.05\%) and receiving floating (SOFR).\\
        \vspace{1em}

        \centering
        \includegraphics[width=\textwidth]{figures/ch23_irs.png}


\end{frame}

\begin{frame}[t]
    \frametitle{2. Types of Swaps}
        \framesubtitle{Interest Rate Swaps}

    Uses of interest rate swaps:
    \begin{enumerate}
        \item \textbf{Asset-Liability Matching:}   For example, banks may 
        \textit{pay fixed} to to convert fixed rate asset (loans)
         to a floating rate asset, which are better matched to their floating rate liabilities.
        \item \textbf{Adjust Exposure to Interest Rate Risk (with no upfront cash):}   Swaps provide an alternative to bonds and do not 
        require cash upfront.  For example, underfunded pension plans may \textit{receive fixed} to gain exposure to long term interest rates and 
        offset long-dated liabilities.
    \end{enumerate}

\end{frame}

\begin{frame}[t]
    \frametitle{2. Types of Swaps}
    \framesubtitle{Interest Rate Swaps - Pricing and Valuation}

    Fixed rate is set so that PV of both legs are equal:
    $$\text{Fixed Rate} \times  {\sum_{i=1}^{n} \frac{\text{Notional}}{(1+r_i)^{t_i}}} = \sum_{i=1}^{n} \frac{F_i \times \text{Notional}}{(1+r_i)^{t_i}}$$
    Where $F_i$ is the forward rate and $r_i$ is the discount rate for period $i$.
    Notes:
    \begin{itemize}
        \item Swap has zero initial value when fairly priced
        \item After origination, value changes as yield curve shifts ($F_i$ and $r_i$ change)
        \item The value of the swap can be determined by the difference between the 
        value of a fixed rate bond and a floating rate bond.
        \item The received fixed party has exposure similar to that of a bond owner: gains 
        when interest rates fall, loses when interest rates rise
    \end{itemize}

\end{frame}

\begin{frame}
    \frametitle{2. Types of Swaps}
        \framesubtitle{Interest Rate Swaps}

        \centering
        \includegraphics[width=0.7\textwidth]{figures/ch23_swap_spreads.png}

\end{frame}

\begin{frame}[t]
    \frametitle{2. Types of Swaps}
        \framesubtitle{Interest Rate Swaps}

        Reasons why there may be a difference between swap yields and Treasury yields:
        \begin{enumerate}
            \item End users may be different and the market may be segmented  
                \begin{itemize}
                    \item For example, insurance companies \textbf{receive fixed} swaps to hedge liabilities. 
                    Banks may \textbf{pay fixed} to convert fixed rate assets to floating rate assets.
                \end{itemize}
            \item Differences in liquidity
                \begin{itemize}
                    \item Swaps may be more liquidity
                \end{itemize}
            \item Regulatory factors prevent arbitragers from closing the spreads
                \begin{itemize}
                    \item Banks may be required to hold capital against US Treasuries due to the 
                    Single Leverage Ratio.   Swaps do not get the same treatment because they are derivatives.
                \end{itemize}
            \end{enumerate}
        \vspace{1em}
        Most interest rate swaps are now cleared, which has significantly reduced 
        counterparty risk.


\end{frame}

\begin{frame}
    \frametitle{2. Types of Swaps}
        \framesubtitle{Credit Default Swaps}

        Generic Credit Default Swap (CDS)\\
        \vspace{1em}

        \centering
        \includegraphics[width=\textwidth]{figures/ch23_cds.png}

\end{frame}

\begin{frame}
    \frametitle{2. Types of Swaps}
        \framesubtitle{Credit Default Swaps}

        Goldman Sachs CDS\\
        \vspace{1em}
        
        \centering
        \includegraphics[width=\textwidth]{figures/ch23_gs_cds.png}

\end{frame}

\begin{frame}[t]
    \frametitle{3. AIG}
    \framesubtitle{Securitization}

        \centering
        \includegraphics[width=\textwidth]{figures/ch23_sec1.png}

\end{frame}

\begin{frame}[t]
    \frametitle{3. AIG}
    \framesubtitle{Securitization}

        \centering
        \includegraphics[width=\textwidth]{figures/ch23_sec2.png}

\end{frame}

\begin{frame}[t]
    \frametitle{3. AIG}
    \framesubtitle{CDOs}

        \centering
        \includegraphics[width=\textwidth]{figures/ch23_sec3.png}

\end{frame}

\begin{frame}[t]
    \frametitle{3. AIG}
    \framesubtitle{Financial Products}

    AIG was one of the largest insurance company in the world and had over \$1 trillion in assets in 2008. 
    in assets\\
    \vspace{0.5em}
    
    \textbf{Securities Lending Program}
    \begin{itemize}
        \item AIG facilitated lending of securities held by insurance companies
        \item AIG invested in MBS and CDOs in order to enhance returns.
    \end{itemize}

    \textbf{CDS Sales}
    \begin{itemize}
        \item AIG Financial Products sold \$500 billion CDS on a variety of financial assets, including CDOs.
        \item These were bilateral, OTC swaps.  The amount of required collateral was determined, in part, by AIG's rating.  Downgrade forced collateral to be posted.
    \end{itemize}
    
    Losses on these two activities totaled \$50 billion in 2008.

    \blfootnote{Source: McDonald and Paulson, "AIG in Hindsight", NBER}

\end{frame}


\begin{frame}[t]
    \frametitle{3. AIG}
    \framesubtitle{Financial Products}

    \centering
    \includegraphics[width=\textwidth]{figures/ch23_abx.png}

\end{frame}

\begin{frame}[t]
    \frametitle{3. AIG}
    \framesubtitle{Failure and Bailout}

    \begin{itemize}
        \item Lehman Brothers filed for bankruptcy on Mon., Sep 15, 2008
        \item Markets panicked. The S\&P fell by more than 
        4.4\% percent on Monday and then by another 4.7\% on Tuesday.
        \item AIG's credit rating was downgraded, requiring it to post additional collateral on its OTC swaps.
        \item AIG was unable to borrow in short-term debt markets
        \item AIG received an \$85 billion line of credit from the Federal Reserve on Tuesday night
        \begin{itemize}
            \item AIG drew close to \$40 billion in the first week.
            \item The loan was under the Fed's 13(3) authority.   
            \item The Fed accepted the AIG insurance business as collateral
        \end{itemize}
    \end{itemize}

\end{frame}

\begin{frame}[t]
    \frametitle{3. AIG}
    \framesubtitle{Failure and Bailout}

    \begin{itemize}
        \item The Treasury Department received a 79.9\% ownership stake in AIG 
        in exchange for the extraordinary support.
        \item The initial terms of the Fed loan were tough, and further ratings agency downgraded 
        forced AIG to post additional collateral, exacerbating the situation.
        \item A month later TARP funds were used to inject capital into AIG
        \item Total government support for AIG reached \$180 billion: \$70 billion in 
        capital and \$110 billion in loans.
        \item AIG Financial Products paid bonuses of \$165 million in March 2008. 
        This was a public relations nightmare, but the Treasury decided it couldn't 
        void the contracts.
        \item Treasury sold its stake in AIG in December 2012, and made \$7.6 billion 
        profit
    \end{itemize}

\end{frame}

\begin{frame}[t]
    \frametitle{3. AIG}
    \framesubtitle{Failure and Bailout}

    Dodd-Frank Wall Street Reform and Consumer Protection Act of 2010
    \begin{itemize}
        \item Systemically Important Financial Institutions (SIFI) are now regulated by the Federal 
        Reserve.   The Fed can require SIFIs to hold capital.
            \begin{itemize}
                \item AIG did not really have a regulator prior to 2008
                \item AIG was designated as a SIFI from 2013-2017.   
            \end{itemize}
        \item It is now much harder for the Federal to make loans to single financial institutions.   
        Loans should be 'broad based'
        \item Interest rate swaps and some CDX are now cleared (reducing counter party risk)
        \item The FDIC has 'Orderly Liquidation Authority' covering non-bank SIFIs. (Prior to 2008 there was no 
        workable bankrutpcy regime for AIG)
    \end{itemize}

\end{frame}

\begin{frame}[t]
    \frametitle{3. AIG}
    \framesubtitle{Failure and Bailout}

    \centering
    \includegraphics[width=\textwidth]{figures/ch23_aig.png}

\end{frame}

\begin{frame}[t]
    \frametitle{3. AIG}
    \framesubtitle{Failure and Bailout}

    \vspace{1em}

    \begin{quote}
        "The national commitment to the free market lasted one day... It was Monday"
    \end{quote}
    \begin{flushright}
        - Barney Frank, Congressman from MA, September 2008
    \end{flushright}

    \vspace{1em}

    \begin{quote}
        "If there is a single episode in this entire 18 months that has made me more angry, I can't think of one other than AIG"
    \end{quote}
    \begin{flushright}
        - Ben Bernanke, Fed Chair, March 2009
    \end{flushright}

\end{frame}

\end{document}