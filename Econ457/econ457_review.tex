\documentclass{beamer}

\newcommand{\week}{}

\title{Coolest Charts from Econ 457}
\vspace{1em}
\subtitle{\textit{This is not meant to be a comprehensive review for the final, 
but it's a decent place to start...}}
\author{Econ 457}
\date{\week}


% Reference the shared preamble
\setbeamertemplate{frametitle}{
  \vspace{0.5em}
  \insertframetitle
  \par
  \vspace{0.5em}
  \hrule
  \vspace{0.3em}
  {\small\color{gray}\insertframesubtitle}
}

\setbeamertemplate{navigation symbols}{}
\setbeamertemplate{itemize item}{\textbullet} % main bullet: filled dot
\setbeamertemplate{itemize subitem}{\normalsize$\circ$} % sub-bullet: empty dot
\setbeamertemplate{itemize subsubitem}{\scriptsize--} % sub-sub-bullet: dash


% Font changes
\usepackage[scaled=0.92]{helvet}
\renewcommand{\familydefault}{\sfdefault}

% Packages
\usepackage{tikz}
\usepackage{booktabs}
\usepackage{xcolor}
\usepackage{array}           % Enhanced column types for tables
\usepackage{multirow}        % Spanning multiple rows in tables
\usepackage{makecell}        % Line breaks and formatting in table cells
\usepackage{siunitx}         % Proper formatting of numbers and units
\usepackage{amsmath}         % Enhanced math environments
\usepackage{amsfonts}        % Additional math fonts
\usepackage{amssymb}         % Additional math symbols
\usepackage{url}             % Better URL formatting
\usepackage{graphicx}        % Enhanced graphics support
\usepackage{tabularray}
\UseTblrLibrary{booktabs, siunitx, varwidth}
% For financial presentations specifically
\usepackage{eurosym}         % Euro symbol
\usepackage{textcomp}        % Additional text symbols
\usepackage{hyperref}        % Hyperlinks (should be loaded last)

% Define a footnote
\renewcommand{\footnoterule}{\vspace*{-3pt}\hrule width 2in height 0.4pt\vspace*{2.6pt}}

% Define a Foundation Slide
\newenvironment{foundframe}[1][t]{
    \setbeamercolor{background canvas}{bg=gray!8}
    \setbeamercolor{frametitle}{fg=gray!80!black,bg=gray!25}
    \setbeamercolor{framesubtitle}{fg=gray!70!black,bg=gray!15}
    \setbeamercolor{item}{fg=gray!80!black}
    \setbeamercolor{enumerate item}{fg=gray!80!black}
    
    % Modify the frametitle template for this frame type
    \setbeamertemplate{frametitle}{
        \vspace{0.5em}
        \begin{minipage}[t]{0.75\textwidth}
            \insertframetitle
            \par
            \vspace{0.5em}
            \hrule
            \vspace{0.3em}
            {\small\color{gray}\insertframesubtitle}
        \end{minipage}%
        \hfill
        \begin{minipage}[t]{0.2\textwidth}
            \raggedleft
            \colorbox{gray!30}{%
                \scriptsize\bfseries\color{gray!80!black}%
                   \hspace{3pt}\begin{tabular}{c}Foundation\\Material\end{tabular}\hspace{3pt}%
            }
        \end{minipage}
        \vspace{0.3em}
    }
    
    \begin{frame}[#1]
}{
    \end{frame}
}

% Define Practice Slide
\newenvironment{practiceframe}[1][t]{
    \setbeamercolor{background canvas}{bg=white}
    \setbeamercolor{frametitle}{fg=blue!80!black,bg=blue!15}
    \setbeamercolor{framesubtitle}{fg=blue!70!black,bg=blue!10}
    \setbeamercolor{item}{fg=blue!80!black}
    \setbeamercolor{enumerate item}{fg=blue!80!black}
    \setbeamercolor{normal text}{fg=blue!90!black}
    
    % Modify the frametitle template for this frame type
    \setbeamertemplate{frametitle}{
        \vspace{0.5em}
        \begin{minipage}[t]{0.75\textwidth}
            \insertframetitle
            \par
            \vspace{0.5em}
            \hrule
            \vspace{0.3em}
            {\small\color{blue!70!black}\insertframesubtitle}
        \end{minipage}%
        \hfill
        \begin{minipage}[t]{0.2\textwidth}
            \raggedleft
            \colorbox{blue!20}{%
                \scriptsize\bfseries\color{blue!80!black}%
                   \hspace{3pt}\begin{tabular}{c}Practice\\Questions\end{tabular}\hspace{3pt}%
            }
        \end{minipage}
        \vspace{0.3em}
    }
    
    \begin{frame}[#1]
}{
    \end{frame}
}

% Define Excel Slide
\newenvironment{excelframe}[1][t]{
    \setbeamercolor{background canvas}{bg=white}
    \setbeamercolor{frametitle}{fg=blue!80!black,bg=blue!15}
    \setbeamercolor{framesubtitle}{fg=blue!70!black,bg=blue!10}
    \setbeamercolor{item}{fg=blue!80!black}
    \setbeamercolor{enumerate item}{fg=blue!80!black}
    \setbeamercolor{normal text}{fg=blue!90!black}
    
    % Modify the frametitle template for this frame type
    \setbeamertemplate{frametitle}{
        \vspace{0.5em}
        \begin{minipage}[t]{0.75\textwidth}
            \insertframetitle
            \par
            \vspace{0.5em}
            \hrule
            \vspace{0.3em}
            {\small\color{blue!70!black}\insertframesubtitle}
        \end{minipage}%
        \hfill
        \begin{minipage}[t]{0.2\textwidth}
            \raggedleft
            \colorbox{green!10}{%
                \scriptsize\bfseries\color{blue!80!black}%
                   \hspace{3pt}\begin{tabular}{c}MS Excel\end{tabular}\hspace{3pt}%
            }
        \end{minipage}
        \vspace{0.3em}
    }
    
    \begin{frame}[#1]
}{
    \end{frame}
}

% Define Caution Slide
\newenvironment{cautionframe}[1][t]{
    \setbeamercolor{background canvas}{bg=white}
    \setbeamercolor{frametitle}{fg=blue!80!black,bg=blue!15}
    \setbeamercolor{framesubtitle}{fg=blue!70!black,bg=blue!10}
    \setbeamercolor{item}{fg=blue!80!black}
    \setbeamercolor{enumerate item}{fg=blue!80!black}
    \setbeamercolor{normal text}{fg=blue!90!black}
    
    % Modify the frametitle template for this frame type
    \setbeamertemplate{frametitle}{
        \vspace{0.5em}
        \begin{minipage}[t]{0.75\textwidth}
            \insertframetitle
            \par
            \vspace{0.5em}
            \hrule
            \vspace{0.3em}
            {\small\color{blue!70!black}\insertframesubtitle}
        \end{minipage}%
        \hfill
        \begin{minipage}[t]{0.2\textwidth}
            \raggedleft
            \colorbox{red!10}{%
                \scriptsize\bfseries\color{blue!80!black}%
                   \hspace{3pt}\begin{tabular}{c}Caution\end{tabular}\hspace{3pt}%
            }
        \end{minipage}
        \vspace{0.3em}
    }
    
    \begin{frame}[#1]
}{
    \end{frame}
}

% Add to footnotes
\makeatletter
\newcommand\blfootnote[1]{%
  \begingroup
  \renewcommand\thefootnote{}%
  \renewcommand\@makefntext[1]{\raggedright\leftskip=0pt ##1}%
  \footnote{\scriptsize #1}%
  \addtocounter{footnote}{-1}%
  \endgroup
}
\makeatother

% Set the footer -- change 
\setbeamertemplate{footline}{
  \leavevmode%
  \vspace{2ex}
  \hbox{%
    % Left box: Econ 457
    \begin{beamercolorbox}[wd=.4\paperwidth,ht=2.5ex,dp=1ex,left]{author in head/foot}%
      \hspace{1em}Econ 457
    \end{beamercolorbox}%
    % Middle box: Week
    \begin{beamercolorbox}[wd=.2\paperwidth,ht=2.5ex,dp=1ex,center]{date in head/foot}%
      \centering\week
    \end{beamercolorbox}%
    % Right box: Slide numbers
    \begin{beamercolorbox}[wd=.4\paperwidth,ht=2.5ex,dp=1ex,center]{date in head/foot}%
      \hfill\insertframenumber{} 
    \end{beamercolorbox}%
  }%
  \vskip0pt%
}

\begin{document}

\frame{\titlepage}

\begin{frame}[t]
    \frametitle{Econ 457: Material Covered}
    \framesubtitle{}

    \vspace{-1em}

        \begin{table}
        \centering
        \footnotesize  % Global font size for the table
        \begin{tblr}{
            colspec = {Q[l,wd=2cm] Q[c,wd=2cm] Q[l,wd=6cm]}
        }
        \toprule
        Subject & Book Chapters & Sub-topics \\
        \midrule
        Intro & 5 & Measuring Returns, Distribution of Returns, Evaluating Returns \\
        Portfolio Construction & 6, 7, 8 & Capital Allocation, Diversification, Index Model \\
        Market Equilibrium & 9, 10, 11, 12 & CAPM, Fama-French Factors, Efficient Market Hypothesis \\
        Fixed Income & 14, 15, 16 & Prices, Yields, Yield Curve, Duration and Convexity \\
        Equity & 18 & Dividend Discount Models, Price-Earnings Ratios\\
        Derivatives & 20, 21, 22, 23 & Futures, Swaps, Options \\
        \bottomrule
        \end{tblr}
    \end{table}
    \vfill
    \footnotesize
    Note, this is subject to change throughout the semester.

\end{frame}

\begin{frame}[t]
    \frametitle{Distributions of Returns of Financial Assets}
    \framesubtitle{}

    \begin{columns}
        \begin{column}{0.5\textwidth}
            \centering
            \includegraphics[width=1.1\textwidth]{figures/ch5_2_sp500_wdist.png}
        \end{column}

        \begin{column}{0.5\textwidth}
            \centering
            \includegraphics[width=1.1\textwidth]{figures/ch5_2_10y_wdist.png}
        \end{column}
    \end{columns}

\end{frame}

\begin{frame}[t]
    \frametitle{Distributions of Returns of Financial Assets}
    \framesubtitle{}

    \textbf{Galton Board}\\
    https://youtu.be/3m4bxse2JEQ?si=SIhI4vECWSKwkQV-
    \vspace{2em}

    \centering
    \includegraphics[width=0.6\textwidth]{figures/review_galton.png}

\end{frame}

\begin{frame}[t]
    \frametitle{Capital Allocation}
    \framesubtitle{}

    \centering
    \includegraphics[width=0.8\textwidth]{figures/ch6_1_opt.png}

\end{frame}

\begin{frame}
    \frametitle{Diversification}
    \framesubtitle{}

    Two assets: Opportunity set for different covariances\\
    \vspace{1em}
    \centering
    \includegraphics[width=0.7\textwidth]{figures/ch7_1_spytlt_er_cov.png} 

\end{frame}

\begin{frame}
    \frametitle{Diversification}
    \framesubtitle{}

    \centering
    \includegraphics[width=0.9\textwidth]{figures/ch7_1_spytlt_sharpe_opt.png}

\end{frame}

\begin{frame}[t]
    \frametitle{Diversification}
    \framesubtitle{}

    \centering
    \includegraphics[width=0.7\textwidth]{figures/ch7_2_spytltgld_er_std_with_cal.png}

    \blfootnote{Data Source: Yahoo Finance}

\end{frame}

\begin{frame}[t]
    \frametitle{Limits of Diversification}
    \framesubtitle{}

    \centering
    \includegraphics[width=0.7\textwidth]{figures/ch08_1_diversification_benefits.png}
    \vspace{1em}

    \footnotesize
    \raggedright
    Monthly returns since 2010.   Portfolios created from non-randomly selected stocks (previous slide). 
    Stocks added to the portfolio in order.   While a different order of adding stocks may change 
    the chart slightly, the broader point is that it doesn't take all that many stocks to replicate 
    the volatility of the S\&P 500.

\end{frame}

\begin{frame}[t]
    \frametitle{CAPM}
    \framesubtitle{Prices Adjust!}

    \centering
    \includegraphics[width=0.8\textwidth]{figures/ch9_sml_with_alpha.png}

\end{frame}

\begin{frame}[t]
    \frametitle{2. Applied CAPM: Disney (DIS)}
    \framesubtitle{Plot}

    \centering
    \includegraphics[width=0.8\textwidth]{figures/ch9_dissp_20_25_wreg.png}

\end{frame}

\begin{frame}[t]
    \frametitle{Fama-French Factors}
    \framesubtitle{Factor Performance}

    \centering
    \includegraphics[width=0.7\textwidth]{figures/ch10_1_fffactors.png}

    \raggedright

    Conclusion: Both SML and HML are factor portfolios that have significant 
    risk premiums associated with them.   Investors have been rewarded for bearing 
    exposure to these factor portfolios.

    \blfootnote{Data Source: Ken French data library}

\end{frame}

\begin{frame}[t]
    \frametitle{Momentum Factor}
    \framesubtitle{}

    \centering
    \includegraphics[width=0.8\textwidth]{figures/ch10_ff_momentum_logs.png}

    \blfootnote{Data Source: Ken French data library}

\end{frame}

\begin{frame}[t]
    \frametitle{Efficient Markets Hypothesis}
    \framesubtitle{Robert Shiller Chart}

    \centering
    \includegraphics[width=0.8\textwidth]{figures/ch11_1_shiller_jb.png}

\end{frame}

\begin{frame}[t]
    \frametitle{Bond Price and Yields}
    \framesubtitle{}

    \centering
    \includegraphics[width=0.8\textwidth]{figures/ch14_1_30y_priceyield.png}

\end{frame}

\begin{frame}[t]
    \frametitle{Bond Prices}
    \framesubtitle{}

    \centering
    \includegraphics[width=0.8\textwidth]{figures/ch14_1_30y_coupon.png}

\end{frame}

\begin{frame}[t]
    \frametitle{Mortgage Bonds}
    \framesubtitle{Prepayment Risk and Negative Convexity}

    \centering
    \includegraphics[width=0.8\textwidth]{figures/ch14_2_mtg_prepayment.png}

\end{frame}

\begin{frame}[t]
    \frametitle{Treasury Yield Curve}
    \framesubtitle{}

    \centering
    \includegraphics[width=0.8\textwidth]{figures/ch15_1_yc_2025-10-23.png}

\end{frame}

\begin{frame}[t]
    \frametitle{Treasury Yield Curve}
    \framesubtitle{Spot Rates and Forward Rates}

    \centering
    \includegraphics[width=0.9\textwidth]{figures/ch15_1_spfw.png}

\end{frame}

\begin{frame}[t]
    \frametitle{Bond Duration}
    \framesubtitle{}

    \centering
    \includegraphics[width=0.8\textwidth]{figures/ch16_1_priceyield_tangent_10s30s.png}

\end{frame}


\begin{frame}[t]
    \frametitle{Dividend Discount Model}
    \framesubtitle{Dividend Growth}

    \centering
    \includegraphics[width=0.8\textwidth]{figures/ch18_1_div_growth.png}

\end{frame}

\begin{frame}[t]
    \frametitle{Cyclically Adjusted P/E (CAPE)}
    \framesubtitle{Historical Comparisons}

    \centering
    \includegraphics[width=0.8\textwidth]{figures/ch18_2_cape_2.png}

    \blfootnote{Data Source: Robert Shiller, available on his Yale website}

\end{frame}

\begin{frame}[t]
    \frametitle{Cyclically Adjusted P/E (CAPE)}
    \framesubtitle{A Good Measure of Value?}

    \centering
    \includegraphics[width=0.8\textwidth]{figures/ch18_2_cape_3.png}

    \blfootnote{Data Source: Robert Shiller, available on his Yale website}

\end{frame}

\begin{frame}[t]
    \frametitle{Call Options}
    \framesubtitle{}

    \centering
    \includegraphics[width=\textwidth]{figures/ch20_1_buy_call.png}

\end{frame}

\begin{frame}[t]
    \frametitle{Common Option Strategies}
    \framesubtitle{Call Spread (1x2)}

    Payoff at Expiry from \textbf{Call Spread (1x2)}\\
    \vspace{1em}
    \centering
    \includegraphics[width=0.9\textwidth]{figures/ch20_1_call_spreadx2.png}

\end{frame}

\begin{frame}[t]
    \frametitle{Option Valuation}
    \framesubtitle{NVDA calls}

    \centering
    \includegraphics[width=0.8\textwidth]{figures/ch21_nvda_call2.png}

\end{frame}

\begin{frame}[t]
    \frametitle{Option Valuation}
    \framesubtitle{NVDA calls - Change in Volatility}

    \centering
    \includegraphics[width=0.8\textwidth]{figures/ch21_nvda_call4.png}

\end{frame}

\end{document}