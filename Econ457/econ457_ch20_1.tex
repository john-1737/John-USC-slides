\documentclass{beamer}

\newcommand{\week}{Week 13-a}

\title{Options}
\subtitle{Reference: Bodie et al, Ch 20}
\author{Econ 457}
\date{\week}

% Reference the shared preamble
\setbeamertemplate{frametitle}{
  \vspace{0.5em}
  \insertframetitle
  \par
  \vspace{0.5em}
  \hrule
  \vspace{0.3em}
  {\small\color{gray}\insertframesubtitle}
}

\setbeamertemplate{navigation symbols}{}
\setbeamertemplate{itemize item}{\textbullet} % main bullet: filled dot
\setbeamertemplate{itemize subitem}{\normalsize$\circ$} % sub-bullet: empty dot
\setbeamertemplate{itemize subsubitem}{\scriptsize--} % sub-sub-bullet: dash


% Font changes
\usepackage[scaled=0.92]{helvet}
\renewcommand{\familydefault}{\sfdefault}

% Packages
\usepackage{tikz}
\usepackage{booktabs}
\usepackage{xcolor}
\usepackage{array}           % Enhanced column types for tables
\usepackage{multirow}        % Spanning multiple rows in tables
\usepackage{makecell}        % Line breaks and formatting in table cells
\usepackage{siunitx}         % Proper formatting of numbers and units
\usepackage{amsmath}         % Enhanced math environments
\usepackage{amsfonts}        % Additional math fonts
\usepackage{amssymb}         % Additional math symbols
\usepackage{url}             % Better URL formatting
\usepackage{graphicx}        % Enhanced graphics support
\usepackage{tabularray}
\UseTblrLibrary{booktabs, siunitx, varwidth}
% For financial presentations specifically
\usepackage{eurosym}         % Euro symbol
\usepackage{textcomp}        % Additional text symbols
\usepackage{hyperref}        % Hyperlinks (should be loaded last)

% Define a footnote
\renewcommand{\footnoterule}{\vspace*{-3pt}\hrule width 2in height 0.4pt\vspace*{2.6pt}}

% Define a Foundation Slide
\newenvironment{foundframe}[1][t]{
    \setbeamercolor{background canvas}{bg=gray!8}
    \setbeamercolor{frametitle}{fg=gray!80!black,bg=gray!25}
    \setbeamercolor{framesubtitle}{fg=gray!70!black,bg=gray!15}
    \setbeamercolor{item}{fg=gray!80!black}
    \setbeamercolor{enumerate item}{fg=gray!80!black}
    
    % Modify the frametitle template for this frame type
    \setbeamertemplate{frametitle}{
        \vspace{0.5em}
        \begin{minipage}[t]{0.75\textwidth}
            \insertframetitle
            \par
            \vspace{0.5em}
            \hrule
            \vspace{0.3em}
            {\small\color{gray}\insertframesubtitle}
        \end{minipage}%
        \hfill
        \begin{minipage}[t]{0.2\textwidth}
            \raggedleft
            \colorbox{gray!30}{%
                \scriptsize\bfseries\color{gray!80!black}%
                   \hspace{3pt}\begin{tabular}{c}Foundation\\Material\end{tabular}\hspace{3pt}%
            }
        \end{minipage}
        \vspace{0.3em}
    }
    
    \begin{frame}[#1]
}{
    \end{frame}
}

% Define Practice Slide
\newenvironment{practiceframe}[1][t]{
    \setbeamercolor{background canvas}{bg=white}
    \setbeamercolor{frametitle}{fg=blue!80!black,bg=blue!15}
    \setbeamercolor{framesubtitle}{fg=blue!70!black,bg=blue!10}
    \setbeamercolor{item}{fg=blue!80!black}
    \setbeamercolor{enumerate item}{fg=blue!80!black}
    \setbeamercolor{normal text}{fg=blue!90!black}
    
    % Modify the frametitle template for this frame type
    \setbeamertemplate{frametitle}{
        \vspace{0.5em}
        \begin{minipage}[t]{0.75\textwidth}
            \insertframetitle
            \par
            \vspace{0.5em}
            \hrule
            \vspace{0.3em}
            {\small\color{blue!70!black}\insertframesubtitle}
        \end{minipage}%
        \hfill
        \begin{minipage}[t]{0.2\textwidth}
            \raggedleft
            \colorbox{blue!20}{%
                \scriptsize\bfseries\color{blue!80!black}%
                   \hspace{3pt}\begin{tabular}{c}Practice\\Questions\end{tabular}\hspace{3pt}%
            }
        \end{minipage}
        \vspace{0.3em}
    }
    
    \begin{frame}[#1]
}{
    \end{frame}
}

% Define Excel Slide
\newenvironment{excelframe}[1][t]{
    \setbeamercolor{background canvas}{bg=white}
    \setbeamercolor{frametitle}{fg=blue!80!black,bg=blue!15}
    \setbeamercolor{framesubtitle}{fg=blue!70!black,bg=blue!10}
    \setbeamercolor{item}{fg=blue!80!black}
    \setbeamercolor{enumerate item}{fg=blue!80!black}
    \setbeamercolor{normal text}{fg=blue!90!black}
    
    % Modify the frametitle template for this frame type
    \setbeamertemplate{frametitle}{
        \vspace{0.5em}
        \begin{minipage}[t]{0.75\textwidth}
            \insertframetitle
            \par
            \vspace{0.5em}
            \hrule
            \vspace{0.3em}
            {\small\color{blue!70!black}\insertframesubtitle}
        \end{minipage}%
        \hfill
        \begin{minipage}[t]{0.2\textwidth}
            \raggedleft
            \colorbox{green!10}{%
                \scriptsize\bfseries\color{blue!80!black}%
                   \hspace{3pt}\begin{tabular}{c}MS Excel\end{tabular}\hspace{3pt}%
            }
        \end{minipage}
        \vspace{0.3em}
    }
    
    \begin{frame}[#1]
}{
    \end{frame}
}

% Define Caution Slide
\newenvironment{cautionframe}[1][t]{
    \setbeamercolor{background canvas}{bg=white}
    \setbeamercolor{frametitle}{fg=blue!80!black,bg=blue!15}
    \setbeamercolor{framesubtitle}{fg=blue!70!black,bg=blue!10}
    \setbeamercolor{item}{fg=blue!80!black}
    \setbeamercolor{enumerate item}{fg=blue!80!black}
    \setbeamercolor{normal text}{fg=blue!90!black}
    
    % Modify the frametitle template for this frame type
    \setbeamertemplate{frametitle}{
        \vspace{0.5em}
        \begin{minipage}[t]{0.75\textwidth}
            \insertframetitle
            \par
            \vspace{0.5em}
            \hrule
            \vspace{0.3em}
            {\small\color{blue!70!black}\insertframesubtitle}
        \end{minipage}%
        \hfill
        \begin{minipage}[t]{0.2\textwidth}
            \raggedleft
            \colorbox{red!10}{%
                \scriptsize\bfseries\color{blue!80!black}%
                   \hspace{3pt}\begin{tabular}{c}Caution\end{tabular}\hspace{3pt}%
            }
        \end{minipage}
        \vspace{0.3em}
    }
    
    \begin{frame}[#1]
}{
    \end{frame}
}

% Add to footnotes
\makeatletter
\newcommand\blfootnote[1]{%
  \begingroup
  \renewcommand\thefootnote{}%
  \renewcommand\@makefntext[1]{\raggedright\leftskip=0pt ##1}%
  \footnote{\scriptsize #1}%
  \addtocounter{footnote}{-1}%
  \endgroup
}
\makeatother

% Set the footer -- change 
\setbeamertemplate{footline}{
  \leavevmode%
  \vspace{2ex}
  \hbox{%
    % Left box: Econ 457
    \begin{beamercolorbox}[wd=.4\paperwidth,ht=2.5ex,dp=1ex,left]{author in head/foot}%
      \hspace{1em}Econ 457
    \end{beamercolorbox}%
    % Middle box: Week
    \begin{beamercolorbox}[wd=.2\paperwidth,ht=2.5ex,dp=1ex,center]{date in head/foot}%
      \centering\week
    \end{beamercolorbox}%
    % Right box: Slide numbers
    \begin{beamercolorbox}[wd=.4\paperwidth,ht=2.5ex,dp=1ex,center]{date in head/foot}%
      \hfill\insertframenumber{} 
    \end{beamercolorbox}%
  }%
  \vskip0pt%
}

\begin{document}

\frame{\titlepage}

\begin{frame}
    \frametitle{Outline}

    \begin{enumerate}
        \item Put and Call Options
        \item Options v. Stock Investments
        \item Common Option Strategies
        \item Other Options
    \end{enumerate}

\end{frame}

\begin{frame}[t]
    \frametitle{Midterm}
    \framesubtitle{}

    \centering
    \includegraphics[width=0.9\textwidth]{figures/midterm2.png}

\end{frame}

\begin{frame}[t]
    \frametitle{Grades for Econ 457}
    \framesubtitle{}

    \begin{table}
        \centering
        \caption{Grades for Econ 457}
        \begin{tblr}{
            colspec = {Q[l,wd=2cm] Q[c,wd=2.cm]}
        }
        \toprule
        Item & Percent of Total \\
        \midrule
        Homework & 25\% \\
        Mid-term 1 & 20\% \\
        Mid-term 2 & 20\% \\
        Final & 35\% \\
        \bottomrule
        \end{tblr}
    \end{table}

\end{frame}

\begin{frame}[t]
    \frametitle{2. Economics 457}
    \framesubtitle{Material Covered}

    \vspace{-1em}

        \begin{table}
        \centering
        \footnotesize  % Global font size for the table
        \begin{tblr}{
            colspec = {Q[l,wd=2cm] Q[c,wd=2cm] Q[l,wd=6cm]}
        }
        \toprule
        Subject & Book Chapters & Sub-topics \\
        \midrule
        Intro & 5 & Measuring Returns, Distribution of Returns, Evaluating Returns \\
        Portfolio Construction & 6, 7, 8 & Capital Allocation, Diversification, Index Model \\
        Market Equilibrium & 9, 10, 11, 12 & CAPM, Fama-French Factors, Efficient Market Hypothesis \\
        Fixed Income & 14, 15, 16 & Prices, Yields, Yield Curve, Duration and Convexity \\
        Equity & 18 & Dividend Discount Models, Price-Earnings Ratios\\
        Derivatives & 20, 21, 22, 23 & Futures, Swaps, Options \\
        \bottomrule
        \end{tblr}
    \end{table}
    \vfill
    \footnotesize
    Note, this is subject to change throughout the semester.

\end{frame}

\begin{frame}[t]
    \frametitle{1. Put and Call Options}
    \framesubtitle{Definitions}

    \textbf{Call Option}
    \begin{itemize}
        \item Gives its holder the right to \textbf{buy} an security 
        at a specified price on or before some expiration date.
        \item Holder will exercise only if the price of the security 
        is \textbf{above} the specified price.
    \end{itemize}

    \textbf{Put Option}
    \begin{itemize}
        \item Gives its holder the right to \textbf{sell} an security 
        at a specified price on or before some expiration date.
        \item Holder will exercise only if the price of the security 
        is \textbf{below} the specified price.
    \end{itemize}

\end{frame}

\begin{frame}[t]
    \frametitle{1. Put and Call Options}
    \framesubtitle{Standard Notation}

    \textbf{Standard Notation:}
    \begin{itemize}
        \item \textbf{Strike Price (K):} The price at which the option can be exercised
        \item \textbf{Current Price (S):} Current market price of the underlying security
        \item \textbf{Expiration Date (T):} Last date the option can be exercised
        \item \textbf{Time to Expiration ($\tau$):} Time remaining until expiration ($\tau = T - t$)
        \item \textbf{Premium (C, P):} Price paid to purchase the option
        \begin{itemize}
            \item $C$ = Call option premium
            \item $P$ = Put option premium
        \end{itemize}
    \end{itemize}

\end{frame}

\begin{frame}[t]
    \frametitle{1. Put and Call Options}
    \framesubtitle{Terminology}

    At the current price, an option is said to be...
    \begin{itemize}
        \item \textit{In the money} when exercising the option 
        would produce a positive cash flow
        \begin{itemize} 
            \item For a call option, security price is \textit{above} 
            the exercise price ($S-K>0$)
            \item For a put option, security price is \textit{below} 
            the exercise price. ($S-K<0$)
        \end{itemize}
        \item \textit{Out of the money} when exercising the option 
        would produce a negative cash flow
        \item \textit{At the money} the current price is equal to the 
        exercise price.
    \end{itemize}

\end{frame}

\begin{frame}[t]
    \frametitle{1. Put and Call Options}
    \framesubtitle{Call Options}

    The value of a \textbf{call} option at expiration equals:
    
    \vspace{1em}
    
    $$\text{Call Payoff} = \begin{cases} 
    S_T - K - C& \text{if } S_T > K \\
    -C & \text{if } S_T \leq K 
    \end{cases}$$
    
    \vspace{1em}
    
    \textbf{Alternatively written as:}
    $$\text{Call Payoff} = \max(S_T - K, 0) - C$$
    
    \vspace{1em}
    
    \textbf{Where:}
    \begin{itemize}
        \item $S_T$ = Stock price at expiration
        \item $K$ = Strike (exercise) price
        \item $C$ is the price paid for the call option
    \end{itemize}

\end{frame}

\begin{frame}[t]
    \frametitle{1. Put and Call Options}
    \framesubtitle{Call Options}

    \centering
    \includegraphics[width=\textwidth]{figures/ch20_1_buy_call.png}

\end{frame}

\begin{frame}[t]
    \frametitle{1. Put and Call Options}
    \framesubtitle{Call Options}

    Selling an option is sometimes referred to as 'writting' an 
    option.\\
    \vspace{1em}
    The payoff at expire for the seller of an option is the 
    inverse of the payoff for the buyer of an option.\\
    \vspace{1em}
    The value of a \textbf{writing a call} option at expiration equals:
    
    \vspace{1em}
    
    $$\text{Call Writing Payoff} = \begin{cases} 
    -(S_T - K) + C& \text{if } S_T > K \\
    +C & \text{if } S_T \leq K 
    \end{cases}$$
    With the standard notation for $S_T$, $K$, and $C$
    
\end{frame}

\begin{frame}[t]
    \frametitle{1. Put and Call Options}
    \framesubtitle{Call Options}

    \centering
    \includegraphics[width=\textwidth]{figures/ch20_1_sell_call.png}

\end{frame}

\begin{frame}[t]
    \frametitle{1. Put and Call Options}
    \framesubtitle{Put Options}

    The value of a \textbf{put} option at expiration equals:
    
    \vspace{1em}
    
    $$\text{Put Payoff} = \begin{cases} 
    -P& \text{if } S_T > K \\
    (K-S_T) - P & \text{if } S_T \leq K 
    \end{cases}$$
    
    \vspace{1em}
    
    \textbf{Alternatively written as:}
    $$\text{Put Payoff} = \max((K-S_T),0) - P$$
    
    \vspace{1em}
    
    Where $S_T$ = Stock price at expiration, 
    $K$ = Strike (exercise) price, and $P$ is 
    the price paid for the put option

\end{frame}

\begin{frame}[t]
    \frametitle{1. Put and Call Options}
    \framesubtitle{Put Options}

    \centering
    \includegraphics[width=\textwidth]{figures/ch20_1_buy_put.png}

\end{frame}

\begin{frame}[t]
    \frametitle{1. Put and Call Options}
    \framesubtitle{Put Options}

    The value of a \textbf{writing a put} option at expiration equals:
    
    $$\text{Put Writing Payoff} = \begin{cases} 
    P & \text{if } S_T > K \\
    -(K-S_T) + P & \text{if } S_T \leq K 
    \end{cases}$$
    With the standard notation for $S_T$, $K$, and $P$
    
\end{frame}

\begin{frame}[t]
    \frametitle{1. Put and Call Options}
    \framesubtitle{Put Options}

    \centering
    \includegraphics[width=\textwidth]{figures/ch20_1_sell_put.png}

\end{frame}

\begin{frame}[t]
    \frametitle{2. Options v. Stock Investments}
    \framesubtitle{}

    When would you purchase a call option rather than buy shares of stock directly?

    Example:
    \begin{itemize}
        \item Suppose you think a stock, currently selling for \$100, will appreciate
        \item A 1-year maturity call option with an exercise price of \$100 currently 
        sells for \$10 and the interest rate is 3\%
        \item You have \$10,000 to invest
    \end{itemize}

\end{frame}

\begin{frame}[t]
    \frametitle{2. Options v. Stock Investments}
    \framesubtitle{}

    Strategies:
    \begin{enumerate}
        \item[A. ] Invest entirely in stock
        \begin{itemize}
            \item Buy 100 shares, each selling for \$100
        \end{itemize}

        \item[B. ] Invest entirely in at-the-money- call options
        \begin{itemize}
            \item Buy 1,000 calls, each selling for \$10
        \end{itemize}
        
        \item[C. ] Invest in call options and T-Bills
        \begin{itemize}
            \item Buy 100 call options, paying a total of \$1,000
            \item Invest your remaining \$9,000 in 1-year T-Bills to earn 3\% interest
        \end{itemize}
        
    \end{enumerate}

\end{frame}

\begin{frame}[t]
    \frametitle{2. Options v. Stock Investments}
    \framesubtitle{}

        \begin{table}[h]
        \centering
        \caption{Portfolio Value by Portfolio Strategy}
        \begin{tabular}{|p{2cm}|c|c|c|c|c|c|}
        \hline
        & \multicolumn{6}{c|}{\textbf{Stock Price}} \\
        \hline
        \textbf{Portfolio} & \textbf{\$95} & \textbf{\$100} & \textbf{\$105} & \textbf{\$110} & \textbf{\$115} & \textbf{\$120} \\
        \hline
        Portfolio A: All stock & \$9.5k & \$10k & \$10.5k & \$11k & \$11.5k & \$12k \\
        \hline
        Portfolio B: All options & 0 & 0 & 5k & 10k & 15k & 20k \\
        \hline
        Portfolio C: Call + T-bills & 9.27k & 9.27k & 9.77k & 10.27k & 10.77k & 11.27k \\
        \hline
        \end{tabular}
        \end{table}
    
\end{frame}

\begin{frame}[t]
    \frametitle{2. Options v. Stock Investments}
    \framesubtitle{}

    \begin{table}[h]
    \centering
    \caption{Rate of Return by Portfolio Strategy}
    \begin{tabular}{|p{2cm}|c|c|c|c|c|c|}
    \hline
    & \multicolumn{6}{c|}{\textbf{Stock Price}} \\
    \hline
    \textbf{Portfolio} & \textbf{\$95} & \textbf{\$100} & \textbf{\$105} & \textbf{\$110} & \textbf{\$115} & \textbf{\$120} \\
    \hline
    Portfolio A: All stock & -5\% & 0\% & 5\% & 10\% & 15\% & 20\% \\
    \hline
    Portfolio B: All options & -100\% & -100\% & -50\% & 0.0\% & 50\% & 100\% \\
    \hline
    Portfolio C: Call + T-bills & -7.3\% & -7.3\% & -2.3\% & 2.7\% & 7.7\% & 12.7\% \\
    \hline
    \end{tabular}
    \end{table}
        
\end{frame}

\begin{frame}[t]
    \frametitle{3. Common Option Strategies}
    \framesubtitle{}

    \centering
    \includegraphics[width=0.8\textwidth]{figures/ch20_option_strategies.png}

\end{frame}

\begin{frame}[t]
    \frametitle{3. Common Option Strategies}
    \framesubtitle{}

    Options provide \textbf{leverage}.  Portfolio B, which
    gets its exposure through options, 
    performs more than proportionally to changes in the stock price.  
    This can mean large upside, but it can also result in losing 
    all the investment.\\
    \vspace{1em}
    Options can also provide \textbf{insurance}.   Portfolio C, which 
    has options and T-bills, can never lose more than 9.2\% (limited downside). 
    However, some return is necessarily sacrificed for the insurance.   Portfolio 
    C underperforms an all stock portfolio when the stock rises.

\end{frame}

\begin{frame}[t]
    \frametitle{3. Common Option Strategies}
    \framesubtitle{}

    Five common option strategies:
    \begin{enumerate}
        \item Selling a Covered Call
        \item Buying a Protective Put
        \item Call or put spreads
        \item Straddle
        \item Strangle
    \end{enumerate}

\end{frame}

\begin{frame}[t]
    \frametitle{3. Common Option Strategies}
    \framesubtitle{Selling a Covered Call}

    Selling a covered call:
    \begin{itemize}
        \item Own the underlying security
        \item Sell a call
    \end{itemize}

    Considerations:
    \begin{itemize}
        \item Typically the call strike is out of the money at time of sale.
        \item Income from $C$ improves returns around and below strike price
        \item Returns are capped (sell away large gains)
        \item Loss unlimited, although slighty better than outright long, due to $C$
    \end{itemize}

\end{frame}

\begin{frame}[t]
    \frametitle{3. Common Option Strategies}
    \framesubtitle{Selling a Covered Call}

    Payoff at Expiry from \textbf{Selling a Covered Call}\\
    \vspace{1em}
    \centering
    \includegraphics[width=0.9\textwidth]{figures/ch20_1_cvd_call.png}

\end{frame}


\begin{frame}[t]
    \frametitle{3. Common Option Strategies}
    \framesubtitle{Buying a Protective Put}

    Buying a protective put:
    \begin{itemize}
        \item Own the underlying security
        \item Buy a put
    \end{itemize}

    Considerations:
    \begin{itemize}
        \item Typically the put strike is out of the money at time of purchase.
        \item Cost of $P$ detracts from returns around and below strike price
        \item Losses are limited (bought insurance against large losses)
    \end{itemize}

\end{frame}

\begin{frame}[t]
    \frametitle{3. Common Option Strategies}
    \framesubtitle{Buying a Protective Put}

    Payoff at Expiry from \textbf{Buying a Protective Put}\\
    \vspace{1em}
    \centering
    \includegraphics[width=0.9\textwidth]{figures/ch20_1_prot_put.png}

\end{frame}

\begin{frame}[t]
    \frametitle{3. Common Option Strategies}
    \framesubtitle{Call or Put Spreads}

    Buying a Call Spread:
    \begin{itemize}
        \item Buy an out of the money call: $K_1 = S_0 + x_1$
        \item Sell a call further out of the money $K_2 = S_0 + x_2$
        \item Selling the second call reduces the cost: $C_1 - C_2 < C_1$
    \end{itemize}

    Considerations:
    \begin{itemize}
        \item Purchasing options can be expensive, spreads are less expensive.
        \item Often executed as a 1x2: Buy one option, sell two options
        \item Losses are limited (bought insurance against large losses)
    \end{itemize}

\end{frame}

\begin{frame}[t]
    \frametitle{3. Common Option Strategies}
    \framesubtitle{Call Spread (1x1)}

    Payoff at Expiry from \textbf{Call Spread (1x1)}\\
    \vspace{1em}
    \centering
    \includegraphics[width=0.9\textwidth]{figures/ch20_1_call_spreadx1.png}

\end{frame}

\begin{frame}[t]
    \frametitle{3. Common Option Strategies}
    \framesubtitle{Call Spread (1x2)}

    Payoff at Expiry from \textbf{Call Spread (1x2)}\\
    \vspace{1em}
    \centering
    \includegraphics[width=0.9\textwidth]{figures/ch20_1_call_spreadx2.png}

\end{frame}

\begin{frame}[t]
    \frametitle{3. Common Option Strategies}
    \framesubtitle{Straddles}

    Buying a Straddle:
    \begin{itemize}
        \item Buy a call option at the money: $K = S_0$
        \item Buy a put option at the money: $K = S_0$ (same strike price)
        \item Both options have the same expiration date
        \item Total cost: $C + P$
    \end{itemize}

    Considerations:
    \begin{itemize}
        \item Profits from large price movements in either direction
        \item Expensive strategy due to purchasing two at-the-money options
        \item Requires significant price movement to overcome premium costs
        \item Maximum loss is limited to total premiums paid ($C + P$)
        \item Used when expecting high volatility but uncertain about direction
    \end{itemize}

\end{frame}

\begin{frame}[t]
    \frametitle{3. Common Option Strategies}
    \framesubtitle{Buy Straddle}

    Payoff at Expiry from \textbf{Buy Straddle}\\
    \vspace{1em}
    \centering
    \includegraphics[width=0.9\textwidth]{figures/ch20_1_buy_straddle.png}

\end{frame}

\begin{frame}[t]
    \frametitle{3. Common Option Strategies}
    \framesubtitle{Sell Straddle}

    Payoff at Expiry from \textbf{Sell Straddle}\\
    \vspace{1em}
    \centering
    \includegraphics[width=0.9\textwidth]{figures/ch20_1_sell_straddle.png}

\end{frame}


\begin{frame}[t]
    \frametitle{3. Common Option Strategies}
    \framesubtitle{Strangle}

    Buying a Strangle:
    \begin{itemize}
        \item Similar to Straddle, but separate the strikes -- move the put strike down sligthly 
        and the call strike up slightly
    \end{itemize}

    Considerations:
    \begin{itemize}
        \item Similar to straddle -- profit from large price moves in either direction
        \item Less expensive than a straddle (because strikes are out of the money, premium is 
        lower)
        \item Need a large move to profit
    \end{itemize}

\end{frame}

\begin{frame}[t]
    \frametitle{3. Common Option Strategies}
    \framesubtitle{Buy Strangle}

    Payoff at Expiry from \textbf{Buy Strangle}\\
    \vspace{1em}
    \centering
    \includegraphics[width=0.9\textwidth]{figures/ch20_1_buy_strangle.png}

\end{frame}

\begin{frame}[t]
    \frametitle{3. Common Option Strategies}
    \framesubtitle{Sell Strangle}

    Payoff at Expiry from \textbf{Sell Strangle}\\
    \vspace{1em}
    \centering
    \includegraphics[width=0.9\textwidth]{figures/ch20_1_sell_strangle.png}

\end{frame}

\begin{practiceframe}[t]
    \frametitle{5. Practice}
    \framesubtitle{}

    \begin{enumerate}
        \item[1. ] What are the trade-offs facing an investor who is consider buying a put option on an existing 
        portfolio?
        \item[2. ] What are the trade-offs facing an investor who is considering writing a call option on an 
        existing portfolio?
        \item[3. ] You expect that a stock will break out of its recent range.   You don't know which direction. 
        What is a simple options strategy to position for your view?   Assuming that at-the-money puts and calls both 
        cost \$1, how much does the stock need to move in order for your strategy to be profitable.
        \item[4. ] In what ways is owning a corporate bond similar to selling a put option?

    \end{enumerate}

\end{practiceframe}

\begin{practiceframe}[t]
    \frametitle{5. Practice}
    \framesubtitle{}

    \begin{enumerate}
        \item[5. ] You sell a call option with $X=50$ and buy a call option with $X=60$.   The options are on the same 
        stock and have the same expiration.   One of the calls sells for \$3 and the other sells for \$9.   Draw the 
        payoff graph for this strategy at option expiration.   What is the breakeven point for this strategy?   Is the investor 
        bullish or bearish on the stock?
        \item[6. ] An investor purchases a stock fo \$28 and a put for \$0.50 with a strike price of \$35.   The 
        investor also sells a call for \$0.50 with a strike price of \$40.   What are the maximum possible profit 
        and loss for this position?  Draw the profit and loss diagram for this strategy as a function of the stock 
        price at expiration.
    \end{enumerate}

\end{practiceframe}

\end{document}