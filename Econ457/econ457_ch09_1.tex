\documentclass{beamer}

\newcommand{\week}{Week 5-a}

\title{Capital Asset Pricing Model (CAPM)}
\subtitle{Reference: Bodie et al, Ch 9}
\author{Econ 457}
\date{\week}

% Reference the shared preamble
\setbeamertemplate{frametitle}{
  \vspace{0.5em}
  \insertframetitle
  \par
  \vspace{0.5em}
  \hrule
  \vspace{0.3em}
  {\small\color{gray}\insertframesubtitle}
}

\setbeamertemplate{navigation symbols}{}
\setbeamertemplate{itemize item}{\textbullet} % main bullet: filled dot
\setbeamertemplate{itemize subitem}{\normalsize$\circ$} % sub-bullet: empty dot
\setbeamertemplate{itemize subsubitem}{\scriptsize--} % sub-sub-bullet: dash


% Font changes
\usepackage[scaled=0.92]{helvet}
\renewcommand{\familydefault}{\sfdefault}

% Packages
\usepackage{tikz}
\usepackage{booktabs}
\usepackage{xcolor}
\usepackage{array}           % Enhanced column types for tables
\usepackage{multirow}        % Spanning multiple rows in tables
\usepackage{makecell}        % Line breaks and formatting in table cells
\usepackage{siunitx}         % Proper formatting of numbers and units
\usepackage{amsmath}         % Enhanced math environments
\usepackage{amsfonts}        % Additional math fonts
\usepackage{amssymb}         % Additional math symbols
\usepackage{url}             % Better URL formatting
\usepackage{graphicx}        % Enhanced graphics support
\usepackage{tabularray}
\UseTblrLibrary{booktabs, siunitx, varwidth}
% For financial presentations specifically
\usepackage{eurosym}         % Euro symbol
\usepackage{textcomp}        % Additional text symbols
\usepackage{hyperref}        % Hyperlinks (should be loaded last)

% Define a footnote
\renewcommand{\footnoterule}{\vspace*{-3pt}\hrule width 2in height 0.4pt\vspace*{2.6pt}}

% Define a Foundation Slide
\newenvironment{foundframe}[1][t]{
    \setbeamercolor{background canvas}{bg=gray!8}
    \setbeamercolor{frametitle}{fg=gray!80!black,bg=gray!25}
    \setbeamercolor{framesubtitle}{fg=gray!70!black,bg=gray!15}
    \setbeamercolor{item}{fg=gray!80!black}
    \setbeamercolor{enumerate item}{fg=gray!80!black}
    
    % Modify the frametitle template for this frame type
    \setbeamertemplate{frametitle}{
        \vspace{0.5em}
        \begin{minipage}[t]{0.75\textwidth}
            \insertframetitle
            \par
            \vspace{0.5em}
            \hrule
            \vspace{0.3em}
            {\small\color{gray}\insertframesubtitle}
        \end{minipage}%
        \hfill
        \begin{minipage}[t]{0.2\textwidth}
            \raggedleft
            \colorbox{gray!30}{%
                \scriptsize\bfseries\color{gray!80!black}%
                   \hspace{3pt}\begin{tabular}{c}Foundation\\Material\end{tabular}\hspace{3pt}%
            }
        \end{minipage}
        \vspace{0.3em}
    }
    
    \begin{frame}[#1]
}{
    \end{frame}
}

% Define Practice Slide
\newenvironment{practiceframe}[1][t]{
    \setbeamercolor{background canvas}{bg=white}
    \setbeamercolor{frametitle}{fg=blue!80!black,bg=blue!15}
    \setbeamercolor{framesubtitle}{fg=blue!70!black,bg=blue!10}
    \setbeamercolor{item}{fg=blue!80!black}
    \setbeamercolor{enumerate item}{fg=blue!80!black}
    \setbeamercolor{normal text}{fg=blue!90!black}
    
    % Modify the frametitle template for this frame type
    \setbeamertemplate{frametitle}{
        \vspace{0.5em}
        \begin{minipage}[t]{0.75\textwidth}
            \insertframetitle
            \par
            \vspace{0.5em}
            \hrule
            \vspace{0.3em}
            {\small\color{blue!70!black}\insertframesubtitle}
        \end{minipage}%
        \hfill
        \begin{minipage}[t]{0.2\textwidth}
            \raggedleft
            \colorbox{blue!20}{%
                \scriptsize\bfseries\color{blue!80!black}%
                   \hspace{3pt}\begin{tabular}{c}Practice\\Questions\end{tabular}\hspace{3pt}%
            }
        \end{minipage}
        \vspace{0.3em}
    }
    
    \begin{frame}[#1]
}{
    \end{frame}
}

% Define Excel Slide
\newenvironment{excelframe}[1][t]{
    \setbeamercolor{background canvas}{bg=white}
    \setbeamercolor{frametitle}{fg=blue!80!black,bg=blue!15}
    \setbeamercolor{framesubtitle}{fg=blue!70!black,bg=blue!10}
    \setbeamercolor{item}{fg=blue!80!black}
    \setbeamercolor{enumerate item}{fg=blue!80!black}
    \setbeamercolor{normal text}{fg=blue!90!black}
    
    % Modify the frametitle template for this frame type
    \setbeamertemplate{frametitle}{
        \vspace{0.5em}
        \begin{minipage}[t]{0.75\textwidth}
            \insertframetitle
            \par
            \vspace{0.5em}
            \hrule
            \vspace{0.3em}
            {\small\color{blue!70!black}\insertframesubtitle}
        \end{minipage}%
        \hfill
        \begin{minipage}[t]{0.2\textwidth}
            \raggedleft
            \colorbox{green!10}{%
                \scriptsize\bfseries\color{blue!80!black}%
                   \hspace{3pt}\begin{tabular}{c}MS Excel\end{tabular}\hspace{3pt}%
            }
        \end{minipage}
        \vspace{0.3em}
    }
    
    \begin{frame}[#1]
}{
    \end{frame}
}

% Define Caution Slide
\newenvironment{cautionframe}[1][t]{
    \setbeamercolor{background canvas}{bg=white}
    \setbeamercolor{frametitle}{fg=blue!80!black,bg=blue!15}
    \setbeamercolor{framesubtitle}{fg=blue!70!black,bg=blue!10}
    \setbeamercolor{item}{fg=blue!80!black}
    \setbeamercolor{enumerate item}{fg=blue!80!black}
    \setbeamercolor{normal text}{fg=blue!90!black}
    
    % Modify the frametitle template for this frame type
    \setbeamertemplate{frametitle}{
        \vspace{0.5em}
        \begin{minipage}[t]{0.75\textwidth}
            \insertframetitle
            \par
            \vspace{0.5em}
            \hrule
            \vspace{0.3em}
            {\small\color{blue!70!black}\insertframesubtitle}
        \end{minipage}%
        \hfill
        \begin{minipage}[t]{0.2\textwidth}
            \raggedleft
            \colorbox{red!10}{%
                \scriptsize\bfseries\color{blue!80!black}%
                   \hspace{3pt}\begin{tabular}{c}Caution\end{tabular}\hspace{3pt}%
            }
        \end{minipage}
        \vspace{0.3em}
    }
    
    \begin{frame}[#1]
}{
    \end{frame}
}

% Add to footnotes
\makeatletter
\newcommand\blfootnote[1]{%
  \begingroup
  \renewcommand\thefootnote{}%
  \renewcommand\@makefntext[1]{\raggedright\leftskip=0pt ##1}%
  \footnote{\scriptsize #1}%
  \addtocounter{footnote}{-1}%
  \endgroup
}
\makeatother

% Set the footer -- change 
\setbeamertemplate{footline}{
  \leavevmode%
  \vspace{2ex}
  \hbox{%
    % Left box: Econ 457
    \begin{beamercolorbox}[wd=.4\paperwidth,ht=2.5ex,dp=1ex,left]{author in head/foot}%
      \hspace{1em}Econ 457
    \end{beamercolorbox}%
    % Middle box: Week
    \begin{beamercolorbox}[wd=.2\paperwidth,ht=2.5ex,dp=1ex,center]{date in head/foot}%
      \centering\week
    \end{beamercolorbox}%
    % Right box: Slide numbers
    \begin{beamercolorbox}[wd=.4\paperwidth,ht=2.5ex,dp=1ex,center]{date in head/foot}%
      \hfill\insertframenumber{} 
    \end{beamercolorbox}%
  }%
  \vskip0pt%
}

\begin{document}

\frame{\titlepage}

\begin{frame}
    \frametitle{Material So Far}

    \begin{table}
        \begin{tabular}{llrrr}
        \toprule
         Week & Subject & Chapter in \textit{Bodi et al} \\
        \midrule
        1 & Intro  &  - \\
        2 & Measuring Returns & 5 \\
        3 & Capital Allocation & 6 \\
        4 & Optimal Portfolios & 7 \\
        \textcolor{red}{5} & \textcolor{red}{CAPM} & \textcolor{red}{9} \\
        6 & Index Model & 8 \\
        \bottomrule
        \end{tabular}
    \end{table}

\end{frame}

\begin{frame}
    \frametitle{Outline}

    \begin{enumerate}
        \item The Market Portfolio and Equilibriumm
        \item Implications of CAPM Equilibrium
        \begin{itemize}
            \item Equilibrium Condition for Expected Returns
            \item Prices adjust
        \end{itemize}
        \item A Linear Model
        \item Practice: Applications of CAPM
    \end{enumerate}
\end{frame}

\begin{frame}[t]
    \frametitle{1. The Market Portfolio and Equilibrium}
    \framesubtitle{Markowitz Portfolio Optmization (Bodi, Ch 7)}

    \vspace{-1em}
    \centering
    \includegraphics[width=0.7\textwidth]{figures/ch9_cml.png}

    \blfootnote{The Markwotiz model determines optimal weights $w_i$, given 
    $\mathbb{E}[r_i]$, $r_i$, and $\sigma_i$, for securities $i=1...n$.   Markowitz says nothing about
     what the security prices \textit{should} be.   That is where CAPM comes in.}

\end{frame}

\begin{frame}[t]
    \frametitle{1. The Market Portfolio and Equilibrium}
    \framesubtitle{CAPM assumptions}

    \begin{itemize}
    \item Investors are similar
        \begin{itemize} 
            \item Investors are mean-variance optimizers
            \item Investors have the same expectations for asset characteristics (means, variances, correlations)
            \item Investors have the same time horizons
        \end{itemize}
    \item Markets are complete
        \begin{itemize} 
            \item No taxes
            \item No trading costs
            \item Unlimited leverage available at the risk-free rate
            (and therefore short positions allowed)
        \end{itemize}
    \end{itemize}

    These may seem overly restrictive.   Some can be relaxed and CAPM still holds.  
    More importantly, CAPM is, like all models, a useful simplification.

\end{frame}

\begin{frame}[t]
    \frametitle{1. The Market Portfolio and Equilibrium}
    \framesubtitle{Equilibrium}

    Under these assumptions, we can conclude that:
    \begin{itemize}
        \item The market portfolio is optimal.   No portfolio exists with better reward-to-risk.
        \item The market portfolio is an equilibrum.   
    \end{itemize}
    \vspace{1em}

    The \textbf{expected return of the market} is determined by economic factors 
    (wealth, average risk aversion, risk-free rate).\\
    \vspace{1em}

    Moreover, if the market portfolio is an equilibrium, this has strong implications for 
    \textbf{expected return of individual securities}, which is the next part of the lecture.

\end{frame}

\begin{frame}[t]
    \frametitle{1. The Market Portfolio and Equilibrium}
    \framesubtitle{Market Price of Risk}

    Define the \textbf{market price of risk} as the reward-to-risk for investment in the market portfolio:

    $$\frac{\text{Market Risk Premium}}{\text{Market Variance}} = \frac{\mathbb{E}[R_m]}{\sigma_m^2}$$
    Where $\mathbb{E}[R_m] = \mathbb{E}[r_m] - r_f$ is the risk premium and $\sigma_m^2$ is the variance 
    of the market portfolio.\\
    \vspace{1em}
    Notice that this is \textit{not} the same as the Sharpe Ratio, which has the standard deviation in the denominator.  
    Using variance instead makes the math cleaner when we consider the contribution of a single security to a portfolio risk.\\
    \vspace{1em} 

\end{frame}

\begin{frame}[t]
    \frametitle{1. The Market Portfolio and Equilibrium}
    \framesubtitle{Contribution of a single security to market risk}

    The contribution of a security $xyz$ to the variance of the market portfolio is given by:
    $$\sum_{i=1}^{n}w_i Cov(R_{xyz},R_i)$$
    for all $i$ in the market portfolio.  
    Using standard rules of variances and covariances, this can be rewritten as:
    $$w_{xyz}Cov(R_{xyz},R_m)$$
    where $R_m$ is the return of the market portfolio.  
    In words, the contribution of security $xyz$ to the variance 
    of the market portfolio is 
    the Covariance of $xyz$ with the market portfolio (multiplied by the weight, $w_{xyz}$).

\end{frame}


\begin{frame}[t]
    \frametitle{2. Implications of CAPM Equilibrium}
    \framesubtitle{Equilibrium Condition for Expected Returns}

    From Bodi et al, p 119:
    \vspace{1em}

    \begin{quote}
    A basic principal of equilibrium is that all investments should offer the same reward-to-risk ratio.
    If the ratio were better for one investment than another, investors would rearrange their portfolios.
    This would cause prices to adjust until the expected reward-to-risk were equal.
    \textbf{Therefore, we conclude that the reward-to-risk for each security is equal to the reward-to-risk for the market portfolio.}
    \end{quote}

    This provides a nice equilibrium condition that can 
    be used to determine the expected returns of a security.

\end{frame}

\begin{frame}[t]
    \frametitle{2. Implications of CAPM Equilibrium}
    \framesubtitle{Equilibrium Condition for Expected Returns}


    The equilibrium condition says that overweighting stock $xyz$ relative to the market portfolio should neither improve or detract the reward-to-risk, therefore we it must be the case that:\\

    $$\frac{\mathbb{E}[R_{xyz}]}{Cov(R_{xyz},R_m)} = \frac{\mathbb{E}[R_m]}{\sigma_m^2}$$
    
    \vspace{1em}
    The denominator on left hand side is the contribution of $xzy$ to the 
    risk of the market portfolio.   
    The right hand side of the equation is the reward-to-risk for the entire market.\\
    \vspace{1em}
    Continued...

\end{frame}


\begin{frame}[t]
    \frametitle{2. Implications of CAPM Equilibrium}
    \framesubtitle{Equilibrium Condition for Expected Returns}

    ...Continued.   Rearranging:
    $$\mathbb{E}[R_{xyz}] = \frac{Cov(R_{xyz},R_m)}{\sigma_m^2}\mathbb{E}[R_m]$$
    \vspace{1em} 

    The first term on the right hand side is the contribution of $xyz$ to 
    variance of the market portfolio, 
    as a share of the total variance of the market portfolio.\\
    \vspace{1em}
    This term is often referred to as $\beta_{xyz}$.

\end{frame}

\begin{frame}[t]
    \frametitle{2. Implications of CAPM Equilibrium}
    \framesubtitle{Equilibrium Condition for Expected Returns}

    Define the Beta of a security as as:
    $$\beta_{xyz} = \frac{Cov(R_{xyz},R_m)}{\sigma_m^2}$$
    Beta has two valid interpretations:
    \begin{enumerate}
        \item The risk each dollar invested in security $xyz$ contributes to the risk of the market portfolio.
        \item The regression coefficient for $R_{xyz}$ on $R_m$, or the sensitivity of the 
        excess returns of security $xyz$ to variation in the market excess returns.
    \end{enumerate}

\end{frame}

\begin{frame}[t]
    \frametitle{2. Implications of CAPM Equilibrium}
    \framesubtitle{Prices Adjust}

    From Bodi et al, p 119:\\
    \vspace{1em}
    \begin{quote}
    A basic principal of equilibrium is that all investments should offer the same reward-to-risk ratio.
    If the ratio were better for one investment than another, investors would rearrange their portfolios.
    \textbf{This would cause prices to adjust until the expected reward-to-risk were equal.}
    Therefore, we conclude that the reward-to-risk for each security is equal to the reward-to-risk for the market portfolio.
    \end{quote}


    Shorter version: Prices adjust!

\end{frame}

\begin{frame}[t]
    \frametitle{2. Implications of CAPM Equilibrium}
    \framesubtitle{Prices Adjust}

    \centering
    \includegraphics[width=0.8\textwidth]{figures/ch9_sml.png}

\end{frame}


\begin{frame}[t]
    \frametitle{2. Implications of CAPM Equilibrium}
    \framesubtitle{Prices Adjust}

    Draw the Security Market Line using the following considerations:
    \begin{itemize}
        \item If a security is uncorrelated with the market, then overweighting that 
        security will reduce the variance of the resulting portfolio.   
        The return on this security must be equal to the risk-free return, 
        otherwise it would be possible to construct a better portfolio than the market, 
        thereby violating the equilibrium condition.
        \item If a security is perfectly correlated with the market and had a better 
        expected return than the market portfolio, then that would also violate the 
        equilibrium assumption.   Conclude the return on a security with $\beta=1$ must 
        be the same as the market portfolio
        \item The excess return for a security with $\beta \neq 1$ is given 
        by the equilibrium condition to be $\beta \mathbb{E}[R_M]$
    \end{itemize}
    
    \textit{All securities must lie on the Security Market Line (SML)!}

\end{frame}

\begin{frame}[t]
    \frametitle{2. Implications of CAPM Equilibrium}
    \framesubtitle{Prices Adjust}

    What if a security lies off the SML?

    \centering
    \includegraphics[width=0.8\textwidth]{figures/ch9_sml_with_AB.png}

\end{frame}

\begin{frame}[t]
    \frametitle{2. Implications of CAPM Equilibrium}
    \framesubtitle{Prices Adjust}

    What is the reward-to-risk of overweighting security $xyz$ in the market portfolio?
    
    $$
    \begin{aligned}
    \text{Reward:} &\quad \mathbb{E}[R_{xyz}] = \alpha_{xyz} + \beta \cdot \mathbb{E}[R_m] \\
    \text{Risk:} &\quad Cov(R_{xyz},R_m) 
    \end{aligned}
    $$

    Note that the risk is the Covariance with the market portfolio.   
    The risk is NOT the variance of $r_{xyz}$.   
    This is a key component of the CAPM assumptions.   
    The market portfolio is optimal, investors hold that already, 
    everything else is compared to holding the market portfolio.

    \vspace{1em}
    Continued...

\end{frame}

\begin{frame}[t]
    \frametitle{2. Implications of CAPM Equilibrium}
    \framesubtitle{Prices Adjust}

    \footnotesize
    ...Continued As above, let $\beta = \frac{Cov(R_{xyz},R_m)}{\sigma_m^2}$

    \begin{align*}
    \left(\frac{\text{reward}}{\text{risk}}\right)_{xyz}   &= \frac{\mathbb{E}[\alpha_{xyz}] + \beta \cdot \mathbb{E}[R_m]}{Cov(R_{xyz},R_m)}\\
                                        &= \frac{\mathbb{E}[\alpha_{xyz}]}{Cov(R_{xyz},R_m)} +
                                            \frac{\beta \cdot \mathbb{E}[R_m]}{Cov(R_{xyz},R_m)}\\
                                        &= \frac{\mathbb{E}[\alpha_{xyz}]}{Cov(R_{xyz},R_m)} +
                                            \frac{\frac{Cov(R_{xyz},R_m)}{\sigma_m^2} \cdot \mathbb{E}[R_m]}{Cov(R_{xyz},R_m)}\\
                                        &= \frac{\mathbb{E}[\alpha_{xyz}]}{Cov(R_{xyz},R_m)} +
                                            \frac{\mathbb{E}[R_m]}{\sigma_m^2}\\
                                        &= \frac{\mathbb{E}[\alpha_{xyz}]}{Cov(R_{xyz},R_m)} +
                                            \left(\frac{\text{reward}}{\text{risk}}\right)_m
    \end{align*}

    Therefore, if $\alpha_{xyz} < 0$, then adding $xyz$ will decrease the reward-to-risk of an investor's portfolio.   Continued...

\end{frame}

\begin{frame}
    \frametitle{2. Implications of CAPM Equilibrium}
    \framesubtitle{Prices Adjust}

    \centering
    \includegraphics[width=0.8\textwidth]{figures/ch9_sml_with_alpha.png}

\end{frame}

\begin{frame}[t]
    \frametitle{2. Implications of CAPM Equilibrium}
    \framesubtitle{Prices Adjust}

    But, prices adjust! \\
    \vspace{1em}
    If $\alpha_{xyz} < 0$ in expectation, then investors will sell $xyz$.   
    Selling $xyz$ lowers the price (raises the expected return).   
    This will happen until $\alpha_{xyz} = 0$ in expectation.\\
    \vspace{1em}

    Conclude that, in expectation, we must have $\alpha_{xyz} = 0$.

\end{frame}


\begin{frame}[t]
    \frametitle{3. A Linear Model}
    \framesubtitle{}

    The equilibrium condition for CAPM, with a bit of rearranging, gives the following 
    expression for expected excess returns of security $xyz$:
    $$\mathbb{E}[R_{xyz}] = \frac{Cov(R_{xyz},R_m)}{\sigma_m^2}\mathbb{E}[R_m]$$

    Substituting $\beta_{xyz} = \frac{Cov(R_{xyz},R_m)}{\sigma_m^2}$ 
    and $\mathbb{E}[R_m] = \mathbb{E}[r_m] - r_f$, gives the standard 
    formulation of CAPM for expected returns of security $xyz$:\\
    $$\boxed{\mathbb{E}[r_{xyz}] = r_f + \beta_{xyz}(\mathbb{E}[r_m] - r_f)}$$\\
    \vspace{1em} 



\end{frame}

\begin{frame}[t]
    \frametitle{3. A Linear Model}
    \framesubtitle{}


    $$\boxed{\mathbb{E}[r_{xyz}] = r_f + \beta_{xyz}(\mathbb{E}[r_m] - r_f)}$$
    \vspace{1em}

    This expression gives the expected return of stock $xyz$ as a function of the following:\\
    \begin{enumerate}
        \item Risk-free rate ($r_f$)
        \item Expected excess market return ($\mathbb{E}[R_m] = \mathbb{E}[r_m] - r_f$)
        \item $\beta_{xyz}$
    \end{enumerate}
    \vspace{1em}
    Note that all three of those inputs are needed to estimate $\mathbb{E}[r_{xyz}]$

\end{frame}

\begin{practiceframe}[t]
    \frametitle{1. Practice from Last Class}
    \framesubtitle{Beta, SML and Price Changes}

    Consider the following
    \footnotesize
    \begin{table}[h]
    \centering
    \begin{tabular}{cccc}
    \toprule
    \textbf{Scenario} & \textbf{Market Return} & \textbf{Aggressive Stock} & \textbf{Defensive Stock} \\
    \midrule
    Low & 5\% & -2\% & 6\% \\
    \hline
    High & 25\% & 38\% & 12\% \\
    \bottomrule
    \end{tabular}
    \end{table}

    Questions:
    \begin{enumerate}
        \item What is the expected rate of return for each stock, if the two scenarios are equally likely?
        \item What is the Var-Cov matrix, if the two scenarios are equally likely?
        \item What are the betas of the aggressive and defensive stock?
        \item Draw the SML if T-Bill rate is 6\% and the two scenarios are equally likely.
        \item Plot the two securities relative to the SML. What is the alpha?
        \item What is likely to happen to the \textit{price} of the two stocks?
    \end{enumerate}

    \blfootnote{Example from Bodi et al, p308}

\end{practiceframe}

\begin{practiceframe}[t]
    \frametitle{4. Practice: Applications of CAPM}
    \framesubtitle{Expected Stock Price}

    Assume the risk-free rate is 6\% and the expected return on the market is 16\%.
    A stock sells for \$50 today.   
    It will pay a dividend of \$6 per share at the end of the year.   
    Its beta is 1.2.\\
    \vspace{1em}
    Questions:\\
    \begin{enumerate}
        \item What is the expected return for this stock?
        \item What do investors expect the price to be at the end of the year?
    \end{enumerate}

\end{practiceframe}

\begin{practiceframe}[t]
    \frametitle{4. Practice: Applications of CAPM}
    \framesubtitle{Expected return of a specific project}

    The risk-free rate is 8\% and the expected return on the market portfolio is 16\%. 
    A firm considers a project that is expected to have a beta of 1.3.\\
    \vspace{1em}
    Questions:
    \begin{enumerate}
        \item What is the required rate of return on the project?
        \item If the expected IRR of the project is 19\%, should the project be accepted?
    \end{enumerate}
    \vfill
    
    \blfootnote{The expected IRR is the discount rate 
    that sets the present discounted value of the project 
    (including both costs and returns) equal to zero.}
 
\end{practiceframe}

\end{document}
