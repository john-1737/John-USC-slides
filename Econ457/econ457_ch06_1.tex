\documentclass{beamer}

\newcommand{\week}{Week 3-a}

\title{Capital Allocation}
\subtitle{Reference: Bodie et al, Ch 6}
\author{Econ 457}
\date{\week}

% Reference the shared preamble
\setbeamertemplate{frametitle}{
  \vspace{0.5em}
  \insertframetitle
  \par
  \vspace{0.5em}
  \hrule
  \vspace{0.3em}
  {\small\color{gray}\insertframesubtitle}
}

\setbeamertemplate{navigation symbols}{}
\setbeamertemplate{itemize item}{\textbullet} % main bullet: filled dot
\setbeamertemplate{itemize subitem}{\normalsize$\circ$} % sub-bullet: empty dot
\setbeamertemplate{itemize subsubitem}{\scriptsize--} % sub-sub-bullet: dash


% Font changes
\usepackage[scaled=0.92]{helvet}
\renewcommand{\familydefault}{\sfdefault}

% Packages
\usepackage{tikz}
\usepackage{booktabs}
\usepackage{xcolor}
\usepackage{array}           % Enhanced column types for tables
\usepackage{multirow}        % Spanning multiple rows in tables
\usepackage{makecell}        % Line breaks and formatting in table cells
\usepackage{siunitx}         % Proper formatting of numbers and units
\usepackage{amsmath}         % Enhanced math environments
\usepackage{amsfonts}        % Additional math fonts
\usepackage{amssymb}         % Additional math symbols
\usepackage{url}             % Better URL formatting
\usepackage{graphicx}        % Enhanced graphics support
\usepackage{tabularray}
\UseTblrLibrary{booktabs, siunitx, varwidth}
% For financial presentations specifically
\usepackage{eurosym}         % Euro symbol
\usepackage{textcomp}        % Additional text symbols
\usepackage{hyperref}        % Hyperlinks (should be loaded last)

% Define a footnote
\renewcommand{\footnoterule}{\vspace*{-3pt}\hrule width 2in height 0.4pt\vspace*{2.6pt}}

% Define a Foundation Slide
\newenvironment{foundframe}[1][t]{
    \setbeamercolor{background canvas}{bg=gray!8}
    \setbeamercolor{frametitle}{fg=gray!80!black,bg=gray!25}
    \setbeamercolor{framesubtitle}{fg=gray!70!black,bg=gray!15}
    \setbeamercolor{item}{fg=gray!80!black}
    \setbeamercolor{enumerate item}{fg=gray!80!black}
    
    % Modify the frametitle template for this frame type
    \setbeamertemplate{frametitle}{
        \vspace{0.5em}
        \begin{minipage}[t]{0.75\textwidth}
            \insertframetitle
            \par
            \vspace{0.5em}
            \hrule
            \vspace{0.3em}
            {\small\color{gray}\insertframesubtitle}
        \end{minipage}%
        \hfill
        \begin{minipage}[t]{0.2\textwidth}
            \raggedleft
            \colorbox{gray!30}{%
                \scriptsize\bfseries\color{gray!80!black}%
                   \hspace{3pt}\begin{tabular}{c}Foundation\\Material\end{tabular}\hspace{3pt}%
            }
        \end{minipage}
        \vspace{0.3em}
    }
    
    \begin{frame}[#1]
}{
    \end{frame}
}

% Define Practice Slide
\newenvironment{practiceframe}[1][t]{
    \setbeamercolor{background canvas}{bg=white}
    \setbeamercolor{frametitle}{fg=blue!80!black,bg=blue!15}
    \setbeamercolor{framesubtitle}{fg=blue!70!black,bg=blue!10}
    \setbeamercolor{item}{fg=blue!80!black}
    \setbeamercolor{enumerate item}{fg=blue!80!black}
    \setbeamercolor{normal text}{fg=blue!90!black}
    
    % Modify the frametitle template for this frame type
    \setbeamertemplate{frametitle}{
        \vspace{0.5em}
        \begin{minipage}[t]{0.75\textwidth}
            \insertframetitle
            \par
            \vspace{0.5em}
            \hrule
            \vspace{0.3em}
            {\small\color{blue!70!black}\insertframesubtitle}
        \end{minipage}%
        \hfill
        \begin{minipage}[t]{0.2\textwidth}
            \raggedleft
            \colorbox{blue!20}{%
                \scriptsize\bfseries\color{blue!80!black}%
                   \hspace{3pt}\begin{tabular}{c}Practice\\Questions\end{tabular}\hspace{3pt}%
            }
        \end{minipage}
        \vspace{0.3em}
    }
    
    \begin{frame}[#1]
}{
    \end{frame}
}

% Define Excel Slide
\newenvironment{excelframe}[1][t]{
    \setbeamercolor{background canvas}{bg=white}
    \setbeamercolor{frametitle}{fg=blue!80!black,bg=blue!15}
    \setbeamercolor{framesubtitle}{fg=blue!70!black,bg=blue!10}
    \setbeamercolor{item}{fg=blue!80!black}
    \setbeamercolor{enumerate item}{fg=blue!80!black}
    \setbeamercolor{normal text}{fg=blue!90!black}
    
    % Modify the frametitle template for this frame type
    \setbeamertemplate{frametitle}{
        \vspace{0.5em}
        \begin{minipage}[t]{0.75\textwidth}
            \insertframetitle
            \par
            \vspace{0.5em}
            \hrule
            \vspace{0.3em}
            {\small\color{blue!70!black}\insertframesubtitle}
        \end{minipage}%
        \hfill
        \begin{minipage}[t]{0.2\textwidth}
            \raggedleft
            \colorbox{green!10}{%
                \scriptsize\bfseries\color{blue!80!black}%
                   \hspace{3pt}\begin{tabular}{c}MS Excel\end{tabular}\hspace{3pt}%
            }
        \end{minipage}
        \vspace{0.3em}
    }
    
    \begin{frame}[#1]
}{
    \end{frame}
}

% Define Caution Slide
\newenvironment{cautionframe}[1][t]{
    \setbeamercolor{background canvas}{bg=white}
    \setbeamercolor{frametitle}{fg=blue!80!black,bg=blue!15}
    \setbeamercolor{framesubtitle}{fg=blue!70!black,bg=blue!10}
    \setbeamercolor{item}{fg=blue!80!black}
    \setbeamercolor{enumerate item}{fg=blue!80!black}
    \setbeamercolor{normal text}{fg=blue!90!black}
    
    % Modify the frametitle template for this frame type
    \setbeamertemplate{frametitle}{
        \vspace{0.5em}
        \begin{minipage}[t]{0.75\textwidth}
            \insertframetitle
            \par
            \vspace{0.5em}
            \hrule
            \vspace{0.3em}
            {\small\color{blue!70!black}\insertframesubtitle}
        \end{minipage}%
        \hfill
        \begin{minipage}[t]{0.2\textwidth}
            \raggedleft
            \colorbox{red!10}{%
                \scriptsize\bfseries\color{blue!80!black}%
                   \hspace{3pt}\begin{tabular}{c}Caution\end{tabular}\hspace{3pt}%
            }
        \end{minipage}
        \vspace{0.3em}
    }
    
    \begin{frame}[#1]
}{
    \end{frame}
}

% Add to footnotes
\makeatletter
\newcommand\blfootnote[1]{%
  \begingroup
  \renewcommand\thefootnote{}%
  \renewcommand\@makefntext[1]{\raggedright\leftskip=0pt ##1}%
  \footnote{\scriptsize #1}%
  \addtocounter{footnote}{-1}%
  \endgroup
}
\makeatother

% Set the footer -- change 
\setbeamertemplate{footline}{
  \leavevmode%
  \vspace{2ex}
  \hbox{%
    % Left box: Econ 457
    \begin{beamercolorbox}[wd=.4\paperwidth,ht=2.5ex,dp=1ex,left]{author in head/foot}%
      \hspace{1em}Econ 457
    \end{beamercolorbox}%
    % Middle box: Week
    \begin{beamercolorbox}[wd=.2\paperwidth,ht=2.5ex,dp=1ex,center]{date in head/foot}%
      \centering\week
    \end{beamercolorbox}%
    % Right box: Slide numbers
    \begin{beamercolorbox}[wd=.4\paperwidth,ht=2.5ex,dp=1ex,center]{date in head/foot}%
      \hfill\insertframenumber{} 
    \end{beamercolorbox}%
  }%
  \vskip0pt%
}

\begin{document}

\frame{\titlepage}

\begin{frame}[t]
    \frametitle{Economics 457}
    \framesubtitle{Course Outline Covered}

    \vspace{-1em}

        \begin{table}
        \centering
        \footnotesize  % Global font size for the table
        \begin{tblr}{
            colspec = {Q[l,wd=2cm] Q[c,wd=2cm] Q[l,wd=6cm]}
        }
        \toprule
        Subject & Weeks & Sub-topics \\
        \midrule
        Intro & 1 \& 2 & Measuring Returns, Distribution of Returns, Evaluating Returns \\
        \textcolor{red}{Portfolio Construction} 
            & \textcolor{red}{3, 4, 5}
            & \textcolor{red}{Capital Allocation, Diversification, Index Model} \\
        Market Equilibrium & 6 \& 7 & CAPM, Fama-French Factors \\
        Fixed Income & 8 \& 9 & Prices, Yields, Yield Curve, Duration and Convexity \\
        Equity & 10 \& 11 & Dividend Discount Models, Price-Earnings Ratios, Efficient Markets, Equity Risk Premium\\
        Derivatives & 12, 13, 14 & Futures, Swaps, Options \\
        Financial Crisis of 2008 & 15 & Causes, What Happened, Aftermath\\
        \bottomrule
        \end{tblr}
    \end{table}

\end{frame}

\begin{frame}
    \frametitle{Outline}

    \begin{enumerate}
        \item Utility and Indifference Curves
        \item Preferences over Return and Risk
        \item Capital Allocation Line (budget constraint)
        \item Optimization
        \item Practice
    \end{enumerate}
\end{frame}

\begin{foundframe}[t]
    \frametitle{1. Utility and Indifference Curves}
    \framesubtitle{Concave Utility}

    Utility functions are generally \textit{concave} with respect to wealth.\\
    \vspace{1em}
    This reflects \textit{diminishing marginal utility} with respect wealth.\\
    \vspace{1em} 
     Intuitively, if a household goes from, say, making \$50,000 to \$150,000 annually that can 
    result in a material change in standard of living.   In contrast, moving from 
    \$100,000,000 to \$100,100,000 probably doesn't matter as much.

\end{foundframe}

\begin{foundframe}[t]
    \frametitle{1. Utility and Indifference Curves}
    \framesubtitle{Concave Utility}

    \centering
    \includegraphics[width=0.8\textwidth]{figures/ch06_1_concave_utility.png}

\end{foundframe}

\begin{foundframe}[t]
    \frametitle{1. Utility and Indifference Curves}
    \framesubtitle{Expected Values}

   \textbf{Definition:} Expected value is the probability-weighted average of all possible outcomes
    $$E[X] = \sum_{i} p_i \cdot x_i$$
    where $p_i$ is the probability of outcome $x_i$
    \vspace{1em}

    \begin{itemize}
        \item \textbf{Linearity:} $E[aX + bY] = aE[X] + bE[Y]$
        \item \textbf{Constant:} $E[c] = c$ for any constant $c$
    \end{itemize}
    \vspace{1em}

\end{foundframe}

\begin{foundframe}[t]
    \frametitle{1. Utility and Indifference Curves}
    \framesubtitle{Expected Values and Concave Utility}

    It follows from the concavity of the utility function that for a given set of 
    outcomes represented by $x$:
    $$\mathbb{E}[u(x)] < u(\mathbb{E}[x])$$
    In words, the expected utility over a set of uncertain outcomes is lower than 
    the utility of the expected value over those same outcomes.\\
    \vspace{1em}
    In even simpler words: people prefer certain outcomes over uncertain outcomes.\\
    \vspace{1em}
    This is known as 'Jensen's inequality' and holds for any concave function.\\
    \vspace{1em}

\end{foundframe}

\begin{foundframe}[t]
    \frametitle{1. Utility and Indifference Curves}
    \framesubtitle{Concave Utility}

    \centering
    \includegraphics[width=0.8\textwidth]{figures/ch06_1_concave_utility_jensen.png}

\end{foundframe}

\begin{frame}[t]
    \frametitle{Outline}
    \framesubtitle{Quick note on the numbers in this lecture}

    The charts in this lecture assume the following:
    \begin{itemize}
      \item Expected excess returns of the S\&P: 8.5\% 
      \item Expected standard deviation of the S\&P: 12.5\%
      \item Expected Sharpe Ratio of S\&P: ~0.7   
      \item Risk-free rate: 4\%
    \end{itemize}
    \vspace{1em}
    These expectations are roughly in-line with the historic S\&P returns and volatility since 2004.\\
    \vspace{1em}   
    At the same time, they are better than the historic Sharpe Ratio since 1926, which is closer to 0.5.\\
    \vspace{1em}   
    Anyways, this lecture is mostly to illustrate the approach, rather than precisely estimate these numbers.

\end{frame}

\begin{frame}[t]
    \frametitle{2. Preferences over Return and Risk}
    \framesubtitle{Preferences and Utility}

    Standard preferences prefer more return and less volatility.   

    \centering
    \includegraphics[width=0.75\textwidth]{figures/ch6_1_util1.png}

\end{frame}

\begin{frame}[t]
    \frametitle{2. Preferences over Return and Risk}
    \framesubtitle{Indifference Curves}

    An indifference curves trace the risk-return combinations that yield the same utility.  

    \centering
    \includegraphics[width=0.75\textwidth]{figures/ch6_1_util2.png}

\end{frame}

\begin{frame}[t]
    \frametitle{2. Preferences over Return and Risk}
    \framesubtitle{Indifference Curves}

    Assume the following equation for utility:
    $$U = E[r] - 0.5 \cdot A \cdot \sigma^2$$ 

    Utility is increasing in $E[r]$ and decreasing in $\sigma^2$.   The convexity comes from the squared-term.\\
    \vspace{1em}
    To get an indifference curve, set $U$ equal to a constant, then plot the function $E[R] = U - 0.5 \cdot A \cdot \sigma^2$\\
    \vspace{1em}
    What happens for larger (smaller) values of A?   How do the indifference curves change?   

\end{frame}

\begin{frame}[t]
    \frametitle{2. Preferences over Return and Risk}
    \framesubtitle{Indifference Curves}

    Higher indifference curves (up and to the left) correspond to more utility. 

    \centering
    \includegraphics[width=0.75\textwidth]{figures/ch6_1_util3.png}

\end{frame}

\begin{frame}[t]
    \frametitle{3. Capital Allocation Line}
    \framesubtitle{}

    Assume, for the moment, that you have two options: cash or a broad portfolio of risky assets.  
    Accordingly, your only decision is how much of your portfolio to allocate to the risky asset.\\
    \vspace{1em}
    The \textbf{Capital Allocation Line (CAL)}  depicts all possible combiations in risk-return space available from this decision.\\
    \vspace{1em}
    The CAL intersects the y-axis at $y=r_{risk-free}$ and goes through the point $(x=\sigma_{risky},y=(r_{risky}+r_{risk-free}))$

\end{frame}

\begin{frame}[t]
    \frametitle{3. Capital Allocation Line}
    \framesubtitle{Capital Markets Line and the Sharpe Ratio}

    When the portfolio of risky assets is a basket of broad-based stocks, the CAL has a specific name: \textbf{Capital Markets Line (CML)}.\\
    \vspace{1em}
    Note that the slope of the CML is the Sharpe Ratio.

    \centering
    \includegraphics[width=0.65\textwidth]{figures/ch6_1_cml.png}

\end{frame}

\begin{frame}[t]
    \frametitle{4. Optimization}
    \framesubtitle{}

    We can now solve for the optimal portfolio using standard maximization techniques:
    $$\max_{w} \quad U = E[r] - 0.5 \cdot A \cdot \sigma^2$$
    
    Subject to:
    $$E[r] = w \cdot E[r_{\text{risky}}] + (1-w) \cdot r_f$$
    $$\sigma^2 = w^2 \cdot \sigma_{\text{risky}}^2$$
    
    where $w$ is the weight in the risky asset.

\end{frame}

\begin{frame}[t]
    \frametitle{4. Optimization}
    \framesubtitle{}

    To solve this, take the first derivative with respect to $w$:

    \begin{align*}
    \frac{dU}{dw} &= w \cdot E[r_{\text{risky}}] + (1-w) \cdot r_f - 0.5 A \cdot w^2 \cdot \sigma_{\text{risky}}^2\\
    &= 0
    \end{align*}
    
    Solving for $w$:
    $$w^* = \frac{E[r_{\text{risky}}] - r_f}{A \cdot \sigma_{\text{risky}}^2}$$
    Be careful: this is not the Sharpe Ratio.   The denominator has a $\sigma^2$ and also the cofficient $A$.

\end{frame}

\begin{frame}[t]
    \frametitle{4. Optimization}
    \framesubtitle{}

    \centering
    \includegraphics[width=0.8\textwidth]{figures/ch6_max_w.png}

\end{frame}

\begin{frame}[t]
    \frametitle{4. Optimization}
    \framesubtitle{}

    \centering
    \includegraphics[width=0.8\textwidth]{figures/ch6_1_opt.png}

\end{frame}

\end{document}