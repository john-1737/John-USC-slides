\documentclass{beamer}

\newcommand{\week}{Week 14-a}

\title{Futures}
\subtitle{Reference: Bodie et al, Ch 22}
\author{Econ 457}
\date{\week}

% Reference the shared preamble
\setbeamertemplate{frametitle}{
  \vspace{0.5em}
  \insertframetitle
  \par
  \vspace{0.5em}
  \hrule
  \vspace{0.3em}
  {\small\color{gray}\insertframesubtitle}
}

\setbeamertemplate{navigation symbols}{}
\setbeamertemplate{itemize item}{\textbullet} % main bullet: filled dot
\setbeamertemplate{itemize subitem}{\normalsize$\circ$} % sub-bullet: empty dot
\setbeamertemplate{itemize subsubitem}{\scriptsize--} % sub-sub-bullet: dash


% Font changes
\usepackage[scaled=0.92]{helvet}
\renewcommand{\familydefault}{\sfdefault}

% Packages
\usepackage{tikz}
\usepackage{booktabs}
\usepackage{xcolor}
\usepackage{array}           % Enhanced column types for tables
\usepackage{multirow}        % Spanning multiple rows in tables
\usepackage{makecell}        % Line breaks and formatting in table cells
\usepackage{siunitx}         % Proper formatting of numbers and units
\usepackage{amsmath}         % Enhanced math environments
\usepackage{amsfonts}        % Additional math fonts
\usepackage{amssymb}         % Additional math symbols
\usepackage{url}             % Better URL formatting
\usepackage{graphicx}        % Enhanced graphics support
\usepackage{tabularray}
\UseTblrLibrary{booktabs, siunitx, varwidth}
% For financial presentations specifically
\usepackage{eurosym}         % Euro symbol
\usepackage{textcomp}        % Additional text symbols
\usepackage{hyperref}        % Hyperlinks (should be loaded last)

% Define a footnote
\renewcommand{\footnoterule}{\vspace*{-3pt}\hrule width 2in height 0.4pt\vspace*{2.6pt}}

% Define a Foundation Slide
\newenvironment{foundframe}[1][t]{
    \setbeamercolor{background canvas}{bg=gray!8}
    \setbeamercolor{frametitle}{fg=gray!80!black,bg=gray!25}
    \setbeamercolor{framesubtitle}{fg=gray!70!black,bg=gray!15}
    \setbeamercolor{item}{fg=gray!80!black}
    \setbeamercolor{enumerate item}{fg=gray!80!black}
    
    % Modify the frametitle template for this frame type
    \setbeamertemplate{frametitle}{
        \vspace{0.5em}
        \begin{minipage}[t]{0.75\textwidth}
            \insertframetitle
            \par
            \vspace{0.5em}
            \hrule
            \vspace{0.3em}
            {\small\color{gray}\insertframesubtitle}
        \end{minipage}%
        \hfill
        \begin{minipage}[t]{0.2\textwidth}
            \raggedleft
            \colorbox{gray!30}{%
                \scriptsize\bfseries\color{gray!80!black}%
                   \hspace{3pt}\begin{tabular}{c}Foundation\\Material\end{tabular}\hspace{3pt}%
            }
        \end{minipage}
        \vspace{0.3em}
    }
    
    \begin{frame}[#1]
}{
    \end{frame}
}

% Define Practice Slide
\newenvironment{practiceframe}[1][t]{
    \setbeamercolor{background canvas}{bg=white}
    \setbeamercolor{frametitle}{fg=blue!80!black,bg=blue!15}
    \setbeamercolor{framesubtitle}{fg=blue!70!black,bg=blue!10}
    \setbeamercolor{item}{fg=blue!80!black}
    \setbeamercolor{enumerate item}{fg=blue!80!black}
    \setbeamercolor{normal text}{fg=blue!90!black}
    
    % Modify the frametitle template for this frame type
    \setbeamertemplate{frametitle}{
        \vspace{0.5em}
        \begin{minipage}[t]{0.75\textwidth}
            \insertframetitle
            \par
            \vspace{0.5em}
            \hrule
            \vspace{0.3em}
            {\small\color{blue!70!black}\insertframesubtitle}
        \end{minipage}%
        \hfill
        \begin{minipage}[t]{0.2\textwidth}
            \raggedleft
            \colorbox{blue!20}{%
                \scriptsize\bfseries\color{blue!80!black}%
                   \hspace{3pt}\begin{tabular}{c}Practice\\Questions\end{tabular}\hspace{3pt}%
            }
        \end{minipage}
        \vspace{0.3em}
    }
    
    \begin{frame}[#1]
}{
    \end{frame}
}

% Define Excel Slide
\newenvironment{excelframe}[1][t]{
    \setbeamercolor{background canvas}{bg=white}
    \setbeamercolor{frametitle}{fg=blue!80!black,bg=blue!15}
    \setbeamercolor{framesubtitle}{fg=blue!70!black,bg=blue!10}
    \setbeamercolor{item}{fg=blue!80!black}
    \setbeamercolor{enumerate item}{fg=blue!80!black}
    \setbeamercolor{normal text}{fg=blue!90!black}
    
    % Modify the frametitle template for this frame type
    \setbeamertemplate{frametitle}{
        \vspace{0.5em}
        \begin{minipage}[t]{0.75\textwidth}
            \insertframetitle
            \par
            \vspace{0.5em}
            \hrule
            \vspace{0.3em}
            {\small\color{blue!70!black}\insertframesubtitle}
        \end{minipage}%
        \hfill
        \begin{minipage}[t]{0.2\textwidth}
            \raggedleft
            \colorbox{green!10}{%
                \scriptsize\bfseries\color{blue!80!black}%
                   \hspace{3pt}\begin{tabular}{c}MS Excel\end{tabular}\hspace{3pt}%
            }
        \end{minipage}
        \vspace{0.3em}
    }
    
    \begin{frame}[#1]
}{
    \end{frame}
}

% Define Caution Slide
\newenvironment{cautionframe}[1][t]{
    \setbeamercolor{background canvas}{bg=white}
    \setbeamercolor{frametitle}{fg=blue!80!black,bg=blue!15}
    \setbeamercolor{framesubtitle}{fg=blue!70!black,bg=blue!10}
    \setbeamercolor{item}{fg=blue!80!black}
    \setbeamercolor{enumerate item}{fg=blue!80!black}
    \setbeamercolor{normal text}{fg=blue!90!black}
    
    % Modify the frametitle template for this frame type
    \setbeamertemplate{frametitle}{
        \vspace{0.5em}
        \begin{minipage}[t]{0.75\textwidth}
            \insertframetitle
            \par
            \vspace{0.5em}
            \hrule
            \vspace{0.3em}
            {\small\color{blue!70!black}\insertframesubtitle}
        \end{minipage}%
        \hfill
        \begin{minipage}[t]{0.2\textwidth}
            \raggedleft
            \colorbox{red!10}{%
                \scriptsize\bfseries\color{blue!80!black}%
                   \hspace{3pt}\begin{tabular}{c}Caution\end{tabular}\hspace{3pt}%
            }
        \end{minipage}
        \vspace{0.3em}
    }
    
    \begin{frame}[#1]
}{
    \end{frame}
}

% Add to footnotes
\makeatletter
\newcommand\blfootnote[1]{%
  \begingroup
  \renewcommand\thefootnote{}%
  \renewcommand\@makefntext[1]{\raggedright\leftskip=0pt ##1}%
  \footnote{\scriptsize #1}%
  \addtocounter{footnote}{-1}%
  \endgroup
}
\makeatother

% Set the footer -- change 
\setbeamertemplate{footline}{
  \leavevmode%
  \vspace{2ex}
  \hbox{%
    % Left box: Econ 457
    \begin{beamercolorbox}[wd=.4\paperwidth,ht=2.5ex,dp=1ex,left]{author in head/foot}%
      \hspace{1em}Econ 457
    \end{beamercolorbox}%
    % Middle box: Week
    \begin{beamercolorbox}[wd=.2\paperwidth,ht=2.5ex,dp=1ex,center]{date in head/foot}%
      \centering\week
    \end{beamercolorbox}%
    % Right box: Slide numbers
    \begin{beamercolorbox}[wd=.4\paperwidth,ht=2.5ex,dp=1ex,center]{date in head/foot}%
      \hfill\insertframenumber{} 
    \end{beamercolorbox}%
  }%
  \vskip0pt%
}

\begin{document}

\frame{\titlepage}

\begin{frame}
    \frametitle{Outline}

    \begin{enumerate}
        \item Short Sales
        \item Forward Prices
        \item Futures Markets
        \item Practice
    \end{enumerate}

\end{frame}

\begin{frame}[t]
    \frametitle{1. Short Sales}
    \framesubtitle{Mechanics}

    \textbf{Short Sale:} Selling a security you don't own, expecting its price to decline
    \vspace{1em}

    \textbf{Process:}
    \begin{enumerate}
        \item \textbf{Borrow shares} from broker's inventory or another client
        \item \textbf{Sell borrowed shares} immediately at current market price
        \item \textbf{Buy back shares} later to "cover" the short position
        \item \textbf{Return shares} to lender
        \item Pay dividend/interest to share lender during short period
    \end{enumerate}
    \vspace{1em}

    If the price falls (rises) during the time that the security has been borrowed, then 
    the short seller is able to buy it back at a lower (higher) price than it was sold for.  
    This generates a profit (loss) from the short sale.\\

\end{frame}

\begin{frame}[t]
    \frametitle{1. Short Sales}
    \framesubtitle{Example, see Hull p 102}

    \footnotesize
    \textit{Purchase of Shares}
    \begin{itemize}
        \item Purchase 500 shares for \$120 per share: -\$60,000
        \item Receive dividend: +\$500
        \item Sell 500 shares for \$100 per share: +\$50,000
        \item Net profit: -\$9,500
    \end{itemize}

    \textit{Short sale of shares}
    \begin{itemize}
        \item Borrow 500 shares and sell them for \$120: +\$60,000
        \item Pay dividend: -\$500
        \item Buy 500 shares for \$100 per share: -\$50,000
        \item Replace borrowed shares and close short position
        \item Net profit: +\$9,500
    \end{itemize}
    \vspace{1em}

    \normalsize
    \textit{Notes:} The two positions are symmetric.   The short seller must pay the dividend.  
    We have not yet accounted for the cost of borrowing, or interest earned by the short seller 
    from the cash received. 

\end{frame}


\begin{frame}[t]
    \frametitle{1. Short Sales}
    \framesubtitle{Short Squeeze}


    Short sales require borrowing the security.   The security may be available from the broker. 
    In some cases, hard to locate securities may have an associated borrowing fee.\\
    \vspace{1em}
    Most short sales require posting margin.   Short sellers are subject to margin calls if prices move adversely. 
    The position may be closed by the broker if the seller fails to post margin.\\
    \vspace{1em}
    In order to close their position, the short seller must buy back the security.  A \textit{short squeeze} is 
    when the security becomes very expensive, but short sellers must buy it anyways in order to close positions.

\end{frame}

\begin{frame}[t]
    \frametitle{1. Short Sales}
    \framesubtitle{Game Stop}

    There was a short squeeze in GameStop (GME) in January 2021.\\
    \vspace{1em}
    \centering
    \includegraphics[width=0.8\textwidth]{figures/ch22_1_gme.png}

\end{frame}

\begin{frame}[t]
    \frametitle{1. Short Sales}
    \framesubtitle{Word of Caution}

    Short squeezes don't always work, of course.   They are predicated on the supply of 
    stocks to borrow are limited.   If the short sellers are able to borrow more stocks 
    in order to maintain or add to their short position, then the price may go down.   In that case,
    the "squeezer", who is likely leveraged, may have to close her position.\\
    \vspace{1em}
    The financial panic of 1907, which eventually led to the creation of the Federal Reserve, started 
    with large losses to a brokerage house that was finance a failed short squeeze in United Copper stocks.

\end{frame}

\begin{frame}[t]
    \frametitle{2. Forward Prices}
    \framesubtitle{General Formula - No dividends}

    A forward contract is an agreement to buy and sell a security at a specified date 
    in the future at a specified price.   The \textit{forward price} is the agreed upon future price in 
    that transaction.\\
    \vspace{1em}
    For a non-dividend paying security, a general formula for the forward price is:\\
    $$F_0 = S_0 \cdot (1 + r_f)$$
    Where
    \begin{itemize}
        \item $F_0$ = the forward price
        \item $S_0$ = the price of the security today
        \item $r_f$ = the risk-free rate
    \end{itemize}

\end{frame}

\begin{frame}[t]
    \frametitle{2. Forward Prices}
    \framesubtitle{General Formula - Proof}

    The proof is based on a no-aribtrage argument.\\
    \vspace{1em}
    Specifically, if the price were either higher or 
    lower, it would possible to enter into a profitable \textit{riskless} arbitrage.  As we generally 
    assume profitable riskless arbitrages don't exit (somebody will do them if they do!) then the price cannot 
    be either higher or lower than the formula.\\

\end{frame}

\begin{frame}[t]
    \frametitle{2. Forward Prices}
    \framesubtitle{General Formula - Proof}

    \textit{Case 1:} If $F_0 > S_0 \cdot (1 + r_f)$, then the following is a riskless arbitrage:
    \begin{itemize}
        \item Borrow money today at the riskfree rate
        \item Use the money to buy the security for $S_0$
        \item Simultaneously sell the security forward at $F_0$
        \item At maturity, deliver the security you bought into the forward contract, repay the loan
        \item Total profit: $F_0 - S_0 - S_0 \cdot (r_f) > 0$
    \end{itemize}
    \vspace{1em}

    Exercise: what are the specific steps to enter into a riskless arbitrage if $F_0 < S_0 \cdot (1 + r_f)$?   (hint: will require selling the security short)

\end{frame}

\begin{frame}[t]
    \frametitle{2. Forward Prices}
    \framesubtitle{General Formula - Dividends}

    In order to incorporate dividends, remember that the short seller 
    must pay dividends to the lender.  Reconstructing the 
    no-arbitrage argument and including that fact leads to the following:\\
    \vspace{1em} 
    For a dividend/coupon paying security, a general formula for the forward price is:\\
    $$F_0 = S_0 \cdot (1 + r_f) - I$$
    Where
    \begin{itemize}
        \item $F_0$ = the forward price
        \item $S_0$ = the price of the security today
        \item $r_f$ = the risk-free rate
        \item $I$ = income from the security (dividends or coupons)
    \end{itemize}

\end{frame}

\begin{frame}[t]
    \frametitle{2. Forward Prices}
    \framesubtitle{General Formula - Dividends}

    A more general statement of this formula, assuming continuous compounding and allowing for 
    multiple periods is:
    $$F_0 = S_0 \cdot e^{(r_f-q)T}$$
    Where
    \begin{itemize}
        \item $F_0$ = the forward price
        \item $S_0$ = the price of the security today
        \item $r_f$ = the risk-free rate
        \item $q$ = yield from the security
        \item $T$ = time to delivery, expressed in years
    \end{itemize}

\end{frame}

\begin{frame}[t]
    \frametitle{2. Forward Prices}
    \framesubtitle{General Formula - Dividends}

    Here is the same formual in discrete time (not continuous time):
    $$F_0 = S_0 \cdot (1 + r_f - q)^T$$

    Note that when the risk-free rate is greater than the yield, then forward price will 
    be \textit{higher} than the current price.   Conversely, in cases where the yield 
    is greater than the risk-free rate, then the forward price will be \textit{lower} 
    than the current price.

\end{frame}

\begin{frame}[t]
    \frametitle{2. Forward Prices}
    \framesubtitle{Forward v Spot Prices}

    \begin{columns}
        \begin{column}{0.5\textwidth}
            \centering
            \includegraphics[width=\textwidth]{figures/ch22_1_sp_indfut.png}
        \end{column}
        \begin{column}{0.5\textwidth}
            \centering
            \includegraphics[width=\textwidth]{figures/ch22_1_sp_indfut_2.png}
        \end{column}
    \end{columns}

\end{frame}

\begin{frame}[t]
    \frametitle{2. Forward Prices}
    \framesubtitle{Forward v Spot Prices}

    More examples:
    \begin{itemize}
        \item High yielding currencies (i.e. Emerging Markets) will have futures 
        prices that are below the spot prices, in order to account for the 
        yield on the underlying security. When futures prices are \textbf{below} spot prices (due to high yields or convenience value), this is known as "backwardation"
        \item You can think about the storage costs for a commodity as being a factor 
        of the earnings.   Commodities with high storage costs will have lower earnings, 
        therefore higher futures prices.   When commodity futures prices are 
        \textbf{above} spot prices, this is known as "contango"
    \end{itemize}

\end{frame}

\begin{frame}[t]
    \frametitle{2. Forward Prices}
    \framesubtitle{Forward v Spot Prices}

    Forward prices always converge to spot prices at the time of the delivery\\

    \centering
    \includegraphics[width=0.7\textwidth]{figures/ch22_1_contango.png}

\end{frame}

\begin{frame}[t]
    \frametitle{3. Futures Markets}
    \framesubtitle{Futures vs. Forward Contracts}

    \begin{table}
        \centering
        \begin{tabular}{|l|l|l|}
        \hline
        \textbf{Feature} & \textbf{Forward} & \textbf{Futures} \\
        \hline
        Trading & OTC (private) & Exchange-traded \\
        Standardization & Customized & Standardized \\
        Counterparty Risk & Yes (bilateral) & None (clearinghouse) \\
        Margin & Usually none & Daily margin calls \\
        Settlement & At maturity & Daily mark-to-market \\
        Liquidity & Low & High \\
        \hline
        \end{tabular}
    \end{table}

\end{frame}

\begin{frame}[t]
    \frametitle{3. Futures Markets}
    \framesubtitle{History}

    \begin{itemize}
        \item \textbf{1848} Chicago Board of Trade (CBOT) established to facilitate forward trades in grains
        \item \textbf{1972} Chicago Mercantile Exchange (CME) introducs first currency future
        \item \textbf{1974} Commodity Futures and Trading Commission (CFTC) established as the regulator
        \item \textbf{1976} CBOT introduces first interest rate future
        \item \textbf{1982} CME introduces S\&P future (after receiving regulatory approval from the CFTC and SEC)
        \item \textbf{2007} CME and CBOT merge
    \end{itemize}

\end{frame}

\begin{frame}[t]
    \frametitle{3. Futures Markets}
    \framesubtitle{History}

    \centering
    \includegraphics[width=0.8\textwidth]{figures/ch22_1_cme.png}

\end{frame}

\begin{frame}[t]
    \frametitle{3. Futures Markets}
    \framesubtitle{CME}

    CME's "featured products":
    \begin{itemize}
        \item Corn (ZC)
        \item Soybean (ZS)
        \item WTI Crude Oil (CL)
        \item Henry Hub Natural Gas (NG)
        \item S\&P 500 (ES)
        \item Nasdaq-100 (NQ)
        \item Euro FX (6E)
        \item 10-year T-Note (ZN)
        \item SOFR (SR3)
        \item Gold (GC)
        \item Copper (HG)
    \end{itemize}
    There are lots, lots more products...

\end{frame}

\begin{frame}[t]
    \frametitle{3. Futures Markets}
    \framesubtitle{Size of Futures Market}

    \centering
    \includegraphics[width=0.8\textwidth]{figures/ch22_1_bis.png}

\end{frame}


\begin{frame}[t]
    \frametitle{3. Futures Markets}
    \framesubtitle{Details}

    \textit{Futures Contracts Specifications}:
    \begin{enumerate}
        \item What can be delivered into the contract (example: No. 2 Soft Red Winter wheat, Treasury bonds with maturity between 6.5-10 years)
        \item The quantity that must be delivered (example: 5,000 bushels of wheat, \$100,000 face value of Treasury bonds)
        \item Delivery dates (typically a quarterly cycle)
    \end{enumerate}


\end{frame}

\begin{frame}[t]
    \frametitle{3. Futures Markets}
    \framesubtitle{Details}

    \textit{Exchange Traded}: Futures are typically traded and cleared on a centralized exchange.
    \begin{itemize}
        \item The exchange maintains a Central Limit Order Book, where participants have placed limit orders to buy and sell.   Transactions 
        occur automatically when bids and offers are met.
        \item   The exchange maintains margin accounts for all participants.  The exchange requires daily mark-to-market and may require posting 
        additional margin.
        \item  Individual investors transact through their broker, and the broker has an account on the exchange.   
    \end{itemize}

\end{frame}

\begin{practiceframe}[t]
    \frametitle{4. Practice}
    \framesubtitle{Practice}

    \begin{enumerate}
        \item Why might an individual purchase futures contracts rather than 
        the underlying asset?
        \item Are the following statements true or false?
        \begin{itemize}
            \item All else equal, the futures price on a stock index with a 
            high dividend yield should be higher than the futures price on an 
            index with a low dividend yield.
            \item All else equal, the futures price on a high-beta stock should be 
            higher than the futures price on a low-beta stock.
        \end{itemize}
        \item A futures contract on a non-dividend paying stock index with a 
        current value 150 has a maturity of one year.   If the T-Bill rate is 3\%
         what should the futures price be?   What if the maturity of the contract 
         is in 3 years?
    \end{enumerate}

\end{practiceframe}

\end{document}