\documentclass{beamer}

\newcommand{\week}{Week 8-a}

\title{Multifactor Models, Fama-French Factors}
\subtitle{Reference: Bodie et al, Ch 10}
\author{Econ 457}
\date{\week}

% Reference the shared preamble
\setbeamertemplate{frametitle}{
  \vspace{0.5em}
  \insertframetitle
  \par
  \vspace{0.5em}
  \hrule
  \vspace{0.3em}
  {\small\color{gray}\insertframesubtitle}
}

\setbeamertemplate{navigation symbols}{}
\setbeamertemplate{itemize item}{\textbullet} % main bullet: filled dot
\setbeamertemplate{itemize subitem}{\normalsize$\circ$} % sub-bullet: empty dot
\setbeamertemplate{itemize subsubitem}{\scriptsize--} % sub-sub-bullet: dash


% Font changes
\usepackage[scaled=0.92]{helvet}
\renewcommand{\familydefault}{\sfdefault}

% Packages
\usepackage{tikz}
\usepackage{booktabs}
\usepackage{xcolor}
\usepackage{array}           % Enhanced column types for tables
\usepackage{multirow}        % Spanning multiple rows in tables
\usepackage{makecell}        % Line breaks and formatting in table cells
\usepackage{siunitx}         % Proper formatting of numbers and units
\usepackage{amsmath}         % Enhanced math environments
\usepackage{amsfonts}        % Additional math fonts
\usepackage{amssymb}         % Additional math symbols
\usepackage{url}             % Better URL formatting
\usepackage{graphicx}        % Enhanced graphics support
\usepackage{tabularray}
\UseTblrLibrary{booktabs, siunitx, varwidth}
% For financial presentations specifically
\usepackage{eurosym}         % Euro symbol
\usepackage{textcomp}        % Additional text symbols
\usepackage{hyperref}        % Hyperlinks (should be loaded last)

% Define a footnote
\renewcommand{\footnoterule}{\vspace*{-3pt}\hrule width 2in height 0.4pt\vspace*{2.6pt}}

% Define a Foundation Slide
\newenvironment{foundframe}[1][t]{
    \setbeamercolor{background canvas}{bg=gray!8}
    \setbeamercolor{frametitle}{fg=gray!80!black,bg=gray!25}
    \setbeamercolor{framesubtitle}{fg=gray!70!black,bg=gray!15}
    \setbeamercolor{item}{fg=gray!80!black}
    \setbeamercolor{enumerate item}{fg=gray!80!black}
    
    % Modify the frametitle template for this frame type
    \setbeamertemplate{frametitle}{
        \vspace{0.5em}
        \begin{minipage}[t]{0.75\textwidth}
            \insertframetitle
            \par
            \vspace{0.5em}
            \hrule
            \vspace{0.3em}
            {\small\color{gray}\insertframesubtitle}
        \end{minipage}%
        \hfill
        \begin{minipage}[t]{0.2\textwidth}
            \raggedleft
            \colorbox{gray!30}{%
                \scriptsize\bfseries\color{gray!80!black}%
                   \hspace{3pt}\begin{tabular}{c}Foundation\\Material\end{tabular}\hspace{3pt}%
            }
        \end{minipage}
        \vspace{0.3em}
    }
    
    \begin{frame}[#1]
}{
    \end{frame}
}

% Define Practice Slide
\newenvironment{practiceframe}[1][t]{
    \setbeamercolor{background canvas}{bg=white}
    \setbeamercolor{frametitle}{fg=blue!80!black,bg=blue!15}
    \setbeamercolor{framesubtitle}{fg=blue!70!black,bg=blue!10}
    \setbeamercolor{item}{fg=blue!80!black}
    \setbeamercolor{enumerate item}{fg=blue!80!black}
    \setbeamercolor{normal text}{fg=blue!90!black}
    
    % Modify the frametitle template for this frame type
    \setbeamertemplate{frametitle}{
        \vspace{0.5em}
        \begin{minipage}[t]{0.75\textwidth}
            \insertframetitle
            \par
            \vspace{0.5em}
            \hrule
            \vspace{0.3em}
            {\small\color{blue!70!black}\insertframesubtitle}
        \end{minipage}%
        \hfill
        \begin{minipage}[t]{0.2\textwidth}
            \raggedleft
            \colorbox{blue!20}{%
                \scriptsize\bfseries\color{blue!80!black}%
                   \hspace{3pt}\begin{tabular}{c}Practice\\Questions\end{tabular}\hspace{3pt}%
            }
        \end{minipage}
        \vspace{0.3em}
    }
    
    \begin{frame}[#1]
}{
    \end{frame}
}

% Define Excel Slide
\newenvironment{excelframe}[1][t]{
    \setbeamercolor{background canvas}{bg=white}
    \setbeamercolor{frametitle}{fg=blue!80!black,bg=blue!15}
    \setbeamercolor{framesubtitle}{fg=blue!70!black,bg=blue!10}
    \setbeamercolor{item}{fg=blue!80!black}
    \setbeamercolor{enumerate item}{fg=blue!80!black}
    \setbeamercolor{normal text}{fg=blue!90!black}
    
    % Modify the frametitle template for this frame type
    \setbeamertemplate{frametitle}{
        \vspace{0.5em}
        \begin{minipage}[t]{0.75\textwidth}
            \insertframetitle
            \par
            \vspace{0.5em}
            \hrule
            \vspace{0.3em}
            {\small\color{blue!70!black}\insertframesubtitle}
        \end{minipage}%
        \hfill
        \begin{minipage}[t]{0.2\textwidth}
            \raggedleft
            \colorbox{green!10}{%
                \scriptsize\bfseries\color{blue!80!black}%
                   \hspace{3pt}\begin{tabular}{c}MS Excel\end{tabular}\hspace{3pt}%
            }
        \end{minipage}
        \vspace{0.3em}
    }
    
    \begin{frame}[#1]
}{
    \end{frame}
}

% Define Caution Slide
\newenvironment{cautionframe}[1][t]{
    \setbeamercolor{background canvas}{bg=white}
    \setbeamercolor{frametitle}{fg=blue!80!black,bg=blue!15}
    \setbeamercolor{framesubtitle}{fg=blue!70!black,bg=blue!10}
    \setbeamercolor{item}{fg=blue!80!black}
    \setbeamercolor{enumerate item}{fg=blue!80!black}
    \setbeamercolor{normal text}{fg=blue!90!black}
    
    % Modify the frametitle template for this frame type
    \setbeamertemplate{frametitle}{
        \vspace{0.5em}
        \begin{minipage}[t]{0.75\textwidth}
            \insertframetitle
            \par
            \vspace{0.5em}
            \hrule
            \vspace{0.3em}
            {\small\color{blue!70!black}\insertframesubtitle}
        \end{minipage}%
        \hfill
        \begin{minipage}[t]{0.2\textwidth}
            \raggedleft
            \colorbox{red!10}{%
                \scriptsize\bfseries\color{blue!80!black}%
                   \hspace{3pt}\begin{tabular}{c}Caution\end{tabular}\hspace{3pt}%
            }
        \end{minipage}
        \vspace{0.3em}
    }
    
    \begin{frame}[#1]
}{
    \end{frame}
}

% Add to footnotes
\makeatletter
\newcommand\blfootnote[1]{%
  \begingroup
  \renewcommand\thefootnote{}%
  \renewcommand\@makefntext[1]{\raggedright\leftskip=0pt ##1}%
  \footnote{\scriptsize #1}%
  \addtocounter{footnote}{-1}%
  \endgroup
}
\makeatother

% Set the footer -- change 
\setbeamertemplate{footline}{
  \leavevmode%
  \vspace{2ex}
  \hbox{%
    % Left box: Econ 457
    \begin{beamercolorbox}[wd=.4\paperwidth,ht=2.5ex,dp=1ex,left]{author in head/foot}%
      \hspace{1em}Econ 457
    \end{beamercolorbox}%
    % Middle box: Week
    \begin{beamercolorbox}[wd=.2\paperwidth,ht=2.5ex,dp=1ex,center]{date in head/foot}%
      \centering\week
    \end{beamercolorbox}%
    % Right box: Slide numbers
    \begin{beamercolorbox}[wd=.4\paperwidth,ht=2.5ex,dp=1ex,center]{date in head/foot}%
      \hfill\insertframenumber{} 
    \end{beamercolorbox}%
  }%
  \vskip0pt%
}

\begin{document}

\frame{\titlepage}

\begin{frame}
    \frametitle{Outline}

    \begin{enumerate}
        \item Multifactor Models 
            \begin{itemize}
                \item Macroeconomic factors
                \item Arbitrage Pricing Theory
                \item Factor Portfolios and Risk Premiums
            \end{itemize}
        \item Fama and French Factors 
        \item Multifactor Models in Practice (Smart Beta)
    \end{enumerate}
\end{frame}

\begin{frame}[t]
    \frametitle{1. Multifactor Models}
    \framesubtitle{Motivation}

    One takeaway from CAPM was that \textit{only systematic risk would be priced}.   
    A security with higher than average idiosyncratic risk will not have a higher than 
    average expected return.   Only to the extent that the security's risk is correlated with 
    systematic risk will it influence the expected return.\\
    \vspace{1em}
    Up to now we defined "systematic risk" as broad market risk.   
    Idiosyncratic, or company specific, risk could be diversified away and therefore should not 
    be rewarded with higher returns.   
    But the risk of the broad market can not be diversified away, which is why investors should receive 
    higher returns for bearing this risk.\\
    \vspace{1em}
    Today we are going to extend these ideas slightly and introduce \textit{multiple factors}.

\end{frame}


\begin{frame}[t]
    \frametitle{1. Multifactor Models}
    \framesubtitle{}

    The Index model said that security returns were determined by 
    one market factor and then idiosyncratic risk.  
    Let's augment that by adding two additional factors reflecting the macroeconomic 
    environment.\\
    \vspace{1em}
    For example, add linear terms to make the risk premium of security $i$ a 
    function of macro surprises (GDP, CPI):
    $$R_{i,t} = \alpha_i + \beta_{m,i} \cdot R_{m,t} + \beta_{GDP,i} \cdot GDP_t + \beta_{CPI,i} \cdot CPI_t + \epsilon_{i,t}$$
    Where $GDP_t$ is the surprise, relative to expectations, of the GDP announcement.

\end{frame}

\begin{frame}[t]
    \frametitle{1. Multifactor Models}
    \framesubtitle{Application 1}

    Suppose this is your model for a security's returns: 
        $$R_{i,t} = \alpha_i + \beta_{m,i} \cdot R_{m,t} + \beta_{GDP,i} \cdot GDP_t + \beta_{CPI,i} \cdot CPI_t + \epsilon_{i,t}$$
    Where $GDP$ and $CPI$ are \textit{surprises}.\\
    \vspace{1em}
    You estimate that $\beta_{GDP,i} = 1.2$.\\
    \vspace{1em}
    Your initial estimate is that the excess return of the security will 
    be 10\%.   If GDP comes in 1 percentage point lower than expected, 
    how should you revise your estimate of the security's expected return?

\end{frame}

\begin{frame}[t]
    \frametitle{1. Multifactor Models}
    \framesubtitle{This isn't CAPM}

    While this may look similar to CAPM, we are not yet able to make the same conclusions with regards to 
    equilibrium pricing. 
    CAPM was motivated by (strong) assumptions about complete markets, shared investor preferences, etc.  
    And the market return was, by assumption in CAPM, the return on the complete portfolio of all investable 
    assets.\\
    \vspace{1em}
    In contrast, here we just chose a few macro factors to include in the model.  While GDP and Inflation 
    are reasonable choices, we could have chosen anything.   
    Because these factors aren't based on the same set of 
    assumptions, and as a consequence can't make the same claims to characterizing the equilibrium, 
    multifactor models need to rely on something else to make statements about pricing.

\end{frame}

\begin{frame}[t]
    \frametitle{1. Multifactor Models}
    \framesubtitle{Arbitrage Pricing Theory}

    Instead of relying on a economic model with equilibrium, these multifactor models instead rely on an
    \textit{arbitrage} argument to make a statement about expected pricing.   This is \textbf{Arbitrage Pricing Theory (APT)}\\
    \vspace{1em}
    An \textit{arbitrage} opportunity is when an investor can make a riskless profit without making a net 
    investment (or exposing herself to risk).   Arbitrages should be quickly closed by traders.  So if you can 
    find two trades with the exact same risk, it's usually safe to assume the expected returns on the two trades 
    should be equal.
    \blfootnote{Obviously it's rare that the risk are exactly the same. 
    We'll see these arguments again in the classes on options and derivatives.}

\end{frame}

\begin{frame}[t]
    \frametitle{1. Multifactor Models}
    \framesubtitle{Security Market Line}

    In multifactor models, we can argue that a no-arbitrage condition leads to the following conclusion:\\ 
    \vspace{1em}
    \begin{quote} 
        Portfolios with equal betas should have equal expected returns, for any beta in a 
        multifactor model.
    \end{quote}
    \vspace{1em}
    This is similar to saying that all securities must lie on the Security Market Line (SML), but now we are 
    talking about a multidimensional SML (harder to visualize).

\end{frame}

\begin{frame}[t]
    \frametitle{1. Multifactor Models}
    \framesubtitle{Security Market Line}
    
    To see why there must be a single SML, consider the following chart where the x-axis is the realization (surprise) of factor F.\\
    \centering
    \includegraphics[width=0.5\textwidth]{figures/ch10_1_2sml.png}

    \raggedright
    In this case, security A is always preferred to security B.   
    An arbitrage strategy of selling B and buying A provides a riskless profit.  
    As this arbitrage will be immediately closed by traders, prices will adjust so that both A and B 
    eventually lie on the same line.

\end{frame}

\begin{frame}[t]
    \frametitle{1. Multifactor Models}
    \framesubtitle{Factor Portfolios and Risk Premiums}

    When we choose the macro factors, we should be careful to choose factors that are difficult to diversify.  
    Idiosyncratic risk is easy to diversify, and therefore should not be rewarded with extra return.   
    Factors that are more difficult to diversify, in contrast, could be rewarded with extra return.\\
    \vspace{1em}
    If we choose the factors appropriately, then we can argue that investors bearing exposure to these macro 
    factors will be rewarded with extra return.   A different way to say that is that the there will be 
    a risk premium associated with the macro factors.\\
    \vspace{1em}

\end{frame}

\begin{frame}[t]
    \frametitle{1. Multifactor Models}
    \framesubtitle{Factor Portfolios and Risk Premiums}
    To implement the APT in a way that is similar to the CAPM, we next construct 
    factor portfolios.   These portfolios may track the evolution of the 
    macrovariable of interest, or perhaps some other aspect of systematic risk that should be rewarded.   
    (We'll have some specific examples in a moment)\\
    \vspace{1em}
    The risk premium associated with the factor is then the excess return of the factor 
    portfolio: $\mathbb{E}[r_F - r_f]$ where $r_F$ is the return on the factor portfolio.

\end{frame}

\begin{frame}[t]
    \frametitle{1. Multifactor Models}
    \framesubtitle{Two Models}

    We now have two different multifactor models.  Be careful because they look similar, but they have slightly different 
    uses.
    \begin{enumerate}
        \item Returns are determined by surprises in macroeconomic factors:
        $$R_i = E(R_i) + \beta_{i1}F_1 + \beta_{i2} F_2 + \epsilon$$
        where $F_1$ and $F_2$ are \textit{surprises} in these factors, relative to expectations.
        \item The fair return on a security is determined by the risk-premiums associated 
        with macroecomonic factors:
        $$E[r_i] = r_f + \beta_{i1}\mathbb{E}[r_1 - r_f] + \beta_{i2}\mathbb{E}[r_2 - r_f]$$
        where $\mathbb{E}[r_1 - r_f]$ is the risk premium associated with a factor portfolio.
    \end{enumerate}

\end{frame}

\begin{frame}[t]
    \frametitle{1. Multifactor Models - Macro Factors}
    \framesubtitle{Application 2}

    Consider the following APT model of returns for a particular stock:
        \begin{table}
            \begin{tabular}{lcc}
                \toprule
                Factor & Factor Beta & Factor Risk Premium \\
                \midrule
                Inflation   & 1.2 & 6\%\\
                Industrial Production & 0.5 & 8\\
                Oil Prices & 0.3 & 3\\
                \bottomrule
            \end{tabular}
        \end{table}
    If T-Bills currently yield 6\%, find the expected rate of return of this stock if the market 
    views it as fairly valued.\\
    \vspace{1em}
    Note that the expected return may subsequently be adjusted due to \textit{surprises} in these variables.

\end{frame}

\begin{frame}[t]
    \frametitle{2. Fama-French Factors}
    \framesubtitle{The Model}

    The most famous multifactor model is the Fama-French model:
    $$
    R_{i,t} = \alpha_i + \beta_{m,i} \cdot R_{m,t} + \beta_{SMB,i} \cdot SMB_t + \beta_{HML,i} \cdot HML_t + \epsilon_{i,t}
    $$
    SML = return of small stocks minus return of large stocks\\
    HML = return of value stocks minus return of growth stocks\\
    \vspace{1em}
    Procedure for creating factor portfolios: Sort firms by size or book-to-market.  
    Create portfolios of the bottom and top deciles.   The SML factor portfolio is the return 
    of small companies (bottom decile) minus the return of the large companies (top decile).
    The HML factor portfolio is the return of the 
    value companies (top decile in book-to-market) minus the growth portfolios (bottom decile in book-to-market).

\end{frame}

\begin{frame}[t]
    \frametitle{2. Fama-French Factors}
    \framesubtitle{Book value v. Market value - Definitions}

    Book Value: The net worth of a common equity according to the firm's balance sheet. 
    Can be calculated as Assets - Liabilities. (also referred to as 'Book Equity' or BE)\\
    \vspace{1em}
    Market Value: The net worth of a common equity according to the market price.  
    Can be calculated by Shares Outstanding X Market Price.  (also referred to as 'Market Equity' or ME)\\
    \vspace{1em}

    Value Stock: A stock with a \textbf{high} BE/ME ratio.  Equivalently, a stock with a \textbf{low} ME/BE ratio.\\
    \vspace{1em}

    Growth Stock: A stock with a \textbf{low} BE/ME ratio.   Equivalently, a stock with a \textbf{high} ME/BE ratio.\\
    \vspace{1em}

\end{frame}

\begin{frame}[t]
    \frametitle{2. Fama-French Factors}
    \framesubtitle{Book value v. Market value - Examples}
    
    \vspace{-1.5em}
    \footnotesize

    \begin{table}
    \caption{Book Equity and Market Equity}
    \begin{tabular}{llll}
    \toprule
     & XOM & MSFT & KHC \\
    \midrule
    Assets (\$bn) & 451.90 & 562.20 & 90.27 \\
    Liabilities (\$bn) & 182.10 & 240.70 & 40.60 \\
    Shares Outstanding (bn) & 4.30 & 7.30 & 1.18 \\
    Price (\$ as of 7-7-25) & 112.00 & 494.40 & 426.50 \\
    \midrule
    Book Equity, BE (\$bn) & 269.80 & 321.50 & 49.67 \\
    Market Equity, ME (\$bn) & 481.60 & 3609.12 & 503.27 \\
    \midrule
    BE/ME & 0.56 & 0.09 & 0.10 \\
    High or Low & High & Low & Low \\
    Price to Book & 1.79 & 11.23 & 10.13 \\
    Value or Growth & Value & Growth & Growth \\
    \bottomrule
    \end{tabular}
    \end{table}

    \vfill
    \footnotesize
    Data Source: Yahoo Finance as of 7-7-2025

\end{frame}

\begin{frame}[t]
    \frametitle{2. Fama-French Factors}
    \framesubtitle{Book value v. Market value - Examples}
    
    \begin{columns}
        \begin{column}{0.5\textwidth}
            \includegraphics[width=\textwidth]{figures/ch_10_XOM_price.png}
        \end{column}
        \begin{column}{0.5\textwidth}
            \includegraphics[width=\textwidth]{figures/ch_10_XOM_logs.png}
        \end{column}
    \end{columns}

\end{frame}

\begin{frame}[t]
    \frametitle{2. Fama-French Factors}
    \framesubtitle{Book value v. Market value - Examples}
    
    \begin{columns}
        \begin{column}{0.5\textwidth}
            \includegraphics[width=\textwidth]{figures/ch_10_MSFT_price.png}
        \end{column}
        \begin{column}{0.5\textwidth}
            \includegraphics[width=\textwidth]{figures/ch_10_MSFT_logs.png}
        \end{column}
    \end{columns}

\end{frame}

\begin{frame}[t]
    \frametitle{2. Fama-French Factors}
    \framesubtitle{Book value v. Market value - Examples}
    
    \begin{columns}
        \begin{column}{0.5\textwidth}
            \includegraphics[width=\textwidth]{figures/ch_10_KHC_price.png}
        \end{column}
        \begin{column}{0.5\textwidth}
            \includegraphics[width=\textwidth]{figures/ch_10_KHC_logs.png}
        \end{column}
    \end{columns}

\end{frame}

\begin{frame}[t]
    \frametitle{2. Fama-French Factors}
    \framesubtitle{Kraft-Heinz (KHC)}

    \begin{columns}
            \begin{column}{0.5\textwidth}
                \centering
                \includegraphics[width=0.6\textwidth]{figures/ch10_buffet_2019.png}
            \end{column}
            \begin{column}{0.5\textwidth}
                \centering
                \includegraphics[width=0.6\textwidth]{figures/ch10_buffet_2025.png}
            \end{column}
    \end{columns}

\end{frame}

\begin{frame}[t]
    \frametitle{2. Fama-French Factors}
    \framesubtitle{Factor Portfolios}

    \footnotesize
    \vspace{-1em}

    \begin{table}
    \caption{Average Returns for Fama-French BE/ME Portfolios, 1926-2025}
    \begin{tabular}{lrrr}
    \toprule
    decade & Lo 10 & Hi 10 & Difference \\
    \midrule
    1926-2025 & 0.91 & 1.30 & 0.39 \\
    \midrule
    1920 & 1.73 & 1.27 & -0.46 \\
    1930 & 0.49 & 1.14 & 0.64 \\
    1940 & 0.62 & 1.73 & 1.12 \\
    1950 & 1.57 & 1.69 & 0.12 \\
    1960 & 0.74 & 0.94 & 0.20 \\
    1970 & 0.27 & 1.36 & 1.09 \\
    1980 & 1.10 & 1.90 & 0.80 \\
    1990 & 1.64 & 1.45 & -0.19 \\
    2000 & -0.16 & 0.85 & 1.01 \\
    2010 & 1.29 & 0.85 & -0.44 \\
    2020 & 1.55 & 0.94 & -0.61 \\
    \bottomrule
    \end{tabular}
    \end{table}

\end{frame}

\begin{frame}[t]
    \frametitle{2. Fama-French Factors}
    \framesubtitle{Factor Performance}

    \centering
    \includegraphics[width=0.7\textwidth]{figures/ch10_1_fffactors.png}

    \raggedright
    Conclusion: Both SML and HML are factor portfolios that have significant 
    risk premiums associated with them.   Investors have been rewarded for bearing 
    exposure to these factor portfolios.

    \blfootnote{Data Source: Ken French data library}

\end{frame}

\begin{frame}[t]
    \frametitle{2. Fama-French Factors}
    \framesubtitle{Application}

    In order to estimate the expected return for stock $i$, estimate the following model:
    $$r_i = \alpha_i + \beta_{im}{r_m - r_f} + \beta_{i,SML} SML + \beta_{i,HML} HML + \epsilon$$
    Then use the estimated risk premiums for $SML$ and $HML$, along with the estimated risk premium 
    of the market, to calculate the expected return of security $r_i$.\\
    \vspace{1em}
    Note that just because there has been a nonzero risk premium on a factor in 
    the past doesn't guarantee there 
    will be a nonzero risk premium on the factor in the future.

\end{frame}

\begin{frame}[t]
    \frametitle{3. Smart Beta}
    \framesubtitle{Blackrock iShares Factors}

    https://www.ishares.com/us/strategies/smart-beta-investing

\end{frame}

\begin{frame}[t]
    \frametitle{3. Smart Beta}
    \framesubtitle{Blackrock iShares Factors}

    \centering
    \includegraphics[width=\textwidth]{figures/ch10_blk_factors.png}

\end{frame}

\begin{frame}[t]
    \frametitle{3. Smart Beta}
    \framesubtitle{Blackrock iShares Factors}

    \centering
    \includegraphics[width=\textwidth]{figures/ch10_blk_factors_perf.png}

\end{frame}

\end{document}