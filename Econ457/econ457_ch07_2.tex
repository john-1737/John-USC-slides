\documentclass{beamer}

\newcommand{\week}{Week 4-b}

\title{Diversification}
\subtitle{Reference: Bodie et al, Ch 7}
\author{Econ 457}
\date{\week}

% Reference the shared preamble
\setbeamertemplate{frametitle}{
  \vspace{0.5em}
  \insertframetitle
  \par
  \vspace{0.5em}
  \hrule
  \vspace{0.3em}
  {\small\color{gray}\insertframesubtitle}
}

\setbeamertemplate{navigation symbols}{}
\setbeamertemplate{itemize item}{\textbullet} % main bullet: filled dot
\setbeamertemplate{itemize subitem}{\normalsize$\circ$} % sub-bullet: empty dot
\setbeamertemplate{itemize subsubitem}{\scriptsize--} % sub-sub-bullet: dash


% Font changes
\usepackage[scaled=0.92]{helvet}
\renewcommand{\familydefault}{\sfdefault}

% Packages
\usepackage{tikz}
\usepackage{booktabs}
\usepackage{xcolor}
\usepackage{array}           % Enhanced column types for tables
\usepackage{multirow}        % Spanning multiple rows in tables
\usepackage{makecell}        % Line breaks and formatting in table cells
\usepackage{siunitx}         % Proper formatting of numbers and units
\usepackage{amsmath}         % Enhanced math environments
\usepackage{amsfonts}        % Additional math fonts
\usepackage{amssymb}         % Additional math symbols
\usepackage{url}             % Better URL formatting
\usepackage{graphicx}        % Enhanced graphics support
\usepackage{tabularray}
\UseTblrLibrary{booktabs, siunitx, varwidth}
% For financial presentations specifically
\usepackage{eurosym}         % Euro symbol
\usepackage{textcomp}        % Additional text symbols
\usepackage{hyperref}        % Hyperlinks (should be loaded last)

% Define a footnote
\renewcommand{\footnoterule}{\vspace*{-3pt}\hrule width 2in height 0.4pt\vspace*{2.6pt}}

% Define a Foundation Slide
\newenvironment{foundframe}[1][t]{
    \setbeamercolor{background canvas}{bg=gray!8}
    \setbeamercolor{frametitle}{fg=gray!80!black,bg=gray!25}
    \setbeamercolor{framesubtitle}{fg=gray!70!black,bg=gray!15}
    \setbeamercolor{item}{fg=gray!80!black}
    \setbeamercolor{enumerate item}{fg=gray!80!black}
    
    % Modify the frametitle template for this frame type
    \setbeamertemplate{frametitle}{
        \vspace{0.5em}
        \begin{minipage}[t]{0.75\textwidth}
            \insertframetitle
            \par
            \vspace{0.5em}
            \hrule
            \vspace{0.3em}
            {\small\color{gray}\insertframesubtitle}
        \end{minipage}%
        \hfill
        \begin{minipage}[t]{0.2\textwidth}
            \raggedleft
            \colorbox{gray!30}{%
                \scriptsize\bfseries\color{gray!80!black}%
                   \hspace{3pt}\begin{tabular}{c}Foundation\\Material\end{tabular}\hspace{3pt}%
            }
        \end{minipage}
        \vspace{0.3em}
    }
    
    \begin{frame}[#1]
}{
    \end{frame}
}

% Define Practice Slide
\newenvironment{practiceframe}[1][t]{
    \setbeamercolor{background canvas}{bg=white}
    \setbeamercolor{frametitle}{fg=blue!80!black,bg=blue!15}
    \setbeamercolor{framesubtitle}{fg=blue!70!black,bg=blue!10}
    \setbeamercolor{item}{fg=blue!80!black}
    \setbeamercolor{enumerate item}{fg=blue!80!black}
    \setbeamercolor{normal text}{fg=blue!90!black}
    
    % Modify the frametitle template for this frame type
    \setbeamertemplate{frametitle}{
        \vspace{0.5em}
        \begin{minipage}[t]{0.75\textwidth}
            \insertframetitle
            \par
            \vspace{0.5em}
            \hrule
            \vspace{0.3em}
            {\small\color{blue!70!black}\insertframesubtitle}
        \end{minipage}%
        \hfill
        \begin{minipage}[t]{0.2\textwidth}
            \raggedleft
            \colorbox{blue!20}{%
                \scriptsize\bfseries\color{blue!80!black}%
                   \hspace{3pt}\begin{tabular}{c}Practice\\Questions\end{tabular}\hspace{3pt}%
            }
        \end{minipage}
        \vspace{0.3em}
    }
    
    \begin{frame}[#1]
}{
    \end{frame}
}

% Define Excel Slide
\newenvironment{excelframe}[1][t]{
    \setbeamercolor{background canvas}{bg=white}
    \setbeamercolor{frametitle}{fg=blue!80!black,bg=blue!15}
    \setbeamercolor{framesubtitle}{fg=blue!70!black,bg=blue!10}
    \setbeamercolor{item}{fg=blue!80!black}
    \setbeamercolor{enumerate item}{fg=blue!80!black}
    \setbeamercolor{normal text}{fg=blue!90!black}
    
    % Modify the frametitle template for this frame type
    \setbeamertemplate{frametitle}{
        \vspace{0.5em}
        \begin{minipage}[t]{0.75\textwidth}
            \insertframetitle
            \par
            \vspace{0.5em}
            \hrule
            \vspace{0.3em}
            {\small\color{blue!70!black}\insertframesubtitle}
        \end{minipage}%
        \hfill
        \begin{minipage}[t]{0.2\textwidth}
            \raggedleft
            \colorbox{green!10}{%
                \scriptsize\bfseries\color{blue!80!black}%
                   \hspace{3pt}\begin{tabular}{c}MS Excel\end{tabular}\hspace{3pt}%
            }
        \end{minipage}
        \vspace{0.3em}
    }
    
    \begin{frame}[#1]
}{
    \end{frame}
}

% Define Caution Slide
\newenvironment{cautionframe}[1][t]{
    \setbeamercolor{background canvas}{bg=white}
    \setbeamercolor{frametitle}{fg=blue!80!black,bg=blue!15}
    \setbeamercolor{framesubtitle}{fg=blue!70!black,bg=blue!10}
    \setbeamercolor{item}{fg=blue!80!black}
    \setbeamercolor{enumerate item}{fg=blue!80!black}
    \setbeamercolor{normal text}{fg=blue!90!black}
    
    % Modify the frametitle template for this frame type
    \setbeamertemplate{frametitle}{
        \vspace{0.5em}
        \begin{minipage}[t]{0.75\textwidth}
            \insertframetitle
            \par
            \vspace{0.5em}
            \hrule
            \vspace{0.3em}
            {\small\color{blue!70!black}\insertframesubtitle}
        \end{minipage}%
        \hfill
        \begin{minipage}[t]{0.2\textwidth}
            \raggedleft
            \colorbox{red!10}{%
                \scriptsize\bfseries\color{blue!80!black}%
                   \hspace{3pt}\begin{tabular}{c}Caution\end{tabular}\hspace{3pt}%
            }
        \end{minipage}
        \vspace{0.3em}
    }
    
    \begin{frame}[#1]
}{
    \end{frame}
}

% Add to footnotes
\makeatletter
\newcommand\blfootnote[1]{%
  \begingroup
  \renewcommand\thefootnote{}%
  \renewcommand\@makefntext[1]{\raggedright\leftskip=0pt ##1}%
  \footnote{\scriptsize #1}%
  \addtocounter{footnote}{-1}%
  \endgroup
}
\makeatother

% Set the footer -- change 
\setbeamertemplate{footline}{
  \leavevmode%
  \vspace{2ex}
  \hbox{%
    % Left box: Econ 457
    \begin{beamercolorbox}[wd=.4\paperwidth,ht=2.5ex,dp=1ex,left]{author in head/foot}%
      \hspace{1em}Econ 457
    \end{beamercolorbox}%
    % Middle box: Week
    \begin{beamercolorbox}[wd=.2\paperwidth,ht=2.5ex,dp=1ex,center]{date in head/foot}%
      \centering\week
    \end{beamercolorbox}%
    % Right box: Slide numbers
    \begin{beamercolorbox}[wd=.4\paperwidth,ht=2.5ex,dp=1ex,center]{date in head/foot}%
      \hfill\insertframenumber{} 
    \end{beamercolorbox}%
  }%
  \vskip0pt%
}

\begin{document}

\frame{\titlepage}

\begin{frame}
    \frametitle{Outline}

    \begin{enumerate}
        \item Diversification - Review
        \begin{itemize}
            \item Set-up
            \item Standard Deviation - Cov term
            \item Maximization - Sharpe Ratio
        \end{itemize}
        \item Optimization
        \item Three Asset Optimization: SPY, TLT, and GLD
        \item Many Asset Optimization
        \item Excel
        \begin{itemize}
            \item Correlations (last week's homework)
            \item Solver (this week's homework)
        \end{itemize}
        \item Practice
    \end{enumerate}
\end{frame}

\begin{frame}[t]
    \frametitle{1. Diversification - Review}
    \framesubtitle{Set up}

    \vspace{-1em}
    \centering
    \includegraphics[width=0.9\textwidth]{figures/ch7_1_spytlt_sharpe_opt.png}

\end{frame}

\begin{frame}[t]
    \frametitle{1. Diversification - Review}
    \framesubtitle{Standard Deviation - COV term}

    \textbf{Generic Case:}\\
    $X$ and $Y$ are \textit{correlated} 
    $$X + Y \sim N(\mu_x+\mu_y,\sigma_x^2 + \sigma_y^2 + 2Cov(X,Y))$$

    \textbf{Specific Case for Portfolio Diversification:}\\
    Combining two assets in a portfolio, $X$ and $Y$,
    where the weights of each asset ($w_x$ and $w_y$) sums to one.
    $$w_xX + w_yY \sim N(w_x\mu_x+w_y\mu_y,w_x^2\sigma_x^2 + w_y^2\sigma_y^2 + 2 w_x w_y Cov(X,Y))$$
    Notice how the weights get treated in the variance term.  Continued...

\end{frame}

\begin{frame}[t]
    \frametitle{1. Diversification - Review}
    \framesubtitle{Standard Deviation - COV term}

    ...Continued\\
    \vspace{1em}
    The general formula for the variance of a portfolio composed of $n$ risky assets:
    $$\sigma_p^2 = \sum_{i=1}^{n}\sum_{j=1}^{n}w_iw_jCov(r_i,r_j)$$
    In the case that the portfolio is equally weighted (all $w=1/n$) this becomes:
    $$\sigma_p^2 = \sum_{i=1}^{n}\frac{1}{n^2}\sigma_i^2 + 
     \sum_{j=1, j \neq i}^{n}\sum_{i=1}^{n}\frac{1}{n^2}Cov(r_i,r_j)$$

\end{frame}

\begin{frame}[t]
    \frametitle{1. Diversification - Review}
    \framesubtitle{Standard Deviation - COV term}

    There is a benefit to diversification for any correlation coefficient $\rho_{x,y}<1$.
    Remember that $\rho_{x,y} = \frac{COV(X,Y)}{\sigma_x \sigma_y}$\\
    \vspace{1em}  
    Consider the case when $w_x = w_y = 0.5$ and $\sigma_x = \sigma_y = \sigma$.   The variance is then given by
    $$0.5^2\sigma^2 + 0.5^2\sigma^2 + 2 * 0.5 * 0.5 * COV(X,Y)$$
    If $\rho_{x,y}=1$ then this evaluates to $0.25\sigma^2 + 0.25\sigma^2 + \frac{\sigma^2}{0.5} = \sigma^2$ 
    and there is no benefit from diversification.\\
    \vspace{1em}
    It follows that if $\rho_{x,y}<1$ then 
    the variance term will be smaller, and therefore will be smaller and there will be 
    a benefit from diversification.\\
    
\end{frame}

\begin{frame}[t]
    \frametitle{1. Diversification - Review}
    \framesubtitle{Standard Deviation - COV term}

    Also note that if $w_x = w_y = 0.5$, $\sigma_x = \sigma_y = \sigma$, and $\rho_{x,y}=0$ then the variance 
    term is given by\\
     $$0.25\sigma^2 + 0.25\sigma^2 = 0.5\sigma^2$$  
    And the standard deviation is given by $\sqrt{0.5}\sigma$.\\
    \vspace{1em}
     Any correlation less than 0 will further reduce the variance and standard deviation terms.

\end{frame}


\begin{frame}[t]
    \frametitle{1. Diversification - Review}
    \framesubtitle{Standard Deviation - COV term}

    \centering
    \includegraphics[width=0.8\textwidth]{figures/ch7_1_spytlt_cov.png} 

\end{frame}

\begin{frame}[t]
    \frametitle{1. Diversification - Review}
    \framesubtitle{Maximization - Sharpe Ratio}

    We define the objective function that we are maximizing to be the Sharpe Ratio of the portfolio:
    $$\frac{\mathbb{E}[r_p] - r_f}{\sigma_p}$$
    How to think about this?  The Sharpe ratio is the expected excess return per unit of volatility.  
    Alternatively, the Sharpe ratio is a measure of the efficiency of the portfolio.\\
    \vspace{1em}
    You could imagine defining the objective function differently, but the Sharpe ratio is standard.

\end{frame}

\begin{foundframe}[t]
    \frametitle{2. Optimization Review}

    \vspace{-1em}
    \textbf{Typical Optimization Steps}
    \begin{enumerate}
        \item Define the objective function 
            (e.g., maximize the Utility or Sharpe Ratio).
        \item Specify constraints 
            (e.g., weights sum to 1, no short selling, minimum/maximum allocations).
        \item If possible, rewrite the objective function in terms of the variables you are choosing 
            (e.g. rewrite $U(E[r],\sigma)$ as $U(w)$ where $w$ is the weight of the risky asset in the portfolio.)
        \item Use optimization tools:
        \begin{itemize}
            \item Take first derivative with respect to choice variable, set derivative to zero, solve for choice variable.
            \item Use Lagrange multipliers
            \item Excel Solver, Python's \texttt{scipy.optimize}, R's \texttt{optim}, etc.
        \end{itemize}
        \item Analyze and interpret the resulting portfolio.
    \end{enumerate}

\end{foundframe}

\begin{foundframe}[t]
    \frametitle{2. Optimization Review}
    \framesubtitle{Maximize Utility in Capital Allocation Problem}

    Rewrite utility as a function of the weight in the risky asset ($w$):
    $$U(w) = w \cdot E[r_{\text{risky}}] + (1-w) \cdot r_f - 0.5 A \cdot w^2 \cdot \sigma_{\text{risky}}^2$$

    \centering
    \includegraphics[width=0.7\textwidth]{figures/ch6_max_w.png}

\end{foundframe}

\begin{foundframe}[t]
    \frametitle{2. Optimization Review}
    \framesubtitle{Maximize Sharpe Ratio in Diversification Problem}

    For a problem with only two assets, we can rewrite Sharpe Ratio as a function of the weight in the SPY ($w_{SPY}$):
    \footnotesize
    \begin{align*}
    &\text{Sharpe Ratio}(w_{SPY}) = \\
    &= \frac{w_{SPY}*\mathbb{E}(R_{SPY}) + (1-w_{SPY})\mathbb{E}(R_{TLT}) - r_f}{w_{SPY}^{2}*\sigma^{2}_{SPY} + (1-w_{SPY})^{2}*\sigma^{2}_{TLT} + 2*(w_{SPY})*(1-w_{SPY})*Cov(r_{SPY},r_{TLT})}
    \end{align*}

    \normalsize
    Then take the first derivative with respect to $w_{SPY}$ and set equal to zero.\\
    \vspace{1em}
    For a portfolio with many assets, you can't simplify the problem in the same way.   
    Need to use a solver (or Lagrange multipliers)

\end{foundframe}

\begin{frame}[t]
    \frametitle{3. Three Asset Optimization: SPY, TLT, and GLD}
    \framesubtitle{Total Return Series}

    \centering
    \includegraphics[width=0.8\textwidth]{figures/ch7_2_spytltgld.png}

    \blfootnote{Data Source: Yahoo Finance}

\end{frame}

\begin{frame}[t]
    \frametitle{3. Three Asset Optimization: SPY, TLT, and GLD}
    \framesubtitle{Sharpe Ratios}

    \footnotesize
    \begin{table}
        \caption{ETF Performance Statitics - Excess Returns}
        \begin{tabular}{llrrr}
        \toprule
        & Start Date & Mean (\%) & Std Dev (\%) & Sharpe Ratio \\
        Ticker &  &  &  &  \\
        \midrule
        SPY & 1993-01 & 8.440 & 14.820 & 0.569 \\
        TLT & 2002-07 & 3.090 & 13.660 & 0.226 \\
        GLD & 2004-11 & 9.180 & 16.680 & 0.550 \\
        EWW & 1996-03 & 9.420 & 26.570 & 0.355 \\
        EWD & 1996-03 & 8.640 & 24.470 & 0.353 \\
        EWH & 1996-03 & 4.630 & 24.100 & 0.192 \\
        EWI & 1996-03 & 6.440 & 23.820 & 0.271 \\
        EWJ & 1996-03 & 1.170 & 17.700 & 0.066 \\
        EWL & 1996-03 & 6.350 & 16.610 & 0.382 \\
        EWP & 1996-03 & 7.980 & 23.650 & 0.338 \\
        \bottomrule
        \end{tabular}
    \end{table}

\end{frame}

\begin{frame}[t]
    \frametitle{3. Three Asset Optimization: SPY, TLT, and GLD}
    \framesubtitle{Expected Return}

    \centering
    \includegraphics[width=0.7\textwidth]{figures/ch7_2_spytltgld_er.png}

    \blfootnote{Data Source: Yahoo Finance}

\end{frame}

\begin{frame}[t]
    \frametitle{3. Three Asset Optimization: SPY, TLT, and GLD}
    \framesubtitle{Standard Deviation}

    \centering
    \includegraphics[width=0.7\textwidth]{figures/ch7_2_spytltgld_std.png}

    \blfootnote{Data Source: Yahoo Finance}

\end{frame}

\begin{frame}[t]
    \frametitle{3. Three Asset Optimization: SPY, TLT, and GLD}
    \framesubtitle{Expected Return and Standard Deviation}

    \centering
    \includegraphics[width=0.7\textwidth]{figures/ch7_2_spytltgld_er_std.png}

    \blfootnote{Data Source: Yahoo Finance}

\end{frame}

\begin{frame}[t]
    \frametitle{3. Three Asset Optimization: SPY, TLT, and GLD}
    \framesubtitle{Capital Allocation Line}

    \centering
    \includegraphics[width=0.7\textwidth]{figures/ch7_2_spytltgld_er_std_with_cal.png}

    \blfootnote{Data Source: Yahoo Finance}

\end{frame}

\begin{frame}[t]
    \frametitle{4. Many Asset Optimization}
    \framesubtitle{Expected Return and Standard Deviations}

    \centering
    \includegraphics[width=0.7\textwidth]{figures/ch7_2_etfs.png}

    \blfootnote{Data Source: Yahoo Finance}

\end{frame}

\begin{frame}[t]
    \frametitle{4. Many Asset Optimization}
    \framesubtitle{Global Minimum Variance Portfolio}

    \centering
    \includegraphics[width=0.7\textwidth]{figures/ch7_2_etfs_with_gmv.png}

    \blfootnote{Data Source: Yahoo Finance}

\end{frame}

\begin{frame}[t]
    \frametitle{4. Many Asset Optimization}
    \framesubtitle{Capital Allocation Line and Optimal Portfolio}

    \centering
    \includegraphics[width=0.7\textwidth]{figures/ch7_2_etfs_with_gmv_cal.png}

    \blfootnote{Data Source: Yahoo Finance}

\end{frame}

\begin{frame}[t]
    \frametitle{4. Many Asset Optimization}
    \framesubtitle{Review}

    Steps for finding the optimal portfolio with many assets:
    \begin{enumerate}
        \item Estimate expected returns and Var-Cov matrix
        \item (Optional: plot various portfolios, find the Global Minimum Variance Portfolio)
        \item Maximize the Sharpe Ratio
        \item Draw the Capital Allocation Line
    \end{enumerate}

\end{frame}

\begin{excelframe}[t]
    \frametitle{5. Excel}
    \framesubtitle{Correlations (last week's homework)}

    \centering
    \includegraphics[width=0.8\textwidth]{figures/ch7_2_correlations.png}

\end{excelframe}

\begin{excelframe}[t]
    \frametitle{5. Excel}
    \framesubtitle{Correlations (last week's homework)}

    Takeaways

    \begin{itemize}
        \item Correlations are based on [monthly] returns (not a total return index).  The levels may look similar -- they all go up -- but it's the changes that matter.
        \item Correlations take out the mean.  Series A-C have a mean increase of 2\%, but that gets removed 
        in the correlation calculation.  Series D has a mean increase of 4\%, but that also gets removed.
        \item Correlations are disproportionately impacted by the observations with the 
        largest moves.   Series A and C are identical except for observation 50.   That single observation, which is the one with 
        the largest change, drives the negative correlation.
    \end{itemize}

\end{excelframe}

\begin{excelframe}[t]
    \frametitle{5. Excel}
    \framesubtitle{Maximization (this week's homework)}

    \centering
    \includegraphics[width=0.9\textwidth]{figures/ch7_1_excel.png}

\end{excelframe}

\begin{practiceframe}[t]
    \frametitle{6. Practice}

    \begin{enumerate}
        \item Stocks offer an expected rate of return of 18\% with a standard 
        deviation of 22\%.   Gold offers an expected return of 10\% with a standard 
        deviation of 30\%.   
        \begin{itemize}
            \item Would anyone hold gold? Demonstrate why or why not with a graph.
            \item Reanswer the question above if the correlation coefficient between gold and 
            stocks equal 1.
            \item Could the set of assumptions in the second part (where correlation equals 1) 
            be an equilibrium?   Why or why not?
        \end{itemize}
        \item True or false: The standard deviation of the portfolio is always 
        equal to the weighted average of the standard deviations of the assets in the portfolio.
    \end{enumerate}

\end{practiceframe}

\begin{practiceframe}[t]
    \frametitle{6. Practice}

    \begin{enumerate}[3]
        \item The correlation coefficients between pairs of stocks are as follows: Corr(A,B) = 0.85,
        Corr(A,C) = 0.6, Corr(A,D) = 0.45.   Each stock has an expected return of 8\% and a standard 
        deviation of 20\%. 
        \begin{itemize}
            \item If your entire portfolio is now composed of stock A and you can 
            add one stock to your portfolio, which do you add?
            \item Would the answer change for more risk averse investors?
            \item Suppose that you could also add T-Bills, which have a rate of 8\%.   How would 
            your portfolio look now?
        \end{itemize}
    \end{enumerate}

\end{practiceframe}

\begin{practiceframe}[t]
    \frametitle{6. Practice}

    \vspace{-1em}
    \footnotesize{
        \begin{enumerate}[4]
        \item Abigail Grace starts with \$900k and then inherits \$100k in 
        portfolio ABC stock.   She has the following estimates:
              \begin{table}
                \caption{Risk and Return Characteristics}
                \begin{tabular}{lrr}
                \toprule
                    & $E[r]$, monthly & $\sigma$\\
                \midrule
                    Original Portfolio & 0.67\% & 2.37\%\\
                ABC & 1.25 & 2.95\\
                \bottomrule
                \end{tabular}
                \end{table}
       The correlation coefficient of ABC with her original portfoio is 0.4.  Answer the following questions:
        \begin{itemize}
            \item Calculate the expected return of the new portfolio, the Covariance of ABC stock returns with her original portfolio, and the 
            standard deviation of her new portfolio.
            \item Grace sells ABC stock and buys T-Bills at a rate of 0.42\% monthly.   Recalculate the expected return and standard deviation.
            \item Grace is considering selling ABC and buying XYZ, which has the same  
            expected return and standard deviation as ABC.   Does it matter whether Grace owns ABC 
            or XZY?   Why or why not?
        \end{itemize}
    \end{enumerate}
    }
\end{practiceframe}

\end{document}
