\documentclass{beamer}

\newcommand{\week}{Week 7-a}

\title{Various Topics}
\subtitle{}
\author{Econ 457}
\date{\week}

% Reference the shared preamble
\setbeamertemplate{frametitle}{
  \vspace{0.5em}
  \insertframetitle
  \par
  \vspace{0.5em}
  \hrule
  \vspace{0.3em}
  {\small\color{gray}\insertframesubtitle}
}

\setbeamertemplate{navigation symbols}{}
\setbeamertemplate{itemize item}{\textbullet} % main bullet: filled dot
\setbeamertemplate{itemize subitem}{\normalsize$\circ$} % sub-bullet: empty dot
\setbeamertemplate{itemize subsubitem}{\scriptsize--} % sub-sub-bullet: dash


% Font changes
\usepackage[scaled=0.92]{helvet}
\renewcommand{\familydefault}{\sfdefault}

% Packages
\usepackage{tikz}
\usepackage{booktabs}
\usepackage{xcolor}
\usepackage{array}           % Enhanced column types for tables
\usepackage{multirow}        % Spanning multiple rows in tables
\usepackage{makecell}        % Line breaks and formatting in table cells
\usepackage{siunitx}         % Proper formatting of numbers and units
\usepackage{amsmath}         % Enhanced math environments
\usepackage{amsfonts}        % Additional math fonts
\usepackage{amssymb}         % Additional math symbols
\usepackage{url}             % Better URL formatting
\usepackage{graphicx}        % Enhanced graphics support
\usepackage{tabularray}
\UseTblrLibrary{booktabs, siunitx, varwidth}
% For financial presentations specifically
\usepackage{eurosym}         % Euro symbol
\usepackage{textcomp}        % Additional text symbols
\usepackage{hyperref}        % Hyperlinks (should be loaded last)

% Define a footnote
\renewcommand{\footnoterule}{\vspace*{-3pt}\hrule width 2in height 0.4pt\vspace*{2.6pt}}

% Define a Foundation Slide
\newenvironment{foundframe}[1][t]{
    \setbeamercolor{background canvas}{bg=gray!8}
    \setbeamercolor{frametitle}{fg=gray!80!black,bg=gray!25}
    \setbeamercolor{framesubtitle}{fg=gray!70!black,bg=gray!15}
    \setbeamercolor{item}{fg=gray!80!black}
    \setbeamercolor{enumerate item}{fg=gray!80!black}
    
    % Modify the frametitle template for this frame type
    \setbeamertemplate{frametitle}{
        \vspace{0.5em}
        \begin{minipage}[t]{0.75\textwidth}
            \insertframetitle
            \par
            \vspace{0.5em}
            \hrule
            \vspace{0.3em}
            {\small\color{gray}\insertframesubtitle}
        \end{minipage}%
        \hfill
        \begin{minipage}[t]{0.2\textwidth}
            \raggedleft
            \colorbox{gray!30}{%
                \scriptsize\bfseries\color{gray!80!black}%
                   \hspace{3pt}\begin{tabular}{c}Foundation\\Material\end{tabular}\hspace{3pt}%
            }
        \end{minipage}
        \vspace{0.3em}
    }
    
    \begin{frame}[#1]
}{
    \end{frame}
}

% Define Practice Slide
\newenvironment{practiceframe}[1][t]{
    \setbeamercolor{background canvas}{bg=white}
    \setbeamercolor{frametitle}{fg=blue!80!black,bg=blue!15}
    \setbeamercolor{framesubtitle}{fg=blue!70!black,bg=blue!10}
    \setbeamercolor{item}{fg=blue!80!black}
    \setbeamercolor{enumerate item}{fg=blue!80!black}
    \setbeamercolor{normal text}{fg=blue!90!black}
    
    % Modify the frametitle template for this frame type
    \setbeamertemplate{frametitle}{
        \vspace{0.5em}
        \begin{minipage}[t]{0.75\textwidth}
            \insertframetitle
            \par
            \vspace{0.5em}
            \hrule
            \vspace{0.3em}
            {\small\color{blue!70!black}\insertframesubtitle}
        \end{minipage}%
        \hfill
        \begin{minipage}[t]{0.2\textwidth}
            \raggedleft
            \colorbox{blue!20}{%
                \scriptsize\bfseries\color{blue!80!black}%
                   \hspace{3pt}\begin{tabular}{c}Practice\\Questions\end{tabular}\hspace{3pt}%
            }
        \end{minipage}
        \vspace{0.3em}
    }
    
    \begin{frame}[#1]
}{
    \end{frame}
}

% Define Excel Slide
\newenvironment{excelframe}[1][t]{
    \setbeamercolor{background canvas}{bg=white}
    \setbeamercolor{frametitle}{fg=blue!80!black,bg=blue!15}
    \setbeamercolor{framesubtitle}{fg=blue!70!black,bg=blue!10}
    \setbeamercolor{item}{fg=blue!80!black}
    \setbeamercolor{enumerate item}{fg=blue!80!black}
    \setbeamercolor{normal text}{fg=blue!90!black}
    
    % Modify the frametitle template for this frame type
    \setbeamertemplate{frametitle}{
        \vspace{0.5em}
        \begin{minipage}[t]{0.75\textwidth}
            \insertframetitle
            \par
            \vspace{0.5em}
            \hrule
            \vspace{0.3em}
            {\small\color{blue!70!black}\insertframesubtitle}
        \end{minipage}%
        \hfill
        \begin{minipage}[t]{0.2\textwidth}
            \raggedleft
            \colorbox{green!10}{%
                \scriptsize\bfseries\color{blue!80!black}%
                   \hspace{3pt}\begin{tabular}{c}MS Excel\end{tabular}\hspace{3pt}%
            }
        \end{minipage}
        \vspace{0.3em}
    }
    
    \begin{frame}[#1]
}{
    \end{frame}
}

% Define Caution Slide
\newenvironment{cautionframe}[1][t]{
    \setbeamercolor{background canvas}{bg=white}
    \setbeamercolor{frametitle}{fg=blue!80!black,bg=blue!15}
    \setbeamercolor{framesubtitle}{fg=blue!70!black,bg=blue!10}
    \setbeamercolor{item}{fg=blue!80!black}
    \setbeamercolor{enumerate item}{fg=blue!80!black}
    \setbeamercolor{normal text}{fg=blue!90!black}
    
    % Modify the frametitle template for this frame type
    \setbeamertemplate{frametitle}{
        \vspace{0.5em}
        \begin{minipage}[t]{0.75\textwidth}
            \insertframetitle
            \par
            \vspace{0.5em}
            \hrule
            \vspace{0.3em}
            {\small\color{blue!70!black}\insertframesubtitle}
        \end{minipage}%
        \hfill
        \begin{minipage}[t]{0.2\textwidth}
            \raggedleft
            \colorbox{red!10}{%
                \scriptsize\bfseries\color{blue!80!black}%
                   \hspace{3pt}\begin{tabular}{c}Caution\end{tabular}\hspace{3pt}%
            }
        \end{minipage}
        \vspace{0.3em}
    }
    
    \begin{frame}[#1]
}{
    \end{frame}
}

% Add to footnotes
\makeatletter
\newcommand\blfootnote[1]{%
  \begingroup
  \renewcommand\thefootnote{}%
  \renewcommand\@makefntext[1]{\raggedright\leftskip=0pt ##1}%
  \footnote{\scriptsize #1}%
  \addtocounter{footnote}{-1}%
  \endgroup
}
\makeatother

% Set the footer -- change 
\setbeamertemplate{footline}{
  \leavevmode%
  \vspace{2ex}
  \hbox{%
    % Left box: Econ 457
    \begin{beamercolorbox}[wd=.4\paperwidth,ht=2.5ex,dp=1ex,left]{author in head/foot}%
      \hspace{1em}Econ 457
    \end{beamercolorbox}%
    % Middle box: Week
    \begin{beamercolorbox}[wd=.2\paperwidth,ht=2.5ex,dp=1ex,center]{date in head/foot}%
      \centering\week
    \end{beamercolorbox}%
    % Right box: Slide numbers
    \begin{beamercolorbox}[wd=.4\paperwidth,ht=2.5ex,dp=1ex,center]{date in head/foot}%
      \hfill\insertframenumber{} 
    \end{beamercolorbox}%
  }%
  \vskip0pt%
}

\begin{document}

\frame{\titlepage}

\begin{frame}
    \frametitle{Outline}

    \begin{enumerate}
        \item Midterm 
        \item Stocks and Bonds Correlations
        \item CAPM and Bond Yields
        \item Regression in Excel
    \end{enumerate}
\end{frame}


\begin{frame}[t]
    \frametitle{1. Midterm}
    \framesubtitle{}

    \centering
    \includegraphics[width=0.8\textwidth]{figures/ch0709_midterm.png}

\end{frame}

\begin{frame}[t]
    \frametitle{1. Midterm}
    \framesubtitle{Grading}

    \begin{table}
        \centering
        \caption{Grades for Econ 457}
        \begin{tblr}{
            colspec = {Q[l,wd=2cm] Q[c,wd=2.cm]}
        }
        \toprule
        Item & Percent of Total \\
        \midrule
        Homework & 25\% \\
        Mid-term 1 & 20\% \\
        Mid-term 2 & 20\% \\
        Final & 35\% \\
        \bottomrule
        \end{tblr}
    \end{table}

\end{frame}

\begin{frame}[t]
    \frametitle{1. Midterm}
    \framesubtitle{Review Topics}

    \begin{enumerate}
        \item Charts for each section
        \item Standard Deviation of a two asset portfolio
        \item EAR 
        \item Variance and Covariance
    \end{enumerate}

\end{frame}

\begin{frame}[t]
    \frametitle{2. Correlations in Practice: SPY and US Treasuries}
    \framesubtitle{Total Return Indices}

    \centering
    \includegraphics[width=0.8\textwidth]{figures/ch7_2_shiller_sp500_gs10_tr.png}

    \blfootnote{Data Source: Robert Shiller}

\end{frame}


\begin{frame}[t]
    \frametitle{2. Correlations in Practice: SPY and US Treasuries}
    \framesubtitle{Total Return Indices - Log Scale}

    \centering
    \includegraphics[width=0.8\textwidth]{figures/ch7_2_shiller_sp500_gs10_tr_logs.png}

    \blfootnote{Data Source: Robert Shiller}

\end{frame}

\begin{frame}[t]
    \frametitle{2. Correlations in Practice: SPY and US Treasuries}
    \framesubtitle{Rolling Correlations}

    \centering
    \includegraphics[width=0.8\textwidth]{figures/ch7_2_sp500_gs10_corr.png}

    \blfootnote{Data Source: Robert Shiller}

\end{frame}

\begin{frame}
    \frametitle{2. Correlations in Practice: SPY and US Treasuries}
    \framesubtitle{Rolling Correlations}

    Preview from fixed income section: \textit{bond prices and yields are inversely related.}\\
    \vspace{1em}

\end{frame}

\begin{frame}[t]
    \frametitle{2. Correlations in Practice: SPY and US Treasuries}
    \framesubtitle{Rolling Correlations}

    What drives the correlation between stocks and bonds?
    \footnotesize
    \begin{itemize}
        \item \textbf{Positive Correlation:}
        \begin{enumerate}
            \item Inflation shocks: Higher (lower) inflation expectations lead to tighter (easier) monetary policy
            \item Higher (lower) discount rates reduce (increase) present value of future cash flows, lowering (raising) both stock and bond prices
        \end{enumerate}
        \item \textbf{Negative Correlation:}
        \begin{enumerate}
            \item Growth shocks: Lower (higher) growth expectations reduce (increase) equity valuations while increasing (decreasing) demand for safe assets
            \item Monetary policy response: Growth concerns lead to easier (tighter) monetary policy, lowering (raising) bond yields while equity markets decline (rise)
            \item Safe haven demand: Economic uncertainty drives investors from risky stocks to safe government bonds
        \end{enumerate}
    \end{itemize}

\end{frame}


\begin{frame}[t]
    \frametitle{2. Correlations in Practice: SPY and US Treasuries}
    \framesubtitle{1970s and 1980s}

    \begin{columns}
        \begin{column}{0.3\textwidth}
            \footnotesize
            \begin{itemize}
                \item Inflation shocks led to higher interest rates and lower equities, contributing to a \textit{positive} correlation.
                \item Lower inflation allowed the Fed to ease policy in the 1980s, again contributing to \textit{positve} correlation
            \end{itemize}
        \end{column}
        \begin{column}{0.7\textwidth}
            \centering
            \includegraphics[width=\textwidth]{figures/ch7_2_sp500_gs10_corr_1970s.png}
        \end{column}
    \end{columns}

\end{frame}

\begin{frame}[t]
    \frametitle{2. Correlations in Practice: SPY and US Treasuries}
    \framesubtitle{2000s}

    \begin{columns}
        \begin{column}{0.3\textwidth}
            \footnotesize
            \begin{itemize}
                \item Low and stable inflation started in the 1990s.   
                \item Growth shocks and safe haven demand \textit{negative} correlation in 2008.
            \end{itemize}
        \end{column}
        \begin{column}{0.7\textwidth}
            \centering
            \includegraphics[width=\textwidth]{figures/ch7_2_sp500_gs10_corr_2000s.png}
        \end{column}
    \end{columns}

\end{frame}

\begin{frame}[t]
    \frametitle{2. Correlations in Practice: SPY and US Treasuries}
    \framesubtitle{2020s}

    \begin{columns}
        \begin{column}{0.3\textwidth}
            \footnotesize
            \begin{itemize}
                \item Safe haven demand drove \textit{negative} correlation in 2020
                \item Inflation shock drove \textit{positive} correlation in 2022
            \end{itemize}
        \end{column}
        \begin{column}{0.7\textwidth}
            \centering
            \includegraphics[width=\textwidth]{figures/ch7_2_sp500_gs10_corr_2020s.png}
        \end{column}
    \end{columns}

\end{frame}

\begin{frame}[t]
    \frametitle{2. Correlations in Practice: SPY and US Treasuries}
    \framesubtitle{Takeaways}

    \begin{enumerate}
        \item Correlations can and do change!   
        \item Correlations reflect the macro environment.
        \item Correctly anticipating the correlations can greatly enhance returns.
        \item Incorrectly anticipating the correlations can cause big problems.
    \end{enumerate}

\end{frame}

\begin{frame}[t]
    \frametitle{3. CAPM and Bond Yields}
    \framesubtitle{2007-2011}

    \centering
    \includegraphics[width=0.9\textwidth]{figures/ch9_tltspy_07_11.png}

\end{frame}

\begin{frame}[t]
    \frametitle{3. CAPM and Bond Yields}
    \framesubtitle{2021-2025}

    \centering
    \includegraphics[width=0.9\textwidth]{figures/ch9_tltspy_21_25.png}

\end{frame}


\begin{frame}[t]
    \frametitle{3. CAPM and Bond Yields}
    \framesubtitle{Rolling Correlations}

    \centering
    \includegraphics[width=0.9\textwidth]{figures/ch9_tltspy_rolling.png}

\end{frame} 

\begin{frame}[t]
    \frametitle{3. CAPM and Bond Yields}
    \framesubtitle{Implications for Bond Yields}

    If bond returns are negatively correlated with the S\&P, then the expected return of bonds will be \textit{less} than the return on the risk-free rate.\\
    \vspace{1em}

    In contrast, a positive correlation between bond returns and S\&P returns would be associated with expected bond returns \textit{higher} than the return on the risk-free rate.\\
    \vspace{1em}
    Conclude: The sharp change in correlations in 2021-2022 likely contributed to the increase in bond yields during that time.

\end{frame}

\begin{frame}[t]
    \frametitle{3. CAPM and Bond Yields}
    \framesubtitle{Term Premium}

    Term premium models attempt to estimate the expected return of Treasury bonds in excess of the risk-free rate over the life of the bond.\\
    \vspace{1em}
    These models depend on a number of assumptions and complicated mathematics.   They are unlikely to be used as trading tools, but nevertheless can be useful in characterizing the drivers of changes in prices and yields.
\end{frame}

\begin{frame}[t]
    \frametitle{3. CAPM and Bond Yields}
    \framesubtitle{Term Premium}

    \centering
    \includegraphics[width=0.8\textwidth]{figures/ch9_termpremium.png}

\end{frame}





\begin{excelframe}[t]
    \frametitle{4. Regression}
    \framesubtitle{}

    \centering
    \includegraphics[width=\textwidth]{figures/ch0709_reg0.png}

\end{excelframe}

\begin{excelframe}[t]
    \frametitle{4. Regression}
    \framesubtitle{}

    \centering
    \includegraphics[width=\textwidth]{figures/ch0709_reg1.png}

\end{excelframe}

\end{document}