\documentclass{beamer}

\newcommand{\week}{Week 2-b}

\title{Evaluating Financial Market Returns}
\subtitle{Reference: Bodie et al, Ch 5}
\author{Econ 457}
\date{\week}

% Reference the shared preamble
\setbeamertemplate{frametitle}{
  \vspace{0.5em}
  \insertframetitle
  \par
  \vspace{0.5em}
  \hrule
  \vspace{0.3em}
  {\small\color{gray}\insertframesubtitle}
}

\setbeamertemplate{navigation symbols}{}
\setbeamertemplate{itemize item}{\textbullet} % main bullet: filled dot
\setbeamertemplate{itemize subitem}{\normalsize$\circ$} % sub-bullet: empty dot
\setbeamertemplate{itemize subsubitem}{\scriptsize--} % sub-sub-bullet: dash


% Font changes
\usepackage[scaled=0.92]{helvet}
\renewcommand{\familydefault}{\sfdefault}

% Packages
\usepackage{tikz}
\usepackage{booktabs}
\usepackage{xcolor}
\usepackage{array}           % Enhanced column types for tables
\usepackage{multirow}        % Spanning multiple rows in tables
\usepackage{makecell}        % Line breaks and formatting in table cells
\usepackage{siunitx}         % Proper formatting of numbers and units
\usepackage{amsmath}         % Enhanced math environments
\usepackage{amsfonts}        % Additional math fonts
\usepackage{amssymb}         % Additional math symbols
\usepackage{url}             % Better URL formatting
\usepackage{graphicx}        % Enhanced graphics support
\usepackage{tabularray}
\UseTblrLibrary{booktabs, siunitx, varwidth}
% For financial presentations specifically
\usepackage{eurosym}         % Euro symbol
\usepackage{textcomp}        % Additional text symbols
\usepackage{hyperref}        % Hyperlinks (should be loaded last)

% Define a footnote
\renewcommand{\footnoterule}{\vspace*{-3pt}\hrule width 2in height 0.4pt\vspace*{2.6pt}}

% Define a Foundation Slide
\newenvironment{foundframe}[1][t]{
    \setbeamercolor{background canvas}{bg=gray!8}
    \setbeamercolor{frametitle}{fg=gray!80!black,bg=gray!25}
    \setbeamercolor{framesubtitle}{fg=gray!70!black,bg=gray!15}
    \setbeamercolor{item}{fg=gray!80!black}
    \setbeamercolor{enumerate item}{fg=gray!80!black}
    
    % Modify the frametitle template for this frame type
    \setbeamertemplate{frametitle}{
        \vspace{0.5em}
        \begin{minipage}[t]{0.75\textwidth}
            \insertframetitle
            \par
            \vspace{0.5em}
            \hrule
            \vspace{0.3em}
            {\small\color{gray}\insertframesubtitle}
        \end{minipage}%
        \hfill
        \begin{minipage}[t]{0.2\textwidth}
            \raggedleft
            \colorbox{gray!30}{%
                \scriptsize\bfseries\color{gray!80!black}%
                   \hspace{3pt}\begin{tabular}{c}Foundation\\Material\end{tabular}\hspace{3pt}%
            }
        \end{minipage}
        \vspace{0.3em}
    }
    
    \begin{frame}[#1]
}{
    \end{frame}
}

% Define Practice Slide
\newenvironment{practiceframe}[1][t]{
    \setbeamercolor{background canvas}{bg=white}
    \setbeamercolor{frametitle}{fg=blue!80!black,bg=blue!15}
    \setbeamercolor{framesubtitle}{fg=blue!70!black,bg=blue!10}
    \setbeamercolor{item}{fg=blue!80!black}
    \setbeamercolor{enumerate item}{fg=blue!80!black}
    \setbeamercolor{normal text}{fg=blue!90!black}
    
    % Modify the frametitle template for this frame type
    \setbeamertemplate{frametitle}{
        \vspace{0.5em}
        \begin{minipage}[t]{0.75\textwidth}
            \insertframetitle
            \par
            \vspace{0.5em}
            \hrule
            \vspace{0.3em}
            {\small\color{blue!70!black}\insertframesubtitle}
        \end{minipage}%
        \hfill
        \begin{minipage}[t]{0.2\textwidth}
            \raggedleft
            \colorbox{blue!20}{%
                \scriptsize\bfseries\color{blue!80!black}%
                   \hspace{3pt}\begin{tabular}{c}Practice\\Questions\end{tabular}\hspace{3pt}%
            }
        \end{minipage}
        \vspace{0.3em}
    }
    
    \begin{frame}[#1]
}{
    \end{frame}
}

% Define Excel Slide
\newenvironment{excelframe}[1][t]{
    \setbeamercolor{background canvas}{bg=white}
    \setbeamercolor{frametitle}{fg=blue!80!black,bg=blue!15}
    \setbeamercolor{framesubtitle}{fg=blue!70!black,bg=blue!10}
    \setbeamercolor{item}{fg=blue!80!black}
    \setbeamercolor{enumerate item}{fg=blue!80!black}
    \setbeamercolor{normal text}{fg=blue!90!black}
    
    % Modify the frametitle template for this frame type
    \setbeamertemplate{frametitle}{
        \vspace{0.5em}
        \begin{minipage}[t]{0.75\textwidth}
            \insertframetitle
            \par
            \vspace{0.5em}
            \hrule
            \vspace{0.3em}
            {\small\color{blue!70!black}\insertframesubtitle}
        \end{minipage}%
        \hfill
        \begin{minipage}[t]{0.2\textwidth}
            \raggedleft
            \colorbox{green!10}{%
                \scriptsize\bfseries\color{blue!80!black}%
                   \hspace{3pt}\begin{tabular}{c}MS Excel\end{tabular}\hspace{3pt}%
            }
        \end{minipage}
        \vspace{0.3em}
    }
    
    \begin{frame}[#1]
}{
    \end{frame}
}

% Define Caution Slide
\newenvironment{cautionframe}[1][t]{
    \setbeamercolor{background canvas}{bg=white}
    \setbeamercolor{frametitle}{fg=blue!80!black,bg=blue!15}
    \setbeamercolor{framesubtitle}{fg=blue!70!black,bg=blue!10}
    \setbeamercolor{item}{fg=blue!80!black}
    \setbeamercolor{enumerate item}{fg=blue!80!black}
    \setbeamercolor{normal text}{fg=blue!90!black}
    
    % Modify the frametitle template for this frame type
    \setbeamertemplate{frametitle}{
        \vspace{0.5em}
        \begin{minipage}[t]{0.75\textwidth}
            \insertframetitle
            \par
            \vspace{0.5em}
            \hrule
            \vspace{0.3em}
            {\small\color{blue!70!black}\insertframesubtitle}
        \end{minipage}%
        \hfill
        \begin{minipage}[t]{0.2\textwidth}
            \raggedleft
            \colorbox{red!10}{%
                \scriptsize\bfseries\color{blue!80!black}%
                   \hspace{3pt}\begin{tabular}{c}Caution\end{tabular}\hspace{3pt}%
            }
        \end{minipage}
        \vspace{0.3em}
    }
    
    \begin{frame}[#1]
}{
    \end{frame}
}

% Add to footnotes
\makeatletter
\newcommand\blfootnote[1]{%
  \begingroup
  \renewcommand\thefootnote{}%
  \renewcommand\@makefntext[1]{\raggedright\leftskip=0pt ##1}%
  \footnote{\scriptsize #1}%
  \addtocounter{footnote}{-1}%
  \endgroup
}
\makeatother

% Set the footer -- change 
\setbeamertemplate{footline}{
  \leavevmode%
  \vspace{2ex}
  \hbox{%
    % Left box: Econ 457
    \begin{beamercolorbox}[wd=.4\paperwidth,ht=2.5ex,dp=1ex,left]{author in head/foot}%
      \hspace{1em}Econ 457
    \end{beamercolorbox}%
    % Middle box: Week
    \begin{beamercolorbox}[wd=.2\paperwidth,ht=2.5ex,dp=1ex,center]{date in head/foot}%
      \centering\week
    \end{beamercolorbox}%
    % Right box: Slide numbers
    \begin{beamercolorbox}[wd=.4\paperwidth,ht=2.5ex,dp=1ex,center]{date in head/foot}%
      \hfill\insertframenumber{} 
    \end{beamercolorbox}%
  }%
  \vskip0pt%
}

\begin{document}

\frame{\titlepage}

\begin{frame}[t]
    \frametitle{Outline}

    Previously we covered measuring total returns (HPR, EAR, etc).   
    Today we will talk about evaluating those returns.   
    The first step is to compare the returns with something.

    \vspace{2em}

    \begin{enumerate}
        \item Inflation \& Real Returns
        \item Risk-free Returns \& Excess Returns (Sharpe Ratio)
        \item Index Returns \& Excess Returns (Information Ratio)
        \item Excel - Pivot Tables
        \item Practice
    \end{enumerate}
\end{frame}

\begin{frame}[t]
    \frametitle{1. Inflation \& Real Returns}
    \framesubtitle{Long Run View}

    \centering{
      \includegraphics[width=0.8\textwidth]{figures/ch5_2_fig_1871-2025.png}
    }
      \tiny{
      \begin{table}
      \begin{tabular}{lr}
      \toprule
      & Annualized Rate \\
      \midrule
      S\&P 500 Total Return & 9.92 \\
      CPI Inflation & 2.16 \\
      S\&P 500 Real TR& 7.82 \\
      \bottomrule
      \end{tabular}
      \end{table}
      }
    \blfootnote{Data Source: Robert Shiller}

\end{frame}

\begin{frame}[t]
    \frametitle{1. Inflation \& Real Returns}
    \framesubtitle{1970}

    \centering{
      \includegraphics[width=0.8\textwidth]{figures/ch5_2_fig_1970s.png}
    }
      \tiny{
      \begin{table}
      \begin{tabular}{lr}
      \toprule
      & Annualized Rate \\
      \midrule
     S\&P 500 Total Return & 6.56 \\
      CPI Inflation & 7.16 \\
      S\&P 500 Real TR & -0.53 \\
      \bottomrule
      \end{tabular}
      \end{table}
      }
    \blfootnote{Data Source: Robert Shiller}

\end{frame}

\begin{frame}[t]
    \frametitle{1. Inflation \& Real Returns}
    \framesubtitle{2021 \& 2022}

    \centering{
      \includegraphics[width=0.8\textwidth]{figures/ch5_2_fig_2021.png}
    }
      \tiny{
      \begin{table}
      \begin{tabular}{lr}
      \toprule
      & Annualized Rate \\
      \midrule
      S\&P 500 Total Return & 5.11 \\
      CPI Inflation & 6.62 \\
      S\&P 500 Real TR & -1.35 \\
      \bottomrule
      \end{tabular}
      \end{table}
      }
    \blfootnote{Data Source: Robert Shiller}

\end{frame}


\begin{frame}[t]
    \frametitle{1. Inflation \& Real Returns}
    \framesubtitle{Some Math}

    Measure Real Returns as Follows:\\
    \vspace{1em}
    \begin{enumerate}
      \item CPI is an Index where Aug 1983 = 100
      \item Dividing a price series by $\text{CPI}/100$ puts everything in "1983 dollars"
      \item For each element in a series, construct a real value:
      $$\text{Real Price (1983 Dollars)}_t = \frac{\text{Price}_t}{\text{CPI}_t/100}$$
      \item Construct a return series first, then adjust to get real values
    \end{enumerate}
  
  \end{frame}

\begin{frame}[t]
    \frametitle{1. Inflation \& Real Returns}
    \framesubtitle{Fisher Equation}
    A common approximation* is:
    $$r_\text{real} \approx r_{\text{nominal}} - \pi$$
    where $r_\text{real}$ is the annualized real rate of return, 
    $r_{\text{nominal}}$ is the annualized nominal rate of return, 
    and $\pi$ is the inflation rate over the relevant period.\\
    \vspace{1em}
    
    \vfill
    \footnotesize
    \rule{2in}{0.4pt}\\
    *As long as $\pi$ is small, we can ignore the cross-term because it is close to zero.
    $$(1 + r_{\text{nominal}}) = (1+r_\text{real})(1 + \pi) = 1 + r_\text{real} + \pi + r_\text{real}*\pi \approx 1 + r_\text{real} + \pi$$
    

\end{frame}

\begin{frame}[t]
    \frametitle{1. Inflation \& Real Returns}
    \framesubtitle{Fisher Equation}

    When used for \textit{expected returns}, this approximation is commonly known as the \textit{Fisher Equation}.\\
    \vspace{1em}
    $$\mathbb{E}[r_\text{real}] \approx \mathbb{E}[r_{\text{nominal}}] - \pi^e$$
    Where $\pi^e$ is expected inflation.\\
    \vspace{1em}
    Remember that the realized return will almost never equal the expected return, as that happens 
    only when the expectations are exactly correct, which is rare.  
  
\end{frame}

\begin{frame}[t]
    \frametitle{1. Inflation \& Real Returns}
    \framesubtitle{Fisher Equation}

    \centering
    \includegraphics[width=0.8\textwidth]{figures/ch5_3_ust_infl.png}

    \blfootnote{Data Source: Federal Reserve, BLS, FRED}

\end{frame}

{
\setbeamertemplate{footline}{}
\begin{frame}[t]
    \frametitle{1. Inflation and Real Returns}
    \framesubtitle{S\&P 500 Total Returns: Nominal and Real}
    
    \vspace{-1em}
    \scriptsize
   \begin{table}
    \caption{S\&P 500 Total Returns by Decade, (\% Annualized)}
    \begin{tabular}{lrrr}
    \toprule
    Decade & Return & CPI Inflation & Real Return \\
    \midrule
    1870s & 8.11 & -2.59 & 10.92 \\
    1880s & 6.29 & -2.38 & 8.52 \\
    1890s & 6.24 & 0.58 & 6.08 \\
    1900s & 10.62 & 2.49 & 8.30 \\
    1910s & 4.99 & 6.65 & -1.31 \\
    1920s & 15.46 & -1.11 & 16.46 \\
    1930s & 4.39 & -1.98 & 6.33 \\
    1940s & 9.47 & 5.39 & 4.30 \\
    1950s & 18.24 & 2.27 & 16.01 \\
    1960s & 8.05 & 2.55 & 5.57 \\
    1970s & 6.56 & 7.16 & -0.53 \\
    1980s & 16.88 & 4.89 & 11.87 \\
    1990s & 17.19 & 2.81 & 14.29 \\
    2000s & 0.38 & 2.50 & -2.12 \\
    2010s & 13.02 & 1.73 & 11.25 \\
    2020s & 13.72 & 4.08 & 9.55 \\
    \bottomrule
    \end{tabular}
    \end{table}

    \blfootnote{Data Source: Robert Shiller, Ken French.  Monthly data annualized.}

\end{frame}
}


\begin{frame}[t]
    \frametitle{1. Inflation \& Real Returns}
    \framesubtitle{Observations}

    Some observations about real returns.\\
    \begin{itemize}
      \item The appropriate price index may vary by investor
        \begin{itemize}        
        \item {\footnotesize For example, endowments may face higher price inflation if the price of health care (what they buy) increases faster than CPI.}
      \end{itemize}
      \item Often investment goals are expressed in real terms.  
      \begin{itemize}        
        \item {\footnotesize For example, "The target return for Harvard's endowed funds is 8\%, which accounts for roughly a 5\% distribution and 3\% growth to maintain purchasing power over time."}
        \item When they report returns, do they report real returns?
      \end{itemize}
      \item Straightforward to measure historic real returns.   Prospective real returns are harder.   Specifically, difficult to measure expected inflation.
      \begin{itemize}
        \item {\footnotesize TIPS are very useful.   Will return to TIPS later in the semester}
      \end{itemize}
    \end{itemize}
\end{frame}

\begin{frame}[t]
    \frametitle{2. Risk-free rates \& Excess Returns}
    Another way to evaluate returns is to compare them against an alternative investment.\\
    \vspace{1em}  
    A natural alternative is the "risk-free rate".  
    One convention is to use the return on Trasury Bills (T-Bills) as the risk-free rate, because 
    T-bills returns have very low volatiliy.\\
    \vspace{1em}
    Note that Treasury \textit{bonds} have price risk, 
    and are therefore not good measures of 'risk-free rates'.
    

\end{frame}

\begin{frame}[t]
    \frametitle{2. Risk-free rates \& Excess Returns}

      \centering
      \includegraphics[width=0.8\textwidth]{figures/ch5_2_fig4.png}

      \blfootnote{Data Source: Federal Reserve}

\end{frame}

\begin{frame}[t]
    \frametitle{2. Risk-free rates \& Excess Returns}
    \framesubtitle{A few words of caution}

    In the discussion about inflation we compared a price series with the CPI.  
    In contrast, here we compare returns of the risky asset with returns on the 
    risk-free asset.\\
    \vspace{1em}

    T-Bill are typically quoted with annualized rates, which need to be adjusted for the 
    appropriate period:
    \begin{itemize}
      \item The 3m return: $(1+\text{T-Bill Rate})^\frac{1}{4}-1$, 
      \item The 1m return: $(1+ \text{T-Bill Rate})^\frac{1}{12}-1$\\
    \end{itemize}


\end{frame}

\begin{frame}[t]
    \frametitle{2. Risk-free rates \& Excess Returns}
    \framesubtitle{Excess Returns of Risky Assets}

    \begin{enumerate}
      \item Calculate the monthly return for the risk-free asset (T-Bills)
      \begin{itemize}
        \item The return each month is determined by the T-Bill rate at the \textit{beginning} of the month.
      \end{itemize}
      \item Create the monthly total return for the risky asset 
        \begin{itemize}
        \item Can use total return series from previous class.  Remember to include income.   Price change reflects the price at the \textit{end} of the month.
      \end{itemize}
      \item For each period, subtract the risk-free return from the return for the risky asset
       to get the "excess return" of the risky asset.
      \item Use the "excess return" in each period to construct a new series for the risky asset
      \item \textit{Note:} Be careful to use the same frequency (e.g. monthly) 
    \end{enumerate}
    
\end{frame}

\begin{frame}[t]
    \frametitle{2. Risk-free rates \& Excess Returns}
    \framesubtitle{Excess Returns of Stocks, cont'd}

    \vspace{2em}
    \begin{minipage}{0.48\textwidth}
      \centering
      % Table
      \includegraphics[width=\textwidth]{figures/ch5_2_fig5.png}
    \end{minipage}
    \hfill
    \begin{minipage}{0.48\textwidth}
      \centering
      % Chart
      \includegraphics[width=\textwidth]{figures/ch5_2_fig6.png}
    \end{minipage}
\end{frame}
  
\begin{frame}[t]
  \frametitle{2. Risk-free rates \& Excess Returns}
  \framesubtitle{Excess Returns of Stocks, cont'd}

  \centering
  \includegraphics[width=0.9\textwidth]{figures/ch5_3_ereturns.png}

\end{frame}

{
\setbeamertemplate{footline}{}
\begin{frame}[t]
    \frametitle{2. Risk-free rates \& Excess Returns}
    \framesubtitle{S\&P 500 Total Returns: Nominal and Real}
    
    \vspace{-1em}
    \scriptsize
    \begin{table}
    \caption{S\&P 500 Total Returns by Decade, (\% Annualized)}
    \begin{tabular}{lrrr}
    \toprule
    Decade & Return & Risk Free Return & Excess Return \\
    \midrule
    1870s & 8.11 & NaN & NaN \\
    1880s & 6.29 & NaN & NaN \\
    1890s & 6.24 & NaN & NaN \\
    1900s & 10.62 & NaN & NaN \\
    1910s & 4.99 & NaN & NaN \\
    1920s & 15.46 & 3.66 & 13.95 \\
    1930s & 4.39 & 0.55 & 5.46 \\
    1940s & 9.47 & 0.41 & 9.92 \\
    1950s & 18.24 & 1.86 & 15.72 \\
    1960s & 8.05 & 3.81 & 4.94 \\
    1970s & 6.56 & 6.14 & 1.18 \\
    1980s & 16.88 & 8.55 & 8.56 \\
    1990s & 17.19 & 4.82 & 12.79 \\
    2000s & 0.38 & 2.72 & -1.70 \\
    2010s & 13.02 & 0.51 & 13.13 \\
    2020s & 13.72 & 2.48 & 10.72 \\
    \bottomrule
    \end{tabular}
    \end{table}

    \blfootnote{Data Source: Robert Shiller, Ken French.  Monthly data annualized.}

\end{frame}
}

\begin{frame}[t]
  \frametitle{2. Risk-free rates \& Excess Returns}
  \framesubtitle{Sharpe Ratio}

  The \textbf{Sharpe Ratio} is defined as:
  $$\frac{E[r] - r_f}{\sigma}$$
  Where $E[r] - r_f$ is the \textit{excess} return of the asset and $\sigma$ is the standard deviation.\\
  \vspace{1em}
  We will see the Sharpe Ratio used in many different contexts in the coming weeks.

\end{frame}

  \begin{frame}[t]
  \frametitle{2. Risk-free rates \& Excess Returns}
  \framesubtitle{Sharpe Ratio, cont'd}

  From 2004 to 2025, the Sharpe Ratio of the S\&P 500 has been:

  \begin{table}
        \caption{S\&P 500 Sharpe Ratio Statistics: 2004-2025}
      \begin{tabular}{lcc}
      \toprule
      Statistic & Value & Percentage \\
      \midrule
      Mean Excess Return (Annualized) & 0.086 & 8.6\% \\
      Standard Deviation (Annualized) & 0.124 & 12.4\% \\
      Sharpe Ratio & 0.692 & -- \\
      \bottomrule
      \end{tabular}
  \end{table}

\end{frame}

{
\setbeamertemplate{footline}{}
\begin{frame}[t]
    \frametitle{2. Risk-free rates \& Excess Returns}
    \framesubtitle{Sharpe Ratio}
    
    \vspace{-1em}
    \scriptsize
    \begin{table}
    \caption{S\&P 500 Total Returns by Decade, (\% Annualized)}
    \begin{tabular}{lrrr}
    \toprule
    Decade & Excess Return & Standard Deviation & Sharpe Ratio \\
    \midrule
    1870s & NaN & 10.86 & NaN \\
    1880s & NaN & 9.62 & NaN \\
    1890s & NaN & 11.84 & NaN \\
    1900s & NaN & 12.70 & NaN \\
    1910s & NaN & 10.51 & NaN \\
    1920s & 13.95 & 15.08 & 0.93 \\
    1930s & 5.46 & 30.63 & 0.18 \\
    1940s & 9.92 & 13.23 & 0.75 \\
    1950s & 15.72 & 10.08 & 1.56 \\
    1960s & 4.94 & 10.31 & 0.48 \\
    1970s & 1.18 & 13.23 & 0.09 \\
    1980s & 8.56 & 12.98 & 0.66 \\
    1990s & 12.79 & 10.45 & 1.22 \\
    2000s & -1.70 & 14.68 & -0.12 \\
    2010s & 13.13 & 9.55 & 1.38 \\
    2020s & 10.72 & 14.15 & 0.76 \\
    \bottomrule
    \end{tabular}
    \end{table}

    \blfootnote{Data Source: Robert Shiller, Ken French.  Monthly data annualized.}

\end{frame}

\begin{frame}[t]
  \frametitle{3. Index Returns \& Excess Returns}
  \framesubtitle{}

  PIMCO's Total Return Fund (PTTRX) is a \$40bn active bond fund that is benchmarked to the Barclay's Aggregate Index.\\
  \vspace{0.5em}
  \centering
  \includegraphics[width=0.6\textwidth]{figures/ch5_3_pttrx.png}

\end{frame}

\begin{frame}[t]
  \frametitle{3. Index Returns \& Excess Returns}
  \framesubtitle{}

  Capital Group's American Growth Fund (AGTHX) is a \$300bn active equity fund that has a growth style but is benchmarked to the S\&P 500 Index.\\
  \vspace{0.5em}
  \centering
  \includegraphics[width=0.6\textwidth]{figures/ch5_3_agthx.png}

\end{frame}

\begin{frame}[t]
  \frametitle{3. Index Returns \& Excess Returns}
  \framesubtitle{The Information Ratio}

  The \textbf{Information Ratio} is defined as
  $$\frac{E[r - r_{index}]}{\sigma_{active}}$$
  Where $E[r - r_{index}]$ is the \textit{excess} return of the portfolio, measured against an appropriate index, and $\sigma_{active}$ is the standard deviation of the excess returns.
  The standard deviation of active returns is often referred to as the \textit{tracking error} of the portfolio.\\
  \vspace{1em}
  This is very similar to the Sharpe Ratio.   The key difference is that here excess returns are calculated against an index, rather than against the risk-free rate.

\end{frame}

\begin{excelframe}[t]
  \frametitle{4. Excel Pivot Tables}
  \framesubtitle{}

    \begin{itemize}
      \item Insert: Pivot Table (I prefer separate sheet)
      \item Drag Fields to the Rows, Columns, Values
      \item If applicable, set the filter on the upper left of the table
      \item Adjust the values using "Field Settings"
    \end{itemize}

\end{excelframe}

\begin{excelframe}[t]
  \frametitle{4. Excel Pivot Tables}
  \framesubtitle{}

  \centering
  \includegraphics[width=0.8\textwidth]{figures/ch5_3_pivot.png}

\end{excelframe}

\begin{practiceframe}[t]
  \frametitle{5. Practice Problems}
  \framesubtitle{Review: HPR, EAR, APR and $r_{cc}$}

  \begin{itemize}
    \item Holding Period Return (HPR)
    $$\frac{\text{Price Change} + \text{Income}}{\text{Initial Price}}$$
    \item Effective Annual Rate (EAR)
    $$(1 + EAR)^T = 1 + HPR$$
    \item Annual Percentage Rate (APR)
    $$1+EAR = \left(1 + \frac{APR}{n}\right)^n$$
    \item Continuously Compounded Rate of Return ($r_{cc}$)
    $$1+EAR = e^{r_{cc}}$$
  \end{itemize}

\end{practiceframe}

\begin{practiceframe}[t]
  \frametitle{5. Practice Problems}
  \framesubtitle{Review: Measures of Return}

      \vspace{-1em}
    \scriptsize
    \begin{table}
    \caption{S\&P 500 Total Returns by Decade, (\% Annualized)}
    \begin{tabular}{lrrrrrr}
    \toprule
    Decade & Return & Geo Mean & CPI & Real Return & Risk Free & Excess Return \\
    \midrule
    1870s & 8.11 & 7.46 & -2.59 & 10.92 & NaN & NaN \\
    1880s & 6.29 & 5.83 & -2.38 & 8.52 & NaN & NaN \\
    1890s & 6.24 & 5.55 & 0.58 & 6.08 & NaN & NaN \\
    1900s & 10.62 & 9.82 & 2.49 & 8.30 & NaN & NaN \\
    1910s & 4.99 & 4.44 & 6.65 & -1.31 & NaN & NaN \\
    1920s & 15.46 & 14.25 & -1.11 & 16.46 & 3.66 & 13.95 \\
    1930s & 4.39 & 0.03 & -1.98 & 6.33 & 0.55 & 5.46 \\
    1940s & 9.47 & 8.58 & 5.39 & 4.30 & 0.41 & 9.92 \\
    1950s & 18.24 & 17.74 & 2.27 & 16.01 & 1.86 & 15.72 \\
    1960s & 8.05 & 7.51 & 2.55 & 5.57 & 3.81 & 4.94 \\
    1970s & 6.56 & 5.69 & 7.16 & -0.53 & 6.14 & 1.18 \\
    1980s & 16.88 & 16.04 & 4.89 & 11.87 & 8.55 & 8.56 \\
    1990s & 17.19 & 16.66 & 2.81 & 14.29 & 4.82 & 12.79 \\
    2000s & 0.38 & -0.73 & 2.50 & -2.12 & 2.72 & -1.70 \\
    2010s & 13.02 & 12.56 & 1.73 & 11.25 & 0.51 & 13.13 \\
    2020s & 13.72 & 12.11 & 4.08 & 9.55 & 2.48 & 10.72 \\
    \bottomrule
    \end{tabular}
    \end{table}

    \blfootnote{Data Source: Robert Shiller, Ken French.  Monthly data annualized.}

  \end{practiceframe}

\begin{practiceframe}[t]
  \frametitle{5. Practice Problems}
  \framesubtitle{}

  \begin{enumerate}
    \item The Fisher equation tells us that the real interest rate approximately
    equals the nominal rate minus the inflationr ate.   Suppose the inflation 
    rate increases from 3\% to 5\%.   Does the Fisher equation imply that this 
    increase will result in a fall in the real interest rate?   Explain.
    \item You have \$5,000 to invest for the next year and are considering three 
    alternatives:
      \begin{itemize}
        \item A money market fund with an average maturity of 30 days, current yield of 3\% per year.
        \item A 1-year savings deposit at a bank offering an interest rate of 4\%
        \item A 20-year US Treasury bond offering a yield to maturity of 5\% per year.
      \end{itemize}
    What role does your forecast for future interest rates play in your decision?
  \end{enumerate}

  \blfootnote{Practice Problems from Bodi et al, page 77}

\end{practiceframe}

\begin{practiceframe}[t]
  \frametitle{5. Practice Problems}
  \framesubtitle{}

  \begin{enumerate}
  \setcounter{enumi}{2}
    \item Your expectations regarding the stock price are as follows:
      \begin{table}
        \begin{tabular}{lrrr}
          \toprule
          Market & Probability & Ending Price & HPR\\
          \midrule
          Boom & 0.35 & \$140 & 44.5\%\\
          Normal & 0.3 & 110 & 14\\
          Recession & 0.35 & 0.8 & -16.5\\
          \bottomrule
        \end{tabular}
      \end{table}
    Compute the mean and standard deviation of the HPR on stocks (see equations 5.11 and 5.12 in book)
    \item The continuously compounded annual return on a stock is normally 
    distributed with mean of -20\% and standard deviation of 30\%.   With 
    95.44\% confidence, we should expect its actual return in any particular 
    year to be between what two values?
  \end{enumerate}

\end{practiceframe}

\begin{practiceframe}[t]
  \frametitle{5. Practice Problems}
  \framesubtitle{}

  \begin{enumerate}
  \setcounter{enumi}{4}
    \item A real-estate property is expected to yield 2\% per quarter (nominal) with a 
    standard deviation of the (effective) quarterly rate of 10\%.  What is the probability 
    of loss on the real estate investment after 10 years?
    \item You invest \$1 million at the beginning of 2028 in a stock-index fund. 
    If the rate of return in 2028 is -40\%, what rate of return in 2029 will be necessary 
    for your portfolio to recover to its original value.
  \end{enumerate}

\end{practiceframe}

\end{document}