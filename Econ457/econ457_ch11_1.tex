\documentclass{beamer}

\newcommand{\week}{Week 8-b}

\title{Efficient Markets}
\subtitle{Reference: Bodie et al, Ch 11 \& 12}
\author{Econ 457}
\date{\week}

% Reference the shared preamble
\setbeamertemplate{frametitle}{
  \vspace{0.5em}
  \insertframetitle
  \par
  \vspace{0.5em}
  \hrule
  \vspace{0.3em}
  {\small\color{gray}\insertframesubtitle}
}

\setbeamertemplate{navigation symbols}{}
\setbeamertemplate{itemize item}{\textbullet} % main bullet: filled dot
\setbeamertemplate{itemize subitem}{\normalsize$\circ$} % sub-bullet: empty dot
\setbeamertemplate{itemize subsubitem}{\scriptsize--} % sub-sub-bullet: dash


% Font changes
\usepackage[scaled=0.92]{helvet}
\renewcommand{\familydefault}{\sfdefault}

% Packages
\usepackage{tikz}
\usepackage{booktabs}
\usepackage{xcolor}
\usepackage{array}           % Enhanced column types for tables
\usepackage{multirow}        % Spanning multiple rows in tables
\usepackage{makecell}        % Line breaks and formatting in table cells
\usepackage{siunitx}         % Proper formatting of numbers and units
\usepackage{amsmath}         % Enhanced math environments
\usepackage{amsfonts}        % Additional math fonts
\usepackage{amssymb}         % Additional math symbols
\usepackage{url}             % Better URL formatting
\usepackage{graphicx}        % Enhanced graphics support
\usepackage{tabularray}
\UseTblrLibrary{booktabs, siunitx, varwidth}
% For financial presentations specifically
\usepackage{eurosym}         % Euro symbol
\usepackage{textcomp}        % Additional text symbols
\usepackage{hyperref}        % Hyperlinks (should be loaded last)

% Define a footnote
\renewcommand{\footnoterule}{\vspace*{-3pt}\hrule width 2in height 0.4pt\vspace*{2.6pt}}

% Define a Foundation Slide
\newenvironment{foundframe}[1][t]{
    \setbeamercolor{background canvas}{bg=gray!8}
    \setbeamercolor{frametitle}{fg=gray!80!black,bg=gray!25}
    \setbeamercolor{framesubtitle}{fg=gray!70!black,bg=gray!15}
    \setbeamercolor{item}{fg=gray!80!black}
    \setbeamercolor{enumerate item}{fg=gray!80!black}
    
    % Modify the frametitle template for this frame type
    \setbeamertemplate{frametitle}{
        \vspace{0.5em}
        \begin{minipage}[t]{0.75\textwidth}
            \insertframetitle
            \par
            \vspace{0.5em}
            \hrule
            \vspace{0.3em}
            {\small\color{gray}\insertframesubtitle}
        \end{minipage}%
        \hfill
        \begin{minipage}[t]{0.2\textwidth}
            \raggedleft
            \colorbox{gray!30}{%
                \scriptsize\bfseries\color{gray!80!black}%
                   \hspace{3pt}\begin{tabular}{c}Foundation\\Material\end{tabular}\hspace{3pt}%
            }
        \end{minipage}
        \vspace{0.3em}
    }
    
    \begin{frame}[#1]
}{
    \end{frame}
}

% Define Practice Slide
\newenvironment{practiceframe}[1][t]{
    \setbeamercolor{background canvas}{bg=white}
    \setbeamercolor{frametitle}{fg=blue!80!black,bg=blue!15}
    \setbeamercolor{framesubtitle}{fg=blue!70!black,bg=blue!10}
    \setbeamercolor{item}{fg=blue!80!black}
    \setbeamercolor{enumerate item}{fg=blue!80!black}
    \setbeamercolor{normal text}{fg=blue!90!black}
    
    % Modify the frametitle template for this frame type
    \setbeamertemplate{frametitle}{
        \vspace{0.5em}
        \begin{minipage}[t]{0.75\textwidth}
            \insertframetitle
            \par
            \vspace{0.5em}
            \hrule
            \vspace{0.3em}
            {\small\color{blue!70!black}\insertframesubtitle}
        \end{minipage}%
        \hfill
        \begin{minipage}[t]{0.2\textwidth}
            \raggedleft
            \colorbox{blue!20}{%
                \scriptsize\bfseries\color{blue!80!black}%
                   \hspace{3pt}\begin{tabular}{c}Practice\\Questions\end{tabular}\hspace{3pt}%
            }
        \end{minipage}
        \vspace{0.3em}
    }
    
    \begin{frame}[#1]
}{
    \end{frame}
}

% Define Excel Slide
\newenvironment{excelframe}[1][t]{
    \setbeamercolor{background canvas}{bg=white}
    \setbeamercolor{frametitle}{fg=blue!80!black,bg=blue!15}
    \setbeamercolor{framesubtitle}{fg=blue!70!black,bg=blue!10}
    \setbeamercolor{item}{fg=blue!80!black}
    \setbeamercolor{enumerate item}{fg=blue!80!black}
    \setbeamercolor{normal text}{fg=blue!90!black}
    
    % Modify the frametitle template for this frame type
    \setbeamertemplate{frametitle}{
        \vspace{0.5em}
        \begin{minipage}[t]{0.75\textwidth}
            \insertframetitle
            \par
            \vspace{0.5em}
            \hrule
            \vspace{0.3em}
            {\small\color{blue!70!black}\insertframesubtitle}
        \end{minipage}%
        \hfill
        \begin{minipage}[t]{0.2\textwidth}
            \raggedleft
            \colorbox{green!10}{%
                \scriptsize\bfseries\color{blue!80!black}%
                   \hspace{3pt}\begin{tabular}{c}MS Excel\end{tabular}\hspace{3pt}%
            }
        \end{minipage}
        \vspace{0.3em}
    }
    
    \begin{frame}[#1]
}{
    \end{frame}
}

% Define Caution Slide
\newenvironment{cautionframe}[1][t]{
    \setbeamercolor{background canvas}{bg=white}
    \setbeamercolor{frametitle}{fg=blue!80!black,bg=blue!15}
    \setbeamercolor{framesubtitle}{fg=blue!70!black,bg=blue!10}
    \setbeamercolor{item}{fg=blue!80!black}
    \setbeamercolor{enumerate item}{fg=blue!80!black}
    \setbeamercolor{normal text}{fg=blue!90!black}
    
    % Modify the frametitle template for this frame type
    \setbeamertemplate{frametitle}{
        \vspace{0.5em}
        \begin{minipage}[t]{0.75\textwidth}
            \insertframetitle
            \par
            \vspace{0.5em}
            \hrule
            \vspace{0.3em}
            {\small\color{blue!70!black}\insertframesubtitle}
        \end{minipage}%
        \hfill
        \begin{minipage}[t]{0.2\textwidth}
            \raggedleft
            \colorbox{red!10}{%
                \scriptsize\bfseries\color{blue!80!black}%
                   \hspace{3pt}\begin{tabular}{c}Caution\end{tabular}\hspace{3pt}%
            }
        \end{minipage}
        \vspace{0.3em}
    }
    
    \begin{frame}[#1]
}{
    \end{frame}
}

% Add to footnotes
\makeatletter
\newcommand\blfootnote[1]{%
  \begingroup
  \renewcommand\thefootnote{}%
  \renewcommand\@makefntext[1]{\raggedright\leftskip=0pt ##1}%
  \footnote{\scriptsize #1}%
  \addtocounter{footnote}{-1}%
  \endgroup
}
\makeatother

% Set the footer -- change 
\setbeamertemplate{footline}{
  \leavevmode%
  \vspace{2ex}
  \hbox{%
    % Left box: Econ 457
    \begin{beamercolorbox}[wd=.4\paperwidth,ht=2.5ex,dp=1ex,left]{author in head/foot}%
      \hspace{1em}Econ 457
    \end{beamercolorbox}%
    % Middle box: Week
    \begin{beamercolorbox}[wd=.2\paperwidth,ht=2.5ex,dp=1ex,center]{date in head/foot}%
      \centering\week
    \end{beamercolorbox}%
    % Right box: Slide numbers
    \begin{beamercolorbox}[wd=.4\paperwidth,ht=2.5ex,dp=1ex,center]{date in head/foot}%
      \hfill\insertframenumber{} 
    \end{beamercolorbox}%
  }%
  \vskip0pt%
}

\begin{document}

\frame{\titlepage}

\begin{frame}
    \frametitle{Outline}

    \begin{enumerate}
        \item Random Walks
        \item Efficient Markets hypothesis
        \item Momentum
        \item Mispricing or Risk Premium?
        \item Practice
    \end{enumerate}

\end{frame}

\begin{foundframe}[t]
    \frametitle{1. Random Walks}
    \framesubtitle{Definition and Formula}

    A \textbf{random walk} is a characterized by the following process:
    $$X_t = X_{t-1} + \epsilon_t$$
    where:
    \begin{itemize}
        \item $X_t$ = value at time $t$
        \item $X_{t-1}$ = value at previous time period
        \item $\epsilon_t$ = random error term (often called "innovation" or "shock")
    \end{itemize}
    \vspace{1em}

    \textbf{Alternative representation:}
    $$\Delta X_t = X_t - X_{t-1} = \epsilon_t$$
    
    In words, the changes in $X$ are purely random and unpredictable.

\end{foundframe}

\begin{foundframe}[t]
    \frametitle{1. Random Walks}
    \framesubtitle{Key Assumptions}

    \textbf{Assumptions about the error term $\epsilon_t$:}
    \vspace{1em}

    \begin{enumerate}
        \item \textbf{Zero mean:} $E[\epsilon_t] = 0$
        \begin{itemize}
            \item On average, there is no systematic upward or downward trend
        \end{itemize}
        \vspace{0.5em}

        \item \textbf{Constant variance:} $\text{Var}(\epsilon_t) = \sigma^2$ for all $t$
        \begin{itemize}
            \item The magnitude of random fluctuations is constant over time
        \end{itemize}
        \vspace{0.5em}

        \item \textbf{No serial correlation:} $\text{Cov}(\epsilon_t, \epsilon_s) = 0$ for $t \neq s$
        \begin{itemize}
            \item Today's shock is independent of past or future shocks
            \item No predictable patterns in the error terms
        \end{itemize}
        \vspace{0.5em}

        \item \textbf{Normal distribution:} $\epsilon_t \sim N(0, \sigma^2)$ (often assumed)
    \end{enumerate}

\end{foundframe}

\begin{foundframe}[t]
    \frametitle{1. Random Walks}
    \framesubtitle{Expected Behavior and Properties}

    \textbf{Properties of a random walk:}

    \begin{itemize}
        \item \textbf{Expected value:} $E[X_t] = X_0$ (constant over time)
        \item \textbf{Variance grows linearly:} $\text{Var}(X_t) = t \cdot \sigma^2$
        \item \textbf{Standard deviation grows as $\sqrt{t}$:} $\text{SD}(X_t) = \sqrt{t} \cdot \sigma$
        \item \textbf{Unpredictability:} Future values cannot be forecasted better than current value
        \item \textbf{Path dependence:} Current value depends on entire history of shocks
        \item \textbf{Non-stationarity:} Statistical properties change over time
        \item \textbf{Persistence:} Effects of shocks are permanent (no mean reversion)
    \end{itemize}
    \vspace{1em}

\end{foundframe}

\begin{foundframe}[t]
    \frametitle{1. Random Walks}
    \framesubtitle{Random Walk with Drift}

    A \textbf{random walk with drift} includes a constant trend component:
    $$X_t = X_{t-1} + \mu + \epsilon_t$$
    where: $\mu$ is drift parameter (constant trend), 
    and $\epsilon_t$ is random error term with $E[\epsilon_t] = 0$
    \vspace{1em}

    \textbf{Properties:}
    \begin{itemize}
        \item \textbf{Expected value:} $E[X_t] = X_0 + t \cdot \mu$ (grows linearly)
        \item \textbf{Variance:} $\text{Var}(X_t) = t \cdot \sigma^2$ (same as before)
        \item \textbf{Trend vs. randomness:} Long-run behavior dominated by drift $\mu$
    \end{itemize}
    \vspace{1em}

\end{foundframe}

\begin{frame}[t]
    \frametitle{2. Efficient Markets Hypothesis}
    \framesubtitle{Car Auctions}

    In an auction, the winner will be the bidder with the highest valuation,
     and the winning price will be set by the second highest valuation + a little bit.
    \begin{itemize}
        \item A used car is being auctioned.   You value the car at \$18k, your friend 
        values it at \$20k.
        \item Bidding starts at \$15k and moves in \$1k increments.  
        \item You both bid at 15, 16, 17, and 18.
        \item At \$19 you drop out, your friend wins the car for \$19 (less than he values it, btw)
    \end{itemize}
    
    It's tempting to think the analogy for markets is that the investors with the highest valuation will 
    always buy the stock.   Tempting, but wrong...

\end{frame}

\begin{frame}[t]
    \frametitle{2. Efficient Markets Hypothesis}
    \framesubtitle{Why markets are not like car auctions}

    If markets are complete, traders can short securities.   A security that is over-valued presents 
    a valuable opportunity.   The amount that an investor would be willing to risk on this opportunity 
    is a function of the investor's certainty and the size of the valuation gap.\\
    \vspace{1em}
    In the context of markets, therefore, prices are set by the investor that values the \textit{trade} 
    the highest, and the trade could either be a long or a short trade.\\
    \vspace{1em}
    If the price was always set by the investor with the highest valuation, prices would consistently be 
    set too high (there are a bunch of overly optimistic investors out there).   But, this would be an 
    opportunity for somebody to make money by being short, and therefore can't be an equilibrium.

\end{frame}

\begin{frame}[t]
    \frametitle{2. Efficient Markets Hypothesis}
    \framesubtitle{}

    The \textbf{Efficient Markets Hypothesis (EMH)} states that
    all available information that could be used to predict 
    stock performance should already be reflected in current stock prices.\\
    \vspace{1em}   
    If true, then future price changes should be 
    unpredictable and stock prices should follow a random walk.\\
    \vspace{1em}
    \begin{itemize}
        \item Prices should adjust (automatically) to fair value.  
        \item Fair value reflects both the cost of money 
        (risk-free rate) as well as a risk premium.   
        \item These factors can introduce \textbf{drift} 
        to the series of stock prices.
        \item Prices can and should adjust to \textbf{new} information (earnings, 
        economic data, etc).
    \end{itemize}
    Put more simply: "Prices are right" and "there is no free lunch" (Barberis and Thaler, 2003)

\end{frame}

\begin{frame}[t]
    \frametitle{2. Efficient Markets Hypothesis}
    \framesubtitle{Three Versions}

    There are three versions of the Efficient Markets Hypothesis
    \begin{enumerate}
        \item \textit{Weak Form:} Stock prices already reflect all information 
        available from history of past prices, trading volume, and short interest.
        \item \textit{Semistrong Form:} All publicly available information 
        regarding the prospects of the firm is already incorporated in stock prices.
        \item \textit{Strong Form:} Stock prices reflect all relevant information, 
        even if only available to insiders.
    \end{enumerate}

\end{frame}


\begin{frame}[t]
    \frametitle{2. Efficient Markets Hypothesis}
    \framesubtitle{Implications for Active Management}

    Implcations:
    \begin{enumerate}
    \item Outperforming the competition is only possible if an investor 
    has better information that isn't already reflected in prices.
    Generating better information is costly. 
    The costs are likely to erode some of the excess performance. 
    Trading costs and fees are likely to further reduce net returns.
    \item The change in market prices is unpredictable.   
    It should not be possible to predict price changes using past information. 
    If price changes are unpredictable, the level of prices should behave like a random walk 
    (potentially a random walk with drift)
    \end{enumerate}

\end{frame}

\begin{frame}[t]
    \frametitle{2. Efficient Markets Hypothesis}
    \framesubtitle{Implications for Active Management}

    \begin{table}
            \caption{Percentage of Actively Managed Mutual Funds}
        \begin{tabular}{lcc}
            \toprule
             & Underperformed S\&P 500 & Outperformed S\&P 500\\
            \midrule
            1 year & 73 & 27\\
            3 year & 65 & 35 \\
            5 year & 87 & 13 \\
            10 year & 86 & 14 \\
            15 year & 88 & 12 \\
            \bottomrule
        \end{tabular}
    \end{table}

    \blfootnote{Source: S\&P Global, 2025}

\end{frame}

\begin{frame}[t]
    \frametitle{2. Efficient Markets Hypothesis}
    \framesubtitle{Robert Shiller Chart}

    \centering
    \includegraphics[width=0.8\textwidth]{figures/ch11_1_shiller.png}

\end{frame}

\begin{frame}[t]
    \frametitle{2. Efficient Markets Hypothesis}
    \framesubtitle{Robert Shiller Chart}

    \centering
    \includegraphics[width=0.8\textwidth]{figures/ch11_1_shiller_jb.png}

\end{frame}

\begin{frame}[t]
    \frametitle{2. Efficient Markets Hypothesis}
    \framesubtitle{Shiller's Defintion of Bubbles}

    In his book \textit{Irrational Exhuberance} Rober Shiller defined a bubble as
    follows:
    
    \vspace{1em}
    
    \begin{quote}
        I define a speculative bubble as a situation in which news of price 
    increases spurs investors enthusiasm, which spreads by pychological contagion 
    from person to person, in the process amplifying stories that might justify 
    the price increases and bringing in a larger and larger class of investors, who, 
    despite doubts about the real value of an investment, are drawn to it partly through 
    envy of others' successes and partly through a gambler's excitement.
    \end{quote}

    \vspace{1em}

    Note all of the words from pyschology: ethusiasm, pychological contagion, envy, and 
    gambler's excitement.\\

\end{frame}

\begin{frame}[t]
    \frametitle{2. Efficient Markets Hypothesis}
    \framesubtitle{Behavioral Economics}

    There are many findings in behavioral finance that may be relevant for investors,
     including
    \begin{itemize}
        \item Overconfidence
        \item Belief Perserverance
        \item Optimism and Wishful Thinking
        \item Representativeness
        \item Conservatism
        \item Endowment Effects
        \item Anchoring
        \item Availability Bias
    \end{itemize}

\end{frame}

\begin{frame}[t]
    \frametitle{2. Efficient Markets Hypothesis}
    \framesubtitle{Behavioral Violations}

    Big idea: investors are systematically irrational \textbf{and} there are limits to 
    arbitrage.   As a consequence, markets regularly misprice securities.  
    Examples include:
    \begin{itemize}
        \item Closed end funds (discounts are correlated across funds, correlated with other 
        asset prices)
        \item Equity carve-outs (pieces don't sum to the whole)
        \item Stocks tend to go up in January and on Mondays
        \item Stocks tend to 'drift' following earnings surprises (prices don't immediately 
       adjust)
        \item Stocks names seem to matter (example: a fund named CUBA held no Cuban assets, but 
        went up when Obama announced a detante with Cuba)
    \end{itemize}

    While these all seem like violations of the efficient market hypothesis, 
    it's not clear how significant or actionable they are.

\end{frame}

\begin{frame}[t]
    \frametitle{2. Efficient Markets Hypothesis}
    \framesubtitle{Samuelson Dictum and Black's Definition of Efficiency}

    Fisher Black said EMH seemed reasonable if we define "efficient" as follows: 
    individual company stock prices are between half true value and twice 
    true value at least 90\% of the time.\\
    \vspace{1em}
    Paul Samuelson reportedly said that markets are "micro efficient" and 
    "macro inefficient".\\ 

    \blfootnote{See Shiller \textit{Finance and the Good Society}, chapter 26 }

\end{frame}

\begin{frame}[t]
    \frametitle{3. Momentum}
    \framesubtitle{Definition}

    Momentum refers to the tendency of assets that have risen in price to continue rising, 
    or, conversely, for assets that have fallen in price to keep falling.\\
    \vspace{1em}
    Constructing a momentum factor is straightforward: 
    the difference between the returns on a portfolio of assets that have recently risen 
    minus the returns on a portfolio of assets that have recently fallen.

\end{frame}

\begin{frame}[t]
    \frametitle{3. Momentum}
    \framesubtitle{Momentum Performance}

    \centering
    \includegraphics[width=0.8\textwidth]{figures/ch10_ff_momentum.png}

    \vfill
    \raggedright
    \footnotesize
    Data Source: Ken French data library

\end{frame}

\begin{frame}[t]
    \frametitle{3. Momentum}
    \framesubtitle{Momentum Performance}

    \centering
    \includegraphics[width=0.8\textwidth]{figures/ch10_ff_momentum_logs.png}

    \vfill
    \raggedright
    \footnotesize
    Data Source: Ken French data library

\end{frame}

\begin{frame}[t]
    \frametitle{3. Momentum}
    \framesubtitle{AQR's Managed Futures Strategy Fund (AQMIX)}

    \centering
    \includegraphics[width=0.6\textwidth]{figures/ch10_aqmix.png}

    \vfill
    \raggedright
    \footnotesize
    Source: AQR website

\end{frame}

\begin{frame}[t]
    \frametitle{3. Momentum}
    \framesubtitle{AQR's Managed Futures Strategy Fund (AQMIX)}

    \centering
    \includegraphics[width=0.6\textwidth]{figures/ch10_aqmix_2.png}

    \vfill
    \raggedright
    \footnotesize
    Source: AQR website

\end{frame}

\begin{frame}[t]
    \frametitle{3. Momentum}
    \framesubtitle{AQR's Managed Futures Strategy Fund (AQMIX)}

    \centering
    \includegraphics[width=0.6\textwidth]{figures/ch10_aqmix_3.png}

    \vfill
    \raggedright
    \footnotesize
    Source: AQR website

\end{frame}

\begin{frame}[t]
    \frametitle{3. Momentum}
    \framesubtitle{AQR's Managed Futures Strategy Fund (AQMIX)}

    \centering
    \includegraphics[width=0.8\textwidth]{figures/ch10_aqmix_perf.png}

    \vfill
    \raggedright
    \footnotesize
    Source: Morningstar website

\end{frame}

\begin{frame}[t]
    \frametitle{4. Mispricing or Risk Premium?}
    \framesubtitle{Outperformance of VALUE strategies}

    Advocates for behavioral finance and efficient markets both agree that 
    value stocks have tended to outperform over time.   They disagree, however, 
    on the reason for the outperformance and the implications.\\
    \vspace{1em}
    \begin{itemize}
        \item \textit{Behavioral:} Investors \textbf{overreact} to new information.  
        This causes them to over-sell on bad news, and the subsequent price recovery 
        leads to outperformance.   
        \item \textit{Efficient Markets:} The performance of value stocks is highly 
        correlated with systemic risk.   These are riskier companies that could and will 
        fail if there were to be a systemic crisis.   Investor deserve a risk premium for 
        bearing exposure to such companies.
    \end{itemize} 

\end{frame}

\begin{frame}[t]
    \frametitle{4. Mispricing or Risk Premium?}
    \framesubtitle{Outperformance of MOMENTUM strategies}

    Similarly, advocates for behavioral finance and efficient markets both agree that 
    momentum strategies have tended to outperform over time.   Again, they disagree, however, 
    on the reason for the outperformance and the implications.\\
    \vspace{1em}
    \begin{itemize}
        \item \textit{Behavioral:} Investors \textbf{underreact} to new information.  
        They have biases that lead to slow updating.   This causes them to price in 
        information slowly.   
        \item \textit{Efficient Markets:} Investors are often hurt when trends break. 
        By definition, momentum strategies will underporm during these periods.  
        Investors deserve a risk premium for bearing this risk.
    \end{itemize} 

\end{frame}

\begin{frame}[t]
    \frametitle{4. Mispricing or Risk Premium?}
    \framesubtitle{Recent Performance of Factors}

    \raggedright
    It's also possible that the factor outperformance won't be repeated.  
    In fact, recently it looks like both factors, momentum and value, have been less 
    effective.\\

    \centering
    \includegraphics[width=0.7\textwidth]{figures/ch10_ff_since2000.png}

    \blfootnote{Data Source: Ken French data library}

\end{frame}

\begin{practiceframe}[t]
    \frametitle{5. Practice}
    \framesubtitle{}

        \begin{enumerate}
            \item Consider the following data for a one-factor economy.   Both portfolios are 
            well diversified.
            \begin{table}
                \begin{tabular}{ccc}
                    \toprule
                    Portfolio & E[r] & Beta\\
                    \midrule
                    A & 12\% & 1.2\\
                    B & 6\% & 0\\
                    \bottomrule
                \end{tabular}
            \end{table}
            Suppose another portfolio, portfolio E, is well diversified with a beta of 0.6 and 
            an expected return of 8\%.  Would an arbitrage opportunity exist?   If so, what would be the 
            arbitrage?
        \end{enumerate}

\end{practiceframe}

\begin{practiceframe}[t]
    \frametitle{5. Practice}
    \framesubtitle{}

        \begin{enumerate}[2]
        \item Suppose that the market can be described by the following three sources of risk, each of which 
        has an associated risk-premium: 
          \begin{table}
            \begin{tabular}{lc}
                \toprule
                Factor & Risk Premium\\
                \midrule
                Industrial Production ($I$) & 6\%\\
                Interest Rates ($R$) & 2\\
                Consumer Confidence ($C$) & 4\\
                \bottomrule
            \end{tabular}
        \end{table}
         You estimate the return on the stock is given by
         $$r = 15 + 1.0 I + 0.5 R + 0.75 C + \epsilon$$
        Where $I$, $R$, and $C$ are surprises in these economic series. 
        Assuming the T-Bill rate is 6\%, find the fair rate of return on this stock
         using the Arbitrage Pricing Theory.   Is the stock over or under valued?
    \end{enumerate}

\end{practiceframe}


\begin{practiceframe}[t]
    \frametitle{2. Practice}
    \framesubtitle{}

        \begin{enumerate}
            \setcounter{enumi}{2} 
            \item A successful firm like Microsoft has consistently generated large 
            profits for years.   Is this a violation of EMH?
            \item "If all securities are fairly priced, all must offer equal expected 
            rates of return."   True or False?   Explain why or why not?
            \item Which of the following would be a viable way to earn \textit{abnormally high} 
            trading profits if markets are efficient:
            \begin{itemize}
                \item Buy shares in companies with low P/E ratios
                \item Buy shares in companies with recent above-average price changes
                \item Buy shares in companies with recent below-average price changes
                \item Buy shares in companies for which you have advanced knowledge of positive changes in the 
                management teams
            \end{itemize}
        \end{enumerate}

\end{practiceframe}

\begin{practiceframe}[t]
    \frametitle{5. Practice}
    \framesubtitle{}

        \begin{enumerate}
            \setcounter{enumi}{5} 
            \item If on any day prices are as likely to increase as they are to decrease, 
            why do investors earn positive returns from the market on average?
            \item A random walk occurs when:
            \begin{itemize}
                \item Stock prices are random but predictable
                \item Stock prices respond slowly to both old and new information
                \item Future price changes are uncorrelated with past price changes
                \item Past information is useful in predicting future prices
            \end{itemize}
            \item Provide a reason why, over an extended period of time, value-stock investing 
            might outperform growth stock investing?   Is this a violation of EMH?
            \item Provide a reason why, over an extended period of time, momentum investing 
            might outperform the broad market?   Is this a violation of EMH?
        \end{enumerate}

\end{practiceframe}

\end{document}