\documentclass{beamer}

\newcommand{\week}{Week 10-a}

\title{Fixed Income, continued}
\subtitle{Reference: Bodie et al, Ch 15}
\author{Econ 457}
\date{\week}

% Reference the shared preamble
\setbeamertemplate{frametitle}{
  \vspace{0.5em}
  \insertframetitle
  \par
  \vspace{0.5em}
  \hrule
  \vspace{0.3em}
  {\small\color{gray}\insertframesubtitle}
}

\setbeamertemplate{navigation symbols}{}
\setbeamertemplate{itemize item}{\textbullet} % main bullet: filled dot
\setbeamertemplate{itemize subitem}{\normalsize$\circ$} % sub-bullet: empty dot
\setbeamertemplate{itemize subsubitem}{\scriptsize--} % sub-sub-bullet: dash


% Font changes
\usepackage[scaled=0.92]{helvet}
\renewcommand{\familydefault}{\sfdefault}

% Packages
\usepackage{tikz}
\usepackage{booktabs}
\usepackage{xcolor}
\usepackage{array}           % Enhanced column types for tables
\usepackage{multirow}        % Spanning multiple rows in tables
\usepackage{makecell}        % Line breaks and formatting in table cells
\usepackage{siunitx}         % Proper formatting of numbers and units
\usepackage{amsmath}         % Enhanced math environments
\usepackage{amsfonts}        % Additional math fonts
\usepackage{amssymb}         % Additional math symbols
\usepackage{url}             % Better URL formatting
\usepackage{graphicx}        % Enhanced graphics support
\usepackage{tabularray}
\UseTblrLibrary{booktabs, siunitx, varwidth}
% For financial presentations specifically
\usepackage{eurosym}         % Euro symbol
\usepackage{textcomp}        % Additional text symbols
\usepackage{hyperref}        % Hyperlinks (should be loaded last)

% Define a footnote
\renewcommand{\footnoterule}{\vspace*{-3pt}\hrule width 2in height 0.4pt\vspace*{2.6pt}}

% Define a Foundation Slide
\newenvironment{foundframe}[1][t]{
    \setbeamercolor{background canvas}{bg=gray!8}
    \setbeamercolor{frametitle}{fg=gray!80!black,bg=gray!25}
    \setbeamercolor{framesubtitle}{fg=gray!70!black,bg=gray!15}
    \setbeamercolor{item}{fg=gray!80!black}
    \setbeamercolor{enumerate item}{fg=gray!80!black}
    
    % Modify the frametitle template for this frame type
    \setbeamertemplate{frametitle}{
        \vspace{0.5em}
        \begin{minipage}[t]{0.75\textwidth}
            \insertframetitle
            \par
            \vspace{0.5em}
            \hrule
            \vspace{0.3em}
            {\small\color{gray}\insertframesubtitle}
        \end{minipage}%
        \hfill
        \begin{minipage}[t]{0.2\textwidth}
            \raggedleft
            \colorbox{gray!30}{%
                \scriptsize\bfseries\color{gray!80!black}%
                   \hspace{3pt}\begin{tabular}{c}Foundation\\Material\end{tabular}\hspace{3pt}%
            }
        \end{minipage}
        \vspace{0.3em}
    }
    
    \begin{frame}[#1]
}{
    \end{frame}
}

% Define Practice Slide
\newenvironment{practiceframe}[1][t]{
    \setbeamercolor{background canvas}{bg=white}
    \setbeamercolor{frametitle}{fg=blue!80!black,bg=blue!15}
    \setbeamercolor{framesubtitle}{fg=blue!70!black,bg=blue!10}
    \setbeamercolor{item}{fg=blue!80!black}
    \setbeamercolor{enumerate item}{fg=blue!80!black}
    \setbeamercolor{normal text}{fg=blue!90!black}
    
    % Modify the frametitle template for this frame type
    \setbeamertemplate{frametitle}{
        \vspace{0.5em}
        \begin{minipage}[t]{0.75\textwidth}
            \insertframetitle
            \par
            \vspace{0.5em}
            \hrule
            \vspace{0.3em}
            {\small\color{blue!70!black}\insertframesubtitle}
        \end{minipage}%
        \hfill
        \begin{minipage}[t]{0.2\textwidth}
            \raggedleft
            \colorbox{blue!20}{%
                \scriptsize\bfseries\color{blue!80!black}%
                   \hspace{3pt}\begin{tabular}{c}Practice\\Questions\end{tabular}\hspace{3pt}%
            }
        \end{minipage}
        \vspace{0.3em}
    }
    
    \begin{frame}[#1]
}{
    \end{frame}
}

% Define Excel Slide
\newenvironment{excelframe}[1][t]{
    \setbeamercolor{background canvas}{bg=white}
    \setbeamercolor{frametitle}{fg=blue!80!black,bg=blue!15}
    \setbeamercolor{framesubtitle}{fg=blue!70!black,bg=blue!10}
    \setbeamercolor{item}{fg=blue!80!black}
    \setbeamercolor{enumerate item}{fg=blue!80!black}
    \setbeamercolor{normal text}{fg=blue!90!black}
    
    % Modify the frametitle template for this frame type
    \setbeamertemplate{frametitle}{
        \vspace{0.5em}
        \begin{minipage}[t]{0.75\textwidth}
            \insertframetitle
            \par
            \vspace{0.5em}
            \hrule
            \vspace{0.3em}
            {\small\color{blue!70!black}\insertframesubtitle}
        \end{minipage}%
        \hfill
        \begin{minipage}[t]{0.2\textwidth}
            \raggedleft
            \colorbox{green!10}{%
                \scriptsize\bfseries\color{blue!80!black}%
                   \hspace{3pt}\begin{tabular}{c}MS Excel\end{tabular}\hspace{3pt}%
            }
        \end{minipage}
        \vspace{0.3em}
    }
    
    \begin{frame}[#1]
}{
    \end{frame}
}

% Define Caution Slide
\newenvironment{cautionframe}[1][t]{
    \setbeamercolor{background canvas}{bg=white}
    \setbeamercolor{frametitle}{fg=blue!80!black,bg=blue!15}
    \setbeamercolor{framesubtitle}{fg=blue!70!black,bg=blue!10}
    \setbeamercolor{item}{fg=blue!80!black}
    \setbeamercolor{enumerate item}{fg=blue!80!black}
    \setbeamercolor{normal text}{fg=blue!90!black}
    
    % Modify the frametitle template for this frame type
    \setbeamertemplate{frametitle}{
        \vspace{0.5em}
        \begin{minipage}[t]{0.75\textwidth}
            \insertframetitle
            \par
            \vspace{0.5em}
            \hrule
            \vspace{0.3em}
            {\small\color{blue!70!black}\insertframesubtitle}
        \end{minipage}%
        \hfill
        \begin{minipage}[t]{0.2\textwidth}
            \raggedleft
            \colorbox{red!10}{%
                \scriptsize\bfseries\color{blue!80!black}%
                   \hspace{3pt}\begin{tabular}{c}Caution\end{tabular}\hspace{3pt}%
            }
        \end{minipage}
        \vspace{0.3em}
    }
    
    \begin{frame}[#1]
}{
    \end{frame}
}

% Add to footnotes
\makeatletter
\newcommand\blfootnote[1]{%
  \begingroup
  \renewcommand\thefootnote{}%
  \renewcommand\@makefntext[1]{\raggedright\leftskip=0pt ##1}%
  \footnote{\scriptsize #1}%
  \addtocounter{footnote}{-1}%
  \endgroup
}
\makeatother

% Set the footer -- change 
\setbeamertemplate{footline}{
  \leavevmode%
  \vspace{2ex}
  \hbox{%
    % Left box: Econ 457
    \begin{beamercolorbox}[wd=.4\paperwidth,ht=2.5ex,dp=1ex,left]{author in head/foot}%
      \hspace{1em}Econ 457
    \end{beamercolorbox}%
    % Middle box: Week
    \begin{beamercolorbox}[wd=.2\paperwidth,ht=2.5ex,dp=1ex,center]{date in head/foot}%
      \centering\week
    \end{beamercolorbox}%
    % Right box: Slide numbers
    \begin{beamercolorbox}[wd=.4\paperwidth,ht=2.5ex,dp=1ex,center]{date in head/foot}%
      \hfill\insertframenumber{} 
    \end{beamercolorbox}%
  }%
  \vskip0pt%
}

\begin{document}

\frame{\titlepage}

\begin{frame}
    \frametitle{Outline}

    \begin{enumerate}
        \item Monetary Policy
        \item Factors Affecting Bond Yields
        \item Treasury Yield Curve
        \item Spot Rates and Forward Rates
        \item Explanations of Yield Curve Slope
        \item Yield Curve as a Recession Indicator?
    \end{enumerate}

\end{frame}

\begin{foundframe}[t]
    \frametitle{1. Monetary Policy}
    \framesubtitle{Federal Reserve Dual Mandate}

    \textbf{Dual Mandate:}
    \begin{itemize}
        \item Maximum employment
        \item Stable prices
    \end{itemize}
    \vspace{1em}

    \textbf{From the Federal Reserve Act, Sec 2A (as ammended in 1977)}
    \begin{quote}
        "The Board of Governors of the Federal Reserve System and the Federal Open Market Committee shall maintain long run growth of the monetary and credit aggregates commensurate with the economy's long run potential to increase production, so as to pro- mote effectively the goals of maximum employment, stable prices, and moderate long-term interest rates."
    \end{quote}
    \vspace{1em}

\end{foundframe}

\begin{foundframe}[t]
    \frametitle{1. Monetary Policy}
    \framesubtitle{The Phillips Curve}

    \textbf{Empirical Phillips Curve:}
    $$\pi_t = \beta_0 \cdot \pi_t^e + \beta_1 \cdot \pi_{t-1} +\beta_2 \cdot \pi_{t-2} + \beta_3 \cdot (SLACK_t) + \beta_4 \cdot (RPIM_t) + \varepsilon_t$$

    \begin{itemize}
        \item $\pi_t$ = inflation rate at time $t$
        \item $\pi_t^e$ = expected inflation
        \item $SLACK_t$ = slack in labor market
        \item $RPIM_t$ = relative price of imported goods
        \item $\varepsilon_t$ = error term
    \end{itemize}
    \vspace{1em}

    \textbf{Observations:}
    \begin{itemize}
        \item Less slack in the labor market leads to higher inflation
        \item Expectations matter 
        \item Supply shocks can cause inflation independent of labor market conditions
    \end{itemize}
    \vspace{1em}

\end{foundframe}

\begin{foundframe}[t]
    \frametitle{1. Monetary Policy}
    \framesubtitle{The Taylor Rule}

    \textbf{Taylor Rule (1993):}
    $$i_t = r^* + \pi_t + 0.5(\pi_t - \pi^*) - 0.5(u_t - u^*)$$

    \begin{itemize}
        \item $i_t$ = nominal federal funds rate
        \item $r^*$ = equilibrium real interest rate ($\approx$ 2\%)
        \item $\pi_t$ = current inflation rate
        \item $\pi^*$ = target inflation rate (2\%)
        \item $u_t$ = unenemployment rate 
        \item $u^*$ = natural rate of unemployment 
    \end{itemize}
    \vspace{1em}

    \textbf{Policy Implications:}
    \begin{itemize}
        \item Raise rates when inflation exceeds target
        \item Raise rates when economy operates above potential
        \item Both inflation and employment gaps matter for policy
    \end{itemize}

\end{foundframe}

\begin{frame}[t]
    \frametitle{2. Factors Affecting Bond Yields}
    \framesubtitle{}

    Factors Affecting Bond Yields:
    \begin{itemize}
        \item \textit{Growth / Jobs}: More (less) growth / job leads to higher (lower) yields
        \item \textit{Inflation}: More (less) inflation leads to higher (lower) yields
        \item \textit{Federal Reserve}: Hawkish (dovish) Fed announcement leads to higher (lower) yields
        \item \textit{Risk prices}: Higher (lower) risk prices leads to higher (lower) yields
    \end{itemize}

    Needs to emphasized that data is judged relative to expectations.   Many of these factors interact with eachother.  
    For example, higher inflation may cause the Fed to be more hawkish.

\end{frame}

\begin{frame}[t]
    \frametitle{2. Factors Affecting Bond Yields}
    \framesubtitle{CPI and Yields in 2022}

    \centering
    \includegraphics[width=\textwidth]{figures/ch15_1_cpi2022.png}

\end{frame}

\begin{frame}[t]
    \frametitle{2. Factors Affecting Bond Yields}
    \framesubtitle{CPI and Yields in 2022}

    \centering
    \includegraphics[width=0.8\textwidth]{figures/ch15_1_cpi10s.png}


\end{frame}

\begin{frame}[t]
    \frametitle{3. Treasury Yield Curve}
    \framesubtitle{}

    \centering
    \includegraphics[width=0.8\textwidth]{figures/ch15_1_yc_2025-10-23.png}

\end{frame}


\begin{frame}[t]
    \frametitle{4. Spot Rate and Forward Rates}
    \framesubtitle{Definitions}

    \textit{Spot rate}: The yield to maturity on zero-coupon bonds.\\
    \vspace{1em}

    \textit{Short rate}: The interest rate for a specifc interval (say, 1 year) available at different points in time.  
    For example, the short rate today is 4.25\% and the expected short rate for next year is 3.5\%.\\
    \vspace{1em}

    \textit{Forward rate}: The breakeven interest rate that would equate the return of a long term 
    bond with that of a strategy of rolling over short term bonds.
    \vspace{1em}

    Flipping the last definition around, the \textit{spot rate} is the 
    geometric average of the forward rates over the relevant period. 
    We don't observe future short rates, or even expectations of future short rates.   We only 
    observe forward rates, and the forward rate may differ from the expected short rate by a risk premium.

\end{frame}

\begin{frame}[t]
    \frametitle{4. Spot Rate and Forward Rates}
    \framesubtitle{Definitions}

    Reminder: yield-to-maturity is the \textit{single} interest rate that equates the 
    present discounted value of future cash flows with today's price.
    \begin{table}[h]
    \centering
    \caption{\textbf{Price and Yield-to-Maturity on zero-coupon bonds}}
    \begin{tabular}{|c|c|c|c|}
    \hline
    \textbf{Maturity} & \textbf{Yield to} & \textbf{Price} & \textbf{Formula} \\
    \textbf{(years)} & \textbf{Maturity (\%)} & & \\
    \hline
    1 & 5\% & \$952.38 & \$1000/1.05 \\
    \hline
    2 & 6\% & \$890.00 & \$1000/1.06$^2$ \\
    \hline
    3 & 7\% & \$816.30 & \$1000/1.07$^3$ \\
    \hline
    4 & 8\% & \$735.03 & \$1000/1.08$^4$ \\
    \hline
    \end{tabular}
    \end{table}

\end{frame}

\begin{frame}[t]
    \frametitle{4. Spot Rates and Forward Rates}
    \framesubtitle{Calculations}

    Using $y_n$ for the spot rate of a bond of maturity $n$ and $r_n$ for the short rate ending at time $n$, we have the following:

    The spot yield curve is the geometric average of the short rates over the relevant period:
    $$
    (1+y_n)^n = \prod_{i=1}^n(1+r_n)
    $$
    You can use the forward short rate to calculate the next spot rate:
    $$
    (1+y_n)^n = (1+y_{n-1})^{n-1} \times (1 + r_n)
    $$
    You can use two adjacent spot rates to calculate the forward short rate:
    $$
    (1+r_n) = \frac{(1+y_n)^n}{(1+y_{n-1})^{n-1}}
    $$

\end{frame}

\begin{frame}[t]
    \frametitle{4. Spot Rates and Forward Rates}
    \framesubtitle{Zero-Coupon Bond Example with Forward Rates}

    \begin{table}[h]
        \centering
        \caption{\textbf{Zero-Coupon Bond Example}}
        \resizebox{\textwidth}{!}{
        \begin{tabular}{|c|c|c|c|c|c|}
        \hline
        \textbf{Mat.} & \textbf{YTM} & \textbf{Fwd} & \textbf{Price} & \textbf{Formula} & \textbf{Fwd Calc} \\
        \textbf{(yrs)} & \textbf{(\%)} & \textbf{(\%)} & & & \\
        \hline
        1 & 5\% & 5.00\% & \$952.38 & \$1000/1.05 & $r_1 = 5\%$ \\
        \hline
        2 & 6\% & 7.01\% & \$890.00 & \$1000/1.06$^2$ & $(1.06)^2/1.05 - 1$ \\
        \hline
        3 & 7\% & 9.02\% & \$816.30 & \$1000/1.07$^3$ & $(1.07)^3/(1.06)^2 - 1$ \\
        \hline
        4 & 8\% & 11.06\% & \$735.03 & \$1000/1.08$^4$ & $(1.08)^4/(1.07)^3 - 1$ \\
        \hline
        \end{tabular}
        }
    \end{table}

\end{frame}

\begin{frame}[t]
    \frametitle{4. Spot Rate and Forward Rates}
    \framesubtitle{}

    \centering
    \includegraphics[width=0.9\textwidth]{figures/ch15_1_spfw.png}

\end{frame}

\begin{frame}[t]
    \frametitle{5. Explanations of Yield Curve Slope}
    \framesubtitle{Summary}

    Possible explanations:
    \begin{enumerate}
        \item Expectations for Federal Reserve Policy
        \item Liquidity Preference Theory / Risk premiums
        \item Market Segmentation / Preferred Habitat
        \item Fed purchases / Quantitative Easing
        \item Other?
    \end{enumerate}
    \vspace{1em}
    Note these are not mutually exclusive.   Each may be valid at different times, different parts of the yield curve.

\end{frame}

\begin{frame}[t]
    \frametitle{5. Explanations of Yield Curve Slope}
    \framesubtitle{Expectations Hypothesis - Fed Dot Plot}

    \centering
    \includegraphics[width=0.8\textwidth]{figures/ch15_dotplot_Sep25.png}

\end{frame}

\begin{frame}[t]
    \frametitle{5. Explanations of Yield Curve Slope}
    \framesubtitle{Expectations Hypothesis - Fed Dot Plot}

    \footnotesize
    \begin{table}[h]
    \centering
    \caption{\textbf{Dot Plot: Forward Rates and Spot Rates}}
    \begin{tabular}{|cccc|}
    \hline
    \textbf{Period} & \textbf{Forward Short Rate} & \textbf{Spot Rate} & \textbf{Market Spot Rate} \\
    \textbf{Ending} & \textbf{From Dot Plot} & \textbf{Implied by Dot Plot} & \textbf{as of Oct 2025} \\
    \hline
    Dec 2025 & 3.6\% & 3.6\% & \\
    Dec 2026 & 3.4\% & 3.5 & 3.6 \\
    Dec 2027 & 3.1\% & 3.4 & 3.46 \\
    Dec 2028 & 3.1\% & 3.3 & 3.5 \\
    \hline
    \end{tabular}
    \end{table}
    \vspace{1em}

\end{frame}

\begin{frame}[t]
    \frametitle{5. Explanations of Yield Curve Slope}
    \framesubtitle{The 20-Year Treasury Bond}

    \textbf{History:}
    \begin{itemize}
        \item \textbf{First issued:} 1993-2004, then discontinued
        \item \textbf{Reintroduced:} May 2020 during COVID-19 pandemic
        \item \textbf{Reason for restart:} Increased government borrowing needs
    \end{itemize}
    \vspace{1em}

    \textbf{Why the 20-year maturity?}
    \begin{itemize}
        \item Fills gap between 10-year and 30-year bonds
        \item Attracts pension funds and insurance companies
        \item Provides duration matching for long-term liabilities
    \end{itemize}
    \vspace{1em}

    Currently the Treasury issues \$16 billion monthly, which is smaller than the amount 
    of 30-year bond issuance (\$25 billion monthly).

\end{frame}

\begin{frame}[t]
    \frametitle{5. Explanations of Yield Curve Slope}
    \framesubtitle{Risk Premiums}

    As we saw in CAPM, the risk premium on an asset is 
    affected by the correlation of the asset returns with 
    the stock market.  Differences across bonds in correlations can 
    lead to differences in risk premiums.\\
    \vspace{1em}
    As we'll discuss in the next lecture, \textit{convexity} in 
    asset returns can also be valuable to investors.

\end{frame}

\begin{frame}[t]
    \frametitle{5. Explanations of Yield Curve Slope}
    \framesubtitle{Risk Premiums}

    Longer term Treasury bonds have more volatility.   Investors generally 
    require compensation for volatility.  Longer term bonds may therefore 
    have higher yields.\\

    \begin{table}
    \caption{Standard Deviation of Treasury Futures Returns}
    \begin{tabular}{lr}
    \toprule
    Future & Annualized Volatility of Returns (\%) \\
    \midrule
    2y Treasury Future & 2.30 \\
    5y Treasury Future & 4.68 \\
    10y Treasury Future & 6.67 \\
    30y Treasury Future & 11.71 \\
    \bottomrule
    \end{tabular}
    \end{table}

    \blfootnote{Data from CBOT, Interactive Brokers, 2023-2025.}

\end{frame}

\begin{frame}[t]
    \frametitle{5. Explanations of Yield Curve Slope}
    \framesubtitle{Risk Premiums}

    Investors also value convexity (to be discussed more next week). 
    Longer term bonds are often more convex.   This can lower yields.\\

    \centering
    \includegraphics[width=0.75\textwidth]{figures/ch15_1_conv.png}

\end{frame}

\begin{frame}[t]
    \frametitle{5. Explanations of Yield Curve Slope}
    \framesubtitle{Market Segmentation / Preferred Habitat}

    \centering
    \includegraphics[width=0.8\textwidth]{figures/ch15_1_20s30s_1.png}

\end{frame}

\begin{frame}[t]
    \frametitle{5. Explanations of Yield Curve Slope}
    \framesubtitle{Market Segmentation / Preferred Habitat}

    \centering
    \includegraphics[width=0.8\textwidth]{figures/ch15_1_20s30s_2.png}

\end{frame}

\begin{frame}[t]
    \frametitle{5. Explanations of Yield Curve Slope}
    \framesubtitle{Quantitative Easing and Operation Twist}

    \textbf{Quantitative Easing (QE):}
    \begin{itemize}
        \item Fed purchases Treasury bonds and MBS
        \item Funded by creating reserves, balance sheet expands
        \item QE1 (2008-'10), QE2 ('10-'11), QE3 ('12-'14), COVID QE ('20-'22)
        \item Lowers long-term interest rates 
        \item Allowing balance sheet to shrink referred to as "quantitative tightening"
    \end{itemize}
    \vspace{1em}

    \textbf{Operation Twist (2011-2012):}
    \begin{itemize}
        \item Fed sold short-term bonds and bought long-term bonds
        \item \textit{Duration-neutral:} Total size of Fed balance sheet unchanged
        \item Flattens the yield curve
    \end{itemize}
    \vspace{1em}

\end{frame}

\begin{frame}[t]
    \frametitle{5. Explanations of Yield Curve Slope}
    \framesubtitle{Fed purchases / Quantitative Easing}

    \vspace{1em}
    
    \centering
    \includegraphics[width=0.75\textwidth]{figures/ch15_1_fed.png}

\end{frame}


\begin{frame}[t]
    \frametitle{5. Explanations of Yield Curve Slope}
    \framesubtitle{Fed purchases / Quantitative Easing}

    \vspace{1em}

    \centering
    \includegraphics[width=0.8\textwidth]{figures/ch15_1.png}

\end{frame}

\begin{frame}[t]
    \frametitle{5. Explanations of Yield Curve Slope}
    \framesubtitle{Other?}

    \vspace{1em}
    Other potential factors:
    \begin{itemize}
        \item Credit risk?  
        \item Issuance expectations?
        \item Other central bank purchases
    \end{itemize}

\end{frame}

\begin{frame}[t]
    \frametitle{6. Yield Curve as a Recession Indicator?}
    \framesubtitle{History}

    The Treasury yield curve has been inverted (i.e. long term rates below short term rates) 
    prior to many previous recessions.\\
    \vspace{1em}

    \centering
    \includegraphics[width=0.8\textwidth]{figures/ch15_1_yc_rec.png}

\end{frame}

\begin{frame}[t]
    \frametitle{6. Yield Curve as a Recession Indicator?}
    \framesubtitle{2023}

    Does an inverted yield curve \textit{predict} a recession?\\
    \vspace{1em}

    \centering
    \includegraphics[width=0.7\textwidth]{figures/ch15_1_yc_2023-09-15.png}

\end{frame}


\end{document}