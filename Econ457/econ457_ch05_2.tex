\documentclass{beamer}

\newcommand{\week}{Week 2-a}

\title{Financial Market Distributions}
\subtitle{Reference: Bodie et al, Ch 5}
\author{Econ 457}
\date{\week}

% Reference the shared preamble
\setbeamertemplate{frametitle}{
  \vspace{0.5em}
  \insertframetitle
  \par
  \vspace{0.5em}
  \hrule
  \vspace{0.3em}
  {\small\color{gray}\insertframesubtitle}
}

\setbeamertemplate{navigation symbols}{}
\setbeamertemplate{itemize item}{\textbullet} % main bullet: filled dot
\setbeamertemplate{itemize subitem}{\normalsize$\circ$} % sub-bullet: empty dot
\setbeamertemplate{itemize subsubitem}{\scriptsize--} % sub-sub-bullet: dash


% Font changes
\usepackage[scaled=0.92]{helvet}
\renewcommand{\familydefault}{\sfdefault}

% Packages
\usepackage{tikz}
\usepackage{booktabs}
\usepackage{xcolor}
\usepackage{array}           % Enhanced column types for tables
\usepackage{multirow}        % Spanning multiple rows in tables
\usepackage{makecell}        % Line breaks and formatting in table cells
\usepackage{siunitx}         % Proper formatting of numbers and units
\usepackage{amsmath}         % Enhanced math environments
\usepackage{amsfonts}        % Additional math fonts
\usepackage{amssymb}         % Additional math symbols
\usepackage{url}             % Better URL formatting
\usepackage{graphicx}        % Enhanced graphics support
\usepackage{tabularray}
\UseTblrLibrary{booktabs, siunitx, varwidth}
% For financial presentations specifically
\usepackage{eurosym}         % Euro symbol
\usepackage{textcomp}        % Additional text symbols
\usepackage{hyperref}        % Hyperlinks (should be loaded last)

% Define a footnote
\renewcommand{\footnoterule}{\vspace*{-3pt}\hrule width 2in height 0.4pt\vspace*{2.6pt}}

% Define a Foundation Slide
\newenvironment{foundframe}[1][t]{
    \setbeamercolor{background canvas}{bg=gray!8}
    \setbeamercolor{frametitle}{fg=gray!80!black,bg=gray!25}
    \setbeamercolor{framesubtitle}{fg=gray!70!black,bg=gray!15}
    \setbeamercolor{item}{fg=gray!80!black}
    \setbeamercolor{enumerate item}{fg=gray!80!black}
    
    % Modify the frametitle template for this frame type
    \setbeamertemplate{frametitle}{
        \vspace{0.5em}
        \begin{minipage}[t]{0.75\textwidth}
            \insertframetitle
            \par
            \vspace{0.5em}
            \hrule
            \vspace{0.3em}
            {\small\color{gray}\insertframesubtitle}
        \end{minipage}%
        \hfill
        \begin{minipage}[t]{0.2\textwidth}
            \raggedleft
            \colorbox{gray!30}{%
                \scriptsize\bfseries\color{gray!80!black}%
                   \hspace{3pt}\begin{tabular}{c}Foundation\\Material\end{tabular}\hspace{3pt}%
            }
        \end{minipage}
        \vspace{0.3em}
    }
    
    \begin{frame}[#1]
}{
    \end{frame}
}

% Define Practice Slide
\newenvironment{practiceframe}[1][t]{
    \setbeamercolor{background canvas}{bg=white}
    \setbeamercolor{frametitle}{fg=blue!80!black,bg=blue!15}
    \setbeamercolor{framesubtitle}{fg=blue!70!black,bg=blue!10}
    \setbeamercolor{item}{fg=blue!80!black}
    \setbeamercolor{enumerate item}{fg=blue!80!black}
    \setbeamercolor{normal text}{fg=blue!90!black}
    
    % Modify the frametitle template for this frame type
    \setbeamertemplate{frametitle}{
        \vspace{0.5em}
        \begin{minipage}[t]{0.75\textwidth}
            \insertframetitle
            \par
            \vspace{0.5em}
            \hrule
            \vspace{0.3em}
            {\small\color{blue!70!black}\insertframesubtitle}
        \end{minipage}%
        \hfill
        \begin{minipage}[t]{0.2\textwidth}
            \raggedleft
            \colorbox{blue!20}{%
                \scriptsize\bfseries\color{blue!80!black}%
                   \hspace{3pt}\begin{tabular}{c}Practice\\Questions\end{tabular}\hspace{3pt}%
            }
        \end{minipage}
        \vspace{0.3em}
    }
    
    \begin{frame}[#1]
}{
    \end{frame}
}

% Define Excel Slide
\newenvironment{excelframe}[1][t]{
    \setbeamercolor{background canvas}{bg=white}
    \setbeamercolor{frametitle}{fg=blue!80!black,bg=blue!15}
    \setbeamercolor{framesubtitle}{fg=blue!70!black,bg=blue!10}
    \setbeamercolor{item}{fg=blue!80!black}
    \setbeamercolor{enumerate item}{fg=blue!80!black}
    \setbeamercolor{normal text}{fg=blue!90!black}
    
    % Modify the frametitle template for this frame type
    \setbeamertemplate{frametitle}{
        \vspace{0.5em}
        \begin{minipage}[t]{0.75\textwidth}
            \insertframetitle
            \par
            \vspace{0.5em}
            \hrule
            \vspace{0.3em}
            {\small\color{blue!70!black}\insertframesubtitle}
        \end{minipage}%
        \hfill
        \begin{minipage}[t]{0.2\textwidth}
            \raggedleft
            \colorbox{green!10}{%
                \scriptsize\bfseries\color{blue!80!black}%
                   \hspace{3pt}\begin{tabular}{c}MS Excel\end{tabular}\hspace{3pt}%
            }
        \end{minipage}
        \vspace{0.3em}
    }
    
    \begin{frame}[#1]
}{
    \end{frame}
}

% Define Caution Slide
\newenvironment{cautionframe}[1][t]{
    \setbeamercolor{background canvas}{bg=white}
    \setbeamercolor{frametitle}{fg=blue!80!black,bg=blue!15}
    \setbeamercolor{framesubtitle}{fg=blue!70!black,bg=blue!10}
    \setbeamercolor{item}{fg=blue!80!black}
    \setbeamercolor{enumerate item}{fg=blue!80!black}
    \setbeamercolor{normal text}{fg=blue!90!black}
    
    % Modify the frametitle template for this frame type
    \setbeamertemplate{frametitle}{
        \vspace{0.5em}
        \begin{minipage}[t]{0.75\textwidth}
            \insertframetitle
            \par
            \vspace{0.5em}
            \hrule
            \vspace{0.3em}
            {\small\color{blue!70!black}\insertframesubtitle}
        \end{minipage}%
        \hfill
        \begin{minipage}[t]{0.2\textwidth}
            \raggedleft
            \colorbox{red!10}{%
                \scriptsize\bfseries\color{blue!80!black}%
                   \hspace{3pt}\begin{tabular}{c}Caution\end{tabular}\hspace{3pt}%
            }
        \end{minipage}
        \vspace{0.3em}
    }
    
    \begin{frame}[#1]
}{
    \end{frame}
}

% Add to footnotes
\makeatletter
\newcommand\blfootnote[1]{%
  \begingroup
  \renewcommand\thefootnote{}%
  \renewcommand\@makefntext[1]{\raggedright\leftskip=0pt ##1}%
  \footnote{\scriptsize #1}%
  \addtocounter{footnote}{-1}%
  \endgroup
}
\makeatother

% Set the footer -- change 
\setbeamertemplate{footline}{
  \leavevmode%
  \vspace{2ex}
  \hbox{%
    % Left box: Econ 457
    \begin{beamercolorbox}[wd=.4\paperwidth,ht=2.5ex,dp=1ex,left]{author in head/foot}%
      \hspace{1em}Econ 457
    \end{beamercolorbox}%
    % Middle box: Week
    \begin{beamercolorbox}[wd=.2\paperwidth,ht=2.5ex,dp=1ex,center]{date in head/foot}%
      \centering\week
    \end{beamercolorbox}%
    % Right box: Slide numbers
    \begin{beamercolorbox}[wd=.4\paperwidth,ht=2.5ex,dp=1ex,center]{date in head/foot}%
      \hfill\insertframenumber{} 
    \end{beamercolorbox}%
  }%
  \vskip0pt%
}

\begin{document}

\frame{\titlepage}

\begin{frame}
    \frametitle{Outline}

    \vspace{2em}

    \begin{enumerate}
        \item The Normal Distribution
        \item Distributions of Financial Market Returnts
        \item Annualizing Weekly Returns
        \item Forecasting
        \item Problems with Using the Normal Distribution
        \item Risk management enhancements
    \end{enumerate}
\end{frame}

\begin{foundframe}[t]
    \frametitle{1. The Normal Distribution}
    \framesubtitle{Some basic facts}

    Notation: 
    $$X \sim N(\mu, \sigma^2)$$
    Where $\mu$ is the mean, $\sigma^2$ is the variance, and $\sigma$ is the standard deviation.\\   
    \vspace{1em}
    The distribution can be characterized by either the variance or the standard deviation.  
    In finance it is common to use the standard deviation, because it is in the same units as the returns, whereas the variance is squared.\\

\end{foundframe}

\begin{foundframe}[t]
    \frametitle{1. The Normal Distribution}
    \framesubtitle{Some basic facts}

    Empirical mean:
    $$\bar{x} = \frac{1}{T}\sum_{t=1}^{T}x_t$$

    Empirical variance:
    $$s^2 = \frac{1}{T-1}\sum_{t=1}^{T}(x_t - \bar{x})^2$$

    Empirical standard deviation:
    $$s = \sqrt{s^2} = \sqrt{\frac{1}{T-1}\sum_{t=1}^{T}(x_t - \bar{x})^2}$$

\end{foundframe}

\begin{foundframe}[t]
    \frametitle{1. The Normal Distribution}
    \framesubtitle{Some basic facts, cont'd}

    Adding a constant $C$ to $X$:\\
    $$X + C \sim N(\mu_x+C,\sigma_x^2)$$

    Multiplying $X$ by a constant $C$:
    $$C \cdot X \sim N(C\mu_x,C^2\sigma_x^2)$$
    (notice the $C^2$ in the variance term)

\end{foundframe}

\begin{foundframe}[t]
    \frametitle{1. The Normal Distribution}
    \framesubtitle{Some basic facts, cont'd}

    Adding two normally distributed variables $X$ and $Y$:\\
    \vspace{1em}
    
    \textbf{Case 1:} $X$ and $Y$ are \textit{independent}
    $$X + Y \sim N(\mu_x+\mu_y,\sigma_x^2 + \sigma_y^2)$$
     
    \textbf{Case 2:} $X$ and $Y$ are \textit{correlated} 
    $$X + Y \sim N(\mu_x+\mu_y,\sigma_x^2 + \sigma_y^2 + 2\sigma_x\sigma_y\rho_{x,y})$$
    
    where $\rho_{x,y} = \frac{\text{Cov}(X,Y)}{\sigma_x\sigma_y}$ is the correlation coefficient.

\end{foundframe}

\begin{foundframe}[t]
    \frametitle{1. The Normal Distribution}
    \framesubtitle{A Sample, n = 100}

    \centering
    \includegraphics[width=0.8\textwidth]{figures/ch5_2_ndist_500.png}

\end{foundframe}

\begin{foundframe}[t]
    \frametitle{1. The Normal Distribution}
    \framesubtitle{Some basic facts, cont'd}

        \centering
        \includegraphics[width=0.8\textwidth]{figures/ch5_2_ndist_1m.png}

\end{foundframe}

\begin{foundframe}[t]
    \frametitle{1. The Normal Distribution}
    \framesubtitle{Some basic facts, cont'd}

    \textbf{The Central Limit Theorm}
    The distribution of the sample mean converges to a normal distribution, regardless of the distribution of the underlying variable.
    \vspace{1em}
    Let $\bar{X} = \frac{1}{T}\sum_i^Tx_i$ be the sample mean, then
    $$\bar{X} \xrightarrow{d} N\left(\mu, \frac{\sigma^2}{T}\right) \text{ as } T \to \infty$$

    The only requirement is that the underlying variables (usually) need to be independent and identically distributed.

\end{foundframe}

\begin{foundframe}[t]
    \frametitle{1. The Normal Distribution}
    \framesubtitle{Some basic facts, cont'd}

    \textbf{Galton Board}\\
    \vspace{2em}

    https://youtu.be/EvHiee7gs9Y


\end{foundframe}

\begin{frame}[t]
    \frametitle{2. Distributions of Financial Market Returns}
    \framesubtitle{S\&P 500}

    \centering
    \includegraphics[width=0.8\textwidth]{figures/ch5_2_sp500_wdist.png}

    \vfill
    \begin{flushleft}
    \footnotesize{Data source: Robert Shiller, www.shillerdata.com}
    \end{flushleft}

\end{frame}

{
\setbeamertemplate{footline}{}
\begin{frame}[t]
    \frametitle{2. Distributions of Financial Market Returns}
    \framesubtitle{S\&P 500 Total Returns}
    
    \vspace{-1em}
    \scriptsize
    \begin{table}
    \caption{S\&P 500 Total Returns by Decade, (\% Annualized)}
    \begin{tabular}{lrr}
    \toprule
    Decade & Return & Standard Deviation \\
    \midrule
    1870s & 8.11 & 10.86 \\
    1880s & 6.29 & 9.62 \\
    1890s & 6.24 & 11.84 \\
    1900s & 10.62 & 12.70 \\
    1910s & 4.99 & 10.51 \\
    1920s & 15.46 & 15.08 \\
    1930s & 4.39 & 30.63 \\
    1940s & 9.47 & 13.23 \\
    1950s & 18.24 & 10.08 \\
    1960s & 8.05 & 10.31 \\
    1970s & 6.56 & 13.23 \\
    1980s & 16.88 & 12.98 \\
    1990s & 17.19 & 10.45 \\
    2000s & 0.38 & 14.68 \\
    2010s & 13.02 & 9.55 \\
    2020s & 13.72 & 14.15 \\
    \bottomrule
    \end{tabular}
    \end{table}

    \blfootnote{Data Source: Robert Shiller, Ken French.  Monthly data annualized.}

\end{frame}
}

\begin{frame}[t]
    \frametitle{2. Distributions of Financial Market Returns}
    \framesubtitle{US Treasury Bonds, 10-year US Treasury}

    \centering
    \includegraphics[width=0.8\textwidth]{figures/ch5_2_10y_wdist.png}
    

    \vfill
    \begin{flushleft}
    \footnotesize{Data source: Robert Shiller, www.shillerdata.com}
    \end{flushleft}

\end{frame}

\begin{frame}[t]
    \frametitle{2. Distributions of Financial Market Returns}
    \framesubtitle{US T-Bills, 1-month}

    \centering
    \includegraphics[width=0.8\textwidth]{figures/ch5_2_tbill_wdist.png}

    \vfill
    \begin{flushleft}
    \footnotesize{Data source: Ken French}
    \end{flushleft}

\end{frame}

\begin{frame}[t]
    \frametitle{2. Distributions of Financial Market Returns}
    \framesubtitle{MXN}

    \centering
    \includegraphics[width=0.6\textwidth]{figures/ch5_2_MXN_wdist.png}

    \vfill
    \begin{flushleft}
    \footnotesize{Data source: Interactive Brokers (USDMXN Cash)}
    \footnotesize{             Only the price return, doesn't include carry}
    \footnotesize{             5 worst months: Mar 2020, Sep 2011, Oct 2008, May 2012, Nov 2008}
    \end{flushleft}

\end{frame}

\begin{frame}[t]
    \frametitle{3. Annualizing Weekly Returns}
    \framesubtitle{}

    If you are given a data at a higher frequency it is often convenient to 
    convert the mean and standard deviation to annual equivalents.\\
    \vspace{1em}
    In order to do so, imagine that each year is made up of \textit{independent} sub-periods.
    Then we can use the following property of the normal distribution:
    \begin{align*}
    x_1 +... + x_T &\sim N(\mu_x+...+\mu_x,\sigma_x^2 +...+ \sigma_x^2)\\
                    &\sim N(T\mu_x,T\sigma_x^2)
    \end{align*}
    Where $x_1$ through $x_T$ are indpendent sub-periods, and T is the number 
    of sub-periods that make up the longer period.  (i.e. 52 weeks in a year, so T=52)  
    We assume the distrubtions of each sub-period 
    are identical and equal to $N(\mu_x$, $\sigma_x^2$).\\

\end{frame}

\begin{frame}[t]
    \frametitle{3. Annualizing Weekly Returns}
    \framesubtitle{}

    To compute annual returns from a series of weekly returns (52 weeks in a year):
    \begin{enumerate}
        \item Compute the mean and standard deviation of the weekly returns
        \item Multiply the estimated mean by 52
        \begin{itemize}
            \item An alternative would be to account for compounding and use $(1+\mu_x)^{52}-1$.
        \end{itemize}
        \item Multiply the standard deviation by $\sqrt{52}$
        \begin{itemize}
            \item Be careful here, this is because the variance increases by $T$,
            so standard deviation increases by $\sqrt{T}$
        \end{itemize}
        
    \end{enumerate}
    \vspace{1em}
    Easy to do something similar for daily returns 
    (multiply standard deviation by $\sqrt{250}$, number of trading days in a year), 
    or monthly returns (multiply standard deviation by $\sqrt{12}$), 
     or quarterly returns (multiply standard deviation by 2).

\end{frame}

\begin{frame}[t]
    \frametitle{3. Annualizing Weekly Returns}
    \framesubtitle{}

    \begin{table}
        \caption{Daily, Monthly, Annual S\&P Returns}
        \begin{tabular}{lrr}
        \toprule
        &  Mean & Standard Deviation \\
        \midrule
        Daily & 0.04 & 1.19 \\
        Daily, annualized & 9.48 & 18.87 \\
        \midrule
        Monthly & 0.75 & 4.25 \\
        Monthly, annualized & 8.99 & 14.73 \\
        \midrule
        Annual & 9.35 & 16.31 \\
        \bottomrule
        \end{tabular}
    \end{table}

    \blfootnote{Data from Interactive Brokers, since 2004}
\end{frame}

\begin{frame}[t]
    \frametitle{4. Forecasting}
    \framesubtitle{}

    Note that the discussion up to now has been about measuring historical returns.  
    In contrast, the questions we usually care about concern expected future 
    returns.\\
    \vspace{1em}

    In general, you should be very careful about using historical returns and 
    distributions to forecast future returns and distributions.\\
    
    \vspace{1em}
    But forecasting is an inescapable requirement for many things in financial markets, 
    so you have to come up with something.  Much of the next few weeks will be focused on 
    how to think about \textit{expected} returns and prices for different securities.

\end{frame}


\begin{frame}[t]
    \frametitle{4. Forecasting}
    \framesubtitle{If you think history is a good guide...}

    If you can convince yourself that history is a good guide for future 
    returns (or if you can find a historical period that is a good guide),
    then can use the empirical mean and standard deviation:

    $$\mathbb{E}(r) = \frac{1}{T} \sum_i^T r_i$$
    
    $$\text{Var}(r) = \frac{1}{T} \sum_i^T (r_i - \mathbb{E}[r])^2$$

    where $r_i$ are \textit{historical} returns measured over $T$ periods.

\end{frame}

\begin{frame}[t]
    \frametitle{4. Forecasting}
    \framesubtitle{If you can guess the probability of scenarios...}

    If we start with a set of estimates for returns in different scenarios and 
    the probabilities of those scenarios, then we can compute the 
    expected value and expected variance as follows:\\

    $$\mathbb{E}(r) = \sum_i p_i \cdot r_i$$
    
    $$\text{Var}(r) = \sum_i p_i \cdot (r_i - \mathbb{E}[r])^2$$
    
    where $p_i$ is the probability and $r_i$ is the return in scenario $i$.\\
    \vspace{1em}
    Note these are analogous to the empirical estimates of the mean and the variance discussed earlier.

\end{frame}


\begin{frame}[t]
    \frametitle{5. Problems with Using the Normal Distribution}
    \framesubtitle{}

    While it may be convenient to use the normal distribution 
    for forecasts of future returns, doing so raises a lot of problems:
    \begin{enumerate}
        \item Observations may not be independent (Central Limit Theorem doesn't hold)
        \item Distributions of financial returns often have 'fat tails'
        \item Historical returns have no meausure of value
        \item The variance of returns seems to change over time
    \end{enumerate}
    \vspace{1em}
    What to do about this?
\end{frame}

\begin{frame}[t]
    \frametitle{5. Problems with Using the Normal Distribution}
    \framesubtitle{Observations are not independent}

    \textbf{Galton Board, v2}\\
    \vspace{2em}
    https://youtu.be/3m4bxse2JEQ?si=SIhI4vECWSKwkQV-

\end{frame}

\begin{frame}[t]
    \frametitle{5. Problems with Using the Normal Distribution}
    \framesubtitle{Fat Tails}

    \vspace{-1.5em}

    \footnotesize
    \begin{columns}
        \begin{column}{0.5\textwidth}
                \begin{table}
                \caption{S\&P Highest and Lowest Monthly Returns Since 1950}
                \begin{tabular}{lrr}
                \toprule
                Date & Return \% & Probability \% \\
                \midrule
                    2009-04 & 12.32 & 0.05 \\
                    1982-09 & 12.10 & 0.07 \\
                    1991-02 & 11.61 & 0.11 \\
                    1998-11 & 10.97 & 0.19 \\
                    1975-02 & 10.81 & 0.23 \\
                    \midrule
                    1962-06 & -11.41 & 0.02 \\
                    1987-10 & -11.85 & 0.01 \\
                    1987-11 & -12.30 & 0.01 \\
                    2020-03 & -18.92 & 0.00 \\
                    2008-10 & -20.19 & 0.00 \\
                \bottomrule
                \end{tabular}
                \end{table}
        \end{column}

        \begin{column}{0.5\textwidth}
            \begin{table}
                \caption{10y UST Highest and Lowest Monthly Returns Since 1950}
                \begin{tabular}{lrr}
                \toprule
                Date & Return \% & Probability \% \\
                \midrule
                1981-11 & 11.23 & 0.00 \\
                1982-10 & 9.86 & 0.00 \\
                2008-12 & 9.74 & 0.00 \\
                1980-05 & 9.15 & 0.00 \\
                1980-04 & 8.87 & 0.00 \\
                \midrule
                2022-09 & -5.18 & 0.16 \\
                2003-07 & -5.30 & 0.13 \\
                2022-04 & -5.42 & 0.10 \\
                1979-10 & -5.85 & 0.05 \\
                1980-02 & -9.56 & 0.00 \\
                \bottomrule
                \end{tabular}
            \end{table}
        \end{column}
    \end{columns}


    \blfootnote{Probabilities are based on estimated normal distribution.  
    n = 900 months, S\&P: $\mu=0.96$ and $\sigma=3.4$, 
    10y UST: $\mu=0.42$ and $\sigma=1.9$}
\end{frame}

\begin{frame}[t]
    \frametitle{5. Problems with Using the Normal Distribution}
    \framesubtitle{Historical returns have no measure of value}

    \textbf{Home prices, 1960-2006}
    \centering
    \includegraphics[width=0.8\textwidth]{figures/ch5_2_hpi_pre2008.png}

\end{frame}

\begin{frame}[t]
    \frametitle{5. Problems with Using the Normal Distribution}
    \framesubtitle{Historical returns have no measure of value}

    \textbf{Home prices, 1960-2025}
    \centering
    \includegraphics[width=0.8\textwidth]{figures/ch5_2_hpi_all.png}

\end{frame}

\begin{frame}[t]
    \frametitle{5. Problems with Using the Normal Distribution}
    \framesubtitle{Variance changes over time}

    \centering
    \includegraphics[width=\textwidth]{figures/ch5_2_rolling_std.png}

\end{frame}

\begin{frame}[t]
    \frametitle{6. Risk management enhancements}
    \framesubtitle{What to do?}

    How financial market participants address these problems:
    \begin{enumerate}
        \item Use a distribution with fatter tails
        \item Value at Risk (VaR)
        \item Scenario Analysis / Stress Tests
    \end{enumerate}

\end{frame}


\begin{frame}[t]
    \frametitle{6. Risk management enhancements}
    \framesubtitle{Value at Risk (VaR) Definition}

    \textbf{Value at Risk (VaR)} measures the maximum expected loss over a specific time period at a given confidence level.
    \vspace{1em}

    \textbf{Definition:}
    \begin{itemize}
        \item VaR answers: "What is the worst loss we can expect with X\% confidence over Y days?"
        \item Example: 1-day 95\% VaR = \$1 million means there's a 5\% chance of losing more than \$1M tomorrow
    \end{itemize}
    \vspace{0.5em}

\end{frame}

\begin{frame}[t]
    \frametitle{6. Risk management enhancements}
    \framesubtitle{Value at Risk (VaR) - Calculation }

    \textbf{Calculation Methods:}
    \begin{enumerate}
        \item \textbf{Parametric (Normal) VaR:} Assumes returns follow normal distribution
        $$\text{VaR} = \mu - z_\alpha \times \sigma \times \sqrt{t}$$
        where $z_\alpha$ is the critical value (e.g., 1.65 for 95\% confidence)
        
        \item \textbf{Historical VaR:} Uses actual historical return distribution
        
        \item \textbf{Monte Carlo VaR:} Simulates thousands of possible outcomes
    \end{enumerate}
    \vspace{0.5em}

    \textbf{Limitations:} Still need to make an assumption about the tails, past patterns? what else?

\end{frame}

\begin{frame}[t]
    \frametitle{6. Risk management enhancements}
    \framesubtitle{Stress Tests}

    \textbf{Stress Testing:} Evaluates portfolio performance under extreme scenarios
    \vspace{1em}

    \textbf{Approach:}
    \begin{itemize}
        \item Apply historical crisis scenarios (2008, 1987, COVID-19)
        \item Test hypothetical extreme events (interest rate shocks, market crashes)
        \item Examine correlation breakdown during crises
    \end{itemize}
    \vspace{1em}

    \textbf{Example:} "What would happen to our portfolio if we had another 2008-style crisis?"\\
    \vspace{1em}
    Useful at the portfolio level, but need to specify correlations in addition to returns of 
    various securities.

\end{frame}

 \end{document}