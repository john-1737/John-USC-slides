\documentclass{beamer}

\newcommand{\week}{Week 13-b}

\title{Options: Valuation}
\subtitle{Reference: Bodie et al, Ch 21}
\author{Econ 457}
\date{\week}

% Reference the shared preamble
\setbeamertemplate{frametitle}{
  \vspace{0.5em}
  \insertframetitle
  \par
  \vspace{0.5em}
  \hrule
  \vspace{0.3em}
  {\small\color{gray}\insertframesubtitle}
}

\setbeamertemplate{navigation symbols}{}
\setbeamertemplate{itemize item}{\textbullet} % main bullet: filled dot
\setbeamertemplate{itemize subitem}{\normalsize$\circ$} % sub-bullet: empty dot
\setbeamertemplate{itemize subsubitem}{\scriptsize--} % sub-sub-bullet: dash


% Font changes
\usepackage[scaled=0.92]{helvet}
\renewcommand{\familydefault}{\sfdefault}

% Packages
\usepackage{tikz}
\usepackage{booktabs}
\usepackage{xcolor}
\usepackage{array}           % Enhanced column types for tables
\usepackage{multirow}        % Spanning multiple rows in tables
\usepackage{makecell}        % Line breaks and formatting in table cells
\usepackage{siunitx}         % Proper formatting of numbers and units
\usepackage{amsmath}         % Enhanced math environments
\usepackage{amsfonts}        % Additional math fonts
\usepackage{amssymb}         % Additional math symbols
\usepackage{url}             % Better URL formatting
\usepackage{graphicx}        % Enhanced graphics support
\usepackage{tabularray}
\UseTblrLibrary{booktabs, siunitx, varwidth}
% For financial presentations specifically
\usepackage{eurosym}         % Euro symbol
\usepackage{textcomp}        % Additional text symbols
\usepackage{hyperref}        % Hyperlinks (should be loaded last)

% Define a footnote
\renewcommand{\footnoterule}{\vspace*{-3pt}\hrule width 2in height 0.4pt\vspace*{2.6pt}}

% Define a Foundation Slide
\newenvironment{foundframe}[1][t]{
    \setbeamercolor{background canvas}{bg=gray!8}
    \setbeamercolor{frametitle}{fg=gray!80!black,bg=gray!25}
    \setbeamercolor{framesubtitle}{fg=gray!70!black,bg=gray!15}
    \setbeamercolor{item}{fg=gray!80!black}
    \setbeamercolor{enumerate item}{fg=gray!80!black}
    
    % Modify the frametitle template for this frame type
    \setbeamertemplate{frametitle}{
        \vspace{0.5em}
        \begin{minipage}[t]{0.75\textwidth}
            \insertframetitle
            \par
            \vspace{0.5em}
            \hrule
            \vspace{0.3em}
            {\small\color{gray}\insertframesubtitle}
        \end{minipage}%
        \hfill
        \begin{minipage}[t]{0.2\textwidth}
            \raggedleft
            \colorbox{gray!30}{%
                \scriptsize\bfseries\color{gray!80!black}%
                   \hspace{3pt}\begin{tabular}{c}Foundation\\Material\end{tabular}\hspace{3pt}%
            }
        \end{minipage}
        \vspace{0.3em}
    }
    
    \begin{frame}[#1]
}{
    \end{frame}
}

% Define Practice Slide
\newenvironment{practiceframe}[1][t]{
    \setbeamercolor{background canvas}{bg=white}
    \setbeamercolor{frametitle}{fg=blue!80!black,bg=blue!15}
    \setbeamercolor{framesubtitle}{fg=blue!70!black,bg=blue!10}
    \setbeamercolor{item}{fg=blue!80!black}
    \setbeamercolor{enumerate item}{fg=blue!80!black}
    \setbeamercolor{normal text}{fg=blue!90!black}
    
    % Modify the frametitle template for this frame type
    \setbeamertemplate{frametitle}{
        \vspace{0.5em}
        \begin{minipage}[t]{0.75\textwidth}
            \insertframetitle
            \par
            \vspace{0.5em}
            \hrule
            \vspace{0.3em}
            {\small\color{blue!70!black}\insertframesubtitle}
        \end{minipage}%
        \hfill
        \begin{minipage}[t]{0.2\textwidth}
            \raggedleft
            \colorbox{blue!20}{%
                \scriptsize\bfseries\color{blue!80!black}%
                   \hspace{3pt}\begin{tabular}{c}Practice\\Questions\end{tabular}\hspace{3pt}%
            }
        \end{minipage}
        \vspace{0.3em}
    }
    
    \begin{frame}[#1]
}{
    \end{frame}
}

% Define Excel Slide
\newenvironment{excelframe}[1][t]{
    \setbeamercolor{background canvas}{bg=white}
    \setbeamercolor{frametitle}{fg=blue!80!black,bg=blue!15}
    \setbeamercolor{framesubtitle}{fg=blue!70!black,bg=blue!10}
    \setbeamercolor{item}{fg=blue!80!black}
    \setbeamercolor{enumerate item}{fg=blue!80!black}
    \setbeamercolor{normal text}{fg=blue!90!black}
    
    % Modify the frametitle template for this frame type
    \setbeamertemplate{frametitle}{
        \vspace{0.5em}
        \begin{minipage}[t]{0.75\textwidth}
            \insertframetitle
            \par
            \vspace{0.5em}
            \hrule
            \vspace{0.3em}
            {\small\color{blue!70!black}\insertframesubtitle}
        \end{minipage}%
        \hfill
        \begin{minipage}[t]{0.2\textwidth}
            \raggedleft
            \colorbox{green!10}{%
                \scriptsize\bfseries\color{blue!80!black}%
                   \hspace{3pt}\begin{tabular}{c}MS Excel\end{tabular}\hspace{3pt}%
            }
        \end{minipage}
        \vspace{0.3em}
    }
    
    \begin{frame}[#1]
}{
    \end{frame}
}

% Define Caution Slide
\newenvironment{cautionframe}[1][t]{
    \setbeamercolor{background canvas}{bg=white}
    \setbeamercolor{frametitle}{fg=blue!80!black,bg=blue!15}
    \setbeamercolor{framesubtitle}{fg=blue!70!black,bg=blue!10}
    \setbeamercolor{item}{fg=blue!80!black}
    \setbeamercolor{enumerate item}{fg=blue!80!black}
    \setbeamercolor{normal text}{fg=blue!90!black}
    
    % Modify the frametitle template for this frame type
    \setbeamertemplate{frametitle}{
        \vspace{0.5em}
        \begin{minipage}[t]{0.75\textwidth}
            \insertframetitle
            \par
            \vspace{0.5em}
            \hrule
            \vspace{0.3em}
            {\small\color{blue!70!black}\insertframesubtitle}
        \end{minipage}%
        \hfill
        \begin{minipage}[t]{0.2\textwidth}
            \raggedleft
            \colorbox{red!10}{%
                \scriptsize\bfseries\color{blue!80!black}%
                   \hspace{3pt}\begin{tabular}{c}Caution\end{tabular}\hspace{3pt}%
            }
        \end{minipage}
        \vspace{0.3em}
    }
    
    \begin{frame}[#1]
}{
    \end{frame}
}

% Add to footnotes
\makeatletter
\newcommand\blfootnote[1]{%
  \begingroup
  \renewcommand\thefootnote{}%
  \renewcommand\@makefntext[1]{\raggedright\leftskip=0pt ##1}%
  \footnote{\scriptsize #1}%
  \addtocounter{footnote}{-1}%
  \endgroup
}
\makeatother

% Set the footer -- change 
\setbeamertemplate{footline}{
  \leavevmode%
  \vspace{2ex}
  \hbox{%
    % Left box: Econ 457
    \begin{beamercolorbox}[wd=.4\paperwidth,ht=2.5ex,dp=1ex,left]{author in head/foot}%
      \hspace{1em}Econ 457
    \end{beamercolorbox}%
    % Middle box: Week
    \begin{beamercolorbox}[wd=.2\paperwidth,ht=2.5ex,dp=1ex,center]{date in head/foot}%
      \centering\week
    \end{beamercolorbox}%
    % Right box: Slide numbers
    \begin{beamercolorbox}[wd=.4\paperwidth,ht=2.5ex,dp=1ex,center]{date in head/foot}%
      \hfill\insertframenumber{} 
    \end{beamercolorbox}%
  }%
  \vskip0pt%
}

\begin{document}

\frame{\titlepage}

\begin{frame}
    \frametitle{Outline}

    \begin{enumerate}
        \item Embedded Options
        \item Market for Options
        \item Put-Call Parity
        \item Option Valuation
            \begin{itemize}
                \item Time Value
                \item Black-Scholes-Merton
                \item Greeks
            \end{itemize}
    \end{enumerate}

\end{frame}

\begin{frame}[t]
    \frametitle{1. Embedded Options}
    \framesubtitle{Callable Bonds}

    A callable bond is a bond that gives the issuer the right (but not the obligation) to redeem the bond before its maturity date at a predetermined price.

    \textbf{Key Features:}
    \begin{itemize}
        \item \textbf{Call Price:} Usually set at par value (\$1,000) or slightly above
        \item \textbf{Call Protection:} Period when bond cannot be called (e.g., first 5 years)
        \item \textbf{Call Schedule:} Specific dates when bond can be called
        \item \textbf{Higher Yield:} Callable bonds typically offer higher yields to compensate investors
    \end{itemize}

    \textbf{Why Do Issuers Call Bonds?}
    \begin{itemize}
        \item Interest rates have declined since issuance
        \item Can refinance debt at lower cost
        \item Improved credit quality allows cheaper financing
    \end{itemize}

\end{frame}

\begin{frame}[t]
    \frametitle{1. Embedded Options}
    \framesubtitle{Callable Bonds}

    \centering
    \includegraphics[width=0.9\textwidth]{figures/ch21_1_callable_debt.png}

\end{frame}

\begin{frame}[t]
    \frametitle{1. Embedded Options}
    \framesubtitle{Callable Bonds - Mortgages}

    \centering
    \includegraphics[width=0.9\textwidth]{figures/ch14_2_mtg_prepayment.png}

\end{frame}


\begin{frame}[t]
    \frametitle{1. Embedded Options}
    \framesubtitle{Convertible Bonds}

    A convertible bond is a bond that gives the bondholder 
    the right (but not the obligation) to convert the bond 
    into a predetermined number of shares of the issuer's common stock.

    \textbf{Key Features:}
    \begin{itemize}
        \item \textbf{Conversion Ratio:} Number of shares per bond (e.g., 25 shares per \$1,000 bond)
        \item \textbf{Conversion Price:} Effective price per share (\$1,000 ÷ 25 = \$40)
        \item \textbf{Conversion Value:} Current stock price × conversion ratio
        \item \textbf{Lower Yield:} Convertibles offer lower yields than straight bonds
    \end{itemize}

    \textbf{Value Components:}
    \begin{itemize}
        \item \textbf{Bond Floor:} Minimum value as a straight bond
        \item \textbf{Option Premium:} Value of conversion feature
        \item Total Value = Bond Floor + Option Premium
    \end{itemize}

\end{frame}

\begin{frame}[t]
    \frametitle{1. Embedded Options}
    \framesubtitle{Convertible Bonds}

    \centering
    \includegraphics[width=0.9\textwidth]{figures/ch21_1_convertible_debt.png}

\end{frame}

\begin{frame}[t]
    \frametitle{1. Embedded Options}
    \framesubtitle{Convertible Bond Arbitrage}

    Convertible arbitrage is a hedge fund strategy that exploits pricing inefficiencies between convertible bonds and their underlying stocks.

    \textbf{Basic Strategy:}
    \begin{itemize}
        \item \textbf{Long Position:} Buy the underpriced convertible bond
        \item \textbf{Short Position:} Short sell the underlying stock (hedge ratio based on delta)
        \item \textbf{Profit Source:} Capture the difference between implied and actual volatility
    \end{itemize}

    Why is the convertible bond underpriced?   Complexity, maybe?   Liquidity, maybe?

\end{frame}

\begin{frame}[t]
    \frametitle{1. Embedded Options}
    \framesubtitle{Warrants}

    A warrant is a security that gives the holder the right to purchase shares of the issuing company at a fixed price (exercise price) for a specified period of time.

    \textbf{Key Features:}
    \begin{itemize}
        \item \textbf{Long Maturity:} Typically 3-5 years (vs. months for options)
        \item \textbf{Issued by Company:} Company creates new shares when exercised
        \item \textbf{Dilution Effect:} Exercise increases total shares outstanding
    \end{itemize}

    \textbf{Warrants vs. Call Options:}
    \begin{itemize}
        \item \textbf{Source:} Warrants issued by company; calls created by investors
        \item \textbf{Dilution:} Warrant exercise dilutes existing shareholders
        \item \textbf{Proceeds:} Exercise price goes to company (warrants) vs. option writer (calls)
    \end{itemize}

\end{frame}

\begin{frame}[t]
    \frametitle{1. Embedded Options}
    \framesubtitle{Corporate Bonds as Short Put Options}

    Corporate bonds can be viewed as equivalent to buying a risk-free bond and selling a put option on the firm's assets.

    \textbf{The Logic:}
    \begin{itemize}
        \item \textbf{Default occurs when:} Firm value falls below debt obligations
        \item \textbf{Bondholders receive:} Min(Face Value, Firm Value)
        \item \textbf{This is equivalent to:} Face Value - Max(0, Face Value - Firm Value)
        \item \textbf{Which equals:} Risk-free bond - Put option on firm value
    \end{itemize}

    \textbf{Implications:}
    \begin{itemize}
        \item Higher firm volatility $\rightarrow$ Higher put value $\rightarrow$ Lower bond value
        \item Corporate bond yield = Risk-free rate + Credit spread
        \item Credit spread compensates for the "short put" position
        \item Explains why bond prices fall when default risk increases
    \end{itemize}

\end{frame}

\begin{frame}[t]
    \frametitle{1. Embedded Options}
    \framesubtitle{Corporate Bonds as Short Put Options}

        \centering
        \includegraphics[width=0.8\textwidth]{figures/ch21_1_corporate_debt.png}

\end{frame}

\begin{frame}[t]
    \frametitle{2. Market For Options}
    \framesubtitle{How Options Are Traded}

    \footnotesize
    \textbf{Exchange-Traded Options (Listed):}
    \begin{itemize}
        \item \textbf{Major Exchanges:} Chicago Board Options Exchange (CBOE), CME Group, NYSE American
        \item \textbf{Standardized:} Fixed expiration dates, strike prices, contract sizes
        \item \textbf{Clearing:} Options Clearing Corporation (OCC) guarantees all trades
        \item \textbf{Liquid:} High volume, tight bid-ask spreads for popular contracts
        \item \textbf{Examples:} SPY, QQQ, individual stock options
    \end{itemize}

    \textbf{Over-the-Counter (OTC) Options:}
    \begin{itemize}
        \item \textbf{Customized:} Tailored strike prices, expiration dates, underlying assets
        \item \textbf{Counterparty Risk:} No central clearing, direct bilateral agreements
        \item \textbf{Less Liquid:} Harder to trade before expiration
        \item \textbf{Examples:} Currency options, commodity options, exotic options
    \end{itemize}

\end{frame}

\begin{frame}[t]
    \frametitle{2. Market For Options}
    \framesubtitle{}

    One month options on the S\&P:\\
    \vspace{1em}
    \centering
    \includegraphics[width=\textwidth]{figures/ch21_calls.png}
        \vspace{1em}
    \centering
    \includegraphics[width=\textwidth]{figures/ch21_puts.png}

\end{frame}


\begin{frame}[t]
    \frametitle{3. Put-Call Parity}
    \framesubtitle{}

    Notice that \textbf{buy call} strategy looks very similar to a \textbf{protective put}.\\
    \vspace{1em}
    \begin{minipage}{0.48\textwidth}
        \centering
        \includegraphics[width=\textwidth]{figures/ch20_1_buy_call.png}
    \end{minipage}
    \hfill
    \begin{minipage}{0.48\textwidth}
        \centering
        \includegraphics[width=\textwidth]{figures/ch20_1_prot_put.png}
    \end{minipage}

\end{frame}

\begin{frame}[t]
    \frametitle{3. Put-Call Parity}
    \framesubtitle{}

    In order to make them exactly equal, we need to combine the call with a zero-coupon 
    bond that returns the value of the strike.\\
    $$C_0 + \frac{X}{(1+r_f)^T} = S_0 + P_0$$
    Where $X$ is the zero-coupon bond with face value equal to the strike price.   One final detail is that 
    the value of the stock ($S_0$) incorporates the right to dividends that are paid before the option expiry.  
    We need to add those divdends to the left hand side to make these equal.
    $$C_0 + \frac{X}{(1+r_f)^T} + PV(\text{dividends})= S_0 + P_0$$
    Rearranging slightly gives us the expression of \textbf{put-call parity}
    $$\boxed{P_0 = C_0 - S_0 + PV(X) + PV(\text{dividends})}$$

\end{frame}

\begin{frame}[t]
    \frametitle{4. Option Valuation}
    \framesubtitle{Time Value}

    The \textbf{intrisic value} of an option is the value of the option at the time of 
    expiry.   The \textbf{time value} of the option is defined as the difference between 
    the value today and the intrinsic value of the option.\\
    \vspace{1em}

    \centering
    \includegraphics[width=0.6\textwidth]{figures/ch21_timevalue.png}

\end{frame}

\begin{frame}[t]
    \frametitle{4. Option Valuation}
    \framesubtitle{NVDA calls}

    \centering
    \includegraphics[width=\textwidth]{figures/ch21_nvda_call.png}

\end{frame}

\begin{frame}[t]
    \frametitle{4. Option Valuation}
    \framesubtitle{NVDA calls}

    \centering
    \includegraphics[width=0.8\textwidth]{figures/ch21_nvda_call2.png}

\end{frame}

\begin{frame}[t]
    \frametitle{4. Option Valuation}
    \framesubtitle{Determinants of Call Option Value}

    \begin{table}
        \centering
        \caption{Determinants of Call Option Value}
        \begin{tblr}{
            colspec = {Q[l,wd=5cm] Q[c,wd=3.cm]}
        }
        \toprule
        If this Variable Increases... & The Call Option... \\
        \midrule
        Stock Price, $S_0$ & Increases \\
        Exercise Price, $X$ & Decreases \\
        Volatility, $\sigma$ & Increases \\
        Time to Expiration, $T$ & Increases \\
        \bottomrule
        \end{tblr}
    \end{table}

\end{frame}

\begin{frame}[t]
    \frametitle{4. Option Valuation}
    \framesubtitle{NVDA calls - Change in Time to Expiry}

    \centering
    \includegraphics[width=0.8\textwidth]{figures/ch21_nvda_call3.png}

\end{frame}

\begin{frame}[t]
    \frametitle{4. Option Valuation}
    \framesubtitle{NVDA calls - Change in Volatility}

    \centering
    \includegraphics[width=0.8\textwidth]{figures/ch21_nvda_call4.png}

\end{frame}

\begin{frame}[t]
    \frametitle{4. Option Valuation}
    \framesubtitle{Black-Scholes-Merton}

    The value of a call option is given by the \textbf{Black-Scholes-Merton} formula:
    $$C = S_0N(d_1) - Ke^{-rT}N(d_2)$$
    Where
    $$d_1 = \frac{ln\left(\frac{S_0}{K}\right) + \left(r+\frac{\sigma^2}{2}\right)T}{\sigma\sqrt{T}}$$
    $$d_2 = d_1 - \sigma\sqrt{T}$$
    And $S_0$ is the spot price, $K$ is the strike price, 
    $r$ is the risk-free rate, $T$ is time to expiry, and $\sigma$ is the 
    implied volatility.   $N()$ is the cdf of the normal distribution.

\end{frame}

\begin{frame}[t]
    \frametitle{4. Option Valuation}
    \framesubtitle{Greeks}

    The \textbf{greeks} of an option are how much the option 
    value changes for a change in one of the input paramaters.

    \begin{table}
        \centering
        \caption{Option Greeks for \textbf{Call} Options}
        \begin{tblr}{
            colspec = {Q[l,wd=2cm] Q[c,wd=5.cm] Q[c,wd=2.cm] }
        }
        \toprule
        Greek & Measures Sensitivity to & Sign \\
        \midrule
        Delta ($\delta$) & Stock price ($S_0$) & Positive \\
        Gamma ($\gamma$) & Delta & Positive \\
        Theta ($\theta$) & Time to Expiry ($T$) & Negative\\
        Vega ($\nu$) & Implied Volatility & Positive \\
        \bottomrule
        \end{tblr}

    \end{table}

        \blfootnote{Note: For puts, delta is negative while gamma and vega have the same positive sign. 
        Theta is typically negative for both calls and puts (time decay)}

\end{frame}

\begin{practiceframe}[t]
    \frametitle{5. Practice}
    \framesubtitle{}

    \footnotesize
    In each of the following questions you are asked to compare two options.   The risk free interest rate for 
    all cases is 4\%.  Assume the stocks on which these options are written pay no dividends.\\
    \vspace{1em}
    \begin{enumerate}
        \item Which put option is written on the stock \textit{with the lower price}?   A, B, or not enough information given.\\
            \begin{table}
            \centering
            \begin{tblr}{
                colspec = {Q[l,wd=1cm] Q[c,wd=1.cm] Q[c,wd=1.cm] Q[c,wd=1.cm] Q[c,wd=1.cm] }
            }
            \toprule
            Put & T & X & $\sigma$ & Price \\
            \midrule
            A & 0.5 & 50 & 0.2 & \$10\\
            B & 0.5 & 50 & 0.25 & \$10\\
            \bottomrule
            \end{tblr}
            \end{table}
            
        \item Which put option is written on the stock \textit{with the lower price}?   A, B, or not enough information given.
            \begin{table}
            \centering
            \begin{tblr}{
                colspec = {Q[l,wd=1cm] Q[c,wd=1.cm] Q[c,wd=1.cm] Q[c,wd=1.cm] Q[c,wd=1.cm] }
            }
            \toprule
            Put & T & X & $\sigma$ & Price \\
            \midrule
            A & 0.5 & 50 & 0.2 & \$10\\
            B & 0.5 & 50 & 0.2 & \$12\\
            \bottomrule
            \end{tblr}
            \end{table}
            
    \end{enumerate}            

\end{practiceframe}

\begin{practiceframe}[t]
    \frametitle{5. Practice}
    \framesubtitle{}

    \footnotesize
    \begin{enumerate}
        \item[3. ] Which call option must have the \textit{lower time to expiration}?   A, B, or not enough information given.
            \begin{table}
            \centering
            \begin{tblr}{
                colspec = {Q[l,wd=1cm] Q[c,wd=1.cm] Q[c,wd=1.cm] Q[c,wd=1.cm] Q[c,wd=1.cm] }
            }
            \toprule
            Call & S & X & $\sigma$ & Price \\
            \midrule
            A & 50 & 50 & 0.2 & \$12\\
            B & 50 & 50 & 0.2 & \$10\\
            \bottomrule
            \end{tblr}
            \end{table}
            
        \item[4 .] Which call option is written on the stock \textit{with higher volatility}?   A, B, or not enough information given.
            \begin{table}
            \centering
            \begin{tblr}{
                colspec = {Q[l,wd=1cm] Q[c,wd=1.cm] Q[c,wd=1.cm] Q[c,wd=1.cm] Q[c,wd=1.cm] }
            }
            \toprule
            Call & T & S & X & Price \\
            \midrule
            A & 0.5 & 50 & 55 & \$10\\
            B & 0.5 & 50 & 50 & \$7\\
            \bottomrule
            \end{tblr}
            \end{table}
            
    \end{enumerate}            

\end{practiceframe}


\begin{practiceframe}[t]
    \frametitle{5. Practice}
    \framesubtitle{}

    \begin{enumerate}
        \item[5. ] Would you expect a \$1 increase in a call option's exercise price to lead to a decrease in the call 
        option's value of more or less than \$1?
        \item[6. ] Is a put option on a high-beta stock worth more than one on a low-beta stock?   The stocks have identical 
        firm specific risk.
        \item[7. ] If the time to expiration falls and the put price rises, what has happened to the put option's 
        implied volatility?
        \item[8. ] What will happen to the delta of a convertible bond as the stock price becomes very large?
    \end{enumerate}

\end{practiceframe}



\end{document}