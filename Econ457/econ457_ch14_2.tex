\documentclass{beamer}

\newcommand{\week}{Week 9-b}

\title{Fixed Income, continued}
\subtitle{Reference: Bodie et al, Ch 14}
\author{Econ 457}
\date{\week}

% Reference the shared preamble
\setbeamertemplate{frametitle}{
  \vspace{0.5em}
  \insertframetitle
  \par
  \vspace{0.5em}
  \hrule
  \vspace{0.3em}
  {\small\color{gray}\insertframesubtitle}
}

\setbeamertemplate{navigation symbols}{}
\setbeamertemplate{itemize item}{\textbullet} % main bullet: filled dot
\setbeamertemplate{itemize subitem}{\normalsize$\circ$} % sub-bullet: empty dot
\setbeamertemplate{itemize subsubitem}{\scriptsize--} % sub-sub-bullet: dash


% Font changes
\usepackage[scaled=0.92]{helvet}
\renewcommand{\familydefault}{\sfdefault}

% Packages
\usepackage{tikz}
\usepackage{booktabs}
\usepackage{xcolor}
\usepackage{array}           % Enhanced column types for tables
\usepackage{multirow}        % Spanning multiple rows in tables
\usepackage{makecell}        % Line breaks and formatting in table cells
\usepackage{siunitx}         % Proper formatting of numbers and units
\usepackage{amsmath}         % Enhanced math environments
\usepackage{amsfonts}        % Additional math fonts
\usepackage{amssymb}         % Additional math symbols
\usepackage{url}             % Better URL formatting
\usepackage{graphicx}        % Enhanced graphics support
\usepackage{tabularray}
\UseTblrLibrary{booktabs, siunitx, varwidth}
% For financial presentations specifically
\usepackage{eurosym}         % Euro symbol
\usepackage{textcomp}        % Additional text symbols
\usepackage{hyperref}        % Hyperlinks (should be loaded last)

% Define a footnote
\renewcommand{\footnoterule}{\vspace*{-3pt}\hrule width 2in height 0.4pt\vspace*{2.6pt}}

% Define a Foundation Slide
\newenvironment{foundframe}[1][t]{
    \setbeamercolor{background canvas}{bg=gray!8}
    \setbeamercolor{frametitle}{fg=gray!80!black,bg=gray!25}
    \setbeamercolor{framesubtitle}{fg=gray!70!black,bg=gray!15}
    \setbeamercolor{item}{fg=gray!80!black}
    \setbeamercolor{enumerate item}{fg=gray!80!black}
    
    % Modify the frametitle template for this frame type
    \setbeamertemplate{frametitle}{
        \vspace{0.5em}
        \begin{minipage}[t]{0.75\textwidth}
            \insertframetitle
            \par
            \vspace{0.5em}
            \hrule
            \vspace{0.3em}
            {\small\color{gray}\insertframesubtitle}
        \end{minipage}%
        \hfill
        \begin{minipage}[t]{0.2\textwidth}
            \raggedleft
            \colorbox{gray!30}{%
                \scriptsize\bfseries\color{gray!80!black}%
                   \hspace{3pt}\begin{tabular}{c}Foundation\\Material\end{tabular}\hspace{3pt}%
            }
        \end{minipage}
        \vspace{0.3em}
    }
    
    \begin{frame}[#1]
}{
    \end{frame}
}

% Define Practice Slide
\newenvironment{practiceframe}[1][t]{
    \setbeamercolor{background canvas}{bg=white}
    \setbeamercolor{frametitle}{fg=blue!80!black,bg=blue!15}
    \setbeamercolor{framesubtitle}{fg=blue!70!black,bg=blue!10}
    \setbeamercolor{item}{fg=blue!80!black}
    \setbeamercolor{enumerate item}{fg=blue!80!black}
    \setbeamercolor{normal text}{fg=blue!90!black}
    
    % Modify the frametitle template for this frame type
    \setbeamertemplate{frametitle}{
        \vspace{0.5em}
        \begin{minipage}[t]{0.75\textwidth}
            \insertframetitle
            \par
            \vspace{0.5em}
            \hrule
            \vspace{0.3em}
            {\small\color{blue!70!black}\insertframesubtitle}
        \end{minipage}%
        \hfill
        \begin{minipage}[t]{0.2\textwidth}
            \raggedleft
            \colorbox{blue!20}{%
                \scriptsize\bfseries\color{blue!80!black}%
                   \hspace{3pt}\begin{tabular}{c}Practice\\Questions\end{tabular}\hspace{3pt}%
            }
        \end{minipage}
        \vspace{0.3em}
    }
    
    \begin{frame}[#1]
}{
    \end{frame}
}

% Define Excel Slide
\newenvironment{excelframe}[1][t]{
    \setbeamercolor{background canvas}{bg=white}
    \setbeamercolor{frametitle}{fg=blue!80!black,bg=blue!15}
    \setbeamercolor{framesubtitle}{fg=blue!70!black,bg=blue!10}
    \setbeamercolor{item}{fg=blue!80!black}
    \setbeamercolor{enumerate item}{fg=blue!80!black}
    \setbeamercolor{normal text}{fg=blue!90!black}
    
    % Modify the frametitle template for this frame type
    \setbeamertemplate{frametitle}{
        \vspace{0.5em}
        \begin{minipage}[t]{0.75\textwidth}
            \insertframetitle
            \par
            \vspace{0.5em}
            \hrule
            \vspace{0.3em}
            {\small\color{blue!70!black}\insertframesubtitle}
        \end{minipage}%
        \hfill
        \begin{minipage}[t]{0.2\textwidth}
            \raggedleft
            \colorbox{green!10}{%
                \scriptsize\bfseries\color{blue!80!black}%
                   \hspace{3pt}\begin{tabular}{c}MS Excel\end{tabular}\hspace{3pt}%
            }
        \end{minipage}
        \vspace{0.3em}
    }
    
    \begin{frame}[#1]
}{
    \end{frame}
}

% Define Caution Slide
\newenvironment{cautionframe}[1][t]{
    \setbeamercolor{background canvas}{bg=white}
    \setbeamercolor{frametitle}{fg=blue!80!black,bg=blue!15}
    \setbeamercolor{framesubtitle}{fg=blue!70!black,bg=blue!10}
    \setbeamercolor{item}{fg=blue!80!black}
    \setbeamercolor{enumerate item}{fg=blue!80!black}
    \setbeamercolor{normal text}{fg=blue!90!black}
    
    % Modify the frametitle template for this frame type
    \setbeamertemplate{frametitle}{
        \vspace{0.5em}
        \begin{minipage}[t]{0.75\textwidth}
            \insertframetitle
            \par
            \vspace{0.5em}
            \hrule
            \vspace{0.3em}
            {\small\color{blue!70!black}\insertframesubtitle}
        \end{minipage}%
        \hfill
        \begin{minipage}[t]{0.2\textwidth}
            \raggedleft
            \colorbox{red!10}{%
                \scriptsize\bfseries\color{blue!80!black}%
                   \hspace{3pt}\begin{tabular}{c}Caution\end{tabular}\hspace{3pt}%
            }
        \end{minipage}
        \vspace{0.3em}
    }
    
    \begin{frame}[#1]
}{
    \end{frame}
}

% Add to footnotes
\makeatletter
\newcommand\blfootnote[1]{%
  \begingroup
  \renewcommand\thefootnote{}%
  \renewcommand\@makefntext[1]{\raggedright\leftskip=0pt ##1}%
  \footnote{\scriptsize #1}%
  \addtocounter{footnote}{-1}%
  \endgroup
}
\makeatother

% Set the footer -- change 
\setbeamertemplate{footline}{
  \leavevmode%
  \vspace{2ex}
  \hbox{%
    % Left box: Econ 457
    \begin{beamercolorbox}[wd=.4\paperwidth,ht=2.5ex,dp=1ex,left]{author in head/foot}%
      \hspace{1em}Econ 457
    \end{beamercolorbox}%
    % Middle box: Week
    \begin{beamercolorbox}[wd=.2\paperwidth,ht=2.5ex,dp=1ex,center]{date in head/foot}%
      \centering\week
    \end{beamercolorbox}%
    % Right box: Slide numbers
    \begin{beamercolorbox}[wd=.4\paperwidth,ht=2.5ex,dp=1ex,center]{date in head/foot}%
      \hfill\insertframenumber{} 
    \end{beamercolorbox}%
  }%
  \vskip0pt%
}

\begin{document}

\frame{\titlepage}

\begin{frame}
    \frametitle{Outline}

    Review of Bond Yields, from previous class\\

    \vspace{1em}
    
    \begin{enumerate}
        \item Geometric Series
        \item Perpetuities and Coupon Bonds
        \item Default Risk and Corporate Bonds
        \item Prepayment Risk and Mortgage Bonds
        \item Indexed bonds and TIPS
        \item Practice
    \end{enumerate}

\end{frame}

\begin{frame}[t]
    \frametitle{0. Yield - Review}
    \framesubtitle{}

    The bond pricing formula is:
     $$
    \text{Bond Price} = \sum_{t=1}^{T} \frac{C}{(1 + y)^t} + \frac{FV}{(1 + y)^T}
    $$

    The bond coupons ($C$) and face value ($FV$) are determined at the time the bond 
    is issued and generally do not change (TIPS are the exception, see last section of this 
    lecture.)\\
    \vspace{1em}
    The bond price and the yield ($y$) are set by the market.   These can and will change frequently 
    due to changes in investor preferences and expectations.   

\end{frame}

\begin{frame}[t]
    \frametitle{0. Yield - Review}
    \framesubtitle{}

    \vspace{-1em}
    \begin{itemize}
        \item The \textbf{yield to maturity} is the single interest rate that 
        makes the present value of a bond's payments equal to its price.   It is 
        the \textit{expected} rate of return that will be earned on the bond if it 
        is bought now and held to maturity, assuming all coupons are reinvested and earn 
        the yield to maturiy.
        \item The \textbf{current yield} is the annual coupon payment divided by the bond 
        price.   As with yield to maturity, this is an indicator of \textit{expected} return of the bond.
        \item The \textbf{holding period return} for a bond is defined in the normal way: 
        income return plus capital gain return.
        \item The \textbf{realized compound return} for a bond incorporates reinvestment of coupon 
        payments.   It will equal the yield to maturity only in the case that 
        coupon payments are reinvested at the same yield to maturity.   If the reinvestment rate 
        changes through the life of the bond (likely) it may not equal the yield.
    \end{itemize}

\end{frame}

\begin{frame}[t]
    \frametitle{0. Yield - Review}
    \framesubtitle{Example}

    Consider a 2-year bond issued at par that has a 10\% annual coupon.  
    \vspace{-1em}
        \begin{itemize}
            \item Verify that if the first coupon is reinvested at a 10\% rate, then 
            the annualized rate of return on the bond is equal to 10\%.
            \item Remember that if the coupon rate is equal to the yield, the price is 
            equal to the par value.
            \item Now, imagine that at the end of year 1, market rates drop to 5\%.  
            \vspace{-1em}
            \begin{itemize}
                \item What is the realized compound return on this bond if the first coupon 
                is reinvested at a 5\% interest rate?
                \item What is the price of the bond at the end of year 1 if the yield falls to 5\%?   (Note: requires Excel or financial calculator)
                \item What is the holding period return on the bond in the first year?   In the 
                second year? (Note: requires Excel or financial calculator)
            \end{itemize}
        \end{itemize}

\end{frame}

\begin{frame}[t]
    \frametitle{0. Yield - Review}
    \framesubtitle{Example}

    \centering
    \includegraphics[width=\textwidth]{figures/ch14_2_xls.png}

\end{frame}

\begin{foundframe}[t]
    \frametitle{1. Geometric Series}
    \framesubtitle{}

    A \textbf{geometric series} is a sum of terms where each term is a constant multiple of the previous term.
    \vspace{1em}

    \textbf{General Form:}
    $$S = a + ar + ar^2 + ar^3 + \ldots = \sum_{n=0}^{\infty} ar^n$$
    where $a$ is the first term and $r$ is the common ratio.
    \vspace{1em}
    
    \textbf{Convergence:} If $|r| < 1$, the infinite series converges to:
    $$S = \frac{a}{1-r}$$
    \vspace{1em}

\end{foundframe}

\begin{foundframe}[t]
    \frametitle{1. Geometric Series}
    \framesubtitle{Proof of Convergence}

    If $|r| < 1$, then $S = \sum_{n=0}^{\infty} ar^n = \frac{a}{1-r}$\\
    \vspace{1em}

    \textbf{Proof:} Multiply both sides by $r$:
    $$rS = ar + ar^2 + ar^3 + \ldots$$
    \vspace{0.5em}

    Subtract: $S - rS = a $ (note that all the other terms cancel)
    \vspace{0.5em}

    Factor out the $S$ on the left hand side: $S(1-r) = a$
    \vspace{0.5em}

    Therefore: $S = \frac{a}{1-r}$
    \vspace{0.5em}

\end{foundframe}

\begin{frame}[t]
    \frametitle{2. Perpetuities and Coupon Bonds}
    \framesubtitle{Value of Perpetuity}

    The present value of perpetuity can be represented by a geometric series:
    $$PV(\text{perpetuity}) = \frac{c}{(1+r)} + \frac{c}{(1+r)^2} + \frac{c}{(1+r)^3} + \ldots$$
    Where $c$ is the payment made in each period and $r$ is the appropriate discount rate.\\
    \vspace{1em}
    
    Assume that $r > 0$ and let $d = \frac{1}{(1+r)} < 1$. Note that the first payment is discounted, 
    so we need to start the series at $i=1$ instead of $i=0$.\\
    \vspace{1em}
    Continued...

\end{frame}

\begin{frame}[t]
    \frametitle{2. Perpetuities and Coupon Bonds}
    \framesubtitle{Value of Perpetuity, Continued}

    ...Contiued.\\
    \vspace{1em}
    With the expression for $d$ and the adjustment to $i=1$ start, we can now use the same proof as the 
    previous slide:
    \footnotesize
    \begin{align*}
    PV(perpetuity) &= c \sum_{i=1}^{\infty} d^i\\
                    &= c \left(\sum_{i=0}^{\infty} d^i - 1\right)\\
                      &= c \left(\frac{1}{1-d} - 1\right) = c\frac{d}{(1-d)}\\
                      &= c \frac{(\frac{1}{1+r})}{(1-\frac{1}{1+r})}\\
                      &= \frac{c}{r}
    \end{align*}

\end{frame}

\begin{frame}[t]
    \frametitle{2. Perpetuities and Coupon Bonds}
    \framesubtitle{Value of Finite Coupon Stream}

    A similar approach gives the following formula for a finite geometric series
    (as before, write out all the terms to $(N-1)$, then multiply by $r$, then subtract and see what cancels.):
    $$\sum_{i=1}^{N} d^i = \frac{1-d^N}{1-d}$$

    We can use this finite geometric series formula to define a pricing formula for a bond that pays coupon
     $C$ for $N$ periods.

    $$PV(\text{coupons}) = \frac{C}{(1+r)} + \frac{C}{(1+r)^2} + \frac{C}{(1+r)^3} + \ldots + \frac{C}{(1+r)^N}$$

\end{frame}

\begin{frame}[t]
    \frametitle{2. Perpetuities and Coupon Bonds}
    \framesubtitle{Bond Valuation Formula}

    Again, substitute $d=\frac{1}{1+r}$ and the present value of the coupons 
    \begin{align*}
        PV(\text{coupons}) &= C \cdot \frac{d(1-d^N)}{1-d}\\
                            &= C \cdot \frac{\frac{1}{1+r}(1-\frac{1}{(1+r)^N})}{1-\frac{1}{1+r}}\\
                            &= C \cdot \frac{1-\frac{1}{(1+r)^N}}{r}
    \end{align*}

    \textbf{Complete Bond Price:} Coupons plus principal repayment
    $$P = PV(\text{coupons}) + PV(\text{principal})$$
    $$P = C \cdot \frac{1-\frac{1}{(1+r)^N}}{r} + \frac{F}{(1+r)^N}$$
    \vspace{1em}

    \end{frame}

\begin{frame}[t]
    \frametitle{2. Perpetuities and Coupon Bonds}
    \framesubtitle{Bond Valuation Formula}

     \textbf{Standard Form of Bond Price Equation:}
    $$P = C \cdot \frac{1-(1+r)^{-N}}{r} + F(1+r)^{-N}$$
    \vspace{1em}

    This was derived using Geometric Series, along with a clever definitions of 
    $d$ to make it look like a geometric series.\\
    \vspace{1em}
    Note that if the coupon rate is equal to the yield (i.e. $C = cFV$ and $c=r$)
    then $P=FV$.

\end{frame}

\begin{frame}[t]
    \frametitle{3. Default Risk and Corporate Bonds}
    \framesubtitle{Security}

    In a bankruptcy proceeding corporate bondholders are paid before equity holders.\\
    \vspace{1em}
    \textbf{Security Hierarchy:}
    \begin{enumerate}
        \item \textit{Senior Secured Debt} - backed by specific collateral
        \item \textit{Senior Unsecured Debt} - general claim on assets
        \item \textit{Subordinated Debt} - paid after senior debt
        \item \textit{Preferred Stock} - hybrid security
        \item \textit{Common Stock} - residual claim
    \end{enumerate}
    \vspace{1em}

\end{frame}

\begin{frame}[t]
    \frametitle{3. Default Risk and Corporate Bonds}
    \framesubtitle{Credit Ratings}

    \textbf{Major Rating Agencies:} Moody's, S\&P, Fitch\\
    \vspace{1em}

    \scriptsize
    \textbf{Rating Categories:}
    \begin{center}
    \begin{tabular}{|c|c|c|l|}
    \hline
    \textbf{Moody's} & \textbf{S\&P} & \textbf{Quality} & \textbf{Description} \\
    \hline
    Aaa & AAA & \multirow{4}{*}{Investment} & Highest quality \\
    Aa & AA & & High quality \\
    A & A & & Upper medium grade \\
    Baa & BBB & & Medium grade \\
    \hline
    Ba & BB & \multirow{5}{*}{Speculative} & Lower medium grade \\
    B & B & & Speculative \\
    Caa & CCC & & Poor standing \\
    Ca & CC & & Highly speculative \\
    C & C/D & & Default \\
    \hline
    \end{tabular}
    \end{center}
    \vspace{1em}

    \normalsize
    \textbf{Investment Grade vs. High Yield:}
    \begin{itemize}
        \item BBB-/Baa3 and above = Investment Grade
        \item BB+/Ba1 and below = High Yield ("Junk Bonds")
    \end{itemize}

\end{frame}

\begin{frame}[t]
    \frametitle{3. Default Risk and Corporate Bonds}
    \framesubtitle{Credit Ratings}

    \begin{columns}
        \begin{column}{0.3\textwidth}
            \centering
            \includegraphics[width=\textwidth]{figures/ch14_2_sp.png}
        \end{column}
        \begin{column}{0.7\textwidth}
            \vspace{1em}
            \includegraphics[width=\textwidth]{figures/ch14_2_sp_2.png}
        \end{column}
    \end{columns}

\end{frame}

\begin{frame}[t]
    \frametitle{3. Default Risk and Corporate Bonds}
    \framesubtitle{Credit Ratings}

    \centering
    \includegraphics[width=0.8\textwidth]{figures/ch14_sp_downgrade.png}

\end{frame}


\begin{frame}[t]
    \frametitle{3. Default Risk and Corporate Bonds}
    \framesubtitle{Default Risk - Example}
    
    \textbf{Example 2}: Start with a 10-year 9\% coupon corporate bond with a price of \$750.   
    Yield to maturity = 12\%.
    What if there is a 20\% chance of the final payment not being made?

    \begin{itemize}
        \item The \textit{expected value} of the final payment is again only \$800.
        \item Instead of substituting $\mathbb{E}[FV]$ for $FV$ in the pricing equation, 
        subtract $\frac{\$100}{(1+y)^{20}}$ from the bond price.
        \item Recalculate the yield using the new bond price and $FV=\$1,000$.   
        \item Yield to maturity is now approx 14\%.
    \end{itemize}

    Continued...

\end{frame}

\begin{frame}[t]
\frametitle{3. Default Risk and Corporate Bonds}
\framesubtitle{Default Risk - Example}


    ...Continued.  
    What if there is a 20\% chance of no payment on the final payment \textit{and} all the coupons?  

    \begin{itemize}
        \item Subtract \texttt{PRICE(today,10y,0.009,14,10,2)} from the bond price.
        \item Recalculate the yield using the new bond price and $FV=\$1,000$
        \item Yield to maturity is now approx 15.5\%.
    \end{itemize}

    \vspace{1em}
    This example illustrates how default risk actually works:  default risk causes the expected value of payments to decline, which leads to lower bond prices, and therefore higher yields.

\end{frame}



\begin{frame}[t]
    \frametitle{3. Default Risk and Corporate Bonds}
    \framesubtitle{Spreads}

    \centering
    \includegraphics[width=\textwidth]{figures/ch14_2_hy.png}
    
\end{frame}

\begin{frame}[t]
\frametitle{3. Default Risk and Corporate Bonds}
\framesubtitle{Spread}

    The \textbf{spread} on a corporate bond is defined to be the difference in yields between a corporate bond
    and a Treasury bond of similar maturity.\\
    \vspace{1em}
    While the spread is primarily due to the default risk, it also may reflect differences in liquidity.  
    Corporate bonds are typically less liquid than US Treasuries (i.e. harder to sell when you need cash).
    Investors may require a discount (lower price, higher yield) to be compensated for lower liquidity.

\end{frame}

\begin{frame}[t]
\frametitle{3. Default Risk and Corporate Bonds}
\framesubtitle{Option Adjusted Spread}

    Spreads on corporate bonds could also be due to embedded options.
    Some corporate bonds have embedded call options.  The issuer can repay before the maturity date.
    This option is valuable to the bond issuer, and therefore requires a lower price (higher yield) for the bond purchaser.   

    \vspace{1em}
    \textbf{Option Adjusted Spread (OAS):}
    \begin{itemize}
        \item Removes the value of embedded options from the spread
        \item Shows the "pure" credit and liquidity risk premium
        \item Calculated using interest rate trees and Monte Carlo simulation
        \item Formula: Nominal Spread - Option Cost = OAS
    \end{itemize}

\end{frame}

\begin{frame}[t]
    \frametitle{3. Default Risk and Corporate Bonds}
    \framesubtitle{Spreads}

    \includegraphics[width=\textwidth]{figures/ch14_2_hyspreads.png}

\end{frame}

\begin{frame}[t]
    \frametitle{4. Prepayment Risk and Mortgages}
    \framesubtitle{Conforming Mortgages}

    In the United States the standard mortgage has the following characteristics:
    \begin{itemize}
        \item 30-year term
        \item 20\% downpayment, 80\% loan-to-value ratio (LTV)
        \item Fixed-rate, self-amortizing (i.e. fixed monthly payments, no balloon payment at maturity)
        \item Ability to refinance (prepay at par) anytime
    \end{itemize}
    Mortgages that meet these characteristics are referred to as 'conforming' mortgages. 
    Conforming mortgages can be purchased by the GSEs (Fannie Mae and Freddie Mac).

\end{frame}


\begin{frame}[t]
    \frametitle{4. Prepayment Risk and Mortgages}
    \framesubtitle{Mortgage Debt Outstanding}

    The median home price in the United States \$419,000 at the end of 2024.
    The average sales price of new homes was \$500,000 in June 2025.   

    \centering
    \includegraphics[width=0.7\textwidth]{figures/ch14_2_mtddebt.png}

\end{frame}

\begin{frame}[t]
    \frametitle{4. Prepayment Risk and Mortgages}
    \framesubtitle{Mortgage Payments}

    Fixed-rate mortgages repay principal each month, instead of requiring a balloon payment at the end of the loan.\\

    \centering
    \includegraphics[width=0.7\textwidth]{figures/ch14_2_mtg_payments.png}

\end{frame}

\begin{frame}[t]
    \frametitle{4. Prepayment Risk and Mortgages}
    \framesubtitle{Mortgage Leverage}

    \begin{columns}
        \begin{column}{0.4\textwidth}
            The typical mortgage requires a 20\% downpayment.
            The remaining 80\% of the cost of the house is borrowed.\\
            \vspace{1em}
            The leverage ratio is 5:1 (total cost / amount of equity).
        \end{column}
        \begin{column}{0.6\textwidth}
            \centering
            \includegraphics[width=\textwidth]{figures/ch6_2_mtg.png}
        \end{column}
    \end{columns}

\end{frame}

\begin{frame}
    \frametitle{4. Prepayment Risk and Mortgages}
    \framesubtitle{Mortgage Leverage}

    Example: 
    \begin{itemize}
        \item You buy a house for \$1,000,000.   
        \item You make a 20\% downpayment (\$200,000)
        \item You finance the remainding \$800,000 with a 7\% mortgage.
    \end{itemize}

    \begin{table}
        \caption{Example Home Ownership Returns}
        \begin{tabular}{lccc}
        \toprule
        Scenario & Price Change & Interest Cost & Return on Equity \\
        \midrule
        HPI +10\% & \$100,000 & \$56,000 & 22.0\% \\
        HPI +20\% & \$200,000 & \$56,000 & 72.0\% \\
        HPI -10\% & -\$100,000 & \$56,000 & -78.0\% \\
        \bottomrule
    \end{tabular}
    \end{table}
\end{frame}

\begin{frame}[t]
    \frametitle{4. Prepayment Risk and Mortgages}
    \framesubtitle{Mortgage - Refinancing and Prepayments}

    Mortgages can be prepaid at any time.  
    In fact, when interest rates fall it is very common for homeowners to 'refinance' their mortgages. 
    Essentially, they are prepaying the old lender and taking out a replacement loan from a new lender.\\
    \vspace{1em}
    While this is very advantageous for the homeowner (they get a lower interest rate), 
    it is equally \textit{disadvantagous} for the lender.   They are losing an asset that was paying a high interest rate, 
    which they are unable to replace because market rates are now lower.   
    This is know as \textbf{prepayment risk} for the lender.

\end{frame}

\begin{frame}[t]
    \frametitle{4. Prepayment Risk and Mortgages}
    \framesubtitle{Mortgage - Refinancing and Prepayments}

    \centering
    \includegraphics[width=0.8\textwidth]{figures/ch14_2_mtg_prepayment.png}

\end{frame}

\begin{frame}[t]
    \frametitle{4. Prepayment Risk and Mortgages}
    \framesubtitle{Mortgage - Refinancing and Prepayments}

    Because this is \textit{disadvantagous} to the lender, the lender will pay a lower 
    price for this bond.   As a consequence, the bond will have a higher yield.\\
    \vspace{1em}
    Mortgages are an example of a \textbf{callable} bond, where the homeowner has the 
    option to call the bond at par.   Some corporate bonds also have embedded call options,
    which give the issuer the right to repay the bond at a specified price.   Even though the 
    price is usually somewhat above par, this option is still valuable to the issuer (and 
    disadvanteagous to the lender.)

\end{frame}

\begin{frame}[t]
    \frametitle{4. Prepayment Risk and Mortgages}
    \framesubtitle{Mortgage - Refinancing and Prepayments}

    \centering
    \includegraphics[width=0.8\textwidth]{figures/ch14_2.png}

\end{frame}

\begin{frame}[t]
    \frametitle{5. Indexed Bonds and TIPS}
    \framesubtitle{Treasury Inflation-Protected Securities (TIPS)}

    \textbf{Definition:} TIPS are U.S. Treasury bonds that adjust both principal and interest payments for inflation
    \vspace{1em}

    \textbf{Key Features:}
    \begin{itemize}
        \item \textbf{Principal Adjustment:} Face value increases with Consumer Price Index (CPI)
        \item \textbf{Fixed Real Rate:} Coupon rate is fixed, but applied to adjusted principal
        \item \textbf{Inflation Protection:} At maturity, receive greater of original or adjusted principal
        \item \textbf{Maturities:} Available in 5, 10, and 30-year terms
    \end{itemize}
    \vspace{1em}

\end{frame}

\begin{frame}[t]
    \frametitle{5. Indexed Bonds and TIPS}
    \framesubtitle{Treasury Inflation-Protected Securities (TIPS)}

    \textbf{Example:}
    \begin{itemize}
        \item \$1,000 TIPS with 2\% coupon, 3\% inflation
        \item Year 1: Principal = \$1,030, Interest = \$20.60
        \item Year 2: Principal = \$1,061, Interest = \$21.22
    \end{itemize}
    \vspace{1em}

  
\end{frame}

\begin{frame}[t]
    \frametitle{5. Indexed Bonds and TIPS}
    \framesubtitle{TIPS - Yields over time}

    \centering
    \includegraphics[width=0.8\textwidth]{figures/ch14_2_tips_1.png}

\end{frame}

\begin{frame}[t]
    \frametitle{5. Indexed Bonds and TIPS}
    \framesubtitle{TIPS - Real Yields}

    Intuition for TIPS yields:\\
    \vspace{1em}

    The return on TIPS will be comprised of the interest payments, 
    the principal payment at maturity, AND the expected increase in the principal value.\\
    \vspace{1em}

    The yield on TIPS assumes no increase in principal.   This is referred to as the 'real yield'.   
    If the realized increase in principal is equal to the breakeven inflation rate, 
    then the return on TIPS will be exactly equal to the return on nominal Treasury bonds.   
    If the increase in principal is above (below) the breakeven inflation rate, 
    then TIPS will outperform (underperfom) nominal Treasury bonds.

\end{frame}

\begin{frame}[t]
    \frametitle{5. Indexed Bonds and TIPS}
    \framesubtitle{TIPS - Breakeven inflation rate}

    \vspace{-2em}
    $$\text{Breakeven Inflation} = \text{Nominal Treasury Yield} - \text{TIPS Real Yield}$$

    \textbf{Interpretation:}
    \begin{itemize}
        \item Market's expectation of average inflation over the bond's life
        \item If actual inflation is greater than breakeven: TIPS outperform nominal bonds
        \item If actual inflation is less than breakeven: Nominal bonds outperform TIPS
        \item Indifference point between TIPS and nominal Treasury investing
    \end{itemize}

    \textbf{Example:} 10-year Treasury yield: 4.5\%, 10-year TIPS yield: 2.0\%, 
    Breakeven inflation: 4.5\% - 2.0\% = 2.5\%

\end{frame}

\begin{frame}[t]
    \frametitle{5. Indexed Bonds and TIPS}
    \framesubtitle{TIPS - Breakevens over time}

    \centering
    \includegraphics[width=0.8\textwidth]{figures/ch14_2_tips_2.png}

\end{frame}

\begin{practiceframe}[t]
    \frametitle{6. Practice}
    \framesubtitle{}

    \begin{enumerate}
        \item Two bonds have identical times to maturity and coupon rates.   One is 
        callable at 105, the other at 110.   Which has the higher yield to maturity?  Why?
        \item Consider a bond with a 10\% coupon and a yield to maturity of 8\%.   If the 
        bond's yield to remains constant, then in one year will the bond price be higher, 
        lower, or unchanged?   Why?
        \item Consider a bond paying a coupon rate of 10\% per year semiannually when the 
        market interest rate is only 4\% per half-year.   The bond has three years until 
        maturity.   (a) Find the bond's price today, (b) find the bond's price six 
        months from now (assuming now change in market interest rate), (b) What is the total 
        (6-month) return on the bond?
    \end{enumerate}

\end{practiceframe}

\begin{practiceframe}[t]
    \frametitle{6. Practice}
    \framesubtitle{}

    \begin{enumerate}[4]
        \item Fill in the table below for the following zero-coupon bonds, all of 
        which have par values of \$1,000
        \begin{table}
                \begin{tblr}{
                    colspec = {lcc},
                    hlines,
                    vlines
                }
                Price & Maturity & Yield to Maturity \\
                \$400 & 20 & -\\
                \$500 & 20 & -\\
                \$500 & 10 & - \\
                - & 10 & 10\% \\
                - & 10 & 8\%\\
                \$400 & - & 8\%\\
            \end{tblr}
        \end{table}
    \end{enumerate}

\end{practiceframe}


\end{document}