\documentclass{beamer}

\newcommand{\week}{Week 10-b}

\title{Fixed Income, continued}
\subtitle{Reference: Bodie et al, Ch 16}
\author{Econ 457}
\date{\week}

% Reference the shared preamble
\setbeamertemplate{frametitle}{
  \vspace{0.5em}
  \insertframetitle
  \par
  \vspace{0.5em}
  \hrule
  \vspace{0.3em}
  {\small\color{gray}\insertframesubtitle}
}

\setbeamertemplate{navigation symbols}{}
\setbeamertemplate{itemize item}{\textbullet} % main bullet: filled dot
\setbeamertemplate{itemize subitem}{\normalsize$\circ$} % sub-bullet: empty dot
\setbeamertemplate{itemize subsubitem}{\scriptsize--} % sub-sub-bullet: dash


% Font changes
\usepackage[scaled=0.92]{helvet}
\renewcommand{\familydefault}{\sfdefault}

% Packages
\usepackage{tikz}
\usepackage{booktabs}
\usepackage{xcolor}
\usepackage{array}           % Enhanced column types for tables
\usepackage{multirow}        % Spanning multiple rows in tables
\usepackage{makecell}        % Line breaks and formatting in table cells
\usepackage{siunitx}         % Proper formatting of numbers and units
\usepackage{amsmath}         % Enhanced math environments
\usepackage{amsfonts}        % Additional math fonts
\usepackage{amssymb}         % Additional math symbols
\usepackage{url}             % Better URL formatting
\usepackage{graphicx}        % Enhanced graphics support
\usepackage{tabularray}
\UseTblrLibrary{booktabs, siunitx, varwidth}
% For financial presentations specifically
\usepackage{eurosym}         % Euro symbol
\usepackage{textcomp}        % Additional text symbols
\usepackage{hyperref}        % Hyperlinks (should be loaded last)

% Define a footnote
\renewcommand{\footnoterule}{\vspace*{-3pt}\hrule width 2in height 0.4pt\vspace*{2.6pt}}

% Define a Foundation Slide
\newenvironment{foundframe}[1][t]{
    \setbeamercolor{background canvas}{bg=gray!8}
    \setbeamercolor{frametitle}{fg=gray!80!black,bg=gray!25}
    \setbeamercolor{framesubtitle}{fg=gray!70!black,bg=gray!15}
    \setbeamercolor{item}{fg=gray!80!black}
    \setbeamercolor{enumerate item}{fg=gray!80!black}
    
    % Modify the frametitle template for this frame type
    \setbeamertemplate{frametitle}{
        \vspace{0.5em}
        \begin{minipage}[t]{0.75\textwidth}
            \insertframetitle
            \par
            \vspace{0.5em}
            \hrule
            \vspace{0.3em}
            {\small\color{gray}\insertframesubtitle}
        \end{minipage}%
        \hfill
        \begin{minipage}[t]{0.2\textwidth}
            \raggedleft
            \colorbox{gray!30}{%
                \scriptsize\bfseries\color{gray!80!black}%
                   \hspace{3pt}\begin{tabular}{c}Foundation\\Material\end{tabular}\hspace{3pt}%
            }
        \end{minipage}
        \vspace{0.3em}
    }
    
    \begin{frame}[#1]
}{
    \end{frame}
}

% Define Practice Slide
\newenvironment{practiceframe}[1][t]{
    \setbeamercolor{background canvas}{bg=white}
    \setbeamercolor{frametitle}{fg=blue!80!black,bg=blue!15}
    \setbeamercolor{framesubtitle}{fg=blue!70!black,bg=blue!10}
    \setbeamercolor{item}{fg=blue!80!black}
    \setbeamercolor{enumerate item}{fg=blue!80!black}
    \setbeamercolor{normal text}{fg=blue!90!black}
    
    % Modify the frametitle template for this frame type
    \setbeamertemplate{frametitle}{
        \vspace{0.5em}
        \begin{minipage}[t]{0.75\textwidth}
            \insertframetitle
            \par
            \vspace{0.5em}
            \hrule
            \vspace{0.3em}
            {\small\color{blue!70!black}\insertframesubtitle}
        \end{minipage}%
        \hfill
        \begin{minipage}[t]{0.2\textwidth}
            \raggedleft
            \colorbox{blue!20}{%
                \scriptsize\bfseries\color{blue!80!black}%
                   \hspace{3pt}\begin{tabular}{c}Practice\\Questions\end{tabular}\hspace{3pt}%
            }
        \end{minipage}
        \vspace{0.3em}
    }
    
    \begin{frame}[#1]
}{
    \end{frame}
}

% Define Excel Slide
\newenvironment{excelframe}[1][t]{
    \setbeamercolor{background canvas}{bg=white}
    \setbeamercolor{frametitle}{fg=blue!80!black,bg=blue!15}
    \setbeamercolor{framesubtitle}{fg=blue!70!black,bg=blue!10}
    \setbeamercolor{item}{fg=blue!80!black}
    \setbeamercolor{enumerate item}{fg=blue!80!black}
    \setbeamercolor{normal text}{fg=blue!90!black}
    
    % Modify the frametitle template for this frame type
    \setbeamertemplate{frametitle}{
        \vspace{0.5em}
        \begin{minipage}[t]{0.75\textwidth}
            \insertframetitle
            \par
            \vspace{0.5em}
            \hrule
            \vspace{0.3em}
            {\small\color{blue!70!black}\insertframesubtitle}
        \end{minipage}%
        \hfill
        \begin{minipage}[t]{0.2\textwidth}
            \raggedleft
            \colorbox{green!10}{%
                \scriptsize\bfseries\color{blue!80!black}%
                   \hspace{3pt}\begin{tabular}{c}MS Excel\end{tabular}\hspace{3pt}%
            }
        \end{minipage}
        \vspace{0.3em}
    }
    
    \begin{frame}[#1]
}{
    \end{frame}
}

% Define Caution Slide
\newenvironment{cautionframe}[1][t]{
    \setbeamercolor{background canvas}{bg=white}
    \setbeamercolor{frametitle}{fg=blue!80!black,bg=blue!15}
    \setbeamercolor{framesubtitle}{fg=blue!70!black,bg=blue!10}
    \setbeamercolor{item}{fg=blue!80!black}
    \setbeamercolor{enumerate item}{fg=blue!80!black}
    \setbeamercolor{normal text}{fg=blue!90!black}
    
    % Modify the frametitle template for this frame type
    \setbeamertemplate{frametitle}{
        \vspace{0.5em}
        \begin{minipage}[t]{0.75\textwidth}
            \insertframetitle
            \par
            \vspace{0.5em}
            \hrule
            \vspace{0.3em}
            {\small\color{blue!70!black}\insertframesubtitle}
        \end{minipage}%
        \hfill
        \begin{minipage}[t]{0.2\textwidth}
            \raggedleft
            \colorbox{red!10}{%
                \scriptsize\bfseries\color{blue!80!black}%
                   \hspace{3pt}\begin{tabular}{c}Caution\end{tabular}\hspace{3pt}%
            }
        \end{minipage}
        \vspace{0.3em}
    }
    
    \begin{frame}[#1]
}{
    \end{frame}
}

% Add to footnotes
\makeatletter
\newcommand\blfootnote[1]{%
  \begingroup
  \renewcommand\thefootnote{}%
  \renewcommand\@makefntext[1]{\raggedright\leftskip=0pt ##1}%
  \footnote{\scriptsize #1}%
  \addtocounter{footnote}{-1}%
  \endgroup
}
\makeatother

% Set the footer -- change 
\setbeamertemplate{footline}{
  \leavevmode%
  \vspace{2ex}
  \hbox{%
    % Left box: Econ 457
    \begin{beamercolorbox}[wd=.4\paperwidth,ht=2.5ex,dp=1ex,left]{author in head/foot}%
      \hspace{1em}Econ 457
    \end{beamercolorbox}%
    % Middle box: Week
    \begin{beamercolorbox}[wd=.2\paperwidth,ht=2.5ex,dp=1ex,center]{date in head/foot}%
      \centering\week
    \end{beamercolorbox}%
    % Right box: Slide numbers
    \begin{beamercolorbox}[wd=.4\paperwidth,ht=2.5ex,dp=1ex,center]{date in head/foot}%
      \hfill\insertframenumber{} 
    \end{beamercolorbox}%
  }%
  \vskip0pt%
}

\begin{document}

\frame{\titlepage}

\begin{frame}
    \frametitle{Outline}

    \begin{enumerate}
        \item Calculus
        \item Duration
        \item Bond Convexity
        \item Applications
    \end{enumerate}

\end{frame}

\begin{foundframe}[t]
    \frametitle{1. Calculus}
    \framesubtitle{}

    The first derivative of a function is equal to the slope of the 
    tangent lines to that function.\\
    \vspace{1em}
    \centering
    \includegraphics[width=0.8\textwidth]{figures/ch16_1_derivative.png}

\end{foundframe}

\begin{foundframe}[t]
    \frametitle{1. Calculus}
    \framesubtitle{Rules for Derivatives}

    \vspace{-0.5em}
    \textbf{Power Rule:}
    $$\frac{d}{dx}[x^n] = nx^{n-1}$$
    \textit{Example:} $\frac{d}{dx}[x^3] = 3x^2$
    \vspace{1em}

    \textbf{Product Rule:}
    $$\frac{d}{dx}[f(x) \cdot g(x)] = f'(x) \cdot g(x) + f(x) \cdot g'(x)$$
    \textit{Example:} $\frac{d}{dx}[x^2 \cdot e^x] = 2x \cdot e^x + x^2 \cdot e^x$
    \vspace{1em}

    \textbf{Chain Rule:}
    $$\frac{d}{dx}[f(g(x))] = f'(g(x)) \cdot g'(x)$$
    \textit{Example:} $\frac{d}{dx}[(1+r)^{-t}] = -(1+r)^{-t-1} \cdot 1 = -t(1+r)^{-t-1}$
    \vspace{1em}

\end{foundframe}

\begin{foundframe}[t]
    \frametitle{1. Calculus}
    \framesubtitle{Estimating the Derivative with Small Changes in X}

    \textbf{Slope Approximation:} For a function $f(x)$, the slope at point $x$ can be approximated using small changes:
    $$\text{Slope} \approx \frac{f(x + \Delta x) - f(x)}{\Delta x}$$
    
    As $\Delta x$ becomes smaller, this approximation becomes more accurate.
    \vspace{1em}

\end{foundframe}

\begin{foundframe}[t]
    \frametitle{1. Calculus}
    \framesubtitle{Second Derivative}

    The second derivative measures the rate of change of the slope of the tangent lines.\\
    \vspace{1em}

    \begin{itemize}
        \item If $f"(x) > 0$ the function is convex
        \item Larger (smaller) values of $f"(x)$ indicate more (less) convexity
        \item If $f"(x) < 0$ the function is concave (function looks like a cave)
    \end{itemize}

\end{foundframe}

\begin{foundframe}[t]
    \frametitle{1. Calculus}
    \framesubtitle{Second Derivative}

    \centering
    \includegraphics[width=\textwidth]{figures/ch16_1_convexity.png}

\end{foundframe}

\begin{frame}[t]
    \frametitle{2. Duration}
    \framesubtitle{Definitions}

    The \textbf{duration} of a bond is the approximate percentage 
    change in the bond price for a 100 basis point change in the 
    yield to maturity.\\
    \vspace{1em}

    \begin{itemize}
        \item \textit{Intuition:} Duration is the first derivative (slope) of the bond price-yield curve.
        \item \textit{Convention:} Duration is positive for most bonds (even though the slope of change in the bond price is negative) and duration is referred to as 'years' (more on this later).
        \item \textit{Example:} For a bond with 20 years of duration, 
        the price will increase by 20\% if yields fall by 100 basis points.
    \end{itemize}

\end{frame}

\begin{frame}[t]
    \frametitle{2. Duration}
    \framesubtitle{Chart of Price-Yield Relationship}

    \centering
    \includegraphics[width=0.8\textwidth]{figures/ch16_1_priceyield_tangent.png}

\end{frame}

\begin{frame}[t]
    \frametitle{2. Duration}
    \framesubtitle{Chart of Price-Yield Relationship}

    \centering
    \includegraphics[width=0.8\textwidth]{figures/ch16_1_priceyield_tangent_10s30s.png}

\end{frame}


\begin{frame}[t]
    \frametitle{2. Duration}
    \framesubtitle{First Derivative, continued}

    In practice, bond duration is often estimated by 'shocking' the bond price equation with a 
    small change in the yield, and then calculating the slope of the bond price-yield 
    relationship in the normal way.\\
    \vspace{1em}

    \textbf{Example for Bond Pricing:}
    \begin{itemize}
        \item Let $P(y)$ = bond price as function of yield $y$
        \item To estimate duration at yield $y = 5\%$:
        \item Calculate $P(5.01\%) - P(4.99\%)$ and divide by $0.02\%$
        \item This gives approximate percentage price change per basis point
    \end{itemize}

\end{frame}

\begin{frame}[t]
    \frametitle{2. Duration}
    \framesubtitle{Modified Duration}

    Define the \textbf{modified duration} of a bond as:
    $$-\frac{P'(y)}{P(y)}$$
    Where $P(y)$ is the bond price given a yield $y$ and $P'(y)$ is the first derivative of the 
    bond price with respect to the yield.\\
    \vspace{1em}
    Note that, by convention, duration is expressed as a percent change (divide by $P(y)$),
    and that duration is positive (multiply by $-1$).

\end{frame}



\begin{frame}[t]
    \frametitle{2. Duration}
    \framesubtitle{Macaulay Duration}

    \textbf{Macaulay Duration} is the weighted average of the time to 
    maturity of the bond's cash flows, where the 
    weight applied to each cash flow is the proportion of 
    the total present value of the bond accounted 
    for by that payment (i.e. the present value of the cash flow divided by the present value of the bond).

    $$
    D = \sum_{t=1}^{T} t \times w_t
    $$
    Where\\
     $t = \text{Time to Maturity}$, $w_t = \frac{CF_t/(1+y)^t}{P}$, $CF_t = \text{Cash flow at time } t$,
    $P = \text{Bond Price}$, $y = \text{Yield to maturity}$. \\

\end{frame}


\begin{frame}[t]
    \frametitle{2. Duration}
    \framesubtitle{Macaulay Duration}

    Macaulay observed the maturity of a bond was not necessarily the most important time-characteristic. 
    Maucaly said: \\
    \vspace{1em}
    \begin{quotation}
        Let us use the word "duration" to signify the essence of the time element in a loan.
    \end{quotation}
    The weights applied to each cash flow are what he meant by the "essence".\\
    \vspace{1em}
    Due to the initial insights of Maucaly regarding the "essense" of the time element, now the convention 
    is to refer to duration in "years" even though that doesn't have any real relevance to the modified 
    duration formulation.

    \blfootnote{See K. Winston, \textit{Quantitative Risk and Portfollio Management} page 92}

\end{frame}

\begin{frame}[t]
    \frametitle{2. Duration}
    \framesubtitle{Macaulay Duration}

    \footnotesize
    Here is how Modified Duration is related to Macaulay Duration:
    \begin{itemize}
        \item Start with generic bond price equation
        $$ P(y) = \sum_{t=1}^{T} \frac{C}{(1+y)^t} + \frac{FV}{(1+y)^T} $$
        \item Write down the Modified Duration: take the first derivative of $P(y)$, divide by $P(y)$, and multiply by $-1$
        $$ \frac{-P'(y)}{P(y)} = \sum_{t=1}^{T} \frac{\frac{t \cdot C}{(1+y)^{(t+1)}}}{P} + T \frac{\frac{FV}{(1+y)^{(T+1)}}}{P} $$
        \item Factor out $\frac{1}{(1+y)}$ from the left hand side
        $$ \frac{-P'(y)}{P(y)} = \frac{1}{(1+y)} \left[  \sum_{t=1}^{T} t\frac{\frac{C}{(1+y)^{t}}}{P} + T \frac{\frac{FV}{(1+y)^{T}}}{P} \right]$$
        \item The term in brackets is the Maucaly Duration from the previous slide, 
        where the weights are $\frac{\frac{C}{(1+y)^{t}}}{P}$
        and $\frac{\frac{FV}{(1+y)^{T}}}{P}$
    \end{itemize}

\end{frame}



\begin{frame}[t]
    \frametitle{2. Duration}
    \framesubtitle{Macaulay Duration and Modified Duration}

    Conclude that:
    $$\text{Modified Duration} = -\frac{P'(y)}{ P(y)} = \frac{\text{Mauculay Duration}}{(1+y)}$$
    Maucaly Duration is usually is very similar to the Modified Duration, especially when $y$ is small.\\
    \vspace{1em}
    In practice its common to hear traders refer to "duration" 
    without specifying which measure.   This usually doesn't matter, because, as we'll see below, 
    these are used for estimates anyways.

\end{frame}

\begin{frame}[t]
    \frametitle{2. Duration}
    \framesubtitle{Rules}

    The following rules for duration can be shown either by thinking about the weighted cash flows, 
    or by taking a derivative.

    \begin{enumerate}
        \item The Macaulay Duration of a zero-coupon bond equals $T$ (Only one cash flow!).  
        The Modified Duration of a zero-coupon bond is $\frac{T}{1+y}$ (take the derivative...).      
        \item The Modified Duration of a perpetuity is equal to $\frac{1}{y}$, and the Macaulay Duration is $\frac{1+y}{y}$
        \item Holding maturity constant, duration is lower when the coupon rate is higher.
        \item Holding the coupon rate constant, duration generally increases with its time to maturity.
        \item Holding other factors constant, duration higher when yield to maturity is lower
    \end{enumerate}

\end{frame}

\begin{frame}[t]
    \frametitle{3. Bond Convexity}
    \framesubtitle{Definitions}

    The \textbf{convexity} of a bond measures the curvature of the bond price-yield relationship.\\
    \vspace{1em}

    \begin{itemize}
        \item \textit{Mathematical Definition:} Convexity is the second derivative of the bond price with respect to yield
        \item \textit{Intuition:} Duration (first derivative) gives the slope; convexity gives the curvature
        \item \textit{Formula:} $$\text{Convexity} = \frac{P''(y)}{P(y)}$$
        \item \textit{Key Property:} Bond prices are generally convex with respect to yields (convexity $>$ 0)
    \end{itemize}
    \vspace{1em}

\end{frame}

\begin{frame}[t]
\frametitle{3. Bond Convexity}
\framesubtitle{Definitions}

    \begin{itemize}
        \item Duration alone underestimates price gains when yields fall
        \item Duration alone overestimates price losses when yields rise
        \item Convexity correction improves price change estimates
    \end{itemize}
    \vspace{1em}
    Convexity is generally valuable to investors.   They are willing to pay more for 
    convex bonds, and, conversely, may require a discount (higher yield) for negatively 
    convex bonds.

\end{frame}

\begin{frame}[t]
    \frametitle{3. Bond Convexity}
    \framesubtitle{Price Change Approximation}

    \textbf{First-Order Approximation (Duration Only):}
    $$\Delta P \approx -D \times P \times \Delta y$$
    
    \textbf{Second-Order Approximation (Duration + Convexity):}
    $$\Delta P \approx -D \times P \times \Delta y + \frac{1}{2} \times C \times P \times (\Delta y)^2$$
    \vspace{1em}

    Where: $D$ = Modified Duration, $C$ = Convexity, $P$ = Current Bond Price,  $\Delta y$ = Change in Yield
    
\end{frame}

\begin{frame}[t]
    \frametitle{3. Bond Convexity}
    \framesubtitle{Mortgage - negative convexity}

    Mortgage-backed securities (MBS) are said to have "negative convexity" due to prepayment risk.\\

    \centering
    \includegraphics[width=0.6\textwidth]{figures/ch14_2_mtg_prepayment.png}

\end{frame}

\begin{frame}[t]
    \frametitle{3. Bond Convexity}
    \framesubtitle{Other sources of convexity}

    MBS are a specific example of a bond with an embedded option. The embedded option changes the convexity profile.\\
    \vspace{1em}
    Callable corporate bonds are often negatively convex for the same reason.\\
    \vspace{1em}
    Options can also be used to explicitly change the convexity profile of a bond portfolio (upcoming lecture)\\
    \vspace{1em}
    The key is to draw the price-yield relationship of the bond (or the portfolio) and think about the curvature.

\end{frame}

\begin{practiceframe}[t]
    \frametitle{4. Two Applications}
    \framesubtitle{Hedging a Bond Portfolio}

    You start with a portfolio of \$1,000,000 in on-the-run 10 year US Treasury bonds 
    (10 years to maturity, 3.5\% coupon, 3.5\% yield) and \$1,000,000 in cash.
    \footnote{Cash has zero duration.}\\
    \vspace{1em}

    You use cash to buy \$1,000,0000 of a high yield corporate bond: 
    4\% coupon, 10 years to maturity, 15\% yield.
    \footnote{Assume the high yield bond is purchased using cash.}\\
    \vspace{1em}

    You want exposure to the default premium of the high yield bond 
    but you do \textbf{NOT} want exposure to the interest rate risk of the high yield bond.\\
    \vspace{1em}

    \textit{Question: How much 10 year US Treasury bonds do you sell to offset the purchase of the corporate bond?}

\end{practiceframe}

\begin{practiceframe}[t]
    \frametitle{4. Two Applications}
    \framesubtitle{Hedging a Bond Portfolio}

    \centering
    \includegraphics[width=\textwidth]{figures/ch16_1_exp_1.png}

\end{practiceframe}

\begin{practiceframe}[t]
    \frametitle{4. Two Applications}
    \framesubtitle{Hedging a Bond Portfolio}

    \centering
    \includegraphics[width=0.8\textwidth]{figures/ch16_1_exp_2.png}

\end{practiceframe}

\begin{practiceframe}[t]
    \frametitle{4. Two Applications}
    \framesubtitle{Estimating Performace of a Bond Portfolio}

    Consider a 30 year zero coupon bond that yields 5\% 
    and a 10 year zero coupon bond that also yield 5\%.
       Assume the yield curve is flat.\\
    \vspace{1em}

    Investors start to worry about a recession, all yields drop to 4.5\%.  \\
    \vspace{1em}

    What is your return on the 30 year bond? On the 10 year bond?\\
    \vspace{1em}

    Then the recession actually occurs, all yields drop to 2.5\%.\\
    \vspace{1em}

    \textit{Question: What is your return on the 30 year bond?   On the 10 year bond?}

\end{practiceframe}

\begin{practiceframe}[t]
    \frametitle{4. Two Applications}
    \framesubtitle{Predicting Performance}

    \centering
    \includegraphics[width=0.8\textwidth]{figures/ch16_1_exp_3.png}

\end{practiceframe}

\begin{practiceframe}[t]
    \frametitle{4. Two Applications}
    \framesubtitle{Predicting Performance}

    \centering
    \includegraphics[width=0.8\textwidth]{figures/ch16_1_exp_4.png}

\end{practiceframe}

\end{document}