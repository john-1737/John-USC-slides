\documentclass{beamer}

\newcommand{\week}{Week 11-b}

\title{Equity Valuation}
\subtitle{Reference: Bodie et al, Ch 18}
\author{Econ 457}
\date{\week}

% Reference the shared preamble
\setbeamertemplate{frametitle}{
  \vspace{0.5em}
  \insertframetitle
  \par
  \vspace{0.5em}
  \hrule
  \vspace{0.3em}
  {\small\color{gray}\insertframesubtitle}
}

\setbeamertemplate{navigation symbols}{}
\setbeamertemplate{itemize item}{\textbullet} % main bullet: filled dot
\setbeamertemplate{itemize subitem}{\normalsize$\circ$} % sub-bullet: empty dot
\setbeamertemplate{itemize subsubitem}{\scriptsize--} % sub-sub-bullet: dash


% Font changes
\usepackage[scaled=0.92]{helvet}
\renewcommand{\familydefault}{\sfdefault}

% Packages
\usepackage{tikz}
\usepackage{booktabs}
\usepackage{xcolor}
\usepackage{array}           % Enhanced column types for tables
\usepackage{multirow}        % Spanning multiple rows in tables
\usepackage{makecell}        % Line breaks and formatting in table cells
\usepackage{siunitx}         % Proper formatting of numbers and units
\usepackage{amsmath}         % Enhanced math environments
\usepackage{amsfonts}        % Additional math fonts
\usepackage{amssymb}         % Additional math symbols
\usepackage{url}             % Better URL formatting
\usepackage{graphicx}        % Enhanced graphics support
\usepackage{tabularray}
\UseTblrLibrary{booktabs, siunitx, varwidth}
% For financial presentations specifically
\usepackage{eurosym}         % Euro symbol
\usepackage{textcomp}        % Additional text symbols
\usepackage{hyperref}        % Hyperlinks (should be loaded last)

% Define a footnote
\renewcommand{\footnoterule}{\vspace*{-3pt}\hrule width 2in height 0.4pt\vspace*{2.6pt}}

% Define a Foundation Slide
\newenvironment{foundframe}[1][t]{
    \setbeamercolor{background canvas}{bg=gray!8}
    \setbeamercolor{frametitle}{fg=gray!80!black,bg=gray!25}
    \setbeamercolor{framesubtitle}{fg=gray!70!black,bg=gray!15}
    \setbeamercolor{item}{fg=gray!80!black}
    \setbeamercolor{enumerate item}{fg=gray!80!black}
    
    % Modify the frametitle template for this frame type
    \setbeamertemplate{frametitle}{
        \vspace{0.5em}
        \begin{minipage}[t]{0.75\textwidth}
            \insertframetitle
            \par
            \vspace{0.5em}
            \hrule
            \vspace{0.3em}
            {\small\color{gray}\insertframesubtitle}
        \end{minipage}%
        \hfill
        \begin{minipage}[t]{0.2\textwidth}
            \raggedleft
            \colorbox{gray!30}{%
                \scriptsize\bfseries\color{gray!80!black}%
                   \hspace{3pt}\begin{tabular}{c}Foundation\\Material\end{tabular}\hspace{3pt}%
            }
        \end{minipage}
        \vspace{0.3em}
    }
    
    \begin{frame}[#1]
}{
    \end{frame}
}

% Define Practice Slide
\newenvironment{practiceframe}[1][t]{
    \setbeamercolor{background canvas}{bg=white}
    \setbeamercolor{frametitle}{fg=blue!80!black,bg=blue!15}
    \setbeamercolor{framesubtitle}{fg=blue!70!black,bg=blue!10}
    \setbeamercolor{item}{fg=blue!80!black}
    \setbeamercolor{enumerate item}{fg=blue!80!black}
    \setbeamercolor{normal text}{fg=blue!90!black}
    
    % Modify the frametitle template for this frame type
    \setbeamertemplate{frametitle}{
        \vspace{0.5em}
        \begin{minipage}[t]{0.75\textwidth}
            \insertframetitle
            \par
            \vspace{0.5em}
            \hrule
            \vspace{0.3em}
            {\small\color{blue!70!black}\insertframesubtitle}
        \end{minipage}%
        \hfill
        \begin{minipage}[t]{0.2\textwidth}
            \raggedleft
            \colorbox{blue!20}{%
                \scriptsize\bfseries\color{blue!80!black}%
                   \hspace{3pt}\begin{tabular}{c}Practice\\Questions\end{tabular}\hspace{3pt}%
            }
        \end{minipage}
        \vspace{0.3em}
    }
    
    \begin{frame}[#1]
}{
    \end{frame}
}

% Define Excel Slide
\newenvironment{excelframe}[1][t]{
    \setbeamercolor{background canvas}{bg=white}
    \setbeamercolor{frametitle}{fg=blue!80!black,bg=blue!15}
    \setbeamercolor{framesubtitle}{fg=blue!70!black,bg=blue!10}
    \setbeamercolor{item}{fg=blue!80!black}
    \setbeamercolor{enumerate item}{fg=blue!80!black}
    \setbeamercolor{normal text}{fg=blue!90!black}
    
    % Modify the frametitle template for this frame type
    \setbeamertemplate{frametitle}{
        \vspace{0.5em}
        \begin{minipage}[t]{0.75\textwidth}
            \insertframetitle
            \par
            \vspace{0.5em}
            \hrule
            \vspace{0.3em}
            {\small\color{blue!70!black}\insertframesubtitle}
        \end{minipage}%
        \hfill
        \begin{minipage}[t]{0.2\textwidth}
            \raggedleft
            \colorbox{green!10}{%
                \scriptsize\bfseries\color{blue!80!black}%
                   \hspace{3pt}\begin{tabular}{c}MS Excel\end{tabular}\hspace{3pt}%
            }
        \end{minipage}
        \vspace{0.3em}
    }
    
    \begin{frame}[#1]
}{
    \end{frame}
}

% Define Caution Slide
\newenvironment{cautionframe}[1][t]{
    \setbeamercolor{background canvas}{bg=white}
    \setbeamercolor{frametitle}{fg=blue!80!black,bg=blue!15}
    \setbeamercolor{framesubtitle}{fg=blue!70!black,bg=blue!10}
    \setbeamercolor{item}{fg=blue!80!black}
    \setbeamercolor{enumerate item}{fg=blue!80!black}
    \setbeamercolor{normal text}{fg=blue!90!black}
    
    % Modify the frametitle template for this frame type
    \setbeamertemplate{frametitle}{
        \vspace{0.5em}
        \begin{minipage}[t]{0.75\textwidth}
            \insertframetitle
            \par
            \vspace{0.5em}
            \hrule
            \vspace{0.3em}
            {\small\color{blue!70!black}\insertframesubtitle}
        \end{minipage}%
        \hfill
        \begin{minipage}[t]{0.2\textwidth}
            \raggedleft
            \colorbox{red!10}{%
                \scriptsize\bfseries\color{blue!80!black}%
                   \hspace{3pt}\begin{tabular}{c}Caution\end{tabular}\hspace{3pt}%
            }
        \end{minipage}
        \vspace{0.3em}
    }
    
    \begin{frame}[#1]
}{
    \end{frame}
}

% Add to footnotes
\makeatletter
\newcommand\blfootnote[1]{%
  \begingroup
  \renewcommand\thefootnote{}%
  \renewcommand\@makefntext[1]{\raggedright\leftskip=0pt ##1}%
  \footnote{\scriptsize #1}%
  \addtocounter{footnote}{-1}%
  \endgroup
}
\makeatother

% Set the footer -- change 
\setbeamertemplate{footline}{
  \leavevmode%
  \vspace{2ex}
  \hbox{%
    % Left box: Econ 457
    \begin{beamercolorbox}[wd=.4\paperwidth,ht=2.5ex,dp=1ex,left]{author in head/foot}%
      \hspace{1em}Econ 457
    \end{beamercolorbox}%
    % Middle box: Week
    \begin{beamercolorbox}[wd=.2\paperwidth,ht=2.5ex,dp=1ex,center]{date in head/foot}%
      \centering\week
    \end{beamercolorbox}%
    % Right box: Slide numbers
    \begin{beamercolorbox}[wd=.4\paperwidth,ht=2.5ex,dp=1ex,center]{date in head/foot}%
      \hfill\insertframenumber{} 
    \end{beamercolorbox}%
  }%
  \vskip0pt%
}

\begin{document}

\frame{\titlepage}

\begin{frame}
    \frametitle{Outline}

    \begin{enumerate}
        \item Buybacks
        \item Free Cash Flow Models
            \begin{itemize}
                \item Modigliani and Miller
            \end{itemize}
        \item Relative Valuation 
            \begin{itemize}
                \item P/E Ratio
            \end{itemize}
        \item Cyclically Adjusted P/E (CAPE)
        \item Practice
    \end{enumerate}

\end{frame}

\begin{frame}[t]
    \frametitle{1. Buybacks}
    \framesubtitle{}

    Recently company managers have preferred to return earnings to shareholders 
    via \textbf{stock buybacks} rather than through dividends.  Buybacks benefit 
    all shareholders through higher prices.\\
    \vspace{1em}
    Advantages of stock buybacks include:
    \begin{itemize}
        \item Investors treat dividends as 'sticky'.  Any changes to dividend policy
        are scrutinized.   In contrast, stock buybacks are more likely to be considered 
        one-offs, which makes them a good way to distribute excess cash.
        \item Management may view its stock price as undervalued.
        \item There may be different tax treatment between dividends and capital gains
        \item Reduces the amount of equity outstanding, which could improve financial ratios (Earnings per share, Return on Equity)
    \end{itemize}

\end{frame}

\begin{frame}[t]
    \frametitle{1. Buybacks}
    \framesubtitle{Controversy and Recent Regulation}

    Despite some advantages, stock buybacks are controversial:\\
    \begin{itemize}
        \item Prioritizes stock price gains over investments (R\&D, capital expenditures)
        \item Can artificially inflate EPS without improving underlying business performance
        \item Benefits executives with stock-based compensation more than other stakeholders
    \end{itemize}
    \vspace{1em}
    Heightened public awareness led to some recent changes in tax and regulatory treatment:
    \begin{itemize}
        \item Inflation Reduction Act (2022): 1\% excise tax on share repurchases by public companies
        \item SEC enhanced disclosure requirements (2023): more detailed reporting on buyback timing and rationale
    \end{itemize}

\end{frame}

\begin{frame}[t]
    \frametitle{1. Buybacks}
    \framesubtitle{Recent Trends}

    \centering
    \includegraphics[width=0.9\textwidth]{figures/ch18_2_buybacks.png}

    \blfootnote{https://www.spglobal.com/spdji/en/documents/index-news-and-announcements/20250917-sp-500-buyback-pr.pdf}

\end{frame}


\begin{frame}[t]
    \frametitle{2. Free Cash Flow Models}
    \framesubtitle{Modigliani and Miller}

    The Modigliani-Miller theorem states that in perfect capital markets, 
    a firm's value is independent of its capital structure (debt vs. equity mix).
     The formal proof relies on some key assumptons, including, for example,
     no taxes, no bankruptcy costs, perfect information and efficient markets.\\
    \vspace{1em}
    For the intuition, note that debt is usually cheaper than equity.  A more 
    leveraged firm will have more debt, which will seemingly reduce its cost of financing.
     However, the more leveraged firm is also riskier, so the interest rate on that debt and the 
     required return on equity will both rise, and, in the model, perfectly offsetting the initial benefit.   
     Therefore the weighted average cost of capital is unchanged, even as the mix of debt and equity changes.

\end{frame}

\begin{frame}[t]
    \frametitle{2. Free Cash Flow Models}
    \framesubtitle{Free Cash Flow to Firm (FCFF)}

    \textbf{Definition:} Cash flow available to all capital providers (debt and equity holders) before any financing costs.\\
    \vspace{1em}
    \textbf{Calculation:}
    \begin{align*}
    FCFF &= \text{EBIT}(1-\text{Tax Rate}) + \text{Depreciation} \\
         &\quad - \text{Capital Expenditures} - \text{Change in Working Capital}
    \end{align*}
    \vspace{1em}
    \textbf{Valuation:}
    $$\text{Firm Value} = \sum_{t=1}^{\infty} \frac{FCFF_t}{(1 + WACC)^t}$$
    Where $WACC$ is the weighted average cost of capital\\
    \vspace{1em}
    \textbf{Use Case:} Total firm valuation; subtract debt to get equity value

\end{frame}

\begin{frame}[t]
    \frametitle{2. Free Cash Flow Models}
    \framesubtitle{Free Cash Flow to Equity (FCFE)}

    \textbf{Definition:} Cash flow available to equity holders after all expenses, taxes, debt payments, and reinvestment needs.\\
    \vspace{1em}
    \textbf{Calculation:}
    \begin{align*}
    FCFE &= FCFF - \text{Interest Expense}(1-\text{Tax Rate}) + \text{Increases in net debt}
    \end{align*}
    \vspace{1em}
    \textbf{Valuation:}
    $$\text{Equity Value} = \sum_{t=1}^{\infty} \frac{FCFE_t}{(1 + r_E)^t}$$
    Where $r_E$ is the required return on equity (cost of equity)\\
    \vspace{1em}
    \textbf{Use Case:} Direct valuation of equity, particularly useful for leveraged firms

\end{frame}



\begin{frame}[t]
    \frametitle{2. Free Cash Flow Models}
    \framesubtitle{Modigliani and Miller}

    Implication of Modigliani and Miller Theorem for Valuation:
    \begin{itemize}
        \item Firm value depends only on underlying cash flows, not financing decisions
        \item Focus valuation on operating performance rather than capital structure
        \item Provides foundation for free cash flow valuation methods
    \end{itemize}
    \vspace{1em}
    Note that in practice, when valuing the same firm using FCFF (discounted at WACC) 
    versus FCFE (discounted at cost of equity), the implied equity values are rarely 
    equal. This suggests that Modigliani Miller assumptions don't hold perfectly in 
    practice - not surprising given the restrictive assumptions.

\end{frame}

\begin{frame}[t]
    \frametitle{3. Relative Valuation}
    \framesubtitle{}

    The dividend discount model valued a company based on its earnings and a discount rate.  
    Using the modeled valuation, an investor could then determine whether the market price was 
    too high or too low, and either sell or buy accordingly.\\
    \vspace{1em}
    An alternative approach would be to compare the market price of a company to the price of 
    either other, similar companies OR compared to the price of that company over time.\\
    \vspace{1em}
    This approach requires \textit{normalizing} the market price by something, in order to account 
    for observable differences across comparison companies.   A common approach is to normalize 
    the price by \textbf{earnings}, although there are other possibilities too.

\end{frame}


\begin{frame}[t]
    \frametitle{3. Relative Valuation}
    \framesubtitle{P/E Ratio}

    Reminder that the value of a perpetuity is given by:
    $$P_0 = \frac{c}{k}$$
    Where $P_0$ is the current price, $c$ is the payment amount, and $k$ is the discount rate.\\
    \vspace{1em}
    The value of a company that has no expected growth is similar to a perpetuity, with the 
    payment amount equal to earnings ($E$).\\
    $$P_0 = \frac{E}{k}$$
    
\end{frame}

\begin{frame}[t]
    \frametitle{3. Relative Valuation}
    \framesubtitle{P/E Ratio}

    Still thinking about companies with no expected growth (i.e. perpetuities), 
    the Price-Earnings Ratio is:
    $$P_0/E = \frac{1}{k}$$
    Because the earnings cancels out.\\
    \vspace{1em}
    For illustrative purposes, assume that $k=12.5\%$.   
    Then the P-E Ratio would be $1/0.125 = 8$.
    \vspace{1em}

    In words: investors should be willing to pay \$8 billion for a company with a 
    required return of 12.5\%, no-growth prospects, and earnings next year of \$1 billion.

\end{frame}

\begin{frame}[t]
    \frametitle{3. Relative Valuation}
    \framesubtitle{P/E Ratio}

    Of course, companies also have growth prospects, which should increase their value.  
    We can define the value of a company as follows:
    \begin{align*}
    P_0 &= \text{PV(no-growth)} + \text{PV(Growth Opportunities)}\\
        &= \frac{E_1}{k} + PVGO
    \end{align*}
    Where $PVGO$ is $PV(Growth Opportunities)$\\

\end{frame}

\begin{frame}[t]
    \frametitle{3. Relative Valuation}
    \framesubtitle{P/E Ratio}

    Dividing this by $E_1$ and doing a little rearranging 
    gives an expression for the P/E Ratio:
    $$
    \frac{P_0}{E_1} = \frac{1}{k} \left(1 + \frac{PVGO}{E/k}\right)
    $$
    The term $\frac{PVGO}{E/k}$ is the ratio of the growth opportunities 
    to the no-growth scenario for the firm.  And the term $\frac{1}{k}$ is 
    what we say earlier about the PE ratio for a no-growth firm.\\
    \vspace{1em}
    In words, the PE ratio is the value of a perpetuity multiplied by 
    one plus the ratio of PVGO to the no-growth value.  Firms will higher 
    PVGOs will have higher PE ratios.   Investors are willing to pay more 
    for companies with better growth prospects.

\end{frame}


\begin{frame}[t]
    \frametitle{3. Relative Valuation}
    \framesubtitle{P/E Ratio - Earnings}

    In order to estimate the P-E Ratio you need to know the earnings.\\
    \vspace{1em}
    Ideally, you would have an estimate for forward earnings.  This could come from 
    company guidance, or analyst consensus, or your own analysis.\\
    \vspace{1em}
    An often used substitute is earnings over the trailing twelve months (ttm).  While not ideal,
     this is easily observed and readily available.

\end{frame}

\begin{frame}[t]
    \frametitle{2. Price/Earnings Ratio}
    \framesubtitle{PE Ratios across Industries}

    \centering
    \includegraphics[width=0.7\textwidth]{figures/ch18_2_pe_ind.png}

    \blfootnote{Data Source: Aswath Damodaran, available on his NYU website}

\end{frame}

\begin{frame}[t]
    \frametitle{2. Price/Earnings Ratio}
    \framesubtitle{PE Ratios in the Auto Industry}

    \footnotesize
    \begin{table}
    \centering
    \caption{PE Ratios in the Auto Industry, July 2025}
    \begin{tabular}{|m{3cm}|>{\centering\arraybackslash}p{1.5cm}|>{\centering\arraybackslash}p{1.8cm}|>{\centering\arraybackslash}p{1.3cm}|>{\centering\arraybackslash}p{1.5cm}|}
    \hline
    & \textbf{Market Cap} & \textbf{Share Price, \$} & \textbf{EPS, TTM} & \textbf{PE Ratio, TTM} \\
    & \textbf{\$ bn} & \textbf{(as of 7-16)} & & \\
    \hline
    Ford (F) & 44.6 & 11.25 & 1.25 & 9.00 \\
    \hline
    General Motors (GM) & 51.1 & 53.33 & 7.16 & 7.45 \\
    \hline
    Honda (HMC) & 41.8 & 30.53 & 3.64 & 8.39 \\
    \hline
    Tesla (TSLA) & 1,037 & 321.88 & 1.74 & \textbf{184.99} \\
    \hline
    BYD Company (BYD) & 172.6 & 93.73 & 4.30 & 21.77 \\
    \hline
    \end{tabular}
    \end{table}

    \blfootnote{Data from Yahoo Finance, as of 7-16-2025.}

\end{frame}

\begin{frame}[t]
    \frametitle{2. Price/Earnings Ratio}
    \framesubtitle{Do PE Ratios Predict Earnings Growth?}

    \begin{columns}
        \begin{column}{0.5\textwidth}
            \centering
            \includegraphics[width=\textwidth]{figures/ch18_2_conpepsi_pe.png}
        \end{column}
        \begin{column}{0.5\textwidth}
            \centering
            \includegraphics[width=\textwidth]{figures/ch18_2_conpepsi.png}
        \end{column}
    \end{columns}
\end{frame}

\begin{frame}[t]
    \frametitle{4. Cyclically Adjusted P/E (CAPE)}
    \framesubtitle{}

    \centering
    \includegraphics[width=0.8\textwidth]{figures/ch18_2_cape_1.png}

    \blfootnote{Data Source: Robert Shiller, available on his Yale website}

\end{frame}

\begin{frame}[t]
    \frametitle{4. Cyclically Adjusted P/E (CAPE)}
    \framesubtitle{Definition}

    Define the Cyclically Adjusted P-E Ratio (CAPE) as the "stock price divided by 
    average real earnings over the previous 10 year."\\
    \vspace{1em}
    \begin{itemize}
        \item Adjusted for inflation (real earnings)
        \item 10 years is long enough to smooth over business cycles
    \end{itemize}
    \vspace{1em}
    Accordingly, we would expect the CAPE to smoother than the PE ratio.

\end{frame}


\begin{frame}[t]
    \frametitle{4. Cyclically Adjusted P/E (CAPE)}
    \framesubtitle{Historical CAPE}

    \centering
    \includegraphics[width=0.8\textwidth]{figures/ch18_2_cape_2.png}

    \blfootnote{Data Source: Robert Shiller, available on his Yale website}

\end{frame}


\begin{frame}[t]
    \frametitle{4. Cyclically Adjusted P/E (CAPE)}
    \framesubtitle{A Good Measure of Value?}

    The inverse of the PE is the \textit{earnings yield}: $E/P$.\\
    \vspace{1em} 
    Define the \textit{excess CAPE yield} as 
    $$\frac{1}{CAPE} - (\text{10y real UST yield})$$
    \vspace{1em}
    Is this a good measure of value?   When CAPE is high (low), the excess CAPE yield is 
    low (high) and suggests future returns will also be low (high).\\

\end{frame}

\begin{frame}[t]
    \frametitle{4. Cyclically Adjusted P/E (CAPE)}
    \framesubtitle{A Good Measure of Value?}

    \centering
    \includegraphics[width=0.8\textwidth]{figures/ch18_2_cape_3.png}

    \blfootnote{Data Source: Robert Shiller, available on his Yale website}

\end{frame}


\begin{frame}[t]
    \frametitle{4. Cyclically Adjusted P/E (CAPE)}
    \framesubtitle{Barclays}

    \centering
    \includegraphics[width=0.8\textwidth]{figures/ch18_2_barclays.png}

\end{frame}

\begin{frame}[t]
    \frametitle{4. Cyclically Adjusted P/E (CAPE)}
    \framesubtitle{Barclays}

    \centering
    \includegraphics[width=\textwidth]{figures/ch18_2_barclays2.png}

\end{frame}

\begin{practiceframe}[t]
    \frametitle{5. Practice}
    \framesubtitle{}

    \begin{enumerate}
        \item Jand, Inc. currently pays a dividend of \$1.22, which is 
        expected to grow indefinitely at 5\%.   If the current value of 
        Jand's shares based on teh constant-growth dividend discount model 
        is \$32.03, what is the required rate of return?
        \item Computer stocks current provide an expected rate of return 
        of 16\%.   MBI, a large computer compnay will pay a year-end dividend of \$2 
        per share.   If the stock is selling at \$50 per share, what must be the market's 
        expectation for the dividend growth rate?
        \item If the dividend growth forecast is revised down to 5\%, what will have to 
        the price of stock?  What (qualitatively) will happen to the price-earnings ratio?
    \end{enumerate}

\end{practiceframe}

\begin{practiceframe}[t]
    \frametitle{5. Practice}
    \framesubtitle{}

    \begin{enumerate}
        \item[4. ] MF Corp has an ROE of 16\% and a plowback ratio of 50\%.   If the coming 
        year's earnings are expected to be \$2 per share, at what price will the stock 
        sell?   The market capitalization rate is 12\%.   What price do you expect it 
        to sell at in 3 years?
        \item[5. ] Sisters Copr. expects to earn \$6 per share next year.  The firm's 
        ROE is 15\% and its plowback ratio is 60\%.   If the firm's market capitalization 
        rate is 10\%, what is the present value of its growth opportunities?
        \item[6. ] The FI Corporation's dividends per share are expected to 
        grow indefinitely by 5\% per year.  If this year's year-end dividend is \$8 and 
        the market capitalization rate is 10\% per year, what must be the current stock price?  
        If the expected earnings per share are \$12, what is the implied value of the ROE on 
        future investment opportunities?   How much is the market paying per share for future 
        growth?
    \end{enumerate}

\end{practiceframe}

\begin{practiceframe}[t]
    \frametitle{5. Practice}
    \framesubtitle{}

    \begin{enumerate}
        \item[7. ] The Duo Growth Company just paid a dividend of \$1 per share.  The 
        dividend is expected to grow at a rate of 25\% per share for the next three years 
        and then to level off to a 5\% per year forever.   You think the appropriate 
        market capitalization rate is 20\% per year.   
        \begin{itemize}
            \item What is the estimate of the intrinsic value of the stock?
            \item If the market price is equal to this intrinsic value, what is the dividend 
            yield?
            \item What do you expect the price to be one year from now?
            \item If the implied capital gain consistent with your esimate of the dividend 
            yield and the market capitalization rate?
        \end{itemize}
    \end{enumerate}

\end{practiceframe}

\end{document}