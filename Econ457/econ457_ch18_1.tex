\documentclass{beamer}

\newcommand{\week}{Week 11-a}

\title{Equity Valuation}
\subtitle{Reference: Bodie et al, Ch 18}
\author{Econ 457}
\date{\week}

% Reference the shared preamble
\setbeamertemplate{frametitle}{
  \vspace{0.5em}
  \insertframetitle
  \par
  \vspace{0.5em}
  \hrule
  \vspace{0.3em}
  {\small\color{gray}\insertframesubtitle}
}

\setbeamertemplate{navigation symbols}{}
\setbeamertemplate{itemize item}{\textbullet} % main bullet: filled dot
\setbeamertemplate{itemize subitem}{\normalsize$\circ$} % sub-bullet: empty dot
\setbeamertemplate{itemize subsubitem}{\scriptsize--} % sub-sub-bullet: dash


% Font changes
\usepackage[scaled=0.92]{helvet}
\renewcommand{\familydefault}{\sfdefault}

% Packages
\usepackage{tikz}
\usepackage{booktabs}
\usepackage{xcolor}
\usepackage{array}           % Enhanced column types for tables
\usepackage{multirow}        % Spanning multiple rows in tables
\usepackage{makecell}        % Line breaks and formatting in table cells
\usepackage{siunitx}         % Proper formatting of numbers and units
\usepackage{amsmath}         % Enhanced math environments
\usepackage{amsfonts}        % Additional math fonts
\usepackage{amssymb}         % Additional math symbols
\usepackage{url}             % Better URL formatting
\usepackage{graphicx}        % Enhanced graphics support
\usepackage{tabularray}
\UseTblrLibrary{booktabs, siunitx, varwidth}
% For financial presentations specifically
\usepackage{eurosym}         % Euro symbol
\usepackage{textcomp}        % Additional text symbols
\usepackage{hyperref}        % Hyperlinks (should be loaded last)

% Define a footnote
\renewcommand{\footnoterule}{\vspace*{-3pt}\hrule width 2in height 0.4pt\vspace*{2.6pt}}

% Define a Foundation Slide
\newenvironment{foundframe}[1][t]{
    \setbeamercolor{background canvas}{bg=gray!8}
    \setbeamercolor{frametitle}{fg=gray!80!black,bg=gray!25}
    \setbeamercolor{framesubtitle}{fg=gray!70!black,bg=gray!15}
    \setbeamercolor{item}{fg=gray!80!black}
    \setbeamercolor{enumerate item}{fg=gray!80!black}
    
    % Modify the frametitle template for this frame type
    \setbeamertemplate{frametitle}{
        \vspace{0.5em}
        \begin{minipage}[t]{0.75\textwidth}
            \insertframetitle
            \par
            \vspace{0.5em}
            \hrule
            \vspace{0.3em}
            {\small\color{gray}\insertframesubtitle}
        \end{minipage}%
        \hfill
        \begin{minipage}[t]{0.2\textwidth}
            \raggedleft
            \colorbox{gray!30}{%
                \scriptsize\bfseries\color{gray!80!black}%
                   \hspace{3pt}\begin{tabular}{c}Foundation\\Material\end{tabular}\hspace{3pt}%
            }
        \end{minipage}
        \vspace{0.3em}
    }
    
    \begin{frame}[#1]
}{
    \end{frame}
}

% Define Practice Slide
\newenvironment{practiceframe}[1][t]{
    \setbeamercolor{background canvas}{bg=white}
    \setbeamercolor{frametitle}{fg=blue!80!black,bg=blue!15}
    \setbeamercolor{framesubtitle}{fg=blue!70!black,bg=blue!10}
    \setbeamercolor{item}{fg=blue!80!black}
    \setbeamercolor{enumerate item}{fg=blue!80!black}
    \setbeamercolor{normal text}{fg=blue!90!black}
    
    % Modify the frametitle template for this frame type
    \setbeamertemplate{frametitle}{
        \vspace{0.5em}
        \begin{minipage}[t]{0.75\textwidth}
            \insertframetitle
            \par
            \vspace{0.5em}
            \hrule
            \vspace{0.3em}
            {\small\color{blue!70!black}\insertframesubtitle}
        \end{minipage}%
        \hfill
        \begin{minipage}[t]{0.2\textwidth}
            \raggedleft
            \colorbox{blue!20}{%
                \scriptsize\bfseries\color{blue!80!black}%
                   \hspace{3pt}\begin{tabular}{c}Practice\\Questions\end{tabular}\hspace{3pt}%
            }
        \end{minipage}
        \vspace{0.3em}
    }
    
    \begin{frame}[#1]
}{
    \end{frame}
}

% Define Excel Slide
\newenvironment{excelframe}[1][t]{
    \setbeamercolor{background canvas}{bg=white}
    \setbeamercolor{frametitle}{fg=blue!80!black,bg=blue!15}
    \setbeamercolor{framesubtitle}{fg=blue!70!black,bg=blue!10}
    \setbeamercolor{item}{fg=blue!80!black}
    \setbeamercolor{enumerate item}{fg=blue!80!black}
    \setbeamercolor{normal text}{fg=blue!90!black}
    
    % Modify the frametitle template for this frame type
    \setbeamertemplate{frametitle}{
        \vspace{0.5em}
        \begin{minipage}[t]{0.75\textwidth}
            \insertframetitle
            \par
            \vspace{0.5em}
            \hrule
            \vspace{0.3em}
            {\small\color{blue!70!black}\insertframesubtitle}
        \end{minipage}%
        \hfill
        \begin{minipage}[t]{0.2\textwidth}
            \raggedleft
            \colorbox{green!10}{%
                \scriptsize\bfseries\color{blue!80!black}%
                   \hspace{3pt}\begin{tabular}{c}MS Excel\end{tabular}\hspace{3pt}%
            }
        \end{minipage}
        \vspace{0.3em}
    }
    
    \begin{frame}[#1]
}{
    \end{frame}
}

% Define Caution Slide
\newenvironment{cautionframe}[1][t]{
    \setbeamercolor{background canvas}{bg=white}
    \setbeamercolor{frametitle}{fg=blue!80!black,bg=blue!15}
    \setbeamercolor{framesubtitle}{fg=blue!70!black,bg=blue!10}
    \setbeamercolor{item}{fg=blue!80!black}
    \setbeamercolor{enumerate item}{fg=blue!80!black}
    \setbeamercolor{normal text}{fg=blue!90!black}
    
    % Modify the frametitle template for this frame type
    \setbeamertemplate{frametitle}{
        \vspace{0.5em}
        \begin{minipage}[t]{0.75\textwidth}
            \insertframetitle
            \par
            \vspace{0.5em}
            \hrule
            \vspace{0.3em}
            {\small\color{blue!70!black}\insertframesubtitle}
        \end{minipage}%
        \hfill
        \begin{minipage}[t]{0.2\textwidth}
            \raggedleft
            \colorbox{red!10}{%
                \scriptsize\bfseries\color{blue!80!black}%
                   \hspace{3pt}\begin{tabular}{c}Caution\end{tabular}\hspace{3pt}%
            }
        \end{minipage}
        \vspace{0.3em}
    }
    
    \begin{frame}[#1]
}{
    \end{frame}
}

% Add to footnotes
\makeatletter
\newcommand\blfootnote[1]{%
  \begingroup
  \renewcommand\thefootnote{}%
  \renewcommand\@makefntext[1]{\raggedright\leftskip=0pt ##1}%
  \footnote{\scriptsize #1}%
  \addtocounter{footnote}{-1}%
  \endgroup
}
\makeatother

% Set the footer -- change 
\setbeamertemplate{footline}{
  \leavevmode%
  \vspace{2ex}
  \hbox{%
    % Left box: Econ 457
    \begin{beamercolorbox}[wd=.4\paperwidth,ht=2.5ex,dp=1ex,left]{author in head/foot}%
      \hspace{1em}Econ 457
    \end{beamercolorbox}%
    % Middle box: Week
    \begin{beamercolorbox}[wd=.2\paperwidth,ht=2.5ex,dp=1ex,center]{date in head/foot}%
      \centering\week
    \end{beamercolorbox}%
    % Right box: Slide numbers
    \begin{beamercolorbox}[wd=.4\paperwidth,ht=2.5ex,dp=1ex,center]{date in head/foot}%
      \hfill\insertframenumber{} 
    \end{beamercolorbox}%
  }%
  \vskip0pt%
}

\begin{document}

\frame{\titlepage}

\begin{frame}
    \frametitle{Outline}

    \begin{enumerate}
        \item Dividend Discount Model (DDM) 
            \begin{itemize}
                \item Special Case: Gordon Growth Model
            \end{itemize}
        \item DDM Observations:
            \begin{itemize}
                \item Expected capital gain rate $=g$
                \item If $g=0$, then DDM gives present value of no growth (like a perpetuity)
            \end{itemize}
        \item DDM Details:
            \begin{itemize}
                \item Dividends
                \item Discount Rate
                \item Growth
            \end{itemize}
        \item Multi-stage Dividend Discount Models
        \item Practice
    \end{enumerate}

\end{frame}



\begin{frame}[t]
    \frametitle{1. Dividend Discount Model}
    \framesubtitle{}

    Generic Asset Pricing Equation:
    $$
    \text{Value of Asset} = \sum_{t=1}^{T}\frac{E(\text{Cash flow}_t)}{(1+r)^t}
    $$

    \textit{Question}: What to use for the estimated cash flows?\\
    \textit{Answer}: Depends on the financial instrument.   Bond cash flows are specified at the time the contract is written.   Equities 
    are more complicated.

    \vspace{1em}

    \textit{Question}: What rate should to use to discount future cash flows?\\
    \textit{Answer}: It depends on the nature of the cash flows! Riskier future cash flows should be discounted at a higher rate.

\end{frame}

\begin{frame}[t]
    \frametitle{1. Dividend Discount Model}
    \framesubtitle{}

    \textbf{Dividend Discount Model}

    $$
    V_0 = \frac{D_1}{1+k} + \frac{D_2}{(1+k)^2} + \frac{D_3}{(1+k)^3} + \ldots
    $$
    Where $V_0 = \text{current value}$, $D_t = \text{dividend at time } t$, $k = \text{required return on equity}$\\
    \vspace{1em}
    In words, the dividend discount model says that the value of equity today is the present discounted value of all expected future dividends.

\end{frame}

\begin{frame}[t]
    \frametitle{1. Dividend Discount Model}
    \framesubtitle{Gordon Growth Model}

    A specific case of the Dividend Discount Model is the Gordon Growth Model.   
    The Gordon Growth model makes two crucial assumptions:\\
    \begin{enumerate}
        \item Dividends grow at a constant rate $g$ 
        \item The dividend growth rate is less than the discount rate: $g<k$.
    \end{enumerate}

    \begin{align*}
    D_1 = D_0 * (1+g)\\
    D_2 = D_1 * (1+g)\\
    \ldots\\
    D_t = D_{t-1} * (1+g)
    \end{align*}


\end{frame}

\begin{frame}[t]
    \frametitle{1. Dividend Discount Model}
    \framesubtitle{Gordon Growth Model}

    $$
    V_0 = \frac{D_0(1+g)}{1+k} + \frac{D_0(1+g)^2}{(1+k)^2} + \frac{D_0(1+g)^3}{(1+k)^3} + \ldots
    $$
    We can rewrite this is a simpler way using the tricks from geometric series that we discussed two weeks ago.   
    Note that rewriting as geometric series requires using some clever definitions for $a$ and $r$.\\
    \vspace{1em}

    Let $a = \frac{D_0(1+g)}{1+k}$ and $r = \frac{1+g}{1+k}$.   
    
    \begin{align*}
    V_0 &= a + ar + ar^2 + \ldots\\
    &= \sum_{t=0}^{\infty}ar^t = \frac{a}{1-r}
    \end{align*}

\end{frame}


\begin{frame}[t]
    \frametitle{1. Dividend Discount Model}
    \framesubtitle{Gordon Growth Model}

    Substituting back the definitions of $a = \frac{D_0(1+g)}{1+k}$ and $r = \frac{1}{1+k}$ yields:
    
    \begin{align*}
    V_0 &= \frac{\frac{D_0(1+g)}{1+k}}{1-\frac{1+g}{1+k}} = \frac{\frac{D_0(1+g)}{1+k}}{\frac{1+k}{1+k}-\frac{1+g}{1+k}}\\
    &= \frac{D_0(1+g)}{k-g}\\ 
    &= \frac{D_1}{k-g}\\
    \end{align*}
    
    Where second to last step canceled the $(1+k)$ term, and the last step used the fact that $D1 = D_0(1+g)$

\end{frame}


\begin{frame}[t]
    \frametitle{1. Dividend Discount Model}
    \framesubtitle{Gordon Growth Model}

    This is the \textbf{Gordon Growth Model}:
    $$\boxed{V_0 = \frac{D_1}{k-g}}$$

    From this equation, note the three following intuitive facts:
    \begin{enumerate}
        \item A higher dividend in the next period ($D_1$) raises the value of equity
        \item A higher growth rate of dividends ($g$) raises the value of equity
        \item A lower required return on equity ($k$) raises the value of equity
    \end{enumerate}

\end{frame}

\begin{frame}[t]
    \frametitle{1. Dividend Discount Model}
    \framesubtitle{Gordon Growth Model - Four Notes}

    Four notes on the Gordon Growth Model.
    \begin{enumerate}
        \item Be sure to use $D_1$ in the numerator!
        \item The required return on equity ($k$) is an input to the Gordon Growth Model.   
        You cannot calculate $V_0$ without it.   
        The value for $k$ is tricky, as it may be different for different companies, 
        due to expected volatility and correlations.   
        In contrast, when we were valuing US Treasury bonds, we used interest rates as the discount rate.   
        Those interest rates are relatively standard and readily observable.   
    \end{enumerate}

\end{frame}

\begin{frame}[t]
    \frametitle{1. Dividend Discount Model}
    \framesubtitle{Gordon Growth Model - Four Notes}

    \begin{enumerate}
        \item[3.] Why do we assume that $g<k$?  A geometric series does not converge if $r>1$.   
        If the series does not converge, then the value is infinite.   
        Intuitively, if the growth rate of the dividends is higher than discount rate, 
        then the present value of future dividends will be 
        increasingly large and the present value will be infinity.  
        This assumption is crucial for the simplified representation in the Gordon Growth Model, 
        although it is unrealistic in the real world, 
        as there are many companies with very fast growth.   
        \item[4.] It is straightforward to modify it to value equity for a company 
        that does not pay dividends: use Free Cash Flow to Equity instead of dividends.   
        It is also straightforward to modify it to value an entire company: 
        use Free Cash Flow to the Firm and the Weighted Average Cost of Capital as the discount rate.
    \end{enumerate}

\end{frame}

\begin{frame}[t]
    \frametitle{2. DDM Observations}
    \framesubtitle{Capital Gains Rate}

    Note that this model implies the equity price will grow at the same rate as dividends will grow.
    $$
    \text{Growth Rate of V} = \frac{V_1-V_0}{V_0} =   \frac{V_1}{V_0} - 1
    $$
    Where $V_0 = \frac{D_1}{k-g} \text{ and } V_1 = \frac{D_2}{k-g}$

    \begin{align*}
    \text{Growth Rate of V} &= \frac{\frac{D_2}{k-g}}{\frac{D_1}{k-g}} - 1\\
    &= \frac{D_2}{D_1} - 1\\
    &= (1+g) - 1 = g
    \end{align*}

\end{frame}

\begin{frame}[t]
    \frametitle{2. DDM Observations}
    \framesubtitle{Capital Gains Rate}

    The growth rate of the equity price is the \textbf{capital gains yield} and the 
    \textbf{dividend yield} is the dividend divided by the price.\\

    We can therefore write:

    \begin{align*}
        E(r) &= \text{Dividend Yield} + \text{Capital Gain Yield}\\
             &= \frac{D_1}{P_0} + \frac{P_1 - P_0}{P_0}\\
             &= \frac{D_1}{P_0} + g
    \end{align*} 

\end{frame}

\begin{frame}[t]
    \frametitle{2. DDM Observations}
    \framesubtitle{PVGO}

    If there is no growth in the firm, the GGM becomes a perpetuity:

    $$V_0 = \frac{E_1}{k}$$

    The \textbf{present value of growth opportunities (PVGO)} is the difference between the value of the stock 
    and the value of the perpuity that represents the no-growth scenario.

    \begin{align*}
        Price &= \text{No-growth value of firm} + PVGO\\
        P_0 &= \frac{E_1}{k} + PVGO
    \end{align*}

\end{frame}



\begin{frame}[t]
    \frametitle{3. DDM Details}
    \framesubtitle{Discount Rates}

    Generally, we'd expect to have higher discount rates for riskier assets.   But, the exact definition of risk 
    matters.   Riskiness of the firm?   Risk as defined by standard deviation (historic or estimated?)?\\
    \vspace{1em}

    \textit{We can use CAPM for the discount rate!}.  CAPM incorporates both the riskiness 
    of the security and the correlation with broad market risk.   
    Moreover, if we use CAPM as the discount rate, then the 
    estimated value of the stock will be produce the CAPM-based expected return.   If the price is equal to this estimated value, 
    then the price would be "fair."\\
    \vspace{1em}
    The discount rate for equities is therefore often referred to as the \textbf{required return}.   A common term for the consensus
     required return is the \textbf{market capitalization rate}.

\end{frame}

\begin{frame}[t]
    \frametitle{3. DDM Details}
    \framesubtitle{CAPM and the Required Return on Equity}

    As a reminder, the CAPM incorporates both the volatility of an asset's total 
    returns AND the correlation of the total returns between the asset and the broad market.\\
    \vspace{1em}

    Inputs to CAPM:
    \begin{enumerate}
        \item Beta to the equity market, where $\beta = \frac{Cov(X,Mkt)}{\sigma_X\sigma_Mkt}$
        \item Risk-free rate: $r_f$
        \item Expected market excess returns: $E[r_{Mkt}] - r_f$
    \end{enumerate}

    Using these inputs, the required return $k$ is given by:
    $$
    k = E[r] = r_f + \beta(E[r_{Mkt} - r_f])
    $$

\end{frame}

\begin{frame}[t]
    \frametitle{3. DDM Details}
    \framesubtitle{Discount Rates}

    \begin{table}
        \centering
        \caption{\textbf{Discount Rates Used for Various Valuations}}
        \begin{tabular}{|p{1.5cm}|m{1.5cm}|m{1.5cm}|m{1.5cm}|m{1.5cm}|}
        \hline
        & \textbf{Treasury Bonds} & \textbf{Corp Bonds} & \textbf{Equity} & \textbf{Firm} \\
        \hline
        \textbf{Rate} & Yield to Maturity & Yield to Maturity & Required Return on Equity & Weighted Ave Cost of Capital \\
        \hline
        \textbf{Symbol} & $y_T$ & $y_C$ & $k$ & $r_{WACC}$ \\
        \hline
        \end{tabular}
    \end{table}

    \begin{align*}
    \text{UST yields} &< \text{Corp bond yields} \\
    &< \text{Weighted Average Cost of Capital}\\
    &< \text{Required Return for Equity} 
    \end{align*}

\end{frame}

\begin{frame}[t]
    \frametitle{3. DDM Details}
    \framesubtitle{Dividends}

    \textbf{Dividends} are occassional payments made by a company to its shareholders. 
    Dividends are discretionary (no contractual obligation to pay).\\
    \vspace{1em}

    \textit{Why should we use dividends for valuation?}
    \begin{itemize}
        \item Theoretically, all profits must be paid out to shareholders at some point
        \item It's a starting point that can be augment easily.
    \end{itemize}

    \textit{Why shouldn't we use dividends for valuation?}
    \begin{itemize}
        \item Companies can also distribute cash through share buybacks
        \item What about companies that don't pay dividends?
    \end{itemize}

\end{frame}

\begin{frame}[t]
    \frametitle{3. DDM Details}
    \framesubtitle{Bank Dividends and Regulatory Policy}

    Banks face unique regulatory constraints on dividend payments, especially during crises:

    \begin{table}
        \centering
        \begin{tabular}{|l|l|l|}
        \hline
        \textbf{Bank} & \textbf{Crisis Period} & \textbf{Dividend Action} \\
        \hline
        Bank of America & 2009-2011 & Cut from \$0.64 to \$0.01 \\
        \hline
        Wells Fargo & 2020-2021 & Reduced from \$0.51 to \$0.10 \\
        \hline
        \end{tabular}
    \end{table}

    \textbf{Federal Reserve Policy During COVID-19 (2020):}
    \begin{itemize}
        \item Suspended share buybacks for large banks (Q2-Q4 2020)
        \item Capped dividend payments at previous quarter levels
        \item Required stress testing before any dividend increases
        \item Policy lifted gradually in 2021 as economic conditions improved
    \end{itemize}

    \textbf{Key Insight:} Bank dividends are subject to regulatory oversight, making them less predictable than typical corporate dividends.

\end{frame}

\begin{frame}[t]
    \frametitle{3. DDM Details}
    \framesubtitle{Dividend - Suspensions}

    Many well-known companies have temporarily suspended dividends during difficult periods. 
    They may resume dividends when conditions improve:

    \begin{table}
        \centering
        \begin{tabular}{|l|l|l|}
        \hline
        \textbf{Company} & \textbf{Suspension Period} & \textbf{Reason} \\
        \hline
        Apple (AAPL) & 1995-2012 & Focus on growth/innovation \\
        \hline
        General Motors & 2008-2014 & Bankruptcy/restructuring \\
        \hline
        Ford Motor & 2006-2012 & Auto industry crisis \\
        \hline
        Delta Air Lines & 2020-2022 & COVID-19 pandemic \\
        \hline
        Disney (DIS) & 2020-2021 & COVID-19 pandemic \\
        \hline
        \end{tabular}
    \end{table}

\end{frame}

\begin{frame}[t]
    \frametitle{3. DDM Details}
    \framesubtitle{Dividend Growth}

    Company managers choose how to use earnings.   They can distribute the earnings to 
    the company's owners through dividends.  The share of earnings paid out as dividends 
    is referred to as the \textbf{dividend payout ratio}.\\
    \vspace{1em}
    Alternatively, they can reinvest in the business.   The share of earnings reinvested 
    in the business is referred to as either the \textbf{earnings retention ratio} or 
    the \textbf{plowback ratio (b)}.\\
    \vspace{1em}
    While a higher dividend payout ratio may increase dividends in the near term, 
    growth requires reinvestment and a higher earnings retentio ratio \textit{may} increase 
    earnings over the long term.

\end{frame}

\begin{frame}[t]
    \frametitle{3. DDM Details}
    \framesubtitle{Dividend Growth}

    \centering
    \includegraphics[width=0.8\textwidth]{figures/ch18_1_div_growth.png}

\end{frame}

\begin{frame}[t]
    \frametitle{3. DDM Details}
    \framesubtitle{Dividend Growth}

    How much growth is generated by reinvesting in the business?   It depends on the 
    business.   Some businesses have really good investment opportunities, while others 
    do not.\\
    \vspace{1em}
    Specifically, the \textbf{return on equity (ROE)} for some businesses is higher than 
    others.   The amount of growth generated by investing in the business is given by:\\
    $$g = ROE \times b$$
    Remember that the Gordon Growth Model assumes that $g$ can be sustained indefinitely, so 
    we are considering reinvestment opportunities that will always be available (not one-offs).  
    For many companies this establishes an upperbound on $b$.

\end{frame}

\begin{frame}[t]
    \frametitle{3. DDM Details}
    \framesubtitle{Dividend Growth}

    Now we are in a position to evaluate how the stock price will respond to different 
    reinvestment policies.   We can write $D_1 = E_1(1-b)$ where $E_1$ is earnings.\\
    \vspace{1em}
    $$P_0 = \frac{E_1(1-b)}{k - ROE \times b}$$
    Notes:
    \begin{enumerate}
        \item If $ROE = k$ then the denominator becomes $k(1-b)$ 
        and the price reduces to $P_0 = \frac{E_1}{k}$.  The price is 
        therefore independent of $b$ and equaly to a perpetuity.   
        Intuitively, investors are 
        indifferent between the company reinvesting or returning the 
        earnings, because the returns are the same either way.
    \end{enumerate}

\end{frame}

\begin{frame}[t]
\frametitle{3. DDM Details}
\framesubtitle{Dividend Growth}

    \begin{enumerate}
        \item[2.] If $ROE > k$ then the company has attractive reinvestment 
        opportunities.   Investors will prefer that it pursues those opportunities, 
        so a higher $b$ will lead to a higher stock price (like the dashed line in the preceding chart)
        \item[3.] If $ROE < k$ then the company does not has attractive reinvestment 
        opportunities.   Investors will prefer that it return earnings through higher dividends, 
        so a lower $b$ will lead to a higher stock price (like the solid line in the preceding chart)
    \end{enumerate}

\end{frame}

\begin{frame}[t]
    \frametitle{4. Multi-stage Models}
    \framesubtitle{}

    Firms move through life cycles  with different dividend profiles.\\
    \vspace{1em}

    \textit{Early Stage}
    \begin{itemize} 
        \item Many opportunities for profitable reinvestment in the company.
        \item Payout ratios are low.
        \item Growth is correspondingly rapid.
    \end{itemize}

    \textit{Mature Stage}
    \begin{itemize}
        \item Opportunities for reinvestment may become harder to find.
        \item Production capacity is enough to meet market demand.
        \item Competitors enter the market.
        \item Firms may distribute a higher fraction of earnings.
        \item \textit{If dividend growth ($g$) stabilizes and $g<k$ then can apply Gordon Growth Model}
    \end{itemize}

\end{frame}

\begin{frame}[t]
    \frametitle{4. Multi-stage Models}
    \framesubtitle{}

    Model the two stages separately:
    $$
    V_0 = [\text{PV of Dividends in Early Stage}] + [\text{PV of Mature Stage}]
    $$
    Be careful to apply the correct discount rates to get the present values
    $$
    V_0 = \left[\frac{D_1}{1+k} + \frac{D_2}{(1+k)^2} + \ldots + \frac{D_H}{(1+k)^H}\right] + [\frac{V_H}{(1+k)^H}]
    $$
    And because the Gordon Growth Model assumptions are satisfied during the mature stage, we can use:
    $$
    V_H = \frac{D_H(1+g)}{k-g}
    $$
    $$
    V_0 = \left[\frac{D_1}{1+k} + \frac{D_2}{(1+k)^2} + \ldots + \frac{D_H}{(1+k)^H}\right] + [\frac{\frac{D_H(1+g)}{k-g}}{(1+k)^H}]
    $$

\end{frame}

\begin{practiceframe}[t]
    \frametitle{5. Practice}
    \framesubtitle{Smilewhite}

    Use a two-stage dividend discount model to value the company SmileWhite.   
    The beta of SmileWhite to the market is 1.15, the risk-free rate is 4.5\%,
     and the expected market return is 14.5\%.  
     SmileWhite just paid a dividend of \$1.72 per share.   
     SmileWhite is expected to increase its dividend by 12\%
      per year for the next three years.  
      After that SmileWhite is expected to increase its dividend by 9\% per year.\\
      \vspace{1em}

\end{practiceframe}

\begin{practiceframe}[t]
    \frametitle{5. Practice}
    \framesubtitle{Smilewhite}

      Approach:
      \begin{enumerate}
        \item Use CAPM to estiamte the required rate of return on equity.
        \item Value the dividends during the high growth period
        \item Value the mature stage using the Gordon Growth Model
        \item Present value of dividends and the \textit{discounted} horizon value.
      \end{enumerate}

      \blfootnote{See Bodi et al, problem 8, page 624}

\end{practiceframe}

\begin{practiceframe}[t]
    \frametitle{5. Practice}
    \framesubtitle{Smilewhite - Step 1}
    Use CAPM to estimate the required return on equity.
    Inputs:
    \begin{itemize}
        \item $\beta_{SmileWhite} = 1.15$
        \item $r_f = 0.045$
        \item $E[r_M] = 0.145$
    \end{itemize}
    \vspace{1em}

    CAPM:
    \begin{align*}
    r_{\text{SmileWhite}} &= r_f + \beta_{SmileWhite} * [E[r_M] - r_f]\\
     &= 0.045 + 1.15*0.1\\
     &= 0.16
    \end{align*}

\end{practiceframe}

\begin{practiceframe}[t]
    \frametitle{5. Practice}
    \framesubtitle{Smilewhite - Step 2}

    Value the dividends during the high growth period, making sure to apply the correct discount rate to each one.

    \begin{align*}
    \text{PV of divs} &= \\
     &=\frac{D_0*(1+0.12)}{(1+0.16)} + \frac{D_0(1+0.12)^2}{(1+0.16)^2} + \frac{D_0(1+0.12)^3}{(1+0.16)^3}\\
     &= \frac{\$1.72*(1+0.12)}{(1+0.16)} + \frac{\$1.72(1+0.12)^2}{(1+0.16)^2} + \frac{\$1.72(1+0.12)^3}{(1+0.16)^3}\\
     &= 1.66 + 1.60 + 1.55\\
    &= 4.81
    \end{align*}

\end{practiceframe}

\begin{practiceframe}[t]
    \frametitle{5. Practice}
    \framesubtitle{Smilewhite - Step 3}

    Use the Gordon Growth Model to determine the value once dividend growth stabilizes 
    (i.e. once the Gordon Growth Model assumptions have been satisfied).   
    This is the Horizon Value $V_H$.
    \begin{align*}
    V_H &= \frac{D_3(1+g)}{k-g}\\
    V_H &= \frac{\$2.63}{0.16-0.09}\\
    V_H &= \frac{\$2.63}{0.7}\\
    V_H &= \$37.57 = \text{Horizon Value}
    \end{align*}

\end{practiceframe}

\begin{practiceframe}[t]
    \frametitle{5. Practice}
    \framesubtitle{Smilewhite - Step 4}

    Today's present value is the sum of the present value of the dividends 
    during the high growth period AND the \textit{discounted} horizon value.
    \begin{align*}
    V_0 &= \text{PV of high growth dividends} + \frac{Horizon Value}{(1+k)^3}\\
    V_0 &= \$4.81 + \frac{\$37.57}{1.16^3}\\
    V_0 &= \$4.81 + \$24.07 = \$28.89
    \end{align*}

\end{practiceframe}

\end{document}