\documentclass{beamer}

\newcommand{\week}{Week 3-b}

\title{Capital Allocation}
\subtitle{Reference: Bodie et al, Ch 6}
\author{Econ 457}
\date{\week}

% Reference the shared preamble
\setbeamertemplate{frametitle}{
  \vspace{0.5em}
  \insertframetitle
  \par
  \vspace{0.5em}
  \hrule
  \vspace{0.3em}
  {\small\color{gray}\insertframesubtitle}
}

\setbeamertemplate{navigation symbols}{}
\setbeamertemplate{itemize item}{\textbullet} % main bullet: filled dot
\setbeamertemplate{itemize subitem}{\normalsize$\circ$} % sub-bullet: empty dot
\setbeamertemplate{itemize subsubitem}{\scriptsize--} % sub-sub-bullet: dash


% Font changes
\usepackage[scaled=0.92]{helvet}
\renewcommand{\familydefault}{\sfdefault}

% Packages
\usepackage{tikz}
\usepackage{booktabs}
\usepackage{xcolor}
\usepackage{array}           % Enhanced column types for tables
\usepackage{multirow}        % Spanning multiple rows in tables
\usepackage{makecell}        % Line breaks and formatting in table cells
\usepackage{siunitx}         % Proper formatting of numbers and units
\usepackage{amsmath}         % Enhanced math environments
\usepackage{amsfonts}        % Additional math fonts
\usepackage{amssymb}         % Additional math symbols
\usepackage{url}             % Better URL formatting
\usepackage{graphicx}        % Enhanced graphics support
\usepackage{tabularray}
\UseTblrLibrary{booktabs, siunitx, varwidth}
% For financial presentations specifically
\usepackage{eurosym}         % Euro symbol
\usepackage{textcomp}        % Additional text symbols
\usepackage{hyperref}        % Hyperlinks (should be loaded last)

% Define a footnote
\renewcommand{\footnoterule}{\vspace*{-3pt}\hrule width 2in height 0.4pt\vspace*{2.6pt}}

% Define a Foundation Slide
\newenvironment{foundframe}[1][t]{
    \setbeamercolor{background canvas}{bg=gray!8}
    \setbeamercolor{frametitle}{fg=gray!80!black,bg=gray!25}
    \setbeamercolor{framesubtitle}{fg=gray!70!black,bg=gray!15}
    \setbeamercolor{item}{fg=gray!80!black}
    \setbeamercolor{enumerate item}{fg=gray!80!black}
    
    % Modify the frametitle template for this frame type
    \setbeamertemplate{frametitle}{
        \vspace{0.5em}
        \begin{minipage}[t]{0.75\textwidth}
            \insertframetitle
            \par
            \vspace{0.5em}
            \hrule
            \vspace{0.3em}
            {\small\color{gray}\insertframesubtitle}
        \end{minipage}%
        \hfill
        \begin{minipage}[t]{0.2\textwidth}
            \raggedleft
            \colorbox{gray!30}{%
                \scriptsize\bfseries\color{gray!80!black}%
                   \hspace{3pt}\begin{tabular}{c}Foundation\\Material\end{tabular}\hspace{3pt}%
            }
        \end{minipage}
        \vspace{0.3em}
    }
    
    \begin{frame}[#1]
}{
    \end{frame}
}

% Define Practice Slide
\newenvironment{practiceframe}[1][t]{
    \setbeamercolor{background canvas}{bg=white}
    \setbeamercolor{frametitle}{fg=blue!80!black,bg=blue!15}
    \setbeamercolor{framesubtitle}{fg=blue!70!black,bg=blue!10}
    \setbeamercolor{item}{fg=blue!80!black}
    \setbeamercolor{enumerate item}{fg=blue!80!black}
    \setbeamercolor{normal text}{fg=blue!90!black}
    
    % Modify the frametitle template for this frame type
    \setbeamertemplate{frametitle}{
        \vspace{0.5em}
        \begin{minipage}[t]{0.75\textwidth}
            \insertframetitle
            \par
            \vspace{0.5em}
            \hrule
            \vspace{0.3em}
            {\small\color{blue!70!black}\insertframesubtitle}
        \end{minipage}%
        \hfill
        \begin{minipage}[t]{0.2\textwidth}
            \raggedleft
            \colorbox{blue!20}{%
                \scriptsize\bfseries\color{blue!80!black}%
                   \hspace{3pt}\begin{tabular}{c}Practice\\Questions\end{tabular}\hspace{3pt}%
            }
        \end{minipage}
        \vspace{0.3em}
    }
    
    \begin{frame}[#1]
}{
    \end{frame}
}

% Define Excel Slide
\newenvironment{excelframe}[1][t]{
    \setbeamercolor{background canvas}{bg=white}
    \setbeamercolor{frametitle}{fg=blue!80!black,bg=blue!15}
    \setbeamercolor{framesubtitle}{fg=blue!70!black,bg=blue!10}
    \setbeamercolor{item}{fg=blue!80!black}
    \setbeamercolor{enumerate item}{fg=blue!80!black}
    \setbeamercolor{normal text}{fg=blue!90!black}
    
    % Modify the frametitle template for this frame type
    \setbeamertemplate{frametitle}{
        \vspace{0.5em}
        \begin{minipage}[t]{0.75\textwidth}
            \insertframetitle
            \par
            \vspace{0.5em}
            \hrule
            \vspace{0.3em}
            {\small\color{blue!70!black}\insertframesubtitle}
        \end{minipage}%
        \hfill
        \begin{minipage}[t]{0.2\textwidth}
            \raggedleft
            \colorbox{green!10}{%
                \scriptsize\bfseries\color{blue!80!black}%
                   \hspace{3pt}\begin{tabular}{c}MS Excel\end{tabular}\hspace{3pt}%
            }
        \end{minipage}
        \vspace{0.3em}
    }
    
    \begin{frame}[#1]
}{
    \end{frame}
}

% Define Caution Slide
\newenvironment{cautionframe}[1][t]{
    \setbeamercolor{background canvas}{bg=white}
    \setbeamercolor{frametitle}{fg=blue!80!black,bg=blue!15}
    \setbeamercolor{framesubtitle}{fg=blue!70!black,bg=blue!10}
    \setbeamercolor{item}{fg=blue!80!black}
    \setbeamercolor{enumerate item}{fg=blue!80!black}
    \setbeamercolor{normal text}{fg=blue!90!black}
    
    % Modify the frametitle template for this frame type
    \setbeamertemplate{frametitle}{
        \vspace{0.5em}
        \begin{minipage}[t]{0.75\textwidth}
            \insertframetitle
            \par
            \vspace{0.5em}
            \hrule
            \vspace{0.3em}
            {\small\color{blue!70!black}\insertframesubtitle}
        \end{minipage}%
        \hfill
        \begin{minipage}[t]{0.2\textwidth}
            \raggedleft
            \colorbox{red!10}{%
                \scriptsize\bfseries\color{blue!80!black}%
                   \hspace{3pt}\begin{tabular}{c}Caution\end{tabular}\hspace{3pt}%
            }
        \end{minipage}
        \vspace{0.3em}
    }
    
    \begin{frame}[#1]
}{
    \end{frame}
}

% Add to footnotes
\makeatletter
\newcommand\blfootnote[1]{%
  \begingroup
  \renewcommand\thefootnote{}%
  \renewcommand\@makefntext[1]{\raggedright\leftskip=0pt ##1}%
  \footnote{\scriptsize #1}%
  \addtocounter{footnote}{-1}%
  \endgroup
}
\makeatother

% Set the footer -- change 
\setbeamertemplate{footline}{
  \leavevmode%
  \vspace{2ex}
  \hbox{%
    % Left box: Econ 457
    \begin{beamercolorbox}[wd=.4\paperwidth,ht=2.5ex,dp=1ex,left]{author in head/foot}%
      \hspace{1em}Econ 457
    \end{beamercolorbox}%
    % Middle box: Week
    \begin{beamercolorbox}[wd=.2\paperwidth,ht=2.5ex,dp=1ex,center]{date in head/foot}%
      \centering\week
    \end{beamercolorbox}%
    % Right box: Slide numbers
    \begin{beamercolorbox}[wd=.4\paperwidth,ht=2.5ex,dp=1ex,center]{date in head/foot}%
      \hfill\insertframenumber{} 
    \end{beamercolorbox}%
  }%
  \vskip0pt%
}

\begin{document}

\frame{\titlepage}

\begin{frame}
    \frametitle{Outline}

    \begin{enumerate}
        \item Capital Allocation: Review
        \item Capital Allocation: Practice
        \item Leverage and Extending the CAL
        \item Leverage in Practice
        \item Margin Calls
        \item LTCM 1998
        \item Excel
    \end{enumerate}
\end{frame}

\begin{frame}[t]
    \frametitle{1. Capital Allocation: Review}
    \framesubtitle{}

    \centering
    \includegraphics[width=0.8\textwidth]{figures/ch6_1_opt.png}

\end{frame}

\begin{practiceframe}[t]
    \frametitle{2. Capital Allocaiton: Practice}

    \vspace{-1em}
    You manage a risky portfolio with expected rate of return of 12\% and a standard deviation of 28\%. 
    The T-Bill rate is 2\%.
    \begin{enumerate}
        \item You clinet chooses to invest 70\% of a portfolio in your fund and 30\% in a risk-free money 
        market fund.   What are the expected return and standard deviation on her portfolio?
        \item What is the reward-to-volatility (Sharpe) ratio of her portfolio?   Of your risky portfolio?   (Hint: draw the CAL, what is the slope?  Where is your client's portfolio?)
        \item Suppose your client instead chooses to invest a proportion $y$ that maximizes the expected return, 
        subject to the constraint that the complete portfolio standard deviation will not exeed 12\%.   What is in the invested proportion $y$?   What is the 
        expected return?
        \item Your client's degree of risk aversion is $A = 3.5$.   What proportion $y$ should they invest in your fund?
    \end{enumerate}

\end{practiceframe}

\begin{practiceframe}[t]
    \frametitle{2. Capital Allocaiton: Practice}

    \vspace{-1em}
    Consider a risky portfolio with $\mathbb{E}[r_p]=8\%$ and $\sigma_p=15\%$.   The risk free rate is 2\%.
    \begin{enumerate}
    \setcounter{enumi}{4}
        \item Your client wants to invest a proportion of her budget in this fund to get a 
        a rate of return on her overall portfolio of 5\%.   What proportion of her total portfolio 
        should she invest in the risky fund?
        \item What will the standard deviation be on her total portfolio?
        \item Another client wants the highest return possible, subject to the constraint that you 
        limit the standard deviation to be no more than 12\%.   Which client is more risk averse?
    \end{enumerate}

\end{practiceframe}


\begin{frame}[t]
    \frametitle{3. Leverage and Extending the CAL}
    \framesubtitle{Risk Lovers}

    \centering
    \includegraphics[width=0.8\textwidth]{figures/ch6_2_risk_lovers.png}

\end{frame}

\begin{frame}[t]
    \frametitle{3. Leverage and Extending the CAL}
    \framesubtitle{Risk Lovers}

    \centering
    \includegraphics[width=0.8\textwidth]{figures/ch6_max_risk_lovers_w.png}

\end{frame}

\begin{frame}[t]
    \frametitle{3. Leverage and Extending the CAL}
    \framesubtitle{Cost of Leverage}

    Leverage involves borrowing money in order purchase more assets.
    Leverage makes it possible to attain higher expected returns, with additional risk.\\  
    \vspace{1em}
    Typically borrowing has an associated cost.
    The cost depends on the riskiness of the assets being purchased (more on this below).
    For a broad basket of equities, the cost of borrowing is typically close to the risk-free rate.
    \vspace{1em}

\end{frame}

\begin{frame}[t]
    \frametitle{3. Leverage and Extending the CAL}
    \framesubtitle{Cost of Leverage}

    The slope of the line to the right of the CML is $\frac{E[r_{risky}] - 2 \cdot r_f}{\sigma_{risky}}$,
    in order to account for the cost of borrowing.

    \centering
    \includegraphics[width=0.75\textwidth]{figures/ch6_1_cal_wleverage.png}

\end{frame}

\begin{frame}[t]
    \frametitle{3. Leverage and Extending the CAL}
    \framesubtitle{Graphic Representation and Optimization}

    Extending the CAL matters for risk-loving prefernces (low $A$ values).
    Can consider leverage in cases when $w^*=100\%$.

    \centering
    \includegraphics[width=0.75\textwidth]{figures/ch6_1_cal_wleverage_opt.png}

\end{frame}

\begin{cautionframe}[t]
    \frametitle{3. Leverage and Extending the CAL}
    \framesubtitle{Assumptions in CAL Model}

    Be careful, the capital allocation process described so far makes 
    two very strong assumptions, 
    both of which are based, in part, on a assumption of normally distributed returns:
    \begin{enumerate}
        \item Investor preferences are determined by \textit{only} the mean and variance of returns.   
        Investors are "mean-variance optimizers."
        \item The cost of leverage is \textit{only} the cost of borrowing (here $2*r_{rf}$).
    \end{enumerate}
    In fact, investors probably also care about the probability of bankruptcy (probability in the tails).  
    The true cost of leverage may be that it introduces a vulnerability  
    that can exacerbate losses.

\end{cautionframe}

\begin{frame}[t]
    \frametitle{4. Leverage in Practice}
    \framesubtitle{Examples}

    Three prominent examples of leverage in practice:

    \begin{enumerate}
        \item Housing
        \item Banking
        \item Buying Assets on Margin
    \end{enumerate}

\end{frame}

\begin{frame}[t]
    \frametitle{4. Leverage in Practice}
    \framesubtitle{Housing}

    \begin{columns}
        \begin{column}{0.4\textwidth}
            The typical mortgage requires a 20\% downpayment.
            The remaining 80\% of the cost of the house is borrowed.\\
            \vspace{1em}
            The leverage ratio is 5:1 (total cost / amount of equity).
        \end{column}
        \begin{column}{0.6\textwidth}
            \centering
            \includegraphics[width=\textwidth]{figures/ch6_2_mtg.png}
        \end{column}
    \end{columns}

\end{frame}

\begin{frame}
    \frametitle{4. Leverage in Practice}
    \framesubtitle{Housing, cont'd}

    Example: 
    \begin{itemize}
        \item You buy a house for \$1,000,000.   
        \item You make a 20\% downpayment (\$200,000)
        \item You finance the remainding \$800,000 with a 7\% mortgage.
    \end{itemize}

    \begin{table}
        \caption{Example Home Ownership Returns}
        \begin{tabular}{lccc}
        \toprule
        Scenario & Price Change & Interest Cost & Return on Equity \\
        \midrule
        HPI +10\% & \$100,000 & \$56,000 & 22.0\% \\
        HPI +20\% & \$200,000 & \$56,000 & 72.0\% \\
        HPI -10\% & -\$100,000 & \$56,000 & -78.0\% \\
        \bottomrule
    \end{tabular}
    \end{table}
\end{frame}

\begin{frame}[t]
    \frametitle{4. Leverage in Practice}
    \framesubtitle{Banking}

    Banks are required to be financed with $\sim$10\% equity capital.\\
    \vspace{1em}
    While it varies widely with the bank, the remainder is financed with a mix of deposits, which are really short term loans, and unsecured debt.\\
    \vspace{1em}
    Banks are typically levered around 10:1 (total assets / equity capital)

\end{frame}

\begin{frame}[t]
    \frametitle{4. Leverage in Practice}
    \framesubtitle{Banking, cont'd}

    \footnotesize
    The core business of banking is making loans to households and businesses.
    The rate on these loans is, say, around $r_f + 500 bps$ (it varies widely with the credit quality, maturity, etc).
    If the bank finances these loans using unsecured debt, the cost of borrowing is, say, around $r_f + 200 bps$ (again, varies widely).
    The banks' net interest margin on the loan is $300 bps$.  The bank must also account for charge-offs and costs associated with making loans.
    Let's say those are an additional $200 bps$.   The total return for the bank, after expenses, is $100 bps$.  \textit{This is not a particularly high return.}\\
    \vspace{1em}
    The banks return on equity, on the other hand, is much higher.  Because the leverage ratio is 10:1, if the bank makes $100 bps$, then the equity holder would make $10\%$!\\
    \vspace{1em}
    While this example is purely for illustration, and reality is more complex, it does reflect the basics of the banking business model.

\end{frame}

\begin{frame}[t]
    \frametitle{4. Leverage in Practice}
    \framesubtitle{Banking, cont'd}

    \centering
    \includegraphics[width=1\textwidth]{figures/ch6_2_fdic.png}

\end{frame}

\begin{frame}[t]
    \frametitle{5. Margin Calls}
    \framesubtitle{Buying Assets on Margin}

    When investors borrow money to purchase asssets, 
    the lender (often a broker) typically requires them to 
    post collateral or 'margin', which acts as surety for the lender.\\
    \vspace{1em}
    The margin requirements may vary with the riskiness of the assets being purchased.  
    For common stocks the requirements may be 50\% initial margin 
    and then 30\% maintenance margin.\\
    \vspace{1em}
    If the stocks that were purchased fall in price, the value of the posted margin declines as well.   
    If the value falls below the maintenance margin, the investor is forced to post additional margin, or the broker will close the position.
    This is known as a "margin call".

\end{frame}

\begin{frame}[t]
    \frametitle{5. Margin Calls}
    \framesubtitle{Buying Assets on Margin}

    Margin is calculated as the owners' equity divided by the value of the assets owned in the 
    account.\\
    \vspace{1em}
    If you have \$10,000 of equity to fund a margin account, assuming a 50\% initial margin, you can 
    buy \$20,000 worth of assets.\\
    \vspace{1em}
    In order to see where the margin call would happen, note that maintenance margin requirement 
    can be expressed as:
    $$\frac{\$20,000 * (1+r) - \$10,000}{\$20,000 * (1+r)} > 30\%$$
    If that requirement is violated, you must either deposit more equity or the positions will be 
    closed at a loss.

\end{frame}

\begin{frame}[t]
    \frametitle{5. Margin Calls}
    \framesubtitle{Buying Assets on Margin, example}

    \footnotesize
    You buy \$100 of stock by borrowing \$50 from your broker.   The following table summarizes your results, ignorning interest costs for simplicity:\\
    \begin{table}
        \caption{Margin Trading Example: \$100 Stock, \$50 Equity, \$50 Borrowed}
        \begin{tabular}{lcccc}
        \toprule
        Scenario & Stock Value & Equity Value & Return on Equity & Margin Call \\
        \midrule
        Initial & \$100 & \$50 & -- & NO \\
        Stock +20\% & \$120 & \$70 & 40.0\% & NO \\
        Stock +10\% & \$110 & \$60 & 20.0\% & NO \\
        Stock -10\% & \$90 & \$40 & -20.0\% & NO \\
        Stock -20\% & \$80 & \$30 & -40.0\% & NO \\
        Stock -30\% & \$70 & \$20 & -60.0\% & YES \\
        Stock -40\% & \$60 & \$10 & -80.0\% & YES \\
        \bottomrule
        \end{tabular}
    \end{table}

\end{frame}

\begin{frame}[t]
    \frametitle{6. LTCM in 1998}
    \framesubtitle{}

    \centering
    \includegraphics[width=0.9\textwidth]{figures/ltcm.jpg}

    \blfootnote{Source: \textit{When Genius Failed}, Roger Lowenstein, Random House, 2000}

\end{frame}

\begin{frame}[t]
    \frametitle{6. LTCM in 1998}
    \framesubtitle{}

    \vspace{-2em}

    \begin{table}
    \caption{LTCM Losses by Position}
    \begin{tabular}{lc}
    \toprule
    Trade & Loss\\
    \midrule
    Russia and Emerging Markets & \$430 million\\
    Directional Trades & \$371 million\\
    Equity Pairs & \$286 million\\
    Yield Curve & \$215 million\\
    S\&P Stocks & \$203 million\\
    High Yield Bond Arbitrage & \$100 million\\
    Merger Arbitrage & roughly even\\
    \midrule
    Swap Spreads & \$1,600 million\\
    Equity Volatility & \$1,300 million\\
    \bottomrule
    \end{tabular}
    \end{table}

    \blfootnote{Source: \textit{When Genius Failed}, Roger Lowenstein, Random House, 2000, page 235}

\end{frame}

\begin{frame}[t]
    \frametitle{6. LTCM in 1998}
    \framesubtitle{}

    \centering
    \includegraphics[width=0.9\textwidth]{figures/ch6_vix.png}

    \blfootnote{VIX Index is calculated by the CBOE.  Data from Yahoo Finance}

\end{frame}

\begin{frame}[t]
    \frametitle{6. LTCM in 1998}
    \framesubtitle{}

    \centering
    \includegraphics[width=0.9\textwidth]{figures/ch6_vix_ltcm.png}

    \blfootnote{VIX Index is calculated by the CBOE.  Data from Yahoo Finance}

\end{frame}


\begin{foundframe}[t]
    \frametitle{REVIEW: The Normal Distribution}
    \framesubtitle{Empirical Estimates}

    Empirical mean:
    $$\bar{x} = \frac{1}{T}\sum_{t=1}^{T}x_t$$

    Empirical variance:
    $$s^2 = \frac{1}{T-1}\sum_{t=1}^{T}(x_t - \bar{x})^2$$

    Empirical standard deviation:
    $$s = \sqrt{s^2} = \sqrt{\frac{1}{T-1}\sum_{t=1}^{T}(x_t - \bar{x})^2}$$

\end{foundframe}

\begin{foundframe}[t]
    \frametitle{REVIEW: The Normal Distribution}
    \framesubtitle{Empirical Estimates}

    Empirical covariance:
    $$\operatorname{Cov}(x,y) = \frac{1}{T-1}\sum_{t=1}^{T}(x_t - \bar{x})(y_t - \bar{y})$$

\end{foundframe}

\begin{excelframe}[t]
    \frametitle{7. Excel}
    \framesubtitle{}

    \begin{itemize}
        \item AVERAGE()
        \item STDEV()
        \item VAR.P()
        \item COVARIANCE.P()
        \item SUMPRODUCT()
    \end{itemize}

\end{excelframe}

\end{document}