\documentclass{beamer}

\newcommand{\week}{Week 4-a}

\title{Diversification}
\subtitle{Reference: Bodie et al, Ch 7}
\author{Econ 457}
\date{\week}

% Reference the shared preamble
\setbeamertemplate{frametitle}{
  \vspace{0.5em}
  \insertframetitle
  \par
  \vspace{0.5em}
  \hrule
  \vspace{0.3em}
  {\small\color{gray}\insertframesubtitle}
}

\setbeamertemplate{navigation symbols}{}
\setbeamertemplate{itemize item}{\textbullet} % main bullet: filled dot
\setbeamertemplate{itemize subitem}{\normalsize$\circ$} % sub-bullet: empty dot
\setbeamertemplate{itemize subsubitem}{\scriptsize--} % sub-sub-bullet: dash


% Font changes
\usepackage[scaled=0.92]{helvet}
\renewcommand{\familydefault}{\sfdefault}

% Packages
\usepackage{tikz}
\usepackage{booktabs}
\usepackage{xcolor}
\usepackage{array}           % Enhanced column types for tables
\usepackage{multirow}        % Spanning multiple rows in tables
\usepackage{makecell}        % Line breaks and formatting in table cells
\usepackage{siunitx}         % Proper formatting of numbers and units
\usepackage{amsmath}         % Enhanced math environments
\usepackage{amsfonts}        % Additional math fonts
\usepackage{amssymb}         % Additional math symbols
\usepackage{url}             % Better URL formatting
\usepackage{graphicx}        % Enhanced graphics support
\usepackage{tabularray}
\UseTblrLibrary{booktabs, siunitx, varwidth}
% For financial presentations specifically
\usepackage{eurosym}         % Euro symbol
\usepackage{textcomp}        % Additional text symbols
\usepackage{hyperref}        % Hyperlinks (should be loaded last)

% Define a footnote
\renewcommand{\footnoterule}{\vspace*{-3pt}\hrule width 2in height 0.4pt\vspace*{2.6pt}}

% Define a Foundation Slide
\newenvironment{foundframe}[1][t]{
    \setbeamercolor{background canvas}{bg=gray!8}
    \setbeamercolor{frametitle}{fg=gray!80!black,bg=gray!25}
    \setbeamercolor{framesubtitle}{fg=gray!70!black,bg=gray!15}
    \setbeamercolor{item}{fg=gray!80!black}
    \setbeamercolor{enumerate item}{fg=gray!80!black}
    
    % Modify the frametitle template for this frame type
    \setbeamertemplate{frametitle}{
        \vspace{0.5em}
        \begin{minipage}[t]{0.75\textwidth}
            \insertframetitle
            \par
            \vspace{0.5em}
            \hrule
            \vspace{0.3em}
            {\small\color{gray}\insertframesubtitle}
        \end{minipage}%
        \hfill
        \begin{minipage}[t]{0.2\textwidth}
            \raggedleft
            \colorbox{gray!30}{%
                \scriptsize\bfseries\color{gray!80!black}%
                   \hspace{3pt}\begin{tabular}{c}Foundation\\Material\end{tabular}\hspace{3pt}%
            }
        \end{minipage}
        \vspace{0.3em}
    }
    
    \begin{frame}[#1]
}{
    \end{frame}
}

% Define Practice Slide
\newenvironment{practiceframe}[1][t]{
    \setbeamercolor{background canvas}{bg=white}
    \setbeamercolor{frametitle}{fg=blue!80!black,bg=blue!15}
    \setbeamercolor{framesubtitle}{fg=blue!70!black,bg=blue!10}
    \setbeamercolor{item}{fg=blue!80!black}
    \setbeamercolor{enumerate item}{fg=blue!80!black}
    \setbeamercolor{normal text}{fg=blue!90!black}
    
    % Modify the frametitle template for this frame type
    \setbeamertemplate{frametitle}{
        \vspace{0.5em}
        \begin{minipage}[t]{0.75\textwidth}
            \insertframetitle
            \par
            \vspace{0.5em}
            \hrule
            \vspace{0.3em}
            {\small\color{blue!70!black}\insertframesubtitle}
        \end{minipage}%
        \hfill
        \begin{minipage}[t]{0.2\textwidth}
            \raggedleft
            \colorbox{blue!20}{%
                \scriptsize\bfseries\color{blue!80!black}%
                   \hspace{3pt}\begin{tabular}{c}Practice\\Questions\end{tabular}\hspace{3pt}%
            }
        \end{minipage}
        \vspace{0.3em}
    }
    
    \begin{frame}[#1]
}{
    \end{frame}
}

% Define Excel Slide
\newenvironment{excelframe}[1][t]{
    \setbeamercolor{background canvas}{bg=white}
    \setbeamercolor{frametitle}{fg=blue!80!black,bg=blue!15}
    \setbeamercolor{framesubtitle}{fg=blue!70!black,bg=blue!10}
    \setbeamercolor{item}{fg=blue!80!black}
    \setbeamercolor{enumerate item}{fg=blue!80!black}
    \setbeamercolor{normal text}{fg=blue!90!black}
    
    % Modify the frametitle template for this frame type
    \setbeamertemplate{frametitle}{
        \vspace{0.5em}
        \begin{minipage}[t]{0.75\textwidth}
            \insertframetitle
            \par
            \vspace{0.5em}
            \hrule
            \vspace{0.3em}
            {\small\color{blue!70!black}\insertframesubtitle}
        \end{minipage}%
        \hfill
        \begin{minipage}[t]{0.2\textwidth}
            \raggedleft
            \colorbox{green!10}{%
                \scriptsize\bfseries\color{blue!80!black}%
                   \hspace{3pt}\begin{tabular}{c}MS Excel\end{tabular}\hspace{3pt}%
            }
        \end{minipage}
        \vspace{0.3em}
    }
    
    \begin{frame}[#1]
}{
    \end{frame}
}

% Define Caution Slide
\newenvironment{cautionframe}[1][t]{
    \setbeamercolor{background canvas}{bg=white}
    \setbeamercolor{frametitle}{fg=blue!80!black,bg=blue!15}
    \setbeamercolor{framesubtitle}{fg=blue!70!black,bg=blue!10}
    \setbeamercolor{item}{fg=blue!80!black}
    \setbeamercolor{enumerate item}{fg=blue!80!black}
    \setbeamercolor{normal text}{fg=blue!90!black}
    
    % Modify the frametitle template for this frame type
    \setbeamertemplate{frametitle}{
        \vspace{0.5em}
        \begin{minipage}[t]{0.75\textwidth}
            \insertframetitle
            \par
            \vspace{0.5em}
            \hrule
            \vspace{0.3em}
            {\small\color{blue!70!black}\insertframesubtitle}
        \end{minipage}%
        \hfill
        \begin{minipage}[t]{0.2\textwidth}
            \raggedleft
            \colorbox{red!10}{%
                \scriptsize\bfseries\color{blue!80!black}%
                   \hspace{3pt}\begin{tabular}{c}Caution\end{tabular}\hspace{3pt}%
            }
        \end{minipage}
        \vspace{0.3em}
    }
    
    \begin{frame}[#1]
}{
    \end{frame}
}

% Add to footnotes
\makeatletter
\newcommand\blfootnote[1]{%
  \begingroup
  \renewcommand\thefootnote{}%
  \renewcommand\@makefntext[1]{\raggedright\leftskip=0pt ##1}%
  \footnote{\scriptsize #1}%
  \addtocounter{footnote}{-1}%
  \endgroup
}
\makeatother

% Set the footer -- change 
\setbeamertemplate{footline}{
  \leavevmode%
  \vspace{2ex}
  \hbox{%
    % Left box: Econ 457
    \begin{beamercolorbox}[wd=.4\paperwidth,ht=2.5ex,dp=1ex,left]{author in head/foot}%
      \hspace{1em}Econ 457
    \end{beamercolorbox}%
    % Middle box: Week
    \begin{beamercolorbox}[wd=.2\paperwidth,ht=2.5ex,dp=1ex,center]{date in head/foot}%
      \centering\week
    \end{beamercolorbox}%
    % Right box: Slide numbers
    \begin{beamercolorbox}[wd=.4\paperwidth,ht=2.5ex,dp=1ex,center]{date in head/foot}%
      \hfill\insertframenumber{} 
    \end{beamercolorbox}%
  }%
  \vskip0pt%
}

\begin{document}

\frame{\titlepage}

\begin{frame}
    \frametitle{Outline}

    \begin{enumerate}
        \item Review: Data Construction
        \item Sum of Two Random Variables
        \item SPY and TLT
        \item Optimization: SPY, TLT and Cash
    \end{enumerate}
\end{frame}

\begin{frame}[t]
    \frametitle{1. Review: Data Construction}
    \framesubtitle{}

    Total Return (\%) = Price Return (\%) + Dividend Yield (\%)\\
    \vspace{1em}
    Where the Dividend Yield = Income / (Initial Price).\\
    \vspace{1em}

    Compute Total Return Index:
    \begin{enumerate}
      \item Match income series to price series
      \item Compute Total Return for Each Period: $r_{tr}$
      \item Create an Index starting at 100, subsequent periods = $i_t = (1+r_{tr}) \cdot i_{t-1}$
    \end{enumerate}

\end{frame}

\begin{frame}[t]
    \frametitle{1. Review Data Construction}
    \framesubtitle{Dividend Yield}

    \begin{columns}
        \begin{column}{0.5\textwidth}
            \centering
            \includegraphics[width=1.1\textwidth]{figures/ch7_1_SPY_pdiv}
        \end{column}
        \begin{column}{0.5\textwidth}
            \centering
            \includegraphics[width=1.1\textwidth]{figures/ch7_1_TLT_pdiv}
        \end{column}
    \end{columns}

    \vfill
    \footnotesize
    Data Source: Yahoo finance\\
    Dividend yield is calculated using the 12-month \textit{trailing} dividends and the current price.   While this is standard practice, it is also not great. 

\end{frame}

\begin{frame}[t]
    \frametitle{1. Review: Data Construction}
    \framesubtitle{Sharpe Ratio}

    \textbf{Sharpe Ratio} = Excess Returns / Std Dev of Excess Returns\\
    $$\frac{E[r] - r_f}{\sigma}$$
    Where $E[r] - r_f$ is the \textit{excess} return of the asset and $\sigma$ is the standard deviation.\\
    \vspace{1em}
    Compute Historical Sharpe Ratios:
    \begin{enumerate}
      \item Match Total Return Series to Risk-free Return Series
      \item For each period, compute Excess Returns = Total Return (\%) - Risk-Free Return (\%)
      \item Compute Mean and Standard Deviation of Excess Returns
      \item Sharpe Ration = Mean / Standard Deviation
    \end{enumerate}

\end{frame}

\begin{frame}[t]
    \frametitle{1. Review: Data Construction}
    \framesubtitle{Sharpe Ratio}

    \footnotesize
    \begin{table}
        \caption{ETF Performance Statitics - Excess Returns}
        \begin{tabular}{llrrr}
        \toprule
        & Start Date & Mean (\%) & Std Dev (\%) & Sharpe Ratio \\
        Ticker &  &  &  &  \\
        \midrule
        SPY & 1993-01 & 8.440 & 14.820 & 0.569 \\
        TLT & 2002-07 & 3.090 & 13.660 & 0.226 \\
        GLD & 2004-11 & 9.180 & 16.680 & 0.550 \\
        EWW & 1996-03 & 9.420 & 26.570 & 0.355 \\
        EWD & 1996-03 & 8.640 & 24.470 & 0.353 \\
        EWH & 1996-03 & 4.630 & 24.100 & 0.192 \\
        EWI & 1996-03 & 6.440 & 23.820 & 0.271 \\
        EWJ & 1996-03 & 1.170 & 17.700 & 0.066 \\
        EWL & 1996-03 & 6.350 & 16.610 & 0.382 \\
        EWP & 1996-03 & 7.980 & 23.650 & 0.338 \\
        \bottomrule
        \end{tabular}
    \end{table}

\end{frame}

\begin{foundframe}[t]
    \frametitle{2. Sum of Two Random Variable}
    \framesubtitle{}

    Adding two normally distributed variables $X$ and $Y$:\\
    \vspace{1em}
    
    \textbf{Case 1:} $X$ and $Y$ are \textit{independent}
    $$X + Y \sim N(\mu_x+\mu_y,\sigma_x^2 + \sigma_y^2)$$
     
    \textbf{Case 2:} $X$ and $Y$ are \textit{correlated} 
    $$X + Y \sim N(\mu_x+\mu_y,\sigma_x^2 + \sigma_y^2 + 2Cov(X,Y))$$
    
    \vfill
    Note, in a previou lecture we saw this using the correlation coefficient: $\rho_{x,y} = \frac{\text{Cov}(X,Y)}{\sigma_x\sigma_y}$.   

\end{foundframe}

\begin{foundframe}[t]
    \frametitle{2. Sum of Two Random Variable}
    \framesubtitle{Variance-Covariance Matrix}

    The \textbf{Variance-Covariance matrix} of $X$ and $Y$ is as follows:
    $$\boldsymbol{\Sigma} = \begin{pmatrix}
    \sigma_x^2 & \text{Cov}(X,Y) \\
    \text{Cov}(X,Y) & \sigma_y^2
    \end{pmatrix}$$

    where:
    \begin{itemize}
        \item $\sigma_x^2$ is the variance of $X$
        \item $\sigma_y^2$ is the variance of $Y$  
        \item $\text{Cov}(X,Y)$ is the covariance between $X$ and $Y$
    \end{itemize}

\end{foundframe}

\begin{foundframe}[t]
    \frametitle{2. Sum of Two Random Variable}
    \framesubtitle{Empirical Variance-Covariance Matrix}

    \footnotesize

    Given observations $(x_1, y_1), (x_2, y_2), \ldots, (x_T, y_T)$, the empirical variance-covariance matrix is:

    $$\hat{\boldsymbol{\Sigma}} = \begin{pmatrix}
    \hat{\sigma}_x^2 & \widehat{\text{Cov}}(X,Y) \\
    \widehat{\text{Cov}}(X,Y) & \hat{\sigma}_y^2
    \end{pmatrix}$$

    where:
    \begin{align}
    \hat{\sigma}_x^2 &= \frac{1}{T-1}\sum_{t=1}^{T}(x_t - \bar{x})^2\\
    \hat{\sigma}_y^2 &= \frac{1}{T-1}\sum_{t=1}^{T}(y_t - \bar{y})^2\\
    \widehat{\text{Cov}}(X,Y) &= \frac{1}{T-1}\sum_{t=1}^{T}(x_t - \bar{x})(y_t - \bar{y})
    \end{align}

    and $\bar{x} = \frac{1}{T}\sum_{t=1}^{T}x_t$, $\bar{y} = \frac{1}{T}\sum_{t=1}^{T}y_t$

\end{foundframe}

\begin{frame}[t]
    \frametitle{SPY and TLT}
    \framesubtitle{Portfolio Construction}

    You are constructing a portfolio with SPY and TLT.   
    For the moment, your only choice is how much to allocate to SPY.   
    The remainder is allocated to TLT.\\
    \vspace{1em}
    A note on terminology: The \textit{risk premium} is the expected 
    excess return of a security: $\mathbb{E}[r_{SPY}] - r_{f}$ The historic 
    average excess returns is often used as an estimate of the risk premium.

\end{frame}

\begin{frame}
    \frametitle{SPY and TLT}
    \framesubtitle{Total Returns}

    \centering
    \includegraphics[width=0.9\textwidth]{figures/ch7_sp_tlt_totalreturn.png}

\end{frame}

\begin{frame}[t]
    \frametitle{SPY and TLT}
    \framesubtitle{Empirical Means, Variance-Covariance Matrix}

    \textbf{Summary Statistics:}

    $$\text{Excess Returns:} [\mu_{SPX},\mu_{TLT}] = [9.4,3.1]$$
    
    $$\hat{\boldsymbol{\Sigma}} = \begin{pmatrix}
    \hat{\sigma}_{SPY}^2 & \widehat{\text{Cov}}(SPY,TLT) \\
    \widehat{\text{Cov}}(SPY,TLT) & \hat{\sigma}_{TLT}^2
    \end{pmatrix} = \begin{pmatrix}
    219.61 & -21.09 \\
    -21.09 & 186.68
    \end{pmatrix}$$

    \vspace{2em}
    Assume that the risk-free rate is 4\%.

\end{frame}

\begin{frame}[t]
    \frametitle{SPY and TLT}
    \framesubtitle{Empirical Means, Variance-Covariance Matrix}

    In words...\\
    \vspace{1em}

    \textbf{Summary Statistics:}
    \begin{itemize}
        \item Risk Premium (SPY): $\mathbb{E}[r_{SPY}] - r_{f} = 9.4\%$
        \item Risk Premium (TLT): $\mathbb{E}[r_{TLT}] - r_{f} = 3.1\%$
         \item Expected Return (SPY): $\mathbb{E}[r_{SPY}] - r_{f} = 13.4\%$
        \item Expected Return (TLT): $\mathbb{E}[r_{TLT}] - r_{f} = 7.1\%$
        \item Standard Deviation (SPY): $\hat{\sigma}_{SPY} = 14.8\%$
        \item Standard Deviation (TLT): $\hat{\sigma}_{TLT} = 13.6\%$
        \item Covariance: $\widehat{\text{Cov}}(SPY,TLT) = -21.09$
        \item Correlation: $\hat{\rho}_{SPY,TLT} = -0.104$
    \end{itemize}

\end{frame}


\begin{frame}
    \frametitle{SPY and TLT}
    \framesubtitle{Portfolio Construction, cont'd}

    \centering
    \includegraphics[width=0.9\textwidth]{figures/ch7_1_spytlt_er.png} 

\end{frame}

\begin{frame}
    \frametitle{SPY and TLT}
    \framesubtitle{Portfolio Construction, cont'd}

    \centering
    \includegraphics[width=0.9\textwidth]{figures/ch7_1_spytlt_std.png} 

\end{frame}

\begin{frame}
    \frametitle{SPY and TLT}
    \framesubtitle{Portfolio Construction, cont'd}

    \centering
    \includegraphics[width=0.9\textwidth]{figures/ch7_1_spytlt_cov.png} 

\end{frame}

\begin{frame}
    \frametitle{SPY and TLT}
    \framesubtitle{Portfolio Construction, cont'd}

    \centering
    \includegraphics[width=0.9\textwidth]{figures/ch7_1_spytlt_er_cov.png} 

\end{frame}

\begin{frame}
    \frametitle{Optimization: SPY, TLT, and Cash}
    \framesubtitle{Step 1: Maximize the Sharpe Ratio}

    Maximize the Sharpe ratio of the portfolio, which is given by
    $$
    \max_{w_i} S_p = \frac{\mathbb{E}(r_p) - r_f}{\sigma_p}
    $$
    s.t.
    $$
    w_{SPY} + w_{TLT} = 1
    $$
    \footnotesize
    Substitute expressions for $\mathbb{E}(r_p)$, $\sigma_p$ and $w_{TLT}$ 
    $$
    \max_{w_{SPY}} = \frac{w_{SPY}*\mathbb{E}(R_{SPY})+(1-w_{SPY})\mathbb{E}(R_{TLT}) - r_f}{w_{SPY}^{2}*\sigma^{2}_{SPY} + (1-w_{SPY})^{2}*\sigma^{2}_{TLT} + 2*(w_{SPY})*(1-w_{SPY})*Cov(r_{SPY},r_{TLT})}
    $$
    and then take the derivative with respect to $w$, set the derivative equal to zero, and finally solve for $w$.

\end{frame}

\begin{frame}
    \frametitle{Optimization: SPY, TLT, and Cash}
    \framesubtitle{Step 1: Maximize the Sharpe Ratio}

    \centering
    \includegraphics[width=0.9\textwidth]{figures/ch7_1_spytlt_sharpe.png}

\end{frame}

\begin{frame}[t]
    \frametitle{Optimization: SPY, TLT, and Cash}
    \framesubtitle{Step 1: Maximize the Sharpe Ratio}

    Note that the Sharpe Ratio of the portfolio with $w = 0.67$ is \textit{higher}
    than the Sharpe Ratio of the S\&P:

    \begin{table}
        \caption{Sharpe Ratios}
        \begin{tabular}{lrrr}
            \toprule
            & Risk Premium & Std Dev & Sharpe Ratio\\
            \midrule
            SPY & 9.4 & 14.8 & 0.63\\
            TLT & 3.1 & 13.6 & 0.22\\
            \midrule
            67\% SPY & & &\\
            + 33\% TLT & 7.35 & 10.51 & 0.69 \\
            \bottomrule
        \end{tabular}
    \end{table}

\end{frame}

\begin{frame}[t]
    \frametitle{Optimization: SPY, TLT, and Cash}
    \framesubtitle{Step 2: Decide on Cash Allocation}

    Now that you have determined the optimal mix of SPY and TLT, 
    the next decision is how much risk you want to take in your 
    portfolio.   This is accomplished using the capital allocation 
    process we discussed last week, with the only difference being that 
    you are now allocating between cash and the optimal portfolio, 
    rather than cash and the S\&P.\\
    \vspace{1em}
    Note that the allocation line is once again the \textit{capital allocation line (CAL)}, 
    as the Capital Markets Line (CML) only referred to the case where the 
    risky portfolio was the S\&P.

\end{frame}

\begin{frame}
    \frametitle{Optimization: SPY, TLT, and Cash}
    \framesubtitle{Step 2: Decide on Cash Allocation}

    \centering
    \includegraphics[width=0.9\textwidth]{figures/ch7_1_spytlt_sharpe_opt.png}

\end{frame}

\begin{frame}[t]
    \frametitle{Optimization: SPY, TLT, and Cash}
    \framesubtitle{Review}

    In order to find the optimal allocation between three assets (SPY, TLT, and cash), follow these steps:

    \begin{enumerate}
        \item Estimate return and risk characteristics
        \item Find optimal allocation between SPY and TLT, maximize the Sharpe Ratio
        \item Find optimal allocation to cash, using indifference curves
    \end{enumerate}

\end{frame}

\end{document}
