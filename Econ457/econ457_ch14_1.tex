\documentclass{beamer}

\newcommand{\week}{Week 9-a}

\title{Fixed Income}
\subtitle{Reference: Bodie et al, Ch 14}
\author{Econ 457}
\date{\week}

% Reference the shared preamble
\setbeamertemplate{frametitle}{
  \vspace{0.5em}
  \insertframetitle
  \par
  \vspace{0.5em}
  \hrule
  \vspace{0.3em}
  {\small\color{gray}\insertframesubtitle}
}

\setbeamertemplate{navigation symbols}{}
\setbeamertemplate{itemize item}{\textbullet} % main bullet: filled dot
\setbeamertemplate{itemize subitem}{\normalsize$\circ$} % sub-bullet: empty dot
\setbeamertemplate{itemize subsubitem}{\scriptsize--} % sub-sub-bullet: dash


% Font changes
\usepackage[scaled=0.92]{helvet}
\renewcommand{\familydefault}{\sfdefault}

% Packages
\usepackage{tikz}
\usepackage{booktabs}
\usepackage{xcolor}
\usepackage{array}           % Enhanced column types for tables
\usepackage{multirow}        % Spanning multiple rows in tables
\usepackage{makecell}        % Line breaks and formatting in table cells
\usepackage{siunitx}         % Proper formatting of numbers and units
\usepackage{amsmath}         % Enhanced math environments
\usepackage{amsfonts}        % Additional math fonts
\usepackage{amssymb}         % Additional math symbols
\usepackage{url}             % Better URL formatting
\usepackage{graphicx}        % Enhanced graphics support
\usepackage{tabularray}
\UseTblrLibrary{booktabs, siunitx, varwidth}
% For financial presentations specifically
\usepackage{eurosym}         % Euro symbol
\usepackage{textcomp}        % Additional text symbols
\usepackage{hyperref}        % Hyperlinks (should be loaded last)

% Define a footnote
\renewcommand{\footnoterule}{\vspace*{-3pt}\hrule width 2in height 0.4pt\vspace*{2.6pt}}

% Define a Foundation Slide
\newenvironment{foundframe}[1][t]{
    \setbeamercolor{background canvas}{bg=gray!8}
    \setbeamercolor{frametitle}{fg=gray!80!black,bg=gray!25}
    \setbeamercolor{framesubtitle}{fg=gray!70!black,bg=gray!15}
    \setbeamercolor{item}{fg=gray!80!black}
    \setbeamercolor{enumerate item}{fg=gray!80!black}
    
    % Modify the frametitle template for this frame type
    \setbeamertemplate{frametitle}{
        \vspace{0.5em}
        \begin{minipage}[t]{0.75\textwidth}
            \insertframetitle
            \par
            \vspace{0.5em}
            \hrule
            \vspace{0.3em}
            {\small\color{gray}\insertframesubtitle}
        \end{minipage}%
        \hfill
        \begin{minipage}[t]{0.2\textwidth}
            \raggedleft
            \colorbox{gray!30}{%
                \scriptsize\bfseries\color{gray!80!black}%
                   \hspace{3pt}\begin{tabular}{c}Foundation\\Material\end{tabular}\hspace{3pt}%
            }
        \end{minipage}
        \vspace{0.3em}
    }
    
    \begin{frame}[#1]
}{
    \end{frame}
}

% Define Practice Slide
\newenvironment{practiceframe}[1][t]{
    \setbeamercolor{background canvas}{bg=white}
    \setbeamercolor{frametitle}{fg=blue!80!black,bg=blue!15}
    \setbeamercolor{framesubtitle}{fg=blue!70!black,bg=blue!10}
    \setbeamercolor{item}{fg=blue!80!black}
    \setbeamercolor{enumerate item}{fg=blue!80!black}
    \setbeamercolor{normal text}{fg=blue!90!black}
    
    % Modify the frametitle template for this frame type
    \setbeamertemplate{frametitle}{
        \vspace{0.5em}
        \begin{minipage}[t]{0.75\textwidth}
            \insertframetitle
            \par
            \vspace{0.5em}
            \hrule
            \vspace{0.3em}
            {\small\color{blue!70!black}\insertframesubtitle}
        \end{minipage}%
        \hfill
        \begin{minipage}[t]{0.2\textwidth}
            \raggedleft
            \colorbox{blue!20}{%
                \scriptsize\bfseries\color{blue!80!black}%
                   \hspace{3pt}\begin{tabular}{c}Practice\\Questions\end{tabular}\hspace{3pt}%
            }
        \end{minipage}
        \vspace{0.3em}
    }
    
    \begin{frame}[#1]
}{
    \end{frame}
}

% Define Excel Slide
\newenvironment{excelframe}[1][t]{
    \setbeamercolor{background canvas}{bg=white}
    \setbeamercolor{frametitle}{fg=blue!80!black,bg=blue!15}
    \setbeamercolor{framesubtitle}{fg=blue!70!black,bg=blue!10}
    \setbeamercolor{item}{fg=blue!80!black}
    \setbeamercolor{enumerate item}{fg=blue!80!black}
    \setbeamercolor{normal text}{fg=blue!90!black}
    
    % Modify the frametitle template for this frame type
    \setbeamertemplate{frametitle}{
        \vspace{0.5em}
        \begin{minipage}[t]{0.75\textwidth}
            \insertframetitle
            \par
            \vspace{0.5em}
            \hrule
            \vspace{0.3em}
            {\small\color{blue!70!black}\insertframesubtitle}
        \end{minipage}%
        \hfill
        \begin{minipage}[t]{0.2\textwidth}
            \raggedleft
            \colorbox{green!10}{%
                \scriptsize\bfseries\color{blue!80!black}%
                   \hspace{3pt}\begin{tabular}{c}MS Excel\end{tabular}\hspace{3pt}%
            }
        \end{minipage}
        \vspace{0.3em}
    }
    
    \begin{frame}[#1]
}{
    \end{frame}
}

% Define Caution Slide
\newenvironment{cautionframe}[1][t]{
    \setbeamercolor{background canvas}{bg=white}
    \setbeamercolor{frametitle}{fg=blue!80!black,bg=blue!15}
    \setbeamercolor{framesubtitle}{fg=blue!70!black,bg=blue!10}
    \setbeamercolor{item}{fg=blue!80!black}
    \setbeamercolor{enumerate item}{fg=blue!80!black}
    \setbeamercolor{normal text}{fg=blue!90!black}
    
    % Modify the frametitle template for this frame type
    \setbeamertemplate{frametitle}{
        \vspace{0.5em}
        \begin{minipage}[t]{0.75\textwidth}
            \insertframetitle
            \par
            \vspace{0.5em}
            \hrule
            \vspace{0.3em}
            {\small\color{blue!70!black}\insertframesubtitle}
        \end{minipage}%
        \hfill
        \begin{minipage}[t]{0.2\textwidth}
            \raggedleft
            \colorbox{red!10}{%
                \scriptsize\bfseries\color{blue!80!black}%
                   \hspace{3pt}\begin{tabular}{c}Caution\end{tabular}\hspace{3pt}%
            }
        \end{minipage}
        \vspace{0.3em}
    }
    
    \begin{frame}[#1]
}{
    \end{frame}
}

% Add to footnotes
\makeatletter
\newcommand\blfootnote[1]{%
  \begingroup
  \renewcommand\thefootnote{}%
  \renewcommand\@makefntext[1]{\raggedright\leftskip=0pt ##1}%
  \footnote{\scriptsize #1}%
  \addtocounter{footnote}{-1}%
  \endgroup
}
\makeatother

% Set the footer -- change 
\setbeamertemplate{footline}{
  \leavevmode%
  \vspace{2ex}
  \hbox{%
    % Left box: Econ 457
    \begin{beamercolorbox}[wd=.4\paperwidth,ht=2.5ex,dp=1ex,left]{author in head/foot}%
      \hspace{1em}Econ 457
    \end{beamercolorbox}%
    % Middle box: Week
    \begin{beamercolorbox}[wd=.2\paperwidth,ht=2.5ex,dp=1ex,center]{date in head/foot}%
      \centering\week
    \end{beamercolorbox}%
    % Right box: Slide numbers
    \begin{beamercolorbox}[wd=.4\paperwidth,ht=2.5ex,dp=1ex,center]{date in head/foot}%
      \hfill\insertframenumber{} 
    \end{beamercolorbox}%
  }%
  \vskip0pt%
}

\begin{document}

\frame{\titlepage}

\begin{frame}
    \frametitle{Outline}

    \begin{enumerate}
        \item Time Value of Money
        \item Bond Characteristics
        \item Bond Prices
        \item Bond Yields
        \item Excel Formulas
        \item Numerical Soliving Methods
    \end{enumerate}
\end{frame}

\begin{frame}[t]
    \frametitle{1. Time Value of Money}
    \framesubtitle{Basic Concept}

    A dollar today is worth more than a dollar tomorrow.\\
    \vspace{1em}

    \begin{itemize}
        \item \textit{Opportunity Cost:} Money can be invested to earn returns
        \item \textit{Risk:} Future payments are uncertain, and people value uncertain things less
        \item \textit{Inflation:} Purchasing power decreases over time
    \end{itemize}

    $$FV = PV \times (1 + r)^t$$
    Where:
    \begin{itemize}
        \item $FV$ = Future Value
        \item $PV$ = Present Value
        \item $r$ = Interest/discount rate
        \item $t$ = Time periods
    \end{itemize}

\end{frame}

\begin{frame}[t]
    \frametitle{1. Time Value of Money}
    \framesubtitle{Present vs. Future Value}

    \textbf{Future Value (Compounding):}
    $$FV = PV \times (1 + r)^t$$
    \textit{Example:} \$1,000 invested at 5\% for 3 years
    $$FV = 1,000 \times (1.05)^3 = \$1,157.63$$

    \textbf{Present Value (Discounting):}
    $$PV = \frac{FV}{(1 + r)^t}$$
    \textit{Example:} What is \$1,157.63 received in 3 years worth today at 5\%?
    $$PV = \frac{1,157.63}{(1.05)^3} = \$1,000$$

\end{frame}

\begin{frame}[t]
    \frametitle{1. Time Value of Money}
    \framesubtitle{The Rule of 72}

    \textbf{The Rule of 72}: To find how long it takes for money to double at a given rate of return:
    $$\text{Years to Double} \approx \frac{72}{\text{Rate of Return (\%)}}$$
 

    \textbf{Examples:}
    \begin{itemize}
        \item At 6\% interest: $\frac{72}{6} = 12$ years to double
        \item At 8\% interest: $\frac{72}{8} = 9$ years to double
        \item At 12\% interest: $\frac{72}{12} = 6$ years to double
    \end{itemize}
    \vspace{1em}

\end{frame}

\begin{frame}[t]
    \frametitle{1. Time Value of Money}
    \framesubtitle{The Rule of 72}

    For money to double: $FV = 2 \times PV$
    $$2 \times PV = PV \times (1 + r)^t$$
    $$2 = (1 + r)^t$$

    Taking natural logarithm of both sides:
    $$\ln(2) = t \times \ln(1 + r)$$
    $$t = \frac{\ln(2)}{\ln(1 + r)}$$

    \begin{itemize}
        \item $\ln(2) \approx 0.693$
        \item For small $r$: $\ln(1 + r) \approx r$
        \item Therefore: $t \approx \frac{0.693}{r} = \frac{69.3}{r \times 100}$
        \item Rule of 72 uses 72 instead of 69.3 for easier mental math
    \end{itemize}

\end{frame}

\begin{frame}[t]
    \frametitle{1. Time Value of Money}
    \framesubtitle{The Rule of 72}

    At 6\% annually:
    $$FV = 1,000 \times (1.06)^{12} = \$2,012$$
    Close to exactly double.\\
    \vspace{1em}
    Rule of 72 works best for small values of r, say below 15\%.

\end{frame}

\begin{frame}[t]
    \frametitle{2. Bond Characteristics}
    \framesubtitle{Principal}

    At maturity the bond issuer repays the principal of the bond.\\
    \vspace{1em}
    Also referred to as the 'par value' or 'face value' of the bond.\\
    \vspace{1em}
    Most bonds -- including US Treasury bonds and corporate bonds -- 
    typically have par values of \$1,000.  
    For the purposes of examples and Excel formulas, \$100 is also commonly used for the par value.
    
\end{frame}


\begin{frame}[t]
    \frametitle{2. Bond Characteristics}
    \framesubtitle{Coupons}

    The bond issuer makes regular payments to bond holder.  
    These payments are called 'coupon payments.'   The coupon rate usually doesn't change over the life of the bond (hence the term 'fixed income').\\
    $$\text{Coupon} = \text{Coupon Rate (\%)} \cdot \text{Par Value}$$
    Bonds commonly pay coupons 'semi-annually', or twice per year.\\
    \vspace{1em}
    \begin{itemize}
        \item Semi-annual coupon payment = $\frac{\text{Annual Coupon Rate}}{2} \times \text{Par Value}$
        \item Bond math often assumes semi-annual compounding.   
    \end{itemize}

\end{frame}

\begin{frame}[t]
    \frametitle{2. Bond Characteristics}
    \framesubtitle{Coupons}

    \centering
    \includegraphics[width=0.8\textwidth]{figures/ch14_coupons.png}

\end{frame}

\begin{frame}[t]
    \frametitle{2. Bond Characteristics}
    \framesubtitle{Security - US Treasury Bonds}

    US Treasury bonds are  Backed by the full faith and credit of the US government.  Often referred to as "risk-free".\\
    \vspace{1em}
    
    This refers only to \underline{default risk}\\
    \vspace{1em}
    
    Still subject to other risks:
        \begin{itemize}
            \item \textbf{Interest rate risk:} Bond prices fall when rates rise
            \item \textbf{Inflation risk:} Real purchasing power may decline
        \end{itemize}

\end{frame}


\begin{frame}[t]
    \frametitle{2. Bond Characteristics}
    \framesubtitle{Prices}

    Bond prices are set in the market.\\
    \vspace{1em}
    The \textbf{Clean Price} is the quoted market price and \textit{excludes} accrued since the last coupon payment.\\
    \vspace{1em}
    The \textbf{Dirty Price} is the total price paid by the buyer and \textit{includes} the value of interest accrued since the last coupon payment.\\
    \vspace{1em}
     $$\text{Dirty Price} = \text{Clean Price} + \text{Accrued Interest}$$
    \vspace{1em}
    Bond quotes in newspapers/Bloomberg refer to the clean price.

\end{frame}

\begin{frame}[t]
    \frametitle{2. Bond Characteristics}
    \framesubtitle{US Treasury Auctions}

    \textbf{Auction Frequency:}
    \begin{itemize}
        \item \textit{Bills (4w, 8w, 13w, 26w, 52w):} Weekly
        \item \textit{Notes (2y, 3y, 5y, 7y, 10y):} Monthly
        \item \textit{Bonds (20y, 30y):} Monthly
        \item \textit{TIPS:} Quarterly for most maturities
    \end{itemize}
    \vspace{1em}

    \textbf{Auction Timeline:}
    \begin{itemize}
        \item \textit{Announcement:} 3-5 business days before auction
        \item \textit{Auction day:} Bids due by 1:00 PM ET
        \item \textit{Results:} Released within 30 minutes
        \item \textit{Settlement:} Next business day (bills) or 2-3 days (notes/bonds)
    \end{itemize}
    \vspace{1em}

    \textbf{Auction Sizes:} Typically \$40-60 billion per auction for popular maturities (10y, 30y).

\end{frame}

\begin{frame}
    \frametitle{2. Bond Characteristics}
    \framesubtitle{US Treasury Auctions}

    \centering
    \includegraphics[width=0.8\textwidth]{figures/ch14_ust_1.png}

\end{frame}

\begin{frame}
    \frametitle{2. Bond Characteristics}
    \framesubtitle{US Treasury Auctions}

    \centering
    \includegraphics[width=0.8\textwidth]{figures/ch14_ust_3.png}

\end{frame}

\begin{frame}
    \frametitle{3. Bond Prices}
    \framesubtitle{Generic Pricing Formula}

    Generic pricing formula:
    $$
    \text{Bond Price} = \sum_{t=1}^{T} \frac{C}{(1 + y)^t} + \frac{FV}{(1 + y)^T}
    $$

    where $C$ is the coupon amount and $FV$ is the payment due at maturity.\\
    \vspace{1em}

    Calculating the price requires the following inputs:
    \begin{enumerate}
        \item Number of coupon payments ($T$)
        \item Discount rate or yield ($y$).   Typically this the the market interest rate on the bond.
        \item The par value ($FV$)
        \item The coupon amount ($C$), usually Coupon Rate x Face Value.
    \end{enumerate}

\end{frame}

\begin{frame}[t]
    \frametitle{3. Bond Prices}
    \framesubtitle{Generic Pricing Formula}

    Things to remember:
    \begin{itemize}
        \item The $y$ in this equation is more precisely the 'yield to maturity.'
        \item Bodi et al use $r$ for the discount rate, 
    whereas I use $y$ for the yield.   They are the same thing. 
        \item  Be careful to account for semi-annual coupon payments
        \item Be careful to distinguish between coupon payments and 
    the market interest rates.
    \end{itemize}

\end{frame}

\begin{frame}[t]
    \frametitle{3. Bond Prices}
    \framesubtitle{Zero Coupon Bonds}

    Zero coupon bonds have no coupons, and therefore easier math. \\
    $$\text{Zero Coupon Price} = \frac{FV}{(1 + r)^T}$$  
    \vspace{4em}
    %\vspace*{\fill}
    \vfill
    \rule{0.3\textwidth}{0.4pt}
    %\vspace{0.2em}
    \scriptsize

    Zero coupon bonds have other nice characteristics too, as we'll discuss in the lecture on duration and convexity.  
    Zero coupon Treasury bonds are also referred to as "Ps" or as "STRIPS" because they can be created by "stripping" a normal, coupon bond into separate parts.\\

\end{frame}

\begin{frame}[t]
    \frametitle{3. Bond Prices}
    \framesubtitle{Zero Coupon Bonds}

    \centering
    \includegraphics[width=0.8\textwidth]{figures/ch14_1_30y_zeros.png}

\end{frame}

\begin{frame}[t]
    \frametitle{3. Bond Prices}
    \framesubtitle{Prices and Yields are Inversely Related}

    \textbf{Bond prices and yields are inversely related}\\
    \vspace{1em}
    
    \textit{Math}: The yield ($r$) appears in the denominator of the bond pricing equation.\\
    $$\text{Bond Price} = \sum_{t=1}^{T} \frac{C}{(1 + r)^t} + \frac{FV}{(1 + r)^T}$$
    \vspace{1em}
    
    \textit{Intuition}: The price is the present value.   
    Higher yields (discount rates) reduce the present value of the future bond payments, leading to lower bond prices today.\\

\end{frame}

\begin{frame}[t]
    \frametitle{3. Bond Prices}
    \framesubtitle{Coupon Bonds}

    When the yield is equal to the coupon rate, the bond price is equal to the par value.\\
    \vspace{1em}

    Proof uses geometric series (we'll see this in the next lecture).\\
    \vspace{1em}  
    The intuition is that when the coupons equal the yield, 
    the discount rate and the coupon payments (roughly) cancel eachother out.

\end{frame}

\begin{frame}[t]
    \frametitle{3. Bond Prices}
    \framesubtitle{Coupon Bonds}

    \centering
    \includegraphics[width=0.8\textwidth]{figures/ch14_1_30y_coupon.png}

\end{frame}

\begin{frame}[t]
    \frametitle{3. Bond Prices}
    \framesubtitle{Coupon Bonds}

    \centering
    \includegraphics[width=0.8\textwidth]{figures/ch14_1_30y_coupon_2.png}

\end{frame}

\begin{frame}[t]
    \frametitle{3. Bond Prices}
    \framesubtitle{Pull to par}

    At maturity, the bond price is equal to the face value.\\
    \vspace{1em}
    The tendency of the bond price to approach face value as the bond
    approaches maturity is the \textit{pull to par}.  Note that tendency in the graphs on prevous slides.
    
\end{frame}

\begin{frame}[t]
    \frametitle{3. Bond Prices}
    \framesubtitle{Pull to par}

    \centering
    \includegraphics[width=0.5\textwidth]{figures/ch14_1.png}

\end{frame}

\begin{frame}[t]
    \frametitle{4. Bond Yields}
    \framesubtitle{Formula and calculations}

    If the bond price is known, the bond yield can be calculated 
    using the pricing equation and solving for $y$:
   $$\text{Bond Price} = \sum_{t=1}^{T} \frac{C}{(1 + y)^t} + \frac{FV}{(1 + y)^T}$$
    In most cases, there is no closed form expression for $y$.   Numerical solving methods must be used instead.

\end{frame}

\begin{frame}[t]
    \frametitle{4. Bond Yields}
    \framesubtitle{Yields v Prices}

    \centering
    \includegraphics[width=0.8\textwidth]{figures/ch14_1_30y_priceyield.png}

\end{frame}

\begin{frame}[t]
    \frametitle{4. Bond Yields}
    \framesubtitle{Yields v Prices}

    \centering
    \includegraphics[width=0.8\textwidth]{figures/ch14_1_4pct_priceyield.png}

\end{frame}

\begin{frame}[t]
    \frametitle{4. Bond Yields}
    \framesubtitle{Yields v Prices}

    \textit{Bond yields and prices are inversely related.}\\
    \vspace{1em}

    Intuition:
    \begin{enumerate}
        \item Yield in denominator
        \item Reinvestment risk
        \item Yield is similar expected return, one way to get higher E[r] is lower (current)  prices
    \end{enumerate}

\end{frame}

\begin{excelframe}[t]
    \frametitle{5. Excel}
    \framesubtitle{PRICE()}

 \texttt{=PRICE(settlement, maturity, rate, yld, redemption, frequency, [basis])}
    \vspace{1em}

    \begin{itemize}
        \item \texttt{settlement}: Settlement date (when you buy the bond)
        \item \texttt{maturity}: Maturity date (when bond expires)
        \item \texttt{rate}: Annual coupon rate (as decimal: 0.05 for 5\%)
        \item \texttt{yld}: Annual yield to maturity (as decimal: 0.06 for 6\%)
        \item \texttt{redemption}: Redemption value per \$100 face value (usually 100)
        \item \texttt{frequency}: Coupon payments per year (1, 2, or 4)
    \end{itemize}
    \vspace{1em}

    \textbf{Optional Argument:} \texttt{[basis]}: Day count convention (0 = 30/360, 1 = actual/actual, etc.)

\end{excelframe}

\begin{excelframe}[t]
    \frametitle{5. Excel}
    \framesubtitle{YIELD()}

    \texttt{=YIELD(settlement, maturity, rate, pr, redemption, frequency, [basis])}
    \vspace{1em}

    \begin{itemize}
        \item \texttt{settlement}: Settlement date (when you buy the bond)
        \item \texttt{maturity}: Maturity date (when bond expires)
        \item \texttt{rate}: Annual coupon rate (as decimal: 0.05 for 5\%)
        \item \texttt{pr}: Bond price per \$100 face value (e.g., 98.5)
        \item \texttt{redemption}: Redemption value per \$100 face value (usually 100)
        \item \texttt{frequency}: Coupon payments per year (1, 2, or 4)
    \end{itemize}
    \vspace{1em}

    \textbf{Optional Argument:} \texttt{[basis]}: Day count convention (0 = 30/360, 1 = actual/actual, etc.)
   
\end{excelframe}

\begin{excelframe}[t]
    \frametitle{5. Excel}
    \framesubtitle{Notes}

    Yield to maturity is returned as a decimal: 0.06 for 6\%.\\
    \vspace{1em}
    Price is returned as as a percent of par (i.e. for 98 for a bond with par \$1,000 and price of \$980.)\\
    \vspace{1em}
    Excel returns the CLEAN price\\

\end{excelframe}

\begin{excelframe}[t]
    \frametitle{5. Excel}
    \framesubtitle{Notes}

    Coupon and Interest Rates are expected to be \textit{annual rates}.  Convention is that compounding is semi-annual, so \texttt{frequency=2}.\\
    \vspace{1em}
    Excel's calculation is:
    $$P = \sum_{t=1}^{T} \frac{C/\text{freq}}{(1 + y/\text{freq})^t} + \frac{FV}{(1 + y/\text{freq})^T}$$

    Also, Excel deals with the fractions in time, which is relevant in two places:

    \begin{itemize}
        \item Need to compute accrued.  Formula above gives you dirty price.  Clean price = dirty price - accrued.   
        \item The discount rates should accurately reflect time to maturity.   For example, par value should be discounted by $1/(1+y)^T$ where $T = \text{exact time to maturity}$
    \end{itemize}

\end{excelframe}

\begin{foundframe}[t]
    \frametitle{6. Numerical Solving Methods}
    \framesubtitle{Overview}

    \textbf{Problem:} Many equations cannot be solved analytically
    $$f(x) = 0$$
    
    Examples: polynomials of degree $\geq 5$, transcendental equations, complex financial models
    \vspace{1em}

    \textbf{Goal:} Find $x^*$ such that $f(x^*) = 0$ (or very close to zero)
    \vspace{1em}

    \textbf{General Approach:}
    \begin{enumerate}
        \item Start with an initial guess $x_0$
        \item Use an iterative algorithm to improve the guess
        \item Stop when solution is "close enough" to the true answer
    \end{enumerate}
    \vspace{1em}

\end{foundframe}

\begin{foundframe}[t]
    \frametitle{6. Numerical Solving Methods}
    \framesubtitle{Newton Raphson Method}

    \textbf{Method:} Start with initial guess, iteratively improve using derivatives
    $$y_{n+1} = y_n - \frac{f(y_n)}{f'(y_n)}$$
    
    Where $f(y) = \text{Bond Price}(y) - \text{Market Price}$
    \vspace{1em}

    \textbf{Algorithm:}
    \begin{enumerate}
        \item Start with initial guess (e.g., $y_0 = 0.05$)
        \item Calculate $f(y_n)$ and $f'(y_n)$
        \item Update: $y_{n+1} = y_n - \frac{f(y_n)}{f'(y_n)}$
        \item Repeat until $|f(y_n)| < \text{tolerance}$
    \end{enumerate}

\end{foundframe}


\end{document}